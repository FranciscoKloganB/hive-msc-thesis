%% Generated by Sphinx.
\def\sphinxdocclass{report}
\documentclass[letterpaper,10pt,english]{sphinxmanual}
\ifdefined\pdfpxdimen
   \let\sphinxpxdimen\pdfpxdimen\else\newdimen\sphinxpxdimen
\fi \sphinxpxdimen=.75bp\relax

\PassOptionsToPackage{warn}{textcomp}
\usepackage[utf8]{inputenc}
\ifdefined\DeclareUnicodeCharacter
% support both utf8 and utf8x syntaxes
  \ifdefined\DeclareUnicodeCharacterAsOptional
    \def\sphinxDUC#1{\DeclareUnicodeCharacter{"#1}}
  \else
    \let\sphinxDUC\DeclareUnicodeCharacter
  \fi
  \sphinxDUC{00A0}{\nobreakspace}
  \sphinxDUC{2500}{\sphinxunichar{2500}}
  \sphinxDUC{2502}{\sphinxunichar{2502}}
  \sphinxDUC{2514}{\sphinxunichar{2514}}
  \sphinxDUC{251C}{\sphinxunichar{251C}}
  \sphinxDUC{2572}{\textbackslash}
\fi
\usepackage{cmap}
\usepackage[T1]{fontenc}
\usepackage{amsmath,amssymb,amstext}
\usepackage{babel}



\usepackage{times}
\expandafter\ifx\csname T@LGR\endcsname\relax
\else
% LGR was declared as font encoding
  \substitutefont{LGR}{\rmdefault}{cmr}
  \substitutefont{LGR}{\sfdefault}{cmss}
  \substitutefont{LGR}{\ttdefault}{cmtt}
\fi
\expandafter\ifx\csname T@X2\endcsname\relax
  \expandafter\ifx\csname T@T2A\endcsname\relax
  \else
  % T2A was declared as font encoding
    \substitutefont{T2A}{\rmdefault}{cmr}
    \substitutefont{T2A}{\sfdefault}{cmss}
    \substitutefont{T2A}{\ttdefault}{cmtt}
  \fi
\else
% X2 was declared as font encoding
  \substitutefont{X2}{\rmdefault}{cmr}
  \substitutefont{X2}{\sfdefault}{cmss}
  \substitutefont{X2}{\ttdefault}{cmtt}
\fi


\usepackage[Bjarne]{fncychap}
\usepackage{sphinx}

\fvset{fontsize=\small}
\usepackage{geometry}


% Include hyperref last.
\usepackage{hyperref}
% Fix anchor placement for figures with captions.
\usepackage{hypcap}% it must be loaded after hyperref.
% Set up styles of URL: it should be placed after hyperref.
\urlstyle{same}

\addto\captionsenglish{\renewcommand{\contentsname}{Navbar}}

\usepackage{sphinxmessages}
\setcounter{tocdepth}{3}
\setcounter{secnumdepth}{3}


\title{Hives}
\date{Oct 23, 2020}
\release{1.6.0rc1}
\author{Francisco Barros}
\newcommand{\sphinxlogo}{\vbox{}}
\renewcommand{\releasename}{Release}
\makeindex
\begin{document}

\ifdefined\shorthandoff
  \ifnum\catcode`\=\string=\active\shorthandoff{=}\fi
  \ifnum\catcode`\"=\active\shorthandoff{"}\fi
\fi

\pagestyle{empty}
\sphinxmaketitle
\pagestyle{plain}
\sphinxtableofcontents
\pagestyle{normal}
\phantomsection\label{\detokenize{index::doc}}



\chapter{Quickstart}
\label{\detokenize{quickstartdocs:quickstart}}\label{\detokenize{quickstartdocs::doc}}

\section{Technology}
\label{\detokenize{quickstartdocs:technology}}
This simulator uses Python 3.7.7. You are free to use any version you desire,
but we do not guarantee the simulator will work under such conditions.
Any version launched before 3.7.x will not run this project due to retro
compatibility errors. We recommend using an IDE such as PyCharm or equivalent
for easier code inspection, usage and overall faster and, more stable workflows.


\subsection{Installation \sphinxhyphen{} Part I}
\label{\detokenize{quickstartdocs:installation-part-i}}\begin{enumerate}
\sphinxsetlistlabels{\arabic}{enumi}{enumii}{}{.}%
\item {} 
Download and install Python 3.7.x or higher:

\end{enumerate}

\begin{DUlineblock}{0em}
\item[] \sphinxhyphen{} \sphinxurl{https://www.python.org/downloads/release/python-377/}
\end{DUlineblock}
\begin{enumerate}
\sphinxsetlistlabels{\arabic}{enumi}{enumii}{}{.}%
\setcounter{enumi}{1}
\item {} 
Clone our repository at:

\end{enumerate}

\begin{DUlineblock}{0em}
\item[] \sphinxhyphen{} \sphinxurl{https://github.com/FranciscoKloganB/hive-msc-thesis}
\end{DUlineblock}
\begin{enumerate}
\sphinxsetlistlabels{\arabic}{enumi}{enumii}{}{.}%
\setcounter{enumi}{2}
\item {} 
We recommended using JetBrains’ IDEs, but you can skip this step:

\end{enumerate}

\begin{DUlineblock}{0em}
\item[] \sphinxhyphen{} \sphinxurl{https://www.jetbrains.com/pycharm/download}
\end{DUlineblock}

4. Create a virtual environment of your choosing, two example guides are
linked below:

\begin{DUlineblock}{0em}
\item[] \sphinxhyphen{} \sphinxurl{https://packaging.python.org/guides/installing-using-pip-and-virtual-environments/}
\item[] \sphinxhyphen{} \sphinxurl{https://www.jetbrains.com/help/pycharm/creating-virtual-environment.html}
\end{DUlineblock}

5. Navigate to \sphinxcode{\sphinxupquote{hive}} folder located at the root of your recently cloned
project:

\begin{DUlineblock}{0em}
\item[] \sphinxhyphen{} \sphinxcode{\sphinxupquote{\$ cd hive}}
\end{DUlineblock}

6. Install project dependencies by opening your terminal and inserting the
command:

\begin{DUlineblock}{0em}
\item[] \sphinxhyphen{} \sphinxcode{\sphinxupquote{\$ pip install \sphinxhyphen{}r requirements.txt}}
\end{DUlineblock}

The previous steps complete the setup of your Hives project. If you have or
can obtain licenses for \sphinxhref{https://www.mosek.com/products/academic-licenses/}{Mosek} or \sphinxhref{https://https://www.mathworks.com/products/matlab.html}{MatLab}, read {\hyperref[\detokenize{quickstartdocs:installation-part-ii}]{\sphinxcrossref{Installation \sphinxhyphen{} Part II}}},
otherwise read {\hyperref[\detokenize{quickstartdocs:disabling-licensed-components}]{\sphinxcrossref{Disabling Licensed Components}}}.


\subsection{Installation \sphinxhyphen{} Part II}
\label{\detokenize{quickstartdocs:installation-part-ii}}
Throughout the development of the project, a handful of convex optimization
problems had to be solved. We used the \sphinxhref{https://www.cvxpy.org/}{CVXPY} package to tackle that issue;
unfortunately, the solvers available to \sphinxhref{https://www.python.org/}{Python} are few and not very powerful,
specially when it comes to open\sphinxhyphen{}source ones. For semi\sphinxhyphen{}definite programming
problems with utilized \sphinxhref{https://www.mosek.com/products/academic-licenses/}{Mosek}. We let CVXPY select the solver for global
optimization problems, from among the pool of installed solvers. We also use
\sphinxhref{https://yalmip.github.io/solver/bmibnb/}{BMIBNB} solver from \sphinxhref{https://https://www.mathworks.com/products/matlab.html}{MatLab} (through the \sphinxhref{https://www.mathworks.com/help/matlab/matlab-engine-for-python.html}{MatLabEngine}) using \sphinxhref{https://yalmip.github.io/}{YALMIP} because
the latter supports non\sphinxhyphen{}convex constraints.

To use MOSEK along with CVXPY follow the installation instructions linked below:
\begin{enumerate}
\sphinxsetlistlabels{\arabic}{enumi}{enumii}{}{.}%
\item {} 
Mosek licensing quick start.

\end{enumerate}

\begin{DUlineblock}{0em}
\item[] \sphinxhyphen{} \sphinxurl{https://docs.mosek.com/9.2/licensing/quickstart.html}
\end{DUlineblock}
\begin{enumerate}
\sphinxsetlistlabels{\arabic}{enumi}{enumii}{}{.}%
\setcounter{enumi}{1}
\item {} 
Installing Mosek on your python environment.

\end{enumerate}

\begin{DUlineblock}{0em}
\item[] \sphinxhyphen{} \sphinxurl{https://docs.mosek.com/9.2/pythonapi/install-interface.html}
\end{DUlineblock}

For the MatLab Engine to work you need to have
\sphinxhref{https://www.mathworks.com/products/new\_products/latest\_features.html}{MatLab R2020a}
or higher installed on your machine with a valid license. After you installing
and validating the software, you should
\sphinxhref{https://yalmip.github.io/tutorial/installation/}{install YALMIP}. BMIBNB
is bundled with YALMIP by default and no further action is required.


\subsection{Disabling Licensed Components}
\label{\detokenize{quickstartdocs:disabling-licensed-components}}
As explained in the previous session, some licensed components were needed
during the development of our research, which we bundled with the simulator
source code for demonstrative purposes. Concerning \sphinxhref{https://www.mosek.com/products/academic-licenses/}{Mosek}, you do not need to
take any action. Any modules’ functions that use the
\sphinxhref{https://docs.mosek.com/9.2/pythonapi/index.html}{Mosek Optimizer API}
through \sphinxhref{https://www.cvxpy.org/}{CVXPY} check if the package is installed and properly licensed before
using it in favor of other open\sphinxhyphen{}source solvers. Concerning \sphinxhref{https://https://www.mathworks.com/products/matlab.html}{MatLab} you should
not need any further action either \sphinxhyphen{} our modules deal both with invalid
licenses and Pythons’
\sphinxhref{https://docs.python.org/3.7/library/exceptions.html\#AttributeError}{AttributeError}
transparently when invoking \sphinxhref{https://www.mathworks.com/help/matlab/matlab-engine-for-python.html}{MatLabEngine} methods as a result of our singleton,
thread\sphinxhyphen{}safe, implementation of {\hyperref[\detokenize{app.domain.helpers:app.domain.helpers.matlab_utils.MatlabEngineContainer}]{\sphinxcrossref{\sphinxcode{\sphinxupquote{MatlabEngineContainer}}}}}.


\section{Usage}
\label{\detokenize{quickstartdocs:usage}}
A typical usage of the Hives simulator would include the following sequence of
commands (see {\hyperref[\detokenize{scriptdocs:id1}]{\sphinxcrossref{\DUrole{std,std-ref}{Scripts and Flags}}}} section for option details), responding
accordingly to any prompts that appear on your command line terminal:
\begin{quote}

\sphinxcode{\sphinxupquote{\$ cd hive/app}}

\sphinxcode{\sphinxupquote{\$ python simfile\_generator.py \sphinxhyphen{}\sphinxhyphen{}file=test01.json}}

\sphinxcode{\sphinxupquote{\$ python hive\_simulation.py \sphinxhyphen{}\sphinxhyphen{}file=test01.json \sphinxhyphen{}\sphinxhyphen{}iters=30 \sphinxhyphen{}\sphinxhyphen{}epochs=720}}
\end{quote}


\chapter{Scripts and Flags}
\label{\detokenize{scriptdocs:scripts-and-flags}}\label{\detokenize{scriptdocs::doc}}
This section lists all available flags (options) with which each python script
can be ran with. All scripts should be run with the following pattern:
\begin{quote}

\sphinxcode{\sphinxupquote{\$ python \textless{}script\textgreater{} \textless{}options\textgreater{}}}
\end{quote}

Flags whose \sphinxstylestrong{Argument Type} is \sphinxcode{\sphinxupquote{void}} do not require arg values to be
passed when specified in \sphinxstyleemphasis{\textless{}options\textgreater{}}. All other types require an arg value to be
given following the option if specified. Flags whose \sphinxstylestrong{Default Value} is
\sphinxcode{\sphinxupquote{None}} are mandatory to be included in \sphinxstyleemphasis{\textless{}options\textgreater{}}, unless an alternative flag
such as \sphinxcode{\sphinxupquote{\sphinxhyphen{}\sphinxhyphen{}directory}} in \sphinxstyleemphasis{hive\_simulation.py} is given. Example:

\begin{sphinxVerbatim}[commandchars=\\\{\}]
\PYGZdl{} python hive\PYGZus{}simulation.py \PYGZhy{}d \PYGZhy{}i 30 \PYGZhy{}\PYGZhy{}epochs=360
\end{sphinxVerbatim}

Args of type \sphinxstyleemphasis{List{[}str{]}} should be inputed as comma seperated list of string
without blank spaces. Example:

\begin{sphinxVerbatim}[commandchars=\\\{\}]
\PYGZdl{} python mixing\PYGZus{}rate\PYGZus{}sampler.py \PYGZhy{}f a\PYGZus{}func,another\PYGZus{}func,yet\PYGZus{}another\PYGZus{}func
\end{sphinxVerbatim}


\begin{savenotes}\sphinxattablestart
\centering
\begin{tabulary}{\linewidth}[t]{|T|T|T|T|T|}
\hline
\sphinxstartmulticolumn{5}%
\begin{varwidth}[t]{\sphinxcolwidth{5}{5}}
\sphinxstyletheadfamily \sphinxstyleemphasis{simfile\_generator.py}
\par
\vskip-\baselineskip\vbox{\hbox{\strut}}\end{varwidth}%
\sphinxstopmulticolumn
\\
\hline
\sphinxstylestrong{Long option}
&
\sphinxstylestrong{Short Option}
&
\sphinxstylestrong{Argument Type}
&
\sphinxstylestrong{Default Value}
&
\sphinxstylestrong{Description}
\\
\hline
\textendash{}file
&
\sphinxhyphen{}f
&
str
&
\sphinxcode{\sphinxupquote{None}}
&
Creates a simulation file with the specified name inside {\hyperref[\detokenize{app:app.environment_settings.SIMULATION_ROOT}]{\sphinxcrossref{\sphinxcode{\sphinxupquote{SIMULATION\_ROOT}}}}}.
\\
\hline
\end{tabulary}
\par
\sphinxattableend\end{savenotes}


\begin{savenotes}\sphinxattablestart
\centering
\begin{tabulary}{\linewidth}[t]{|T|T|T|T|T|}
\hline
\sphinxstartmulticolumn{5}%
\begin{varwidth}[t]{\sphinxcolwidth{5}{5}}
\sphinxstyletheadfamily \sphinxstyleemphasis{hive\_simulation.py}
\par
\vskip-\baselineskip\vbox{\hbox{\strut}}\end{varwidth}%
\sphinxstopmulticolumn
\\
\hline
\sphinxstylestrong{Long option}
&
\sphinxstylestrong{Short Option}
&
\sphinxstylestrong{Argument Type}
&
\sphinxstylestrong{Default Value}
&
\sphinxstylestrong{Description}
\\
\hline
\textendash{}directory
&
\sphinxhyphen{}d
&
void
&
\sphinxcode{\sphinxupquote{False}}
&
Executes all simulation files available inside {\hyperref[\detokenize{app:app.environment_settings.SIMULATION_ROOT}]{\sphinxcrossref{\sphinxcode{\sphinxupquote{SIMULATION\_ROOT}}}}} folder.
\\
\hline
\textendash{}file
&
\sphinxhyphen{}f
&
str
&
\sphinxcode{\sphinxupquote{None}}
&
Executes the specified file, located inside {\hyperref[\detokenize{app:app.environment_settings.SIMULATION_ROOT}]{\sphinxcrossref{\sphinxcode{\sphinxupquote{SIMULATION\_ROOT}}}}}. Extension must be included. This argument is ignored if \sphinxcode{\sphinxupquote{\sphinxhyphen{}d}} flag is set.
\\
\hline
\textendash{}iters
&
\sphinxhyphen{}i
&
int
&
\sphinxcode{\sphinxupquote{1}}
&
How many times each simulation file is executed.
\\
\hline
\textendash{}epochs
&
\sphinxhyphen{}e
&
int
&
\sphinxcode{\sphinxupquote{480}}
&
Number of discrete time steps (epochs) each executed simulation lasts.
\\
\hline
\textendash{}threading
&
\sphinxhyphen{}t
&
int
&
\sphinxcode{\sphinxupquote{0}}
&
Number of threads the script will create. Each thread runs one simulation at a time.
\\
\hline
\textendash{}master\_server
&
\sphinxhyphen{}m
&
str
&
\sphinxcode{\sphinxupquote{"SGMaster"}}
&
Name of the class that manages the systems’ \sphinxcode{\sphinxupquote{cluster\_groups}} and reads the simulation file (See {\hyperref[\detokenize{app.domain:module-app.domain.master_servers}]{\sphinxcrossref{\sphinxcode{\sphinxupquote{app.domain.master\_servers}}}}})
\\
\hline
\textendash{}cluster\_group
&
\sphinxhyphen{}c
&
str
&
\sphinxcode{\sphinxupquote{"SGClusterExt"}}
&
Name of the class that represents a \sphinxcode{\sphinxupquote{cluster\_group}} (See {\hyperref[\detokenize{app.domain:module-app.domain.cluster_groups}]{\sphinxcrossref{\sphinxcode{\sphinxupquote{app.domain.cluster\_groups}}}}}).
\\
\hline
\textendash{}network\_node
&
\sphinxhyphen{}n
&
str
&
\sphinxcode{\sphinxupquote{"SGNodeExt"}}
&
Name of the class that represents a \sphinxcode{\sphinxupquote{network\_node}} (See {\hyperref[\detokenize{app.domain:module-app.domain.network_nodes}]{\sphinxcrossref{\sphinxcode{\sphinxupquote{app.domain.network\_nodes}}}}}).
\\
\hline
\end{tabulary}
\par
\sphinxattableend\end{savenotes}


\begin{savenotes}\sphinxattablestart
\centering
\begin{tabulary}{\linewidth}[t]{|T|T|T|T|T|}
\hline
\sphinxstartmulticolumn{5}%
\begin{varwidth}[t]{\sphinxcolwidth{5}{5}}
\sphinxstyletheadfamily \sphinxstyleemphasis{mixing\_rate\_sampler.py}
\par
\vskip-\baselineskip\vbox{\hbox{\strut}}\end{varwidth}%
\sphinxstopmulticolumn
\\
\hline
\sphinxstylestrong{Long option}
&
\sphinxstylestrong{Short Option}
&
\sphinxstylestrong{Argument Type}
&
\sphinxstylestrong{Default Value}
&
\sphinxstylestrong{Description}
\\
\hline
\textendash{}network\_sizes
&
\sphinxhyphen{}n
&
Tuple{[}int{]}
&
\sphinxcode{\sphinxupquote{(8, 16)}}
&
The network sizes to be tested.
\\
\hline
\textendash{}samples
&
\sphinxhyphen{}s
&
int
&
\sphinxcode{\sphinxupquote{30}}
&
Number of random adjacency matrices and desired steady\sphinxhyphen{}state vectors should be generated for each network size.
\\
\hline
\textendash{}module
&
\sphinxhyphen{}m
&
str
&
\sphinxcode{\sphinxupquote{"domain.helpers.matrices"}}
&
Name of the module where \sphinxcode{\sphinxupquote{\sphinxhyphen{}f}} flag functions are located.
\\
\hline
\textendash{}functions
&
\sphinxhyphen{}f
&
List{[}str{]}
&
\sphinxcode{\sphinxupquote{{[}"new\_mh\_transition\_matrix", "new\_sdp\_mh\_transition\_matrix", "new\_go\_transition\_matrix", "new\_mgo\_transition\_matrix"{]}}}
&
Name of functions that will use \sphinxcode{\sphinxupquote{\sphinxhyphen{}s}} generated samples to create Markov transition matrices whose mixing rate will be stored in JSON’s output.
\\
\hline
\textendash{}allow\_sloops
&
\sphinxhyphen{}a
&
int
&
\sphinxcode{\sphinxupquote{1}}
&
Indicates if \sphinxcode{\sphinxupquote{\sphinxhyphen{}s}} flag generated adjacency matrices can have \sphinxcode{\sphinxupquote{1}} values in diagonal entries.
\\
\hline
\textendash{}enforce\_sloops
&
\sphinxhyphen{}e
&
int
&
\sphinxcode{\sphinxupquote{1}}
&
Forces \sphinxcode{\sphinxupquote{\sphinxhyphen{}s}} flag generated adjacency matrices to have \sphinxcode{\sphinxupquote{1}} values in diagonal entries. If \sphinxcode{\sphinxupquote{a}} flag is set to \sphinxcode{\sphinxupquote{0}} when \sphinxcode{\sphinxupquote{e}} is set to \sphinxcode{\sphinxupquote{1}}, an error will be raised.
\\
\hline
\end{tabulary}
\par
\sphinxattableend\end{savenotes}


\chapter{App Documentation}
\label{\detokenize{app:app-documentation}}\label{\detokenize{app::doc}}\phantomsection\label{\detokenize{app:module-app}}\index{module@\spxentry{module}!app@\spxentry{app}}\index{app@\spxentry{app}!module@\spxentry{module}}

\section{Subpackages}
\label{\detokenize{app:subpackages}}

\subsection{app.domain}
\label{\detokenize{app.domain:module-app.domain}}\label{\detokenize{app.domain:app-domain}}\label{\detokenize{app.domain::doc}}\index{module@\spxentry{module}!app.domain@\spxentry{app.domain}}\index{app.domain@\spxentry{app.domain}!module@\spxentry{module}}

\subsubsection{Subpackages}
\label{\detokenize{app.domain:subpackages}}

\paragraph{app.domain.helpers}
\label{\detokenize{app.domain.helpers:module-app.domain.helpers}}\label{\detokenize{app.domain.helpers:app-domain-helpers}}\label{\detokenize{app.domain.helpers::doc}}\index{module@\spxentry{module}!app.domain.helpers@\spxentry{app.domain.helpers}}\index{app.domain.helpers@\spxentry{app.domain.helpers}!module@\spxentry{module}}

\subparagraph{Submodules}
\label{\detokenize{app.domain.helpers:submodules}}

\subparagraph{app.domain.helpers.enums}
\label{\detokenize{app.domain.helpers:module-app.domain.helpers.enums}}\label{\detokenize{app.domain.helpers:app-domain-helpers-enums}}\index{module@\spxentry{module}!app.domain.helpers.enums@\spxentry{app.domain.helpers.enums}}\index{app.domain.helpers.enums@\spxentry{app.domain.helpers.enums}!module@\spxentry{module}}
Module with classes that inherit from python’s builtin Enum class.
\index{HttpCodes (class in app.domain.helpers.enums)@\spxentry{HttpCodes}\spxextra{class in app.domain.helpers.enums}}

\begin{fulllineitems}
\phantomsection\label{\detokenize{app.domain.helpers:app.domain.helpers.enums.HttpCodes}}\pysiglinewithargsret{\sphinxbfcode{\sphinxupquote{class }}\sphinxbfcode{\sphinxupquote{HttpCodes}}}{\emph{\DUrole{n}{value}}}{}
Bases: \sphinxhref{https://docs.python.org/3.7/library/enum.html\#enum.Enum}{\sphinxcode{\sphinxupquote{enum.Enum}}}

Enumerator class used to represent HTTP response codes.

The following codes are considered:
\begin{itemize}
\item {} 
DUMMY: Dummy value. Use when no valid HTTP code exists;

\item {} 
OK: Callee accepts the sent file;

\item {} 
BAD\_REQUEST: Callee refuses the integrity of sent file;

\item {} 
NOT\_FOUND: Callee is not a member of the network;

\item {} 
NOT\_ACCEPTABLE: Callee already has a file with same Id;

\item {} 
TIME\_OUT: Message lost in translation;

\item {} 
SERVER\_DOWN: Metadata server is offline;

\end{itemize}
\index{BAD\_REQUEST (HttpCodes attribute)@\spxentry{BAD\_REQUEST}\spxextra{HttpCodes attribute}}

\begin{fulllineitems}
\phantomsection\label{\detokenize{app.domain.helpers:app.domain.helpers.enums.HttpCodes.BAD_REQUEST}}\pysigline{\sphinxbfcode{\sphinxupquote{BAD\_REQUEST}}\sphinxbfcode{\sphinxupquote{: \sphinxhref{https://docs.python.org/3.7/library/functions.html\#int}{int}}}\sphinxbfcode{\sphinxupquote{ = 400}}}
\end{fulllineitems}

\index{DUMMY (HttpCodes attribute)@\spxentry{DUMMY}\spxextra{HttpCodes attribute}}

\begin{fulllineitems}
\phantomsection\label{\detokenize{app.domain.helpers:app.domain.helpers.enums.HttpCodes.DUMMY}}\pysigline{\sphinxbfcode{\sphinxupquote{DUMMY}}\sphinxbfcode{\sphinxupquote{: \sphinxhref{https://docs.python.org/3.7/library/functions.html\#int}{int}}}\sphinxbfcode{\sphinxupquote{ = \sphinxhyphen{}1}}}
\end{fulllineitems}

\index{NOT\_ACCEPTABLE (HttpCodes attribute)@\spxentry{NOT\_ACCEPTABLE}\spxextra{HttpCodes attribute}}

\begin{fulllineitems}
\phantomsection\label{\detokenize{app.domain.helpers:app.domain.helpers.enums.HttpCodes.NOT_ACCEPTABLE}}\pysigline{\sphinxbfcode{\sphinxupquote{NOT\_ACCEPTABLE}}\sphinxbfcode{\sphinxupquote{: \sphinxhref{https://docs.python.org/3.7/library/functions.html\#int}{int}}}\sphinxbfcode{\sphinxupquote{ = 406}}}
\end{fulllineitems}

\index{NOT\_FOUND (HttpCodes attribute)@\spxentry{NOT\_FOUND}\spxextra{HttpCodes attribute}}

\begin{fulllineitems}
\phantomsection\label{\detokenize{app.domain.helpers:app.domain.helpers.enums.HttpCodes.NOT_FOUND}}\pysigline{\sphinxbfcode{\sphinxupquote{NOT\_FOUND}}\sphinxbfcode{\sphinxupquote{: \sphinxhref{https://docs.python.org/3.7/library/functions.html\#int}{int}}}\sphinxbfcode{\sphinxupquote{ = 404}}}
\end{fulllineitems}

\index{OK (HttpCodes attribute)@\spxentry{OK}\spxextra{HttpCodes attribute}}

\begin{fulllineitems}
\phantomsection\label{\detokenize{app.domain.helpers:app.domain.helpers.enums.HttpCodes.OK}}\pysigline{\sphinxbfcode{\sphinxupquote{OK}}\sphinxbfcode{\sphinxupquote{: \sphinxhref{https://docs.python.org/3.7/library/functions.html\#int}{int}}}\sphinxbfcode{\sphinxupquote{ = 200}}}
\end{fulllineitems}

\index{SERVER\_DOWN (HttpCodes attribute)@\spxentry{SERVER\_DOWN}\spxextra{HttpCodes attribute}}

\begin{fulllineitems}
\phantomsection\label{\detokenize{app.domain.helpers:app.domain.helpers.enums.HttpCodes.SERVER_DOWN}}\pysigline{\sphinxbfcode{\sphinxupquote{SERVER\_DOWN}}\sphinxbfcode{\sphinxupquote{: \sphinxhref{https://docs.python.org/3.7/library/functions.html\#int}{int}}}\sphinxbfcode{\sphinxupquote{ = 521}}}
\end{fulllineitems}

\index{TIME\_OUT (HttpCodes attribute)@\spxentry{TIME\_OUT}\spxextra{HttpCodes attribute}}

\begin{fulllineitems}
\phantomsection\label{\detokenize{app.domain.helpers:app.domain.helpers.enums.HttpCodes.TIME_OUT}}\pysigline{\sphinxbfcode{\sphinxupquote{TIME\_OUT}}\sphinxbfcode{\sphinxupquote{: \sphinxhref{https://docs.python.org/3.7/library/functions.html\#int}{int}}}\sphinxbfcode{\sphinxupquote{ = 408}}}
\end{fulllineitems}


\end{fulllineitems}

\index{Status (class in app.domain.helpers.enums)@\spxentry{Status}\spxextra{class in app.domain.helpers.enums}}

\begin{fulllineitems}
\phantomsection\label{\detokenize{app.domain.helpers:app.domain.helpers.enums.Status}}\pysiglinewithargsret{\sphinxbfcode{\sphinxupquote{class }}\sphinxbfcode{\sphinxupquote{Status}}}{\emph{\DUrole{n}{value}}}{}
Bases: \sphinxhref{https://docs.python.org/3.7/library/enum.html\#enum.Enum}{\sphinxcode{\sphinxupquote{enum.Enum}}}

Enumerator that defines connectivity status of a network node.

The following status exist:
\begin{itemize}
\item {} 
SUSPECT: Network node may be offline;

\item {} 
OFFLINE: Network node is offline;

\item {} 
ONLINE: Network node is online;

\end{itemize}
\index{OFFLINE (Status attribute)@\spxentry{OFFLINE}\spxextra{Status attribute}}

\begin{fulllineitems}
\phantomsection\label{\detokenize{app.domain.helpers:app.domain.helpers.enums.Status.OFFLINE}}\pysigline{\sphinxbfcode{\sphinxupquote{OFFLINE}}\sphinxbfcode{\sphinxupquote{: \sphinxhref{https://docs.python.org/3.7/library/functions.html\#int}{int}}}\sphinxbfcode{\sphinxupquote{ = 2}}}
\end{fulllineitems}

\index{ONLINE (Status attribute)@\spxentry{ONLINE}\spxextra{Status attribute}}

\begin{fulllineitems}
\phantomsection\label{\detokenize{app.domain.helpers:app.domain.helpers.enums.Status.ONLINE}}\pysigline{\sphinxbfcode{\sphinxupquote{ONLINE}}\sphinxbfcode{\sphinxupquote{: \sphinxhref{https://docs.python.org/3.7/library/functions.html\#int}{int}}}\sphinxbfcode{\sphinxupquote{ = 3}}}
\end{fulllineitems}

\index{SUSPECT (Status attribute)@\spxentry{SUSPECT}\spxextra{Status attribute}}

\begin{fulllineitems}
\phantomsection\label{\detokenize{app.domain.helpers:app.domain.helpers.enums.Status.SUSPECT}}\pysigline{\sphinxbfcode{\sphinxupquote{SUSPECT}}\sphinxbfcode{\sphinxupquote{: \sphinxhref{https://docs.python.org/3.7/library/functions.html\#int}{int}}}\sphinxbfcode{\sphinxupquote{ = 1}}}
\end{fulllineitems}


\end{fulllineitems}



\subparagraph{app.domain.helpers.exceptions}
\label{\detokenize{app.domain.helpers:module-app.domain.helpers.exceptions}}\label{\detokenize{app.domain.helpers:app-domain-helpers-exceptions}}\index{module@\spxentry{module}!app.domain.helpers.exceptions@\spxentry{app.domain.helpers.exceptions}}\index{app.domain.helpers.exceptions@\spxentry{app.domain.helpers.exceptions}!module@\spxentry{module}}
Module with classes that inherit from Python’s builtin Exception class
and represent domain specific errors.
\index{DistributionShapeError@\spxentry{DistributionShapeError}}

\begin{fulllineitems}
\phantomsection\label{\detokenize{app.domain.helpers:app.domain.helpers.exceptions.DistributionShapeError}}\pysiglinewithargsret{\sphinxbfcode{\sphinxupquote{exception }}\sphinxbfcode{\sphinxupquote{DistributionShapeError}}}{\emph{\DUrole{n}{value}\DUrole{o}{=}\DUrole{default_value}{\textquotesingle{}\textquotesingle{}}}}{}
Bases: \sphinxhref{https://docs.python.org/3.7/library/exceptions.html\#Exception}{\sphinxcode{\sphinxupquote{Exception}}}

Raised when a vector shape is expected to match a matrix’s column or
row shape and does not.
\index{\_\_init\_\_() (DistributionShapeError method)@\spxentry{\_\_init\_\_()}\spxextra{DistributionShapeError method}}

\begin{fulllineitems}
\phantomsection\label{\detokenize{app.domain.helpers:app.domain.helpers.exceptions.DistributionShapeError.__init__}}\pysiglinewithargsret{\sphinxbfcode{\sphinxupquote{\_\_init\_\_}}}{\emph{\DUrole{n}{value}\DUrole{o}{=}\DUrole{default_value}{\textquotesingle{}\textquotesingle{}}}}{}
Initialize self.  See help(type(self)) for accurate signature.

\end{fulllineitems}


\end{fulllineitems}

\index{IllegalArgumentError@\spxentry{IllegalArgumentError}}

\begin{fulllineitems}
\phantomsection\label{\detokenize{app.domain.helpers:app.domain.helpers.exceptions.IllegalArgumentError}}\pysiglinewithargsret{\sphinxbfcode{\sphinxupquote{exception }}\sphinxbfcode{\sphinxupquote{IllegalArgumentError}}}{\emph{\DUrole{n}{value}\DUrole{o}{=}\DUrole{default_value}{\textquotesingle{}\textquotesingle{}}}}{}
Bases: \sphinxhref{https://docs.python.org/3.7/library/exceptions.html\#ValueError}{\sphinxcode{\sphinxupquote{ValueError}}}

Generic error used to indicate a parameter is not valid or expected.
\index{\_\_init\_\_() (IllegalArgumentError method)@\spxentry{\_\_init\_\_()}\spxextra{IllegalArgumentError method}}

\begin{fulllineitems}
\phantomsection\label{\detokenize{app.domain.helpers:app.domain.helpers.exceptions.IllegalArgumentError.__init__}}\pysiglinewithargsret{\sphinxbfcode{\sphinxupquote{\_\_init\_\_}}}{\emph{\DUrole{n}{value}\DUrole{o}{=}\DUrole{default_value}{\textquotesingle{}\textquotesingle{}}}}{}
Initialize self.  See help(type(self)) for accurate signature.

\end{fulllineitems}


\end{fulllineitems}

\index{MatrixError@\spxentry{MatrixError}}

\begin{fulllineitems}
\phantomsection\label{\detokenize{app.domain.helpers:app.domain.helpers.exceptions.MatrixError}}\pysiglinewithargsret{\sphinxbfcode{\sphinxupquote{exception }}\sphinxbfcode{\sphinxupquote{MatrixError}}}{\emph{\DUrole{n}{value}\DUrole{o}{=}\DUrole{default_value}{\textquotesingle{}\textquotesingle{}}}}{}
Bases: \sphinxhref{https://docs.python.org/3.7/library/exceptions.html\#Exception}{\sphinxcode{\sphinxupquote{Exception}}}

Generic Matrix Error, can be used for example when a pandas DataFrame
has shape (0, 0) or is one\sphinxhyphen{}dimensional.
\index{\_\_init\_\_() (MatrixError method)@\spxentry{\_\_init\_\_()}\spxextra{MatrixError method}}

\begin{fulllineitems}
\phantomsection\label{\detokenize{app.domain.helpers:app.domain.helpers.exceptions.MatrixError.__init__}}\pysiglinewithargsret{\sphinxbfcode{\sphinxupquote{\_\_init\_\_}}}{\emph{\DUrole{n}{value}\DUrole{o}{=}\DUrole{default_value}{\textquotesingle{}\textquotesingle{}}}}{}
Initialize self.  See help(type(self)) for accurate signature.

\end{fulllineitems}


\end{fulllineitems}

\index{MatrixNotSquareError@\spxentry{MatrixNotSquareError}}

\begin{fulllineitems}
\phantomsection\label{\detokenize{app.domain.helpers:app.domain.helpers.exceptions.MatrixNotSquareError}}\pysiglinewithargsret{\sphinxbfcode{\sphinxupquote{exception }}\sphinxbfcode{\sphinxupquote{MatrixNotSquareError}}}{\emph{\DUrole{n}{value}\DUrole{o}{=}\DUrole{default_value}{\textquotesingle{}\textquotesingle{}}}}{}
Bases: \sphinxhref{https://docs.python.org/3.7/library/exceptions.html\#Exception}{\sphinxcode{\sphinxupquote{Exception}}}

Raised when a function expects a square (N * N) matrix in any
representation and some other dimensions are given.
\index{\_\_init\_\_() (MatrixNotSquareError method)@\spxentry{\_\_init\_\_()}\spxextra{MatrixNotSquareError method}}

\begin{fulllineitems}
\phantomsection\label{\detokenize{app.domain.helpers:app.domain.helpers.exceptions.MatrixNotSquareError.__init__}}\pysiglinewithargsret{\sphinxbfcode{\sphinxupquote{\_\_init\_\_}}}{\emph{\DUrole{n}{value}\DUrole{o}{=}\DUrole{default_value}{\textquotesingle{}\textquotesingle{}}}}{}
Initialize self.  See help(type(self)) for accurate signature.

\end{fulllineitems}


\end{fulllineitems}



\subparagraph{app.domain.helpers.matlab\_utils}
\label{\detokenize{app.domain.helpers:module-app.domain.helpers.matlab_utils}}\label{\detokenize{app.domain.helpers:app-domain-helpers-matlab-utils}}\index{module@\spxentry{module}!app.domain.helpers.matlab\_utils@\spxentry{app.domain.helpers.matlab\_utils}}\index{app.domain.helpers.matlab\_utils@\spxentry{app.domain.helpers.matlab\_utils}!module@\spxentry{module}}
Module with Matlab related classes.
\index{MatlabEngineContainer (class in app.domain.helpers.matlab\_utils)@\spxentry{MatlabEngineContainer}\spxextra{class in app.domain.helpers.matlab\_utils}}

\begin{fulllineitems}
\phantomsection\label{\detokenize{app.domain.helpers:app.domain.helpers.matlab_utils.MatlabEngineContainer}}\pysigline{\sphinxbfcode{\sphinxupquote{class }}\sphinxbfcode{\sphinxupquote{MatlabEngineContainer}}}
Bases: \sphinxhref{https://docs.python.org/3.7/library/functions.html\#object}{\sphinxcode{\sphinxupquote{object}}}

Singleton class wrapper containing thread safe access to a MatlabEngine.

This class provides a thread\sphinxhyphen{}safe way to access one singleton
matlab engine object when running simulations in threaded mode. Having
one single engine is important, since starting up an engine takes
approximately 12s (machine dependant), not including the time matlab
scripts are executing and data convertions between python and matlab and
back.
\index{eng (MatlabEngineContainer attribute)@\spxentry{eng}\spxextra{MatlabEngineContainer attribute}}

\begin{fulllineitems}
\phantomsection\label{\detokenize{app.domain.helpers:app.domain.helpers.matlab_utils.MatlabEngineContainer.eng}}\pysigline{\sphinxbfcode{\sphinxupquote{eng}}}
A matlab engine instance, which can be used for example for matrix
and vector optimization operations throughout the simulations.

\begin{sphinxadmonition}{note}{Note:}
MatlabEngine objects are not thread safe, thus it
is recommended that you utilize  the a wrapper function that
obtains {\hyperref[\detokenize{app.domain.helpers:app.domain.helpers.matlab_utils.MatlabEngineContainer._LOCK}]{\sphinxcrossref{\sphinxcode{\sphinxupquote{\_LOCK}}}}}, before you send any requests to
\sphinxcode{\sphinxupquote{eng}}.
\end{sphinxadmonition}

\end{fulllineitems}

\index{\_\_init\_\_() (MatlabEngineContainer method)@\spxentry{\_\_init\_\_()}\spxextra{MatlabEngineContainer method}}

\begin{fulllineitems}
\phantomsection\label{\detokenize{app.domain.helpers:app.domain.helpers.matlab_utils.MatlabEngineContainer.__init__}}\pysiglinewithargsret{\sphinxbfcode{\sphinxupquote{\_\_init\_\_}}}{}{}
Instantiates a new MatlabEngineContainer object.

\begin{sphinxadmonition}{note}{Note:}
Do not directly invoke constructor, use {\hyperref[\detokenize{app.domain.helpers:app.domain.helpers.matlab_utils.MatlabEngineContainer.get_instance}]{\sphinxcrossref{\sphinxcode{\sphinxupquote{get\_instance()}}}}}
instead.
\end{sphinxadmonition}
\begin{quote}\begin{description}
\item[{Return type}] \leavevmode
\sphinxhref{https://docs.python.org/3.7/library/constants.html\#None}{None}

\end{description}\end{quote}

\end{fulllineitems}

\index{get\_instance() (MatlabEngineContainer static method)@\spxentry{get\_instance()}\spxextra{MatlabEngineContainer static method}}

\begin{fulllineitems}
\phantomsection\label{\detokenize{app.domain.helpers:app.domain.helpers.matlab_utils.MatlabEngineContainer.get_instance}}\pysiglinewithargsret{\sphinxbfcode{\sphinxupquote{static }}\sphinxbfcode{\sphinxupquote{get\_instance}}}{}{}
Used to obtain a singleton instance of \sphinxcode{\sphinxupquote{MatlabEngineContainer}}.

If one instance already exists that instance is returned,
otherwise a new one is created and returned.
\begin{quote}\begin{description}
\item[{Returns}] \leavevmode
A reference to the existing \sphinxcode{\sphinxupquote{MatlabEngineContainer}}
{\hyperref[\detokenize{app.domain.helpers:app.domain.helpers.matlab_utils.MatlabEngineContainer._instance}]{\sphinxcrossref{\sphinxcode{\sphinxupquote{instance}}}}}.

\item[{Return type}] \leavevmode
{\hyperref[\detokenize{app.domain.helpers:app.domain.helpers.matlab_utils.MatlabEngineContainer}]{\sphinxcrossref{app.domain.helpers.matlab\_utils.MatlabEngineContainer}}}

\end{description}\end{quote}

\end{fulllineitems}

\index{matrix\_global\_opt() (MatlabEngineContainer method)@\spxentry{matrix\_global\_opt()}\spxextra{MatlabEngineContainer method}}

\begin{fulllineitems}
\phantomsection\label{\detokenize{app.domain.helpers:app.domain.helpers.matlab_utils.MatlabEngineContainer.matrix_global_opt}}\pysiglinewithargsret{\sphinxbfcode{\sphinxupquote{matrix\_global\_opt}}}{\emph{\DUrole{n}{a}}, \emph{\DUrole{n}{v\_}}}{}
Constructs an optimized transition matrix using the matlab engine.

Constructs an optimized transition matrix using linear programming
relaxations and convex envelope approximations for the specified steady
state \sphinxcode{\sphinxupquote{v\_}}, this is done by invoke the matlabscript
\sphinxstyleemphasis{matrixGlobalOpt.m} located inside
{\hyperref[\detokenize{app:app.environment_settings.MATLAB_DIR}]{\sphinxcrossref{\sphinxcode{\sphinxupquote{MATLAB\_DIR}}}}}.
\begin{quote}\begin{description}
\item[{Parameters}] \leavevmode\begin{itemize}
\item {} 
\sphinxstyleliteralstrong{\sphinxupquote{a}} (\sphinxhref{https://numpy.org/doc/stable/reference/generated/numpy.ndarray.html\#numpy.ndarray}{\sphinxstyleliteralemphasis{\sphinxupquote{numpy.ndarray}}}) \textendash{} A non\sphinxhyphen{}optimized symmetric adjency matrix.

\item {} 
\sphinxstyleliteralstrong{\sphinxupquote{v\_}} (\sphinxhref{https://numpy.org/doc/stable/reference/generated/numpy.ndarray.html\#numpy.ndarray}{\sphinxstyleliteralemphasis{\sphinxupquote{numpy.ndarray}}}) \textendash{} A stochastic steady state distribution vector.

\end{itemize}

\item[{Returns}] \leavevmode
Markov Matrix with \sphinxcode{\sphinxupquote{v\_}} as steady state distribution and the
respective {\hyperref[\detokenize{app.domain.helpers:app.domain.helpers.matrices.get_mixing_rate}]{\sphinxcrossref{\sphinxcode{\sphinxupquote{mixing rate}}}}} or \sphinxcode{\sphinxupquote{None}}.

\item[{Raises}] \leavevmode
\sphinxstyleliteralstrong{\sphinxupquote{EngineError}} \textendash{} If you do not have a valid MatLab license.

\item[{Return type}] \leavevmode
Any

\end{description}\end{quote}

\end{fulllineitems}

\index{\_LOCK (MatlabEngineContainer attribute)@\spxentry{\_LOCK}\spxextra{MatlabEngineContainer attribute}}

\begin{fulllineitems}
\phantomsection\label{\detokenize{app.domain.helpers:app.domain.helpers.matlab_utils.MatlabEngineContainer._LOCK}}\pysigline{\sphinxbfcode{\sphinxupquote{\_LOCK}}\sphinxbfcode{\sphinxupquote{ = \textless{}unlocked \_thread.RLock object owner=0 count=0\textgreater{}}}}
A re\sphinxhyphen{}entrant lock used to make \sphinxtitleref{eng} shareable by multiple threads.

\end{fulllineitems}

\index{\_instance (MatlabEngineContainer attribute)@\spxentry{\_instance}\spxextra{MatlabEngineContainer attribute}}

\begin{fulllineitems}
\phantomsection\label{\detokenize{app.domain.helpers:app.domain.helpers.matlab_utils.MatlabEngineContainer._instance}}\pysigline{\sphinxbfcode{\sphinxupquote{\_instance}}\sphinxbfcode{\sphinxupquote{: {\hyperref[\detokenize{app.domain.helpers:app.domain.helpers.matlab_utils.MatlabEngineContainer}]{\sphinxcrossref{app.domain.helpers.matlab\_utils.MatlabEngineContainer}}}}}\sphinxbfcode{\sphinxupquote{ = None}}}
A reference to the instance of \sphinxtitleref{MatlabEngineContainer} or \sphinxtitleref{None}.

\end{fulllineitems}


\end{fulllineitems}



\subparagraph{app.domain.helpers.matrices}
\label{\detokenize{app.domain.helpers:module-app.domain.helpers.matrices}}\label{\detokenize{app.domain.helpers:app-domain-helpers-matrices}}\index{module@\spxentry{module}!app.domain.helpers.matrices@\spxentry{app.domain.helpers.matrices}}\index{app.domain.helpers.matrices@\spxentry{app.domain.helpers.matrices}!module@\spxentry{module}}
Module used by \sphinxcode{\sphinxupquote{domain.cluster\_groups.Cluster}} to create transition
matrices for the simulation.

You should implement your own metropolis\sphinxhyphen{}hastings or alternative algorithms
as well as any steady\sphinxhyphen{}state or transition matrix optimization algorithms in
this module.
\index{\_adjency\_matrix\_sdp\_optimization() (in module app.domain.helpers.matrices)@\spxentry{\_adjency\_matrix\_sdp\_optimization()}\spxextra{in module app.domain.helpers.matrices}}

\begin{fulllineitems}
\phantomsection\label{\detokenize{app.domain.helpers:app.domain.helpers.matrices._adjency_matrix_sdp_optimization}}\pysiglinewithargsret{\sphinxbfcode{\sphinxupquote{\_adjency\_matrix\_sdp\_optimization}}}{\emph{\DUrole{n}{a}}}{}
Optimizes a symmetric adjacency matrix using Semidefinite Programming.

The optimization is done with respect to the uniform stochastic vector
with the the same length as the inputed symmetric matrix.

\begin{sphinxadmonition}{note}{Note:}
This function tries to use
\sphinxhref{https://docs.mosek.com/9.2/pythonapi/index.html}{Mosek Solver},
if a valid license is not found, it uses
\sphinxhref{https://github.com/cvxgrp/scs}{SCS Solver} instead.
\end{sphinxadmonition}
\begin{quote}\begin{description}
\item[{Parameters}] \leavevmode
\sphinxstyleliteralstrong{\sphinxupquote{a}} (\sphinxhref{https://numpy.org/doc/stable/reference/generated/numpy.ndarray.html\#numpy.ndarray}{\sphinxstyleliteralemphasis{\sphinxupquote{numpy.ndarray}}}) \textendash{} Any symmetric adjacency matrix.

\item[{Returns}] \leavevmode
The optimal matrix or None if the problem is unfeasible.

\item[{Return type}] \leavevmode
Optional{[}Tuple{[}cvxpy.problems.problem.Problem, cvxpy.expressions.variable.Variable{]}{]}

\end{description}\end{quote}

\end{fulllineitems}

\index{\_construct\_random\_walk\_matrix() (in module app.domain.helpers.matrices)@\spxentry{\_construct\_random\_walk\_matrix()}\spxextra{in module app.domain.helpers.matrices}}

\begin{fulllineitems}
\phantomsection\label{\detokenize{app.domain.helpers:app.domain.helpers.matrices._construct_random_walk_matrix}}\pysiglinewithargsret{\sphinxbfcode{\sphinxupquote{\_construct\_random\_walk\_matrix}}}{\emph{\DUrole{n}{a}}}{}
Builds a random walk matrix over the given adjacency matrix
\begin{quote}\begin{description}
\item[{Parameters}] \leavevmode
\sphinxstyleliteralstrong{\sphinxupquote{a}} (\sphinxhref{https://numpy.org/doc/stable/reference/generated/numpy.ndarray.html\#numpy.ndarray}{\sphinxstyleliteralemphasis{\sphinxupquote{numpy.ndarray}}}) \textendash{} Any adjacency matrix.

\item[{Returns}] \leavevmode
A matrix representing the performed random walk.

\item[{Return type}] \leavevmode
\sphinxhref{https://numpy.org/doc/stable/reference/generated/numpy.ndarray.html\#numpy.ndarray}{numpy.ndarray}

\end{description}\end{quote}

\end{fulllineitems}

\index{\_construct\_rejection\_matrix() (in module app.domain.helpers.matrices)@\spxentry{\_construct\_rejection\_matrix()}\spxextra{in module app.domain.helpers.matrices}}

\begin{fulllineitems}
\phantomsection\label{\detokenize{app.domain.helpers:app.domain.helpers.matrices._construct_rejection_matrix}}\pysiglinewithargsret{\sphinxbfcode{\sphinxupquote{\_construct\_rejection\_matrix}}}{\emph{\DUrole{n}{rw}}, \emph{\DUrole{n}{v\_}}}{}
Builds a matrix of rejection probabilities for a given random walk.
\begin{quote}\begin{description}
\item[{Parameters}] \leavevmode\begin{itemize}
\item {} 
\sphinxstyleliteralstrong{\sphinxupquote{rw}} (\sphinxhref{https://numpy.org/doc/stable/reference/generated/numpy.ndarray.html\#numpy.ndarray}{\sphinxstyleliteralemphasis{\sphinxupquote{numpy.ndarray}}}) \textendash{} a random\_walk over an adjacency matrix

\item {} 
\sphinxstyleliteralstrong{\sphinxupquote{v\_}} (\sphinxhref{https://numpy.org/doc/stable/reference/generated/numpy.ndarray.html\#numpy.ndarray}{\sphinxstyleliteralemphasis{\sphinxupquote{numpy.ndarray}}}) \textendash{} a stochastic desired distribution vector

\end{itemize}

\item[{Returns}] \leavevmode
A matrix whose entries are acceptance probabilities for \sphinxcode{\sphinxupquote{rw}}.

\item[{Return type}] \leavevmode
\sphinxhref{https://numpy.org/doc/stable/reference/generated/numpy.ndarray.html\#numpy.ndarray}{numpy.ndarray}

\end{description}\end{quote}

\end{fulllineitems}

\index{\_get\_diagonal\_entry\_probability\_v1() (in module app.domain.helpers.matrices)@\spxentry{\_get\_diagonal\_entry\_probability\_v1()}\spxextra{in module app.domain.helpers.matrices}}

\begin{fulllineitems}
\phantomsection\label{\detokenize{app.domain.helpers:app.domain.helpers.matrices._get_diagonal_entry_probability_v1}}\pysiglinewithargsret{\sphinxbfcode{\sphinxupquote{\_get\_diagonal\_entry\_probability\_v1}}}{\emph{\DUrole{n}{rw}}, \emph{\DUrole{n}{r}}, \emph{\DUrole{n}{i}}}{}
Helper function used during the metropolis\sphinxhyphen{}hastings algorithm.

Calculates the value that should be assigned to the entry \sphinxcode{\sphinxupquote{(i, i)}} of the
transition matrix being calculated by the metropolis hastings algorithm
by considering the rejection probability over the random walk that was
performed on an adjacency matrix.

\begin{sphinxadmonition}{note}{Note:}
This method does considers element\sphinxhyphen{}wise rejection probabilities
for random walk matrices. If you wish to implement a modification of
the metropolis\sphinxhyphen{}hastings algorithm and you do not utilize rejection
matrices use {\hyperref[\detokenize{app.domain.helpers:app.domain.helpers.matrices._get_diagonal_entry_probability_v2}]{\sphinxcrossref{\sphinxcode{\sphinxupquote{\_get\_diagonal\_entry\_probability\_v2()}}}}} instead.
\end{sphinxadmonition}
\begin{quote}\begin{description}
\item[{Parameters}] \leavevmode\begin{itemize}
\item {} 
\sphinxstyleliteralstrong{\sphinxupquote{rw}} (\sphinxhref{https://numpy.org/doc/stable/reference/generated/numpy.ndarray.html\#numpy.ndarray}{\sphinxstyleliteralemphasis{\sphinxupquote{numpy.ndarray}}}) \textendash{} A random walk over an adjacency matrix.

\item {} 
\sphinxstyleliteralstrong{\sphinxupquote{r}} (\sphinxhref{https://numpy.org/doc/stable/reference/generated/numpy.ndarray.html\#numpy.ndarray}{\sphinxstyleliteralemphasis{\sphinxupquote{numpy.ndarray}}}) \textendash{} A matrix whose entries contain acceptance probabilities for \sphinxcode{\sphinxupquote{rw}}.

\item {} 
\sphinxstyleliteralstrong{\sphinxupquote{i}} (\sphinxhref{https://docs.python.org/3.7/library/functions.html\#int}{\sphinxstyleliteralemphasis{\sphinxupquote{int}}}) \textendash{} The diagonal\sphinxhyphen{}index of \sphinxcode{\sphinxupquote{rw}} where summation needs to
be performed on. E.g.: \sphinxcode{\sphinxupquote{rw{[}i, i{]}}}.

\end{itemize}

\item[{Returns}] \leavevmode
A probability to be inserted at entry \sphinxcode{\sphinxupquote{(i, i)}} of the transition
matrix outputed by the {\hyperref[\detokenize{app.domain.helpers:app.domain.helpers.matrices._metropolis_hastings}]{\sphinxcrossref{\sphinxcode{\sphinxupquote{\_metropolis\_hastings()}}}}}.

\item[{Return type}] \leavevmode
numpy.float64

\end{description}\end{quote}

\end{fulllineitems}

\index{\_get\_diagonal\_entry\_probability\_v2() (in module app.domain.helpers.matrices)@\spxentry{\_get\_diagonal\_entry\_probability\_v2()}\spxextra{in module app.domain.helpers.matrices}}

\begin{fulllineitems}
\phantomsection\label{\detokenize{app.domain.helpers:app.domain.helpers.matrices._get_diagonal_entry_probability_v2}}\pysiglinewithargsret{\sphinxbfcode{\sphinxupquote{\_get\_diagonal\_entry\_probability\_v2}}}{\emph{\DUrole{n}{m}}, \emph{\DUrole{n}{i}}}{}
Helper function used during the metropolis\sphinxhyphen{}hastings algorithm.

Calculates the value that should be assigned to the entry \sphinxcode{\sphinxupquote{(i, i)}} of the
transition matrix being calculated by the metropolis hastings algorithm
by considering the rejection probability over the random walk that was
performed on an adjacency matrix.

\begin{sphinxadmonition}{note}{Note:}
This method does not consider element\sphinxhyphen{}wise rejection probabilities
for random walk matrices. If you wish to implement a modification of
the metropolis\sphinxhyphen{}hastings algorithm and you utilize rejection matrices
use {\hyperref[\detokenize{app.domain.helpers:app.domain.helpers.matrices._get_diagonal_entry_probability_v1}]{\sphinxcrossref{\sphinxcode{\sphinxupquote{\_get\_diagonal\_entry\_probability\_v1()}}}}} instead.
\end{sphinxadmonition}
\begin{quote}\begin{description}
\item[{Parameters}] \leavevmode\begin{itemize}
\item {} 
\sphinxstyleliteralstrong{\sphinxupquote{m}} (\sphinxhref{https://numpy.org/doc/stable/reference/generated/numpy.ndarray.html\#numpy.ndarray}{\sphinxstyleliteralemphasis{\sphinxupquote{numpy.ndarray}}}) \textendash{} The matrix to receive the diagonal entry value.

\item {} 
\sphinxstyleliteralstrong{\sphinxupquote{i}} (\sphinxhref{https://docs.python.org/3.7/library/functions.html\#int}{\sphinxstyleliteralemphasis{\sphinxupquote{int}}}) \textendash{} The diagonal entry index. E.g.: \sphinxcode{\sphinxupquote{m{[}i, i{]}}}.

\end{itemize}

\item[{Returns}] \leavevmode
A probability to be inserted at entry \sphinxcode{\sphinxupquote{(i, i)}} of the transition matrix
outputed by the {\hyperref[\detokenize{app.domain.helpers:app.domain.helpers.matrices._metropolis_hastings}]{\sphinxcrossref{\sphinxcode{\sphinxupquote{\_metropolis\_hastings()}}}}}.

\item[{Return type}] \leavevmode
numpy.float64

\end{description}\end{quote}

\end{fulllineitems}

\index{\_metropolis\_hastings() (in module app.domain.helpers.matrices)@\spxentry{\_metropolis\_hastings()}\spxextra{in module app.domain.helpers.matrices}}

\begin{fulllineitems}
\phantomsection\label{\detokenize{app.domain.helpers:app.domain.helpers.matrices._metropolis_hastings}}\pysiglinewithargsret{\sphinxbfcode{\sphinxupquote{\_metropolis\_hastings}}}{\emph{\DUrole{n}{a}}, \emph{\DUrole{n}{v\_}}, \emph{\DUrole{n}{column\_major\_out}\DUrole{o}{=}\DUrole{default_value}{True}}, \emph{\DUrole{n}{version}\DUrole{o}{=}\DUrole{default_value}{2}}}{}
Constructs a transition matrix using metropolis\sphinxhyphen{}hastings algorithm.

\begin{sphinxadmonition}{note}{Note:}
The input Matrix hould have no transient states/absorbent nodes,
but this is not enforced or verified.
\end{sphinxadmonition}
\begin{quote}\begin{description}
\item[{Parameters}] \leavevmode\begin{itemize}
\item {} 
\sphinxstyleliteralstrong{\sphinxupquote{a}} (\sphinxhref{https://numpy.org/doc/stable/reference/generated/numpy.ndarray.html\#numpy.ndarray}{\sphinxstyleliteralemphasis{\sphinxupquote{numpy.ndarray}}}) \textendash{} A symmetric adjency matrix.

\item {} 
\sphinxstyleliteralstrong{\sphinxupquote{v\_}} (\sphinxhref{https://numpy.org/doc/stable/reference/generated/numpy.ndarray.html\#numpy.ndarray}{\sphinxstyleliteralemphasis{\sphinxupquote{numpy.ndarray}}}) \textendash{} A stochastic vector that is the steady state of the resulting
transition matrix.

\item {} 
\sphinxstyleliteralstrong{\sphinxupquote{column\_major\_out}} (\sphinxhref{https://docs.python.org/3.7/library/functions.html\#bool}{\sphinxstyleliteralemphasis{\sphinxupquote{bool}}}) \textendash{} Indicates whether to return transition\_matrix output
is in row or column major form.

\item {} 
\sphinxstyleliteralstrong{\sphinxupquote{version}} (\sphinxhref{https://docs.python.org/3.7/library/functions.html\#int}{\sphinxstyleliteralemphasis{\sphinxupquote{int}}}) \textendash{} Indicates which version of the algorith should be used
(default is version 2).

\end{itemize}

\item[{Returns}] \leavevmode
An unlabeled transition matrix with steady state \sphinxcode{\sphinxupquote{v\_}}.

\item[{Raises}] \leavevmode\begin{itemize}
\item {} 
{\hyperref[\detokenize{app.domain.helpers:app.domain.helpers.exceptions.DistributionShapeError}]{\sphinxcrossref{\sphinxstyleliteralstrong{\sphinxupquote{DistributionShapeError}}}}} \textendash{} When the length of \sphinxcode{\sphinxupquote{v\_}} is not the same as the matrix \sphinxtitleref{a}.

\item {} 
{\hyperref[\detokenize{app.domain.helpers:app.domain.helpers.exceptions.MatrixNotSquareError}]{\sphinxcrossref{\sphinxstyleliteralstrong{\sphinxupquote{MatrixNotSquareError}}}}} \textendash{} When matrix \sphinxtitleref{a} is not a square matrix.

\end{itemize}

\item[{Return type}] \leavevmode
\sphinxhref{https://numpy.org/doc/stable/reference/generated/numpy.ndarray.html\#numpy.ndarray}{numpy.ndarray}

\end{description}\end{quote}

\end{fulllineitems}

\index{get\_mixing\_rate() (in module app.domain.helpers.matrices)@\spxentry{get\_mixing\_rate()}\spxextra{in module app.domain.helpers.matrices}}

\begin{fulllineitems}
\phantomsection\label{\detokenize{app.domain.helpers:app.domain.helpers.matrices.get_mixing_rate}}\pysiglinewithargsret{\sphinxbfcode{\sphinxupquote{get\_mixing\_rate}}}{\emph{\DUrole{n}{m}}}{}
Calculats the fast mixing rate the input matrix.

The fast mixing rate of matrix \sphinxcode{\sphinxupquote{m}} is the highest eigenvalue that is
smaller than one. If returned value is \sphinxcode{\sphinxupquote{1.0}} than the matrix has transient
states or absorbent nodes and as a result is not a markov matrix.
\begin{quote}\begin{description}
\item[{Parameters}] \leavevmode
\sphinxstyleliteralstrong{\sphinxupquote{m}} (\sphinxhref{https://numpy.org/doc/stable/reference/generated/numpy.ndarray.html\#numpy.ndarray}{\sphinxstyleliteralemphasis{\sphinxupquote{numpy.ndarray}}}) \textendash{} A matrix.

\item[{Returns}] \leavevmode
The highest eigenvalue of \sphinxcode{\sphinxupquote{m}} that is smaller than one or one.

\item[{Return type}] \leavevmode
\sphinxhref{https://docs.python.org/3.7/library/functions.html\#float}{float}

\end{description}\end{quote}

\end{fulllineitems}

\index{is\_connected() (in module app.domain.helpers.matrices)@\spxentry{is\_connected()}\spxextra{in module app.domain.helpers.matrices}}

\begin{fulllineitems}
\phantomsection\label{\detokenize{app.domain.helpers:app.domain.helpers.matrices.is_connected}}\pysiglinewithargsret{\sphinxbfcode{\sphinxupquote{is\_connected}}}{\emph{\DUrole{n}{m}}, \emph{\DUrole{n}{directed}\DUrole{o}{=}\DUrole{default_value}{False}}}{}
Checks if a matrix is connected by counting the number of connected
components.
\begin{quote}\begin{description}
\item[{Parameters}] \leavevmode\begin{itemize}
\item {} 
\sphinxstyleliteralstrong{\sphinxupquote{m}} (\sphinxhref{https://numpy.org/doc/stable/reference/generated/numpy.ndarray.html\#numpy.ndarray}{\sphinxstyleliteralemphasis{\sphinxupquote{numpy.ndarray}}}) \textendash{} The matrix to be verified.

\item {} 
\sphinxstyleliteralstrong{\sphinxupquote{directed}} (\sphinxhref{https://docs.python.org/3.7/library/functions.html\#bool}{\sphinxstyleliteralemphasis{\sphinxupquote{bool}}}) \textendash{} If \sphinxcode{\sphinxupquote{m}} edges are directed, i.e., if \sphinxcode{\sphinxupquote{m}} is an adjency
matrix in which the edges bidirectional. \sphinxcode{\sphinxupquote{False}} means they
are. \sphinxcode{\sphinxupquote{True}} means they are not.

\end{itemize}

\item[{Returns}] \leavevmode
\sphinxcode{\sphinxupquote{True}} if the matrix is a connected graph, else \sphinxcode{\sphinxupquote{False}}.

\item[{Return type}] \leavevmode
\sphinxhref{https://docs.python.org/3.7/library/functions.html\#bool}{bool}

\end{description}\end{quote}

\end{fulllineitems}

\index{is\_symmetric() (in module app.domain.helpers.matrices)@\spxentry{is\_symmetric()}\spxextra{in module app.domain.helpers.matrices}}

\begin{fulllineitems}
\phantomsection\label{\detokenize{app.domain.helpers:app.domain.helpers.matrices.is_symmetric}}\pysiglinewithargsret{\sphinxbfcode{\sphinxupquote{is\_symmetric}}}{\emph{\DUrole{n}{m}}, \emph{\DUrole{n}{tol}\DUrole{o}{=}\DUrole{default_value}{1e\sphinxhyphen{}08}}}{}
Checks if a matrix is symmetric by performing element\sphinxhyphen{}wise equality
comparison on entries of \sphinxcode{\sphinxupquote{m}} and  \sphinxcode{\sphinxupquote{m.T}}.
\begin{quote}\begin{description}
\item[{Parameters}] \leavevmode\begin{itemize}
\item {} 
\sphinxstyleliteralstrong{\sphinxupquote{m}} (\sphinxhref{https://numpy.org/doc/stable/reference/generated/numpy.ndarray.html\#numpy.ndarray}{\sphinxstyleliteralemphasis{\sphinxupquote{numpy.ndarray}}}) \textendash{} The matrix to be verified.

\item {} 
\sphinxstyleliteralstrong{\sphinxupquote{tol}} (\sphinxhref{https://docs.python.org/3.7/library/functions.html\#float}{\sphinxstyleliteralemphasis{\sphinxupquote{float}}}) \textendash{} The tolerance used to verify the entries of the \sphinxcode{\sphinxupquote{m}} (default
is 1e\sphinxhyphen{}8).

\end{itemize}

\item[{Returns}] \leavevmode
\sphinxcode{\sphinxupquote{True}} if the \sphinxcode{\sphinxupquote{m}} is symmetric, else \sphinxcode{\sphinxupquote{False}}.

\item[{Return type}] \leavevmode
\sphinxhref{https://docs.python.org/3.7/library/functions.html\#bool}{bool}

\end{description}\end{quote}

\end{fulllineitems}

\index{make\_connected() (in module app.domain.helpers.matrices)@\spxentry{make\_connected()}\spxextra{in module app.domain.helpers.matrices}}

\begin{fulllineitems}
\phantomsection\label{\detokenize{app.domain.helpers:app.domain.helpers.matrices.make_connected}}\pysiglinewithargsret{\sphinxbfcode{\sphinxupquote{make\_connected}}}{\emph{\DUrole{n}{m}}}{}
Turns a matrix into a connected matrix that could represent a
connected graph.
\begin{quote}\begin{description}
\item[{Parameters}] \leavevmode
\sphinxstyleliteralstrong{\sphinxupquote{m}} (\sphinxhref{https://numpy.org/doc/stable/reference/generated/numpy.ndarray.html\#numpy.ndarray}{\sphinxstyleliteralemphasis{\sphinxupquote{numpy.ndarray}}}) \textendash{} The matrix to be made connected.

\item[{Returns}] \leavevmode
A connected matrix. If \sphinxcode{\sphinxupquote{m}} was symmetric the modified matrix will
also be symmetric.

\item[{Return type}] \leavevmode
\sphinxhref{https://numpy.org/doc/stable/reference/generated/numpy.ndarray.html\#numpy.ndarray}{numpy.ndarray}

\end{description}\end{quote}

\end{fulllineitems}

\index{new\_go\_transition\_matrix() (in module app.domain.helpers.matrices)@\spxentry{new\_go\_transition\_matrix()}\spxextra{in module app.domain.helpers.matrices}}

\begin{fulllineitems}
\phantomsection\label{\detokenize{app.domain.helpers:app.domain.helpers.matrices.new_go_transition_matrix}}\pysiglinewithargsret{\sphinxbfcode{\sphinxupquote{new\_go\_transition\_matrix}}}{\emph{\DUrole{n}{a}}, \emph{\DUrole{n}{v\_}}}{}
Constructs a transition matrix using global optimization techniques.

Constructs an optimized markov matrix using linear programming relaxations
and convex envelope approximations for the specified steady state \sphinxcode{\sphinxupquote{v}}.
Result is only trully optimal if \(normal(Mopt - (1 / len(v)), 2)\)
is equal to the highest Markov Matrix eigenvalue that is smaller than one.
\begin{quote}\begin{description}
\item[{Parameters}] \leavevmode\begin{itemize}
\item {} 
\sphinxstyleliteralstrong{\sphinxupquote{a}} (\sphinxhref{https://numpy.org/doc/stable/reference/generated/numpy.ndarray.html\#numpy.ndarray}{\sphinxstyleliteralemphasis{\sphinxupquote{numpy.ndarray}}}) \textendash{} A non\sphinxhyphen{}optimized symmetric adjency matrix.

\item {} 
\sphinxstyleliteralstrong{\sphinxupquote{v\_}} (\sphinxhref{https://numpy.org/doc/stable/reference/generated/numpy.ndarray.html\#numpy.ndarray}{\sphinxstyleliteralemphasis{\sphinxupquote{numpy.ndarray}}}) \textendash{} A stochastic steady state distribution vector.

\end{itemize}

\item[{Returns}] \leavevmode
Markov Matrix with \sphinxcode{\sphinxupquote{v\_}} as steady state distribution and the
respective mixing rate.

\item[{Return type}] \leavevmode
Tuple{[}Optional{[}\sphinxhref{https://numpy.org/doc/stable/reference/generated/numpy.ndarray.html\#numpy.ndarray}{numpy.ndarray}{]}, \sphinxhref{https://docs.python.org/3.7/library/functions.html\#float}{float}{]}

\end{description}\end{quote}

\end{fulllineitems}

\index{new\_mgo\_transition\_matrix() (in module app.domain.helpers.matrices)@\spxentry{new\_mgo\_transition\_matrix()}\spxextra{in module app.domain.helpers.matrices}}

\begin{fulllineitems}
\phantomsection\label{\detokenize{app.domain.helpers:app.domain.helpers.matrices.new_mgo_transition_matrix}}\pysiglinewithargsret{\sphinxbfcode{\sphinxupquote{new\_mgo\_transition\_matrix}}}{\emph{\DUrole{n}{a}}, \emph{\DUrole{n}{v\_}}}{}
Constructs an optimized transition matrix using the matlab engine.

Constructs an optimized transition matrix using linear programming
relaxations and convex envelope approximations for the specified steady
state \sphinxcode{\sphinxupquote{v}}.
Result is only trully optimal if \(normal(Mopt - (1 / len(v)), 2)\)
is equal to the highest Markov Matrix eigenvalue that is smaller than one.

\begin{sphinxadmonition}{note}{Note:}
This function’s code runs inside a matlab engine because it provides
a non\sphinxhyphen{}convex SDP solver BMIBNB. If you do not have valid matlab
license the output of this function is always \sphinxcode{\sphinxupquote{(None, float(\textquotesingle{}inf\textquotesingle{})}}.
\end{sphinxadmonition}
\begin{quote}\begin{description}
\item[{Parameters}] \leavevmode\begin{itemize}
\item {} 
\sphinxstyleliteralstrong{\sphinxupquote{a}} (\sphinxhref{https://numpy.org/doc/stable/reference/generated/numpy.ndarray.html\#numpy.ndarray}{\sphinxstyleliteralemphasis{\sphinxupquote{numpy.ndarray}}}) \textendash{} A non\sphinxhyphen{}optimized symmetric adjency matrix.

\item {} 
\sphinxstyleliteralstrong{\sphinxupquote{v\_}} (\sphinxhref{https://numpy.org/doc/stable/reference/generated/numpy.ndarray.html\#numpy.ndarray}{\sphinxstyleliteralemphasis{\sphinxupquote{numpy.ndarray}}}) \textendash{} A stochastic steady state distribution vector.

\end{itemize}

\item[{Returns}] \leavevmode
Markov Matrix with \sphinxcode{\sphinxupquote{v\_}} as steady state distribution and the
respective mixing rate.

\item[{Return type}] \leavevmode
Tuple{[}Optional{[}\sphinxhref{https://numpy.org/doc/stable/reference/generated/numpy.ndarray.html\#numpy.ndarray}{numpy.ndarray}{]}, \sphinxhref{https://docs.python.org/3.7/library/functions.html\#float}{float}{]}

\end{description}\end{quote}

\end{fulllineitems}

\index{new\_mh\_transition\_matrix() (in module app.domain.helpers.matrices)@\spxentry{new\_mh\_transition\_matrix()}\spxextra{in module app.domain.helpers.matrices}}

\begin{fulllineitems}
\phantomsection\label{\detokenize{app.domain.helpers:app.domain.helpers.matrices.new_mh_transition_matrix}}\pysiglinewithargsret{\sphinxbfcode{\sphinxupquote{new\_mh\_transition\_matrix}}}{\emph{\DUrole{n}{a}}, \emph{\DUrole{n}{v\_}}}{}
Constructs a transition matrix using metropolis\sphinxhyphen{}hastings.

Constructs a transition matrix using metropolis\sphinxhyphen{}hastings algorithm  for
the specified steady state \sphinxcode{\sphinxupquote{v}}.

\begin{sphinxadmonition}{note}{Note:}
The input Matrix hould have no transient states or absorbent nodes,
but this is not enforced or verified.
\end{sphinxadmonition}
\begin{quote}\begin{description}
\item[{Parameters}] \leavevmode\begin{itemize}
\item {} 
\sphinxstyleliteralstrong{\sphinxupquote{a}} (\sphinxhref{https://numpy.org/doc/stable/reference/generated/numpy.ndarray.html\#numpy.ndarray}{\sphinxstyleliteralemphasis{\sphinxupquote{numpy.ndarray}}}) \textendash{} A symmetric adjency matrix.

\item {} 
\sphinxstyleliteralstrong{\sphinxupquote{v\_}} (\sphinxhref{https://numpy.org/doc/stable/reference/generated/numpy.ndarray.html\#numpy.ndarray}{\sphinxstyleliteralemphasis{\sphinxupquote{numpy.ndarray}}}) \textendash{} A stochastic steady state distribution vector.

\end{itemize}

\item[{Returns}] \leavevmode
Markov Matrix with \sphinxcode{\sphinxupquote{v\_}} as steady state distribution and the
respective mixing rate or \sphinxcode{\sphinxupquote{None, float(\textquotesingle{}inf\textquotesingle{})}} if the problem is
infeasible.

\item[{Return type}] \leavevmode
Tuple{[}\sphinxhref{https://numpy.org/doc/stable/reference/generated/numpy.ndarray.html\#numpy.ndarray}{numpy.ndarray}, \sphinxhref{https://docs.python.org/3.7/library/functions.html\#float}{float}{]}

\end{description}\end{quote}

\end{fulllineitems}

\index{new\_sdp\_mh\_transition\_matrix() (in module app.domain.helpers.matrices)@\spxentry{new\_sdp\_mh\_transition\_matrix()}\spxextra{in module app.domain.helpers.matrices}}

\begin{fulllineitems}
\phantomsection\label{\detokenize{app.domain.helpers:app.domain.helpers.matrices.new_sdp_mh_transition_matrix}}\pysiglinewithargsret{\sphinxbfcode{\sphinxupquote{new\_sdp\_mh\_transition\_matrix}}}{\emph{\DUrole{n}{a}}, \emph{\DUrole{n}{v\_}}}{}
Constructs a transition matrix using semi\sphinxhyphen{}definite programming techniques.

Constructs a transition matrix using metropolis\sphinxhyphen{}hastings algorithm  for
the specified steady state \sphinxcode{\sphinxupquote{v}}. The provided adjacency matrix A is first
optimized with semi\sphinxhyphen{}definite programming techniques for the uniform
distribution vector.
\begin{quote}\begin{description}
\item[{Parameters}] \leavevmode\begin{itemize}
\item {} 
\sphinxstyleliteralstrong{\sphinxupquote{a}} (\sphinxhref{https://numpy.org/doc/stable/reference/generated/numpy.ndarray.html\#numpy.ndarray}{\sphinxstyleliteralemphasis{\sphinxupquote{numpy.ndarray}}}) \textendash{} A non\sphinxhyphen{}optimized symmetric adjency matrix.

\item {} 
\sphinxstyleliteralstrong{\sphinxupquote{v\_}} (\sphinxhref{https://numpy.org/doc/stable/reference/generated/numpy.ndarray.html\#numpy.ndarray}{\sphinxstyleliteralemphasis{\sphinxupquote{numpy.ndarray}}}) \textendash{} A stochastic steady state distribution vector.

\end{itemize}

\item[{Returns}] \leavevmode
Markov Matrix with \sphinxcode{\sphinxupquote{v\_}} as steady state distribution and the
respective mixing rate or \sphinxcode{\sphinxupquote{None, float(\textquotesingle{}inf\textquotesingle{})}} if the problem is
infeasible.

\item[{Return type}] \leavevmode
Tuple{[}Optional{[}\sphinxhref{https://numpy.org/doc/stable/reference/generated/numpy.ndarray.html\#numpy.ndarray}{numpy.ndarray}{]}, \sphinxhref{https://docs.python.org/3.7/library/functions.html\#float}{float}{]}

\end{description}\end{quote}

\end{fulllineitems}

\index{new\_symmetric\_connected\_matrix() (in module app.domain.helpers.matrices)@\spxentry{new\_symmetric\_connected\_matrix()}\spxextra{in module app.domain.helpers.matrices}}

\begin{fulllineitems}
\phantomsection\label{\detokenize{app.domain.helpers:app.domain.helpers.matrices.new_symmetric_connected_matrix}}\pysiglinewithargsret{\sphinxbfcode{\sphinxupquote{new\_symmetric\_connected\_matrix}}}{\emph{\DUrole{n}{size}}, \emph{\DUrole{n}{allow\_sloops}\DUrole{o}{=}\DUrole{default_value}{True}}, \emph{\DUrole{n}{force\_sloops}\DUrole{o}{=}\DUrole{default_value}{True}}}{}
Generates a random symmetric matrix which is also connected.

See {\hyperref[\detokenize{app.domain.helpers:app.domain.helpers.matrices.new_symmetric_matrix}]{\sphinxcrossref{\sphinxcode{\sphinxupquote{new\_symmetric\_matrix()}}}}} and {\hyperref[\detokenize{app.domain.helpers:app.domain.helpers.matrices.make_connected}]{\sphinxcrossref{\sphinxcode{\sphinxupquote{make\_connected()}}}}}.
\begin{quote}\begin{description}
\item[{Parameters}] \leavevmode\begin{itemize}
\item {} 
\sphinxstyleliteralstrong{\sphinxupquote{size}} (\sphinxhref{https://docs.python.org/3.7/library/functions.html\#int}{\sphinxstyleliteralemphasis{\sphinxupquote{int}}}) \textendash{} The length of the square matrix.

\item {} 
\sphinxstyleliteralstrong{\sphinxupquote{allow\_sloops}} (\sphinxhref{https://docs.python.org/3.7/library/functions.html\#bool}{\sphinxstyleliteralemphasis{\sphinxupquote{bool}}}) \textendash{} See {\hyperref[\detokenize{app.domain.helpers:app.domain.helpers.matrices.new_symmetric_matrix}]{\sphinxcrossref{\sphinxcode{\sphinxupquote{new\_symmetric\_matrix()}}}}}
for clarifications.

\item {} 
\sphinxstyleliteralstrong{\sphinxupquote{force\_sloops}} (\sphinxhref{https://docs.python.org/3.7/library/functions.html\#bool}{\sphinxstyleliteralemphasis{\sphinxupquote{bool}}}) \textendash{} See {\hyperref[\detokenize{app.domain.helpers:app.domain.helpers.matrices.new_symmetric_matrix}]{\sphinxcrossref{\sphinxcode{\sphinxupquote{new\_symmetric\_matrix()}}}}}
for clarifications.

\end{itemize}

\item[{Returns}] \leavevmode
A matrix that represents an adjacency matrix that is also connected.

\item[{Return type}] \leavevmode
\sphinxhref{https://numpy.org/doc/stable/reference/generated/numpy.ndarray.html\#numpy.ndarray}{numpy.ndarray}

\end{description}\end{quote}

\end{fulllineitems}

\index{new\_symmetric\_matrix() (in module app.domain.helpers.matrices)@\spxentry{new\_symmetric\_matrix()}\spxextra{in module app.domain.helpers.matrices}}

\begin{fulllineitems}
\phantomsection\label{\detokenize{app.domain.helpers:app.domain.helpers.matrices.new_symmetric_matrix}}\pysiglinewithargsret{\sphinxbfcode{\sphinxupquote{new\_symmetric\_matrix}}}{\emph{\DUrole{n}{size}}, \emph{\DUrole{n}{allow\_sloops}\DUrole{o}{=}\DUrole{default_value}{True}}, \emph{\DUrole{n}{force\_sloops}\DUrole{o}{=}\DUrole{default_value}{True}}}{}
Generates a random symmetric matrix.

The generated adjacency matrix does not have transient state sets or
absorbent nodes and can effectively represent a network topology
with bidirectional connections between {\hyperref[\detokenize{app.domain:app.domain.network_nodes.Node}]{\sphinxcrossref{\sphinxcode{\sphinxupquote{network nodes}}}}}.
\begin{quote}\begin{description}
\item[{Parameters}] \leavevmode\begin{itemize}
\item {} 
\sphinxstyleliteralstrong{\sphinxupquote{size}} (\sphinxhref{https://docs.python.org/3.7/library/functions.html\#int}{\sphinxstyleliteralemphasis{\sphinxupquote{int}}}) \textendash{} The length of the square matrix.

\item {} 
\sphinxstyleliteralstrong{\sphinxupquote{allow\_sloops}} (\sphinxhref{https://docs.python.org/3.7/library/functions.html\#bool}{\sphinxstyleliteralemphasis{\sphinxupquote{bool}}}) \textendash{} Indicates if the generated adjacency matrix allows diagonal
entries representing self\sphinxhyphen{}loops. If \sphinxcode{\sphinxupquote{False}}, then, all diagonal
entries must be zeros. Otherwise, they can be zeros or ones.

\item {} 
\sphinxstyleliteralstrong{\sphinxupquote{force\_sloops}} (\sphinxhref{https://docs.python.org/3.7/library/functions.html\#bool}{\sphinxstyleliteralemphasis{\sphinxupquote{bool}}}) \textendash{} Indicates if the diagonal of the generated matrix should be
filled with ones. If \sphinxcode{\sphinxupquote{False}} valid diagonal entries are
decided by \sphinxcode{\sphinxupquote{allow\_self\_loops}} param. Otherwise, diagonal entries
are filled with ones. If \sphinxcode{\sphinxupquote{allow\_self\_loops}} is \sphinxcode{\sphinxupquote{False}}
and \sphinxcode{\sphinxupquote{enforce\_loops}} is \sphinxcode{\sphinxupquote{True}}, an error is raised.

\end{itemize}

\item[{Returns}] \leavevmode
The adjency matrix representing the connections between a
groups of {\hyperref[\detokenize{app.domain:app.domain.network_nodes.Node}]{\sphinxcrossref{\sphinxcode{\sphinxupquote{network nodes}}}}}.

\item[{Raises}] \leavevmode
{\hyperref[\detokenize{app.domain.helpers:app.domain.helpers.exceptions.IllegalArgumentError}]{\sphinxcrossref{\sphinxstyleliteralstrong{\sphinxupquote{IllegalArgumentError}}}}} \textendash{} When \sphinxcode{\sphinxupquote{allow\_self\_loops}} (\sphinxcode{\sphinxupquote{False}}) conflicts with
    \sphinxcode{\sphinxupquote{enforce\_loops}} (\sphinxcode{\sphinxupquote{True}}).

\item[{Return type}] \leavevmode
\sphinxhref{https://numpy.org/doc/stable/reference/generated/numpy.ndarray.html\#numpy.ndarray}{numpy.ndarray}

\end{description}\end{quote}

\end{fulllineitems}

\index{new\_vector() (in module app.domain.helpers.matrices)@\spxentry{new\_vector()}\spxextra{in module app.domain.helpers.matrices}}

\begin{fulllineitems}
\phantomsection\label{\detokenize{app.domain.helpers:app.domain.helpers.matrices.new_vector}}\pysiglinewithargsret{\sphinxbfcode{\sphinxupquote{new\_vector}}}{\emph{\DUrole{n}{size}}}{}~\begin{quote}\begin{description}
\item[{Parameters}] \leavevmode
\sphinxstyleliteralstrong{\sphinxupquote{size}} (\sphinxhref{https://docs.python.org/3.7/library/functions.html\#int}{\sphinxstyleliteralemphasis{\sphinxupquote{int}}}) \textendash{} 

\item[{Return type}] \leavevmode
\sphinxhref{https://numpy.org/doc/stable/reference/generated/numpy.ndarray.html\#numpy.ndarray}{numpy.ndarray}

\end{description}\end{quote}

\end{fulllineitems}



\subparagraph{app.domain.helpers.smart\_dataclasses}
\label{\detokenize{app.domain.helpers:module-app.domain.helpers.smart_dataclasses}}\label{\detokenize{app.domain.helpers:app-domain-helpers-smart-dataclasses}}\index{module@\spxentry{module}!app.domain.helpers.smart\_dataclasses@\spxentry{app.domain.helpers.smart\_dataclasses}}\index{app.domain.helpers.smart\_dataclasses@\spxentry{app.domain.helpers.smart\_dataclasses}!module@\spxentry{module}}
Module with classes that help to avoid domain class polution by
encapsulating attribute and method behaviour.
\index{FileBlockData (class in app.domain.helpers.smart\_dataclasses)@\spxentry{FileBlockData}\spxextra{class in app.domain.helpers.smart\_dataclasses}}

\begin{fulllineitems}
\phantomsection\label{\detokenize{app.domain.helpers:app.domain.helpers.smart_dataclasses.FileBlockData}}\pysiglinewithargsret{\sphinxbfcode{\sphinxupquote{class }}\sphinxbfcode{\sphinxupquote{FileBlockData}}}{\emph{\DUrole{n}{cluster\_id}}, \emph{\DUrole{n}{name}}, \emph{\DUrole{n}{number}}, \emph{\DUrole{n}{data}}}{}
Bases: \sphinxhref{https://docs.python.org/3.7/library/functions.html\#object}{\sphinxcode{\sphinxupquote{object}}}

Wrapping class for the contents of a file block.

Among other responsabilities \sphinxtitleref{FileBlockData} helps managing simulation
parameters, e.g., replica control such or file block integrity.
\index{cluster\_id (FileBlockData attribute)@\spxentry{cluster\_id}\spxextra{FileBlockData attribute}}

\begin{fulllineitems}
\phantomsection\label{\detokenize{app.domain.helpers:app.domain.helpers.smart_dataclasses.FileBlockData.cluster_id}}\pysigline{\sphinxbfcode{\sphinxupquote{cluster\_id}}}
Unique identifier of the cluster that manages the file block.

\end{fulllineitems}

\index{name (FileBlockData attribute)@\spxentry{name}\spxextra{FileBlockData attribute}}

\begin{fulllineitems}
\phantomsection\label{\detokenize{app.domain.helpers:app.domain.helpers.smart_dataclasses.FileBlockData.name}}\pysigline{\sphinxbfcode{\sphinxupquote{name}}}
The name of the file the file block belongs to.

\end{fulllineitems}

\index{number (FileBlockData attribute)@\spxentry{number}\spxextra{FileBlockData attribute}}

\begin{fulllineitems}
\phantomsection\label{\detokenize{app.domain.helpers:app.domain.helpers.smart_dataclasses.FileBlockData.number}}\pysigline{\sphinxbfcode{\sphinxupquote{number}}}
The number that uniquely identifies the file block.

\end{fulllineitems}

\index{id (FileBlockData attribute)@\spxentry{id}\spxextra{FileBlockData attribute}}

\begin{fulllineitems}
\phantomsection\label{\detokenize{app.domain.helpers:app.domain.helpers.smart_dataclasses.FileBlockData.id}}\pysigline{\sphinxbfcode{\sphinxupquote{id}}}
Concatenation the the \sphinxtitleref{name} and \sphinxtitleref{number}.

\end{fulllineitems}

\index{references (FileBlockData attribute)@\spxentry{references}\spxextra{FileBlockData attribute}}

\begin{fulllineitems}
\phantomsection\label{\detokenize{app.domain.helpers:app.domain.helpers.smart_dataclasses.FileBlockData.references}}\pysigline{\sphinxbfcode{\sphinxupquote{references}}}
Tracks how many references exist to the file block in the
simulation environment. When it reaches 0 the file block ceases
to exist and the simulation fails.

\end{fulllineitems}

\index{replication\_epoch (FileBlockData attribute)@\spxentry{replication\_epoch}\spxextra{FileBlockData attribute}}

\begin{fulllineitems}
\phantomsection\label{\detokenize{app.domain.helpers:app.domain.helpers.smart_dataclasses.FileBlockData.replication_epoch}}\pysigline{\sphinxbfcode{\sphinxupquote{replication\_epoch}}}
When a reference to the file block is lost, i.e., decremented,
a replication epoch that simulates time to copy blocks from one
node to another is assigned to this attribute.
Until a loss occurs and after a loss is recovered,
\sphinxtitleref{recovery\_epoch} is set to positive infinity.

\end{fulllineitems}

\index{data (FileBlockData attribute)@\spxentry{data}\spxextra{FileBlockData attribute}}

\begin{fulllineitems}
\phantomsection\label{\detokenize{app.domain.helpers:app.domain.helpers.smart_dataclasses.FileBlockData.data}}\pysigline{\sphinxbfcode{\sphinxupquote{data}}}
A base64\sphinxhyphen{}encoded string representation of the file block bytes.

\end{fulllineitems}

\index{sha256 (FileBlockData attribute)@\spxentry{sha256}\spxextra{FileBlockData attribute}}

\begin{fulllineitems}
\phantomsection\label{\detokenize{app.domain.helpers:app.domain.helpers.smart_dataclasses.FileBlockData.sha256}}\pysigline{\sphinxbfcode{\sphinxupquote{sha256}}}
The hash value of data resulting from a SHA256 digest.

\end{fulllineitems}

\index{\_\_init\_\_() (FileBlockData method)@\spxentry{\_\_init\_\_()}\spxextra{FileBlockData method}}

\begin{fulllineitems}
\phantomsection\label{\detokenize{app.domain.helpers:app.domain.helpers.smart_dataclasses.FileBlockData.__init__}}\pysiglinewithargsret{\sphinxbfcode{\sphinxupquote{\_\_init\_\_}}}{\emph{\DUrole{n}{cluster\_id}}, \emph{\DUrole{n}{name}}, \emph{\DUrole{n}{number}}, \emph{\DUrole{n}{data}}}{}
Creates an instance of \sphinxtitleref{FileBlockData}.
\begin{quote}\begin{description}
\item[{Parameters}] \leavevmode\begin{itemize}
\item {} 
\sphinxstyleliteralstrong{\sphinxupquote{cluster\_id}} (\sphinxhref{https://docs.python.org/3.7/library/stdtypes.html\#str}{\sphinxstyleliteralemphasis{\sphinxupquote{str}}}) \textendash{} Unique identifier of the cluster that manages the file block.

\item {} 
\sphinxstyleliteralstrong{\sphinxupquote{name}} (\sphinxhref{https://docs.python.org/3.7/library/stdtypes.html\#str}{\sphinxstyleliteralemphasis{\sphinxupquote{str}}}) \textendash{} The name of the file the file block belongs to.

\item {} 
\sphinxstyleliteralstrong{\sphinxupquote{number}} (\sphinxhref{https://docs.python.org/3.7/library/functions.html\#int}{\sphinxstyleliteralemphasis{\sphinxupquote{int}}}) \textendash{} The number that uniquely identifies the file block.

\item {} 
\sphinxstyleliteralstrong{\sphinxupquote{data}} (\sphinxhref{https://docs.python.org/3.7/library/stdtypes.html\#bytes}{\sphinxstyleliteralemphasis{\sphinxupquote{bytes}}}) \textendash{} Actual file block data as a sequence of bytes.

\end{itemize}

\item[{Return type}] \leavevmode
\sphinxhref{https://docs.python.org/3.7/library/constants.html\#None}{None}

\end{description}\end{quote}

\end{fulllineitems}

\index{\_\_str\_\_() (FileBlockData method)@\spxentry{\_\_str\_\_()}\spxextra{FileBlockData method}}

\begin{fulllineitems}
\phantomsection\label{\detokenize{app.domain.helpers:app.domain.helpers.smart_dataclasses.FileBlockData.__str__}}\pysiglinewithargsret{\sphinxbfcode{\sphinxupquote{\_\_str\_\_}}}{}{}
Overrides default string representation of \sphinxtitleref{FileBlockData} instances.
\begin{quote}\begin{description}
\item[{Returns}] \leavevmode
A dictionary representation of the object.

\end{description}\end{quote}

\end{fulllineitems}

\index{can\_replicate() (FileBlockData method)@\spxentry{can\_replicate()}\spxextra{FileBlockData method}}

\begin{fulllineitems}
\phantomsection\label{\detokenize{app.domain.helpers:app.domain.helpers.smart_dataclasses.FileBlockData.can_replicate}}\pysiglinewithargsret{\sphinxbfcode{\sphinxupquote{can\_replicate}}}{\emph{\DUrole{n}{epoch}}}{}
Informs the calling network node if file block needs replication.
\begin{quote}\begin{description}
\item[{Parameters}] \leavevmode
\sphinxstyleliteralstrong{\sphinxupquote{epoch}} (\sphinxhref{https://docs.python.org/3.7/library/functions.html\#int}{\sphinxstyleliteralemphasis{\sphinxupquote{int}}}) \textendash{} Simulation’s current epoch.

\item[{Returns}] \leavevmode
How many times the caller should replicate the block. The network
node knows how many blocks he needs to create and distribute if
returned value is bigger than zero.

\item[{Return type}] \leavevmode
\sphinxhref{https://docs.python.org/3.7/library/functions.html\#int}{int}

\end{description}\end{quote}

\end{fulllineitems}

\index{decrement\_and\_get\_references() (FileBlockData method)@\spxentry{decrement\_and\_get\_references()}\spxextra{FileBlockData method}}

\begin{fulllineitems}
\phantomsection\label{\detokenize{app.domain.helpers:app.domain.helpers.smart_dataclasses.FileBlockData.decrement_and_get_references}}\pysiglinewithargsret{\sphinxbfcode{\sphinxupquote{decrement\_and\_get\_references}}}{}{}
Decreases by one and gets the number of file block references.
\begin{quote}\begin{description}
\item[{Returns}] \leavevmode
The number of file block references existing in the simulation
environment.

\end{description}\end{quote}

\end{fulllineitems}

\index{set\_replication\_epoch() (FileBlockData method)@\spxentry{set\_replication\_epoch()}\spxextra{FileBlockData method}}

\begin{fulllineitems}
\phantomsection\label{\detokenize{app.domain.helpers:app.domain.helpers.smart_dataclasses.FileBlockData.set_replication_epoch}}\pysiglinewithargsret{\sphinxbfcode{\sphinxupquote{set\_replication\_epoch}}}{\emph{\DUrole{n}{epoch}}}{}
Sets the epoch in which replication levels should be restored.

This method tries to assign a new epoch, in the future, at which
recovery should be performed. If the proposition is sooner than the
previous proposition then assignment is accepted, else, it’s rejected.

\begin{sphinxadmonition}{note}{Note:}
This method of calculating the \sphinxtitleref{replication\_epoch} may seem
controversial, but the justification lies in the assumption that
if there are more network nodes monitoring file parts,
than failure detections should be in theory, faster, unless
complex consensus algorithms are being used between volatile
peers, which is not our case. We assume peers only report their
suspicions to a small number of trusted of monitors who then
decide if the reported network node is disconnected, consequently
losing the instance of \sphinxtitleref{FileBlockData} and possibly others.
\end{sphinxadmonition}
\begin{quote}\begin{description}
\item[{Parameters}] \leavevmode
\sphinxstyleliteralstrong{\sphinxupquote{epoch}} (\sphinxhref{https://docs.python.org/3.7/library/functions.html\#int}{\sphinxstyleliteralemphasis{\sphinxupquote{int}}}) \textendash{} Simulation’s current epoch.

\item[{Returns}] \leavevmode
Zero if the current \sphinxtitleref{replication\_epoch} is positive infinity,
otherwise the expected delay\_replication is returned. This value
can be used to log, for example, the average recovery
delay\_replication in a simulation.

\item[{Return type}] \leavevmode
\sphinxhref{https://docs.python.org/3.7/library/functions.html\#float}{float}

\end{description}\end{quote}

\end{fulllineitems}

\index{update\_epochs\_to\_recover() (FileBlockData method)@\spxentry{update\_epochs\_to\_recover()}\spxextra{FileBlockData method}}

\begin{fulllineitems}
\phantomsection\label{\detokenize{app.domain.helpers:app.domain.helpers.smart_dataclasses.FileBlockData.update_epochs_to_recover}}\pysiglinewithargsret{\sphinxbfcode{\sphinxupquote{update\_epochs\_to\_recover}}}{\emph{\DUrole{n}{epoch}}}{}
Update the \sphinxtitleref{replication\_epoch} after a recovery attempt was carried out.

If the recovery attempt performed by some network node successfully
managed to restore the replication levels to the original target, then,
\sphinxtitleref{replication\_epoch} is set to positive infinity, otherwise, another
attempt will be done in the next epoch.
\begin{quote}\begin{description}
\item[{Parameters}] \leavevmode
\sphinxstyleliteralstrong{\sphinxupquote{epoch}} (\sphinxhref{https://docs.python.org/3.7/library/functions.html\#int}{\sphinxstyleliteralemphasis{\sphinxupquote{int}}}) \textendash{} Simulation’s current epoch.

\item[{Return type}] \leavevmode
\sphinxhref{https://docs.python.org/3.7/library/constants.html\#None}{None}

\end{description}\end{quote}

\end{fulllineitems}


\end{fulllineitems}

\index{FileData (class in app.domain.helpers.smart\_dataclasses)@\spxentry{FileData}\spxextra{class in app.domain.helpers.smart\_dataclasses}}

\begin{fulllineitems}
\phantomsection\label{\detokenize{app.domain.helpers:app.domain.helpers.smart_dataclasses.FileData}}\pysiglinewithargsret{\sphinxbfcode{\sphinxupquote{class }}\sphinxbfcode{\sphinxupquote{FileData}}}{\emph{\DUrole{n}{name}}, \emph{\DUrole{n}{sim\_id}\DUrole{o}{=}\DUrole{default_value}{0}}, \emph{\DUrole{n}{origin}\DUrole{o}{=}\DUrole{default_value}{\textquotesingle{}\textquotesingle{}}}}{}
Bases: \sphinxhref{https://docs.python.org/3.7/library/functions.html\#object}{\sphinxcode{\sphinxupquote{object}}}

Holds essential simulation data concerning files being persisted.

FileData is a helper class which has responsabilities such as tracking
how many file block replicas currently existing in a
{\hyperref[\detokenize{app.domain:app.domain.cluster_groups.Cluster}]{\sphinxcrossref{\sphinxcode{\sphinxupquote{cluster group}}}}}
but also keeping simulation events logged in RAM until the simulation
ends, at which point the logs are written to disk.
\index{name (FileData attribute)@\spxentry{name}\spxextra{FileData attribute}}

\begin{fulllineitems}
\phantomsection\label{\detokenize{app.domain.helpers:app.domain.helpers.smart_dataclasses.FileData.name}}\pysigline{\sphinxbfcode{\sphinxupquote{name}}}
The name of the original file.
\begin{quote}\begin{description}
\item[{Type}] \leavevmode
\sphinxhref{https://docs.python.org/3.7/library/stdtypes.html\#str}{str}

\end{description}\end{quote}

\end{fulllineitems}

\index{existing\_replicas (FileData attribute)@\spxentry{existing\_replicas}\spxextra{FileData attribute}}

\begin{fulllineitems}
\phantomsection\label{\detokenize{app.domain.helpers:app.domain.helpers.smart_dataclasses.FileData.existing_replicas}}\pysigline{\sphinxbfcode{\sphinxupquote{existing\_replicas}}}
The number of file parts including blocks that exist for the
named file that exist in the simulation. Updated every epoch.
\begin{quote}\begin{description}
\item[{Type}] \leavevmode
\sphinxhref{https://docs.python.org/3.7/library/functions.html\#int}{int}

\end{description}\end{quote}

\end{fulllineitems}

\index{logger (FileData attribute)@\spxentry{logger}\spxextra{FileData attribute}}

\begin{fulllineitems}
\phantomsection\label{\detokenize{app.domain.helpers:app.domain.helpers.smart_dataclasses.FileData.logger}}\pysigline{\sphinxbfcode{\sphinxupquote{logger}}}
Object that stores captured simulation data. Stored data can be
post\sphinxhyphen{}processed using user defined scripts to create items such
has graphs and figures.
\begin{quote}\begin{description}
\item[{Type}] \leavevmode
{\hyperref[\detokenize{app.domain.helpers:app.domain.helpers.smart_dataclasses.LoggingData}]{\sphinxcrossref{\sphinxcode{\sphinxupquote{LoggingData}}}}}

\end{description}\end{quote}

\end{fulllineitems}

\index{out\_file (FileData attribute)@\spxentry{out\_file}\spxextra{FileData attribute}}

\begin{fulllineitems}
\phantomsection\label{\detokenize{app.domain.helpers:app.domain.helpers.smart_dataclasses.FileData.out_file}}\pysigline{\sphinxbfcode{\sphinxupquote{out\_file}}}
File output stream to where captured data is written in append
mode and to which \sphinxcode{\sphinxupquote{logger}} will be written to at the end of the
simulation.
\begin{quote}\begin{description}
\item[{Type}] \leavevmode
Union{[}\sphinxhref{https://docs.python.org/3.7/library/stdtypes.html\#str}{str}, \sphinxhref{https://docs.python.org/3.7/library/stdtypes.html\#bytes}{bytes}, \sphinxhref{https://docs.python.org/3.7/library/functions.html\#int}{int}{]}

\end{description}\end{quote}

\end{fulllineitems}

\index{\_\_init\_\_() (FileData method)@\spxentry{\_\_init\_\_()}\spxextra{FileData method}}

\begin{fulllineitems}
\phantomsection\label{\detokenize{app.domain.helpers:app.domain.helpers.smart_dataclasses.FileData.__init__}}\pysiglinewithargsret{\sphinxbfcode{\sphinxupquote{\_\_init\_\_}}}{\emph{\DUrole{n}{name}}, \emph{\DUrole{n}{sim\_id}\DUrole{o}{=}\DUrole{default_value}{0}}, \emph{\DUrole{n}{origin}\DUrole{o}{=}\DUrole{default_value}{\textquotesingle{}\textquotesingle{}}}}{}
Creates an instance of \sphinxcode{\sphinxupquote{FileData}}.
\begin{quote}\begin{description}
\item[{Parameters}] \leavevmode\begin{itemize}
\item {} 
\sphinxstyleliteralstrong{\sphinxupquote{name}} (\sphinxhref{https://docs.python.org/3.7/library/stdtypes.html\#str}{\sphinxstyleliteralemphasis{\sphinxupquote{str}}}) \textendash{} Name of the file to be referenced by the \sphinxcode{\sphinxupquote{FileData}} object.

\item {} 
\sphinxstyleliteralstrong{\sphinxupquote{sim\_id}} (\sphinxhref{https://docs.python.org/3.7/library/functions.html\#int}{\sphinxstyleliteralemphasis{\sphinxupquote{int}}}) \textendash{} Identifier that generates unique output file names,
thus guaranteeing that different simulation instances do not
overwrite previous {\hyperref[\detokenize{app.domain.helpers:app.domain.helpers.smart_dataclasses.FileData.out_file}]{\sphinxcrossref{\sphinxcode{\sphinxupquote{output files}}}}}.

\item {} 
\sphinxstyleliteralstrong{\sphinxupquote{origin}} (\sphinxhref{https://docs.python.org/3.7/library/stdtypes.html\#str}{\sphinxstyleliteralemphasis{\sphinxupquote{str}}}) \textendash{} The name of the simulation file name that started
the simulation process. See
{\hyperref[\detokenize{app.domain:app.domain.master_servers.Master}]{\sphinxcrossref{\sphinxcode{\sphinxupquote{Master}}}}} and
{\hyperref[\detokenize{app:module-app.hive_simulation}]{\sphinxcrossref{\sphinxcode{\sphinxupquote{hive\_simulation}}}}}. In addition to the previous,
the origin should somehow include the cluster class name
being run, to differentiate simulations’ output files being
executed by different distributed storage system
implementations.

\end{itemize}

\item[{Return type}] \leavevmode
\sphinxhref{https://docs.python.org/3.7/library/constants.html\#None}{None}

\end{description}\end{quote}

\end{fulllineitems}

\index{fclose() (FileData method)@\spxentry{fclose()}\spxextra{FileData method}}

\begin{fulllineitems}
\phantomsection\label{\detokenize{app.domain.helpers:app.domain.helpers.smart_dataclasses.FileData.fclose}}\pysiglinewithargsret{\sphinxbfcode{\sphinxupquote{fclose}}}{\emph{\DUrole{n}{msg}\DUrole{o}{=}\DUrole{default_value}{None}}}{}
Closes the output stream controlled by the \sphinxcode{\sphinxupquote{FileData}} instance.
\begin{quote}\begin{description}
\item[{Parameters}] \leavevmode
\sphinxstyleliteralstrong{\sphinxupquote{msg}} (\sphinxstyleliteralemphasis{\sphinxupquote{Optional}}\sphinxstyleliteralemphasis{\sphinxupquote{{[}}}\sphinxhref{https://docs.python.org/3.7/library/stdtypes.html\#str}{\sphinxstyleliteralemphasis{\sphinxupquote{str}}}\sphinxstyleliteralemphasis{\sphinxupquote{{]}}}) \textendash{} If filled, a termination message is appended to
{\hyperref[\detokenize{app.domain.helpers:app.domain.helpers.smart_dataclasses.FileData.out_file}]{\sphinxcrossref{\sphinxcode{\sphinxupquote{out\_file}}}}}, before closing it.

\item[{Return type}] \leavevmode
\sphinxhref{https://docs.python.org/3.7/library/constants.html\#None}{None}

\end{description}\end{quote}

\end{fulllineitems}

\index{fwrite() (FileData method)@\spxentry{fwrite()}\spxextra{FileData method}}

\begin{fulllineitems}
\phantomsection\label{\detokenize{app.domain.helpers:app.domain.helpers.smart_dataclasses.FileData.fwrite}}\pysiglinewithargsret{\sphinxbfcode{\sphinxupquote{fwrite}}}{\emph{\DUrole{n}{msg}}}{}
Appends a message to the output stream of \sphinxcode{\sphinxupquote{FileData}}.

The method automatically adds a new line character to \sphinxcode{\sphinxupquote{msg}}.
\begin{quote}\begin{description}
\item[{Parameters}] \leavevmode
\sphinxstyleliteralstrong{\sphinxupquote{msg}} (\sphinxhref{https://docs.python.org/3.7/library/stdtypes.html\#str}{\sphinxstyleliteralemphasis{\sphinxupquote{str}}}) \textendash{} The message to be logged on the {\hyperref[\detokenize{app.domain.helpers:app.domain.helpers.smart_dataclasses.FileData.out_file}]{\sphinxcrossref{\sphinxcode{\sphinxupquote{out\_file}}}}}.

\item[{Return type}] \leavevmode
\sphinxhref{https://docs.python.org/3.7/library/constants.html\#None}{None}

\end{description}\end{quote}

\end{fulllineitems}

\index{jwrite() (FileData method)@\spxentry{jwrite()}\spxextra{FileData method}}

\begin{fulllineitems}
\phantomsection\label{\detokenize{app.domain.helpers:app.domain.helpers.smart_dataclasses.FileData.jwrite}}\pysiglinewithargsret{\sphinxbfcode{\sphinxupquote{jwrite}}}{\emph{\DUrole{n}{cluster}}, \emph{\DUrole{n}{origin}}, \emph{\DUrole{n}{epoch}}}{}
Appends a json string to the output stream of \sphinxcode{\sphinxupquote{FileData}}.

The logged data are all attributes belonging to {\hyperref[\detokenize{app.domain.helpers:app.domain.helpers.smart_dataclasses.FileData.logger}]{\sphinxcrossref{\sphinxcode{\sphinxupquote{logger}}}}}.
\begin{quote}\begin{description}
\item[{Parameters}] \leavevmode\begin{itemize}
\item {} 
\sphinxstyleliteralstrong{\sphinxupquote{cluster}} ({\hyperref[\detokenize{app.domain:app.domain.cluster_groups.Cluster}]{\sphinxcrossref{\sphinxstyleliteralemphasis{\sphinxupquote{domain.cluster\_groups.Cluster}}}}}) \textendash{} The {\hyperref[\detokenize{app.domain:app.domain.cluster_groups.Cluster}]{\sphinxcrossref{\sphinxcode{\sphinxupquote{Cluster}}}}}
object that manages the simulated persistence of the
{\hyperref[\detokenize{app.domain.helpers:app.domain.helpers.smart_dataclasses.FileData.name}]{\sphinxcrossref{\sphinxcode{\sphinxupquote{named file}}}}}.

\item {} 
\sphinxstyleliteralstrong{\sphinxupquote{origin}} (\sphinxhref{https://docs.python.org/3.7/library/stdtypes.html\#str}{\sphinxstyleliteralemphasis{\sphinxupquote{str}}}) \textendash{} The name of the simulation file name that started
the simulation process. See
{\hyperref[\detokenize{app.domain:app.domain.master_servers.Master}]{\sphinxcrossref{\sphinxcode{\sphinxupquote{Master}}}}} and
{\hyperref[\detokenize{app:module-app.hive_simulation}]{\sphinxcrossref{\sphinxcode{\sphinxupquote{hive\_simulation}}}}}.

\item {} 
\sphinxstyleliteralstrong{\sphinxupquote{epoch}} (\sphinxhref{https://docs.python.org/3.7/library/functions.html\#int}{\sphinxstyleliteralemphasis{\sphinxupquote{int}}}) \textendash{} The epoch at which the {\hyperref[\detokenize{app.domain.helpers:app.domain.helpers.smart_dataclasses.FileData.logger}]{\sphinxcrossref{\sphinxcode{\sphinxupquote{logger}}}}} was appended to
{\hyperref[\detokenize{app.domain.helpers:app.domain.helpers.smart_dataclasses.FileData.out_file}]{\sphinxcrossref{\sphinxcode{\sphinxupquote{out\_file}}}}}.

\end{itemize}

\item[{Return type}] \leavevmode
\sphinxhref{https://docs.python.org/3.7/library/constants.html\#None}{None}

\end{description}\end{quote}

\end{fulllineitems}


\end{fulllineitems}

\index{LoggingData (class in app.domain.helpers.smart\_dataclasses)@\spxentry{LoggingData}\spxextra{class in app.domain.helpers.smart\_dataclasses}}

\begin{fulllineitems}
\phantomsection\label{\detokenize{app.domain.helpers:app.domain.helpers.smart_dataclasses.LoggingData}}\pysigline{\sphinxbfcode{\sphinxupquote{class }}\sphinxbfcode{\sphinxupquote{LoggingData}}}
Bases: \sphinxhref{https://docs.python.org/3.7/library/functions.html\#object}{\sphinxcode{\sphinxupquote{object}}}

Logger object that stores simulation events and other data.

\begin{sphinxadmonition}{note}{Note:}
Some attributes might not be documented, but should be straight
forward to understand after inspecting their usage in the source code.
\end{sphinxadmonition}
\index{cswc (LoggingData attribute)@\spxentry{cswc}\spxextra{LoggingData attribute}}

\begin{fulllineitems}
\phantomsection\label{\detokenize{app.domain.helpers:app.domain.helpers.smart_dataclasses.LoggingData.cswc}}\pysigline{\sphinxbfcode{\sphinxupquote{cswc}}}
Indicates how many consecutive steps a file as been in
convergence. Once convergence is not verified by
\sphinxcode{\sphinxupquote{equal\_distributions()}}
this attribute is reset to zero.
\begin{quote}\begin{description}
\item[{Type}] \leavevmode
\sphinxhref{https://docs.python.org/3.7/library/functions.html\#int}{int}

\end{description}\end{quote}

\end{fulllineitems}

\index{largest\_convergence\_window (LoggingData attribute)@\spxentry{largest\_convergence\_window}\spxextra{LoggingData attribute}}

\begin{fulllineitems}
\phantomsection\label{\detokenize{app.domain.helpers:app.domain.helpers.smart_dataclasses.LoggingData.largest_convergence_window}}\pysigline{\sphinxbfcode{\sphinxupquote{largest\_convergence\_window}}}
Stores the largest convergence window that occurred throughout
the simulation, i.e., it stores the highest verified
{\hyperref[\detokenize{app.domain.helpers:app.domain.helpers.smart_dataclasses.LoggingData.cswc}]{\sphinxcrossref{\sphinxcode{\sphinxupquote{cswc}}}}}.
\begin{quote}\begin{description}
\item[{Type}] \leavevmode
\sphinxhref{https://docs.python.org/3.7/library/functions.html\#int}{int}

\end{description}\end{quote}

\end{fulllineitems}

\index{convergence\_set (LoggingData attribute)@\spxentry{convergence\_set}\spxextra{LoggingData attribute}}

\begin{fulllineitems}
\phantomsection\label{\detokenize{app.domain.helpers:app.domain.helpers.smart_dataclasses.LoggingData.convergence_set}}\pysigline{\sphinxbfcode{\sphinxupquote{convergence\_set}}}
Set of consecutive epochs in which convergence was verified.
This list only stores the most up to date convergence set and like
{\hyperref[\detokenize{app.domain.helpers:app.domain.helpers.smart_dataclasses.LoggingData.cswc}]{\sphinxcrossref{\sphinxcode{\sphinxupquote{cswc}}}}} is cleared once convergence is not verified,
after being appended to {\hyperref[\detokenize{app.domain.helpers:app.domain.helpers.smart_dataclasses.LoggingData.convergence_sets}]{\sphinxcrossref{\sphinxcode{\sphinxupquote{convergence\_sets}}}}}.
\begin{quote}\begin{description}
\item[{Type}] \leavevmode
List{[}\sphinxhref{https://docs.python.org/3.7/library/functions.html\#int}{int}{]}

\end{description}\end{quote}

\end{fulllineitems}

\index{convergence\_sets (LoggingData attribute)@\spxentry{convergence\_sets}\spxextra{LoggingData attribute}}

\begin{fulllineitems}
\phantomsection\label{\detokenize{app.domain.helpers:app.domain.helpers.smart_dataclasses.LoggingData.convergence_sets}}\pysigline{\sphinxbfcode{\sphinxupquote{convergence\_sets}}}
Stores all but the most recent {\hyperref[\detokenize{app.domain.helpers:app.domain.helpers.smart_dataclasses.LoggingData.convergence_set}]{\sphinxcrossref{\sphinxcode{\sphinxupquote{convergence\_set}}}}}. If
simulation terminates and {\hyperref[\detokenize{app.domain.helpers:app.domain.helpers.smart_dataclasses.LoggingData.convergence_set}]{\sphinxcrossref{\sphinxcode{\sphinxupquote{convergence\_set}}}}} is not an
empty list, that list will be appended to this one.
\begin{quote}\begin{description}
\item[{Type}] \leavevmode
List{[}List{[}\sphinxhref{https://docs.python.org/3.7/library/functions.html\#int}{int}{]}{]}

\end{description}\end{quote}

\end{fulllineitems}

\index{terminated (LoggingData attribute)@\spxentry{terminated}\spxextra{LoggingData attribute}}

\begin{fulllineitems}
\phantomsection\label{\detokenize{app.domain.helpers:app.domain.helpers.smart_dataclasses.LoggingData.terminated}}\pysigline{\sphinxbfcode{\sphinxupquote{terminated}}}
Indicates the epoch at which the simulation was terminated.
\begin{quote}\begin{description}
\item[{Type}] \leavevmode
\sphinxhref{https://docs.python.org/3.7/library/functions.html\#int}{int}

\end{description}\end{quote}

\end{fulllineitems}

\index{terminated\_messages (LoggingData attribute)@\spxentry{terminated\_messages}\spxextra{LoggingData attribute}}

\begin{fulllineitems}
\phantomsection\label{\detokenize{app.domain.helpers:app.domain.helpers.smart_dataclasses.LoggingData.terminated_messages}}\pysigline{\sphinxbfcode{\sphinxupquote{terminated\_messages}}}
Set of at least one error message that led to the failure
of the simulation or one success message, at
{\hyperref[\detokenize{app.domain.helpers:app.domain.helpers.smart_dataclasses.LoggingData.terminated}]{\sphinxcrossref{\sphinxcode{\sphinxupquote{termination epoch}}}}}.
\begin{quote}\begin{description}
\item[{Type}] \leavevmode
List{[}\sphinxhref{https://docs.python.org/3.7/library/stdtypes.html\#str}{str}{]}

\end{description}\end{quote}

\end{fulllineitems}

\index{successfull (LoggingData attribute)@\spxentry{successfull}\spxextra{LoggingData attribute}}

\begin{fulllineitems}
\phantomsection\label{\detokenize{app.domain.helpers:app.domain.helpers.smart_dataclasses.LoggingData.successfull}}\pysigline{\sphinxbfcode{\sphinxupquote{successfull}}}
When the simulation is {\hyperref[\detokenize{app.domain.helpers:app.domain.helpers.smart_dataclasses.LoggingData.terminated}]{\sphinxcrossref{\sphinxcode{\sphinxupquote{terminated}}}}}, this value is set
to \sphinxcode{\sphinxupquote{True}} if no errors or failures occurred, i.e., if the
simulation managed to persist the file throughout the entire
\sphinxcode{\sphinxupquote{simulation epochs}}.
\begin{quote}\begin{description}
\item[{Type}] \leavevmode
\sphinxhref{https://docs.python.org/3.7/library/functions.html\#bool}{bool}

\end{description}\end{quote}

\end{fulllineitems}

\index{blocks\_corrupted (LoggingData attribute)@\spxentry{blocks\_corrupted}\spxextra{LoggingData attribute}}

\begin{fulllineitems}
\phantomsection\label{\detokenize{app.domain.helpers:app.domain.helpers.smart_dataclasses.LoggingData.blocks_corrupted}}\pysigline{\sphinxbfcode{\sphinxupquote{blocks\_corrupted}}}
The number of {\hyperref[\detokenize{app.domain.helpers:app.domain.helpers.smart_dataclasses.FileBlockData}]{\sphinxcrossref{\sphinxcode{\sphinxupquote{file block replicas}}}}} lost at
each simulation epoch due to disk errors.
\begin{quote}\begin{description}
\item[{Type}] \leavevmode
List{[}\sphinxhref{https://docs.python.org/3.7/library/functions.html\#int}{int}{]}

\end{description}\end{quote}

\end{fulllineitems}

\index{blocks\_existing (LoggingData attribute)@\spxentry{blocks\_existing}\spxextra{LoggingData attribute}}

\begin{fulllineitems}
\phantomsection\label{\detokenize{app.domain.helpers:app.domain.helpers.smart_dataclasses.LoggingData.blocks_existing}}\pysigline{\sphinxbfcode{\sphinxupquote{blocks\_existing}}}
The number of existing {\hyperref[\detokenize{app.domain.helpers:app.domain.helpers.smart_dataclasses.FileBlockData}]{\sphinxcrossref{\sphinxcode{\sphinxupquote{file block replicas}}}}} inside the
{\hyperref[\detokenize{app.domain:module-app.domain.cluster_groups}]{\sphinxcrossref{\sphinxcode{\sphinxupquote{cluster group}}}}} members’ storage
disks at each epoch.
\begin{quote}\begin{description}
\item[{Type}] \leavevmode
List{[}\sphinxhref{https://docs.python.org/3.7/library/functions.html\#int}{int}{]}

\end{description}\end{quote}

\end{fulllineitems}

\index{blocks\_lost (LoggingData attribute)@\spxentry{blocks\_lost}\spxextra{LoggingData attribute}}

\begin{fulllineitems}
\phantomsection\label{\detokenize{app.domain.helpers:app.domain.helpers.smart_dataclasses.LoggingData.blocks_lost}}\pysigline{\sphinxbfcode{\sphinxupquote{blocks\_lost}}}
The number of {\hyperref[\detokenize{app.domain.helpers:app.domain.helpers.smart_dataclasses.FileBlockData}]{\sphinxcrossref{\sphinxcode{\sphinxupquote{file block replicas}}}}} that were
lost at each epoch due to {\hyperref[\detokenize{app.domain:module-app.domain.network_nodes}]{\sphinxcrossref{\sphinxcode{\sphinxupquote{network nodes}}}}} going offline.
\begin{quote}\begin{description}
\item[{Type}] \leavevmode
List{[}\sphinxhref{https://docs.python.org/3.7/library/functions.html\#int}{int}{]}

\end{description}\end{quote}

\end{fulllineitems}

\index{blocks\_moved (LoggingData attribute)@\spxentry{blocks\_moved}\spxextra{LoggingData attribute}}

\begin{fulllineitems}
\phantomsection\label{\detokenize{app.domain.helpers:app.domain.helpers.smart_dataclasses.LoggingData.blocks_moved}}\pysigline{\sphinxbfcode{\sphinxupquote{blocks\_moved}}}
The number of messages containing {\hyperref[\detokenize{app.domain.helpers:app.domain.helpers.smart_dataclasses.FileBlockData}]{\sphinxcrossref{\sphinxcode{\sphinxupquote{file block replicas}}}}} that were
transmited, including those that were not delivered or
acknowledged, at each epoch.
\begin{quote}\begin{description}
\item[{Type}] \leavevmode
List{[}\sphinxhref{https://docs.python.org/3.7/library/functions.html\#int}{int}{]}

\end{description}\end{quote}

\end{fulllineitems}

\index{cluster\_size\_bm (LoggingData attribute)@\spxentry{cluster\_size\_bm}\spxextra{LoggingData attribute}}

\begin{fulllineitems}
\phantomsection\label{\detokenize{app.domain.helpers:app.domain.helpers.smart_dataclasses.LoggingData.cluster_size_bm}}\pysigline{\sphinxbfcode{\sphinxupquote{cluster\_size\_bm}}}
The number of {\hyperref[\detokenize{app.domain:module-app.domain.network_nodes}]{\sphinxcrossref{\sphinxcode{\sphinxupquote{network nodes}}}}}
registered at a  {\hyperref[\detokenize{app.domain:app.domain.cluster_groups.Cluster.members}]{\sphinxcrossref{\sphinxcode{\sphinxupquote{cluster group\textquotesingle{}s members list}}}}},
before the {\hyperref[\detokenize{app.domain:app.domain.cluster_groups.Cluster.membership_maintenance}]{\sphinxcrossref{\sphinxcode{\sphinxupquote{maintenance step}}}}}
of the epoch.
\begin{quote}\begin{description}
\item[{Type}] \leavevmode
List{[}\sphinxhref{https://docs.python.org/3.7/library/functions.html\#int}{int}{]}

\end{description}\end{quote}

\end{fulllineitems}

\index{cluster\_size\_am (LoggingData attribute)@\spxentry{cluster\_size\_am}\spxextra{LoggingData attribute}}

\begin{fulllineitems}
\phantomsection\label{\detokenize{app.domain.helpers:app.domain.helpers.smart_dataclasses.LoggingData.cluster_size_am}}\pysigline{\sphinxbfcode{\sphinxupquote{cluster\_size\_am}}}
The number of {\hyperref[\detokenize{app.domain:module-app.domain.network_nodes}]{\sphinxcrossref{\sphinxcode{\sphinxupquote{network nodes}}}}}
registered at a  {\hyperref[\detokenize{app.domain:app.domain.cluster_groups.Cluster.members}]{\sphinxcrossref{\sphinxcode{\sphinxupquote{cluster group\textquotesingle{}s members list}}}}},
after the {\hyperref[\detokenize{app.domain:app.domain.cluster_groups.Cluster.membership_maintenance}]{\sphinxcrossref{\sphinxcode{\sphinxupquote{maintenance step}}}}}
of the epoch.
\begin{quote}\begin{description}
\item[{Type}] \leavevmode
List{[}\sphinxhref{https://docs.python.org/3.7/library/functions.html\#int}{int}{]}

\end{description}\end{quote}

\end{fulllineitems}

\index{cluster\_status\_bm (LoggingData attribute)@\spxentry{cluster\_status\_bm}\spxextra{LoggingData attribute}}

\begin{fulllineitems}
\phantomsection\label{\detokenize{app.domain.helpers:app.domain.helpers.smart_dataclasses.LoggingData.cluster_status_bm}}\pysigline{\sphinxbfcode{\sphinxupquote{cluster\_status\_bm}}}
Strings describing the health of the {\hyperref[\detokenize{app.domain:app.domain.cluster_groups.Cluster}]{\sphinxcrossref{\sphinxcode{\sphinxupquote{cluster group}}}}} at each epoch,
before the {\hyperref[\detokenize{app.domain:app.domain.cluster_groups.Cluster.membership_maintenance}]{\sphinxcrossref{\sphinxcode{\sphinxupquote{maintenance step}}}}}
of the epoch.
\begin{quote}\begin{description}
\item[{Type}] \leavevmode
List{[}\sphinxhref{https://docs.python.org/3.7/library/stdtypes.html\#str}{str}{]}

\end{description}\end{quote}

\end{fulllineitems}

\index{cluster\_status\_am (LoggingData attribute)@\spxentry{cluster\_status\_am}\spxextra{LoggingData attribute}}

\begin{fulllineitems}
\phantomsection\label{\detokenize{app.domain.helpers:app.domain.helpers.smart_dataclasses.LoggingData.cluster_status_am}}\pysigline{\sphinxbfcode{\sphinxupquote{cluster\_status\_am}}}
Strings describing the health of the {\hyperref[\detokenize{app.domain:app.domain.cluster_groups.Cluster}]{\sphinxcrossref{\sphinxcode{\sphinxupquote{cluster group}}}}} at each epoch,
after the {\hyperref[\detokenize{app.domain:app.domain.cluster_groups.Cluster.membership_maintenance}]{\sphinxcrossref{\sphinxcode{\sphinxupquote{maintenance step}}}}}
of the epoch.
\begin{quote}\begin{description}
\item[{Type}] \leavevmode
List{[}\sphinxhref{https://docs.python.org/3.7/library/stdtypes.html\#str}{str}{]}

\end{description}\end{quote}

\end{fulllineitems}

\index{delay\_replication (LoggingData attribute)@\spxentry{delay\_replication}\spxextra{LoggingData attribute}}

\begin{fulllineitems}
\phantomsection\label{\detokenize{app.domain.helpers:app.domain.helpers.smart_dataclasses.LoggingData.delay_replication}}\pysigline{\sphinxbfcode{\sphinxupquote{delay\_replication}}}
Log of the average time it took to recover one or more lost
{\hyperref[\detokenize{app.domain.helpers:app.domain.helpers.smart_dataclasses.FileBlockData}]{\sphinxcrossref{\sphinxcode{\sphinxupquote{file block replicas}}}}}, at each
epoch.
\begin{quote}\begin{description}
\item[{Type}] \leavevmode
List{[}\sphinxhref{https://docs.python.org/3.7/library/functions.html\#float}{float}{]}

\end{description}\end{quote}

\end{fulllineitems}

\index{delay\_suspects\_detection (LoggingData attribute)@\spxentry{delay\_suspects\_detection}\spxextra{LoggingData attribute}}

\begin{fulllineitems}
\phantomsection\label{\detokenize{app.domain.helpers:app.domain.helpers.smart_dataclasses.LoggingData.delay_suspects_detection}}\pysigline{\sphinxbfcode{\sphinxupquote{delay\_suspects\_detection}}}
Log of the time it took for each suspicious
{\hyperref[\detokenize{app.domain:module-app.domain.network_nodes}]{\sphinxcrossref{\sphinxcode{\sphinxupquote{network node}}}}} to be evicted
from the his {\hyperref[\detokenize{app.domain:module-app.domain.cluster_groups}]{\sphinxcrossref{\sphinxcode{\sphinxupquote{cluster group}}}}}
after having his {\hyperref[\detokenize{app.domain:app.domain.network_nodes.Node.status}]{\sphinxcrossref{\sphinxcode{\sphinxupquote{status}}}}}
changed from online to offline or suspicious.
\begin{quote}\begin{description}
\item[{Type}] \leavevmode
Dict{[}\sphinxhref{https://docs.python.org/3.7/library/functions.html\#int}{int}, \sphinxhref{https://docs.python.org/3.7/library/stdtypes.html\#str}{str}{]}

\end{description}\end{quote}

\end{fulllineitems}

\index{initial\_spread (LoggingData attribute)@\spxentry{initial\_spread}\spxextra{LoggingData attribute}}

\begin{fulllineitems}
\phantomsection\label{\detokenize{app.domain.helpers:app.domain.helpers.smart_dataclasses.LoggingData.initial_spread}}\pysigline{\sphinxbfcode{\sphinxupquote{initial\_spread}}}
Records the strategy used distribute file blocks in the
beggining of the simulation. See
{\hyperref[\detokenize{app.domain:app.domain.cluster_groups.Cluster.spread_files}]{\sphinxcrossref{\sphinxcode{\sphinxupquote{spread\_files()}}}}}.
\begin{quote}\begin{description}
\item[{Type}] \leavevmode
\sphinxhref{https://docs.python.org/3.7/library/stdtypes.html\#str}{str}

\end{description}\end{quote}

\end{fulllineitems}

\index{matrices\_nodes\_degrees (LoggingData attribute)@\spxentry{matrices\_nodes\_degrees}\spxextra{LoggingData attribute}}

\begin{fulllineitems}
\phantomsection\label{\detokenize{app.domain.helpers:app.domain.helpers.smart_dataclasses.LoggingData.matrices_nodes_degrees}}\pysigline{\sphinxbfcode{\sphinxupquote{matrices\_nodes\_degrees}}}
Stores the \sphinxcode{\sphinxupquote{in\sphinxhyphen{}degree}} and \sphinxcode{\sphinxupquote{out\sphinxhyphen{}degree}} of each
{\hyperref[\detokenize{app.domain:module-app.domain.network_nodes}]{\sphinxcrossref{\sphinxcode{\sphinxupquote{network node}}}}} in the
{\hyperref[\detokenize{app.domain:module-app.domain.cluster_groups}]{\sphinxcrossref{\sphinxcode{\sphinxupquote{cluster group}}}}}. One dictionary
is kept in the list for each transition matrix used throughout
the simulation. The integral part of the float value is the
in\sphinxhyphen{}degree, the decimal part is the out\sphinxhyphen{}degree.
\begin{quote}\begin{description}
\item[{Type}] \leavevmode
List{[}Dict{[}\sphinxhref{https://docs.python.org/3.7/library/stdtypes.html\#str}{str}, \sphinxhref{https://docs.python.org/3.7/library/functions.html\#float}{float}{]}{]}

\end{description}\end{quote}

\end{fulllineitems}

\index{off\_node\_count (LoggingData attribute)@\spxentry{off\_node\_count}\spxextra{LoggingData attribute}}

\begin{fulllineitems}
\phantomsection\label{\detokenize{app.domain.helpers:app.domain.helpers.smart_dataclasses.LoggingData.off_node_count}}\pysigline{\sphinxbfcode{\sphinxupquote{off\_node\_count}}}
The number of {\hyperref[\detokenize{app.domain:module-app.domain.network_nodes}]{\sphinxcrossref{\sphinxcode{\sphinxupquote{network nodes}}}}}
whose status changed to offline or suspicious, at each epoch.
\begin{quote}\begin{description}
\item[{Type}] \leavevmode
List{[}\sphinxhref{https://docs.python.org/3.7/library/functions.html\#int}{int}{]}

\end{description}\end{quote}

\end{fulllineitems}

\index{topologies\_goal\_achieved (LoggingData attribute)@\spxentry{topologies\_goal\_achieved}\spxextra{LoggingData attribute}}

\begin{fulllineitems}
\phantomsection\label{\detokenize{app.domain.helpers:app.domain.helpers.smart_dataclasses.LoggingData.topologies_goal_achieved}}\pysigline{\sphinxbfcode{\sphinxupquote{topologies\_goal\_achieved}}}
Stores if a boolean indicating if each topology achieved the
desired density distribution, on average.
\begin{quote}\begin{description}
\item[{Type}] \leavevmode
List{[}\sphinxhref{https://docs.python.org/3.7/library/functions.html\#bool}{bool}{]}

\end{description}\end{quote}

\end{fulllineitems}

\index{topologies\_goal\_distance (LoggingData attribute)@\spxentry{topologies\_goal\_distance}\spxextra{LoggingData attribute}}

\begin{fulllineitems}
\phantomsection\label{\detokenize{app.domain.helpers:app.domain.helpers.smart_dataclasses.LoggingData.topologies_goal_distance}}\pysigline{\sphinxbfcode{\sphinxupquote{topologies\_goal\_distance}}}
Stores magnitude difference between the desired density
distribution and the topologies’ average density distribution.
\begin{quote}\begin{description}
\item[{Type}] \leavevmode
List{[}\sphinxhref{https://docs.python.org/3.7/library/functions.html\#float}{float}{]}

\end{description}\end{quote}

\end{fulllineitems}

\index{transmissions\_failed (LoggingData attribute)@\spxentry{transmissions\_failed}\spxextra{LoggingData attribute}}

\begin{fulllineitems}
\phantomsection\label{\detokenize{app.domain.helpers:app.domain.helpers.smart_dataclasses.LoggingData.transmissions_failed}}\pysigline{\sphinxbfcode{\sphinxupquote{transmissions\_failed}}}
The number of message transmissions that were lost in the
overlay network of a {\hyperref[\detokenize{app.domain:module-app.domain.cluster_groups}]{\sphinxcrossref{\sphinxcode{\sphinxupquote{cluster group}}}}}, at each epoch.
\begin{quote}\begin{description}
\item[{Type}] \leavevmode
List{[}\sphinxhref{https://docs.python.org/3.7/library/functions.html\#int}{int}{]}

\end{description}\end{quote}

\end{fulllineitems}

\index{\_\_init\_\_() (LoggingData method)@\spxentry{\_\_init\_\_()}\spxextra{LoggingData method}}

\begin{fulllineitems}
\phantomsection\label{\detokenize{app.domain.helpers:app.domain.helpers.smart_dataclasses.LoggingData.__init__}}\pysiglinewithargsret{\sphinxbfcode{\sphinxupquote{\_\_init\_\_}}}{}{}
Instanciates a \sphinxcode{\sphinxupquote{LoggingData}} object.
\begin{quote}\begin{description}
\item[{Return type}] \leavevmode
\sphinxhref{https://docs.python.org/3.7/library/constants.html\#None}{None}

\end{description}\end{quote}

\end{fulllineitems}

\index{\_recursive\_len() (LoggingData method)@\spxentry{\_recursive\_len()}\spxextra{LoggingData method}}

\begin{fulllineitems}
\phantomsection\label{\detokenize{app.domain.helpers:app.domain.helpers.smart_dataclasses.LoggingData._recursive_len}}\pysiglinewithargsret{\sphinxbfcode{\sphinxupquote{\_recursive\_len}}}{\emph{\DUrole{n}{item}}}{}
Recusively sums the length of all lists in {\hyperref[\detokenize{app.domain.helpers:app.domain.helpers.smart_dataclasses.LoggingData.convergence_sets}]{\sphinxcrossref{\sphinxcode{\sphinxupquote{convergence\_sets}}}}}.
\begin{quote}\begin{description}
\item[{Parameters}] \leavevmode
\sphinxstyleliteralstrong{\sphinxupquote{item}} (\sphinxstyleliteralemphasis{\sphinxupquote{Any}}) \textendash{} A sub list of {\hyperref[\detokenize{app.domain.helpers:app.domain.helpers.smart_dataclasses.LoggingData.convergence_sets}]{\sphinxcrossref{\sphinxcode{\sphinxupquote{convergence\_sets}}}}} that needs that
as not yet been counted.

\item[{Returns}] \leavevmode
The number of epochs that were registered at the inputed sub list.

\item[{Return type}] \leavevmode
\sphinxhref{https://docs.python.org/3.7/library/functions.html\#int}{int}

\end{description}\end{quote}

\end{fulllineitems}

\index{log\_bandwidth\_units() (LoggingData method)@\spxentry{log\_bandwidth\_units()}\spxextra{LoggingData method}}

\begin{fulllineitems}
\phantomsection\label{\detokenize{app.domain.helpers:app.domain.helpers.smart_dataclasses.LoggingData.log_bandwidth_units}}\pysiglinewithargsret{\sphinxbfcode{\sphinxupquote{log\_bandwidth\_units}}}{\emph{\DUrole{n}{n}}, \emph{\DUrole{n}{epoch}}}{}
Logs the amount of moved file blocks moved at an epoch.
\begin{quote}\begin{description}
\item[{Parameters}] \leavevmode\begin{itemize}
\item {} 
\sphinxstyleliteralstrong{\sphinxupquote{n}} (\sphinxhref{https://docs.python.org/3.7/library/functions.html\#int}{\sphinxstyleliteralemphasis{\sphinxupquote{int}}}) \textendash{} Number of parts moved at epoch.

\item {} 
\sphinxstyleliteralstrong{\sphinxupquote{epoch}} (\sphinxhref{https://docs.python.org/3.7/library/functions.html\#int}{\sphinxstyleliteralemphasis{\sphinxupquote{int}}}) \textendash{} A simulation epoch index.

\end{itemize}

\item[{Return type}] \leavevmode
\sphinxhref{https://docs.python.org/3.7/library/constants.html\#None}{None}

\end{description}\end{quote}

\end{fulllineitems}

\index{log\_corrupted\_file\_blocks() (LoggingData method)@\spxentry{log\_corrupted\_file\_blocks()}\spxextra{LoggingData method}}

\begin{fulllineitems}
\phantomsection\label{\detokenize{app.domain.helpers:app.domain.helpers.smart_dataclasses.LoggingData.log_corrupted_file_blocks}}\pysiglinewithargsret{\sphinxbfcode{\sphinxupquote{log\_corrupted\_file\_blocks}}}{\emph{\DUrole{n}{n}}, \emph{\DUrole{n}{epoch}}}{}
Logs the amount of corrupted file block blocks at an epoch.
\begin{quote}\begin{description}
\item[{Parameters}] \leavevmode\begin{itemize}
\item {} 
\sphinxstyleliteralstrong{\sphinxupquote{n}} (\sphinxhref{https://docs.python.org/3.7/library/functions.html\#int}{\sphinxstyleliteralemphasis{\sphinxupquote{int}}}) \textendash{} Number of corrupted blocks

\item {} 
\sphinxstyleliteralstrong{\sphinxupquote{epoch}} (\sphinxhref{https://docs.python.org/3.7/library/functions.html\#int}{\sphinxstyleliteralemphasis{\sphinxupquote{int}}}) \textendash{} A simulation epoch index.

\end{itemize}

\item[{Return type}] \leavevmode
\sphinxhref{https://docs.python.org/3.7/library/constants.html\#None}{None}

\end{description}\end{quote}

\end{fulllineitems}

\index{log\_existing\_file\_blocks() (LoggingData method)@\spxentry{log\_existing\_file\_blocks()}\spxextra{LoggingData method}}

\begin{fulllineitems}
\phantomsection\label{\detokenize{app.domain.helpers:app.domain.helpers.smart_dataclasses.LoggingData.log_existing_file_blocks}}\pysiglinewithargsret{\sphinxbfcode{\sphinxupquote{log\_existing\_file\_blocks}}}{\emph{\DUrole{n}{n}}, \emph{\DUrole{n}{epoch}}}{}
Logs the amount of existing file blocks in the simulation environment at an epoch.
\begin{quote}\begin{description}
\item[{Parameters}] \leavevmode\begin{itemize}
\item {} 
\sphinxstyleliteralstrong{\sphinxupquote{n}} (\sphinxhref{https://docs.python.org/3.7/library/functions.html\#int}{\sphinxstyleliteralemphasis{\sphinxupquote{int}}}) \textendash{} Number of file blocks in the system.

\item {} 
\sphinxstyleliteralstrong{\sphinxupquote{epoch}} (\sphinxhref{https://docs.python.org/3.7/library/functions.html\#int}{\sphinxstyleliteralemphasis{\sphinxupquote{int}}}) \textendash{} A simulation epoch index.

\end{itemize}

\item[{Return type}] \leavevmode
\sphinxhref{https://docs.python.org/3.7/library/constants.html\#None}{None}

\end{description}\end{quote}

\end{fulllineitems}

\index{log\_fail() (LoggingData method)@\spxentry{log\_fail()}\spxextra{LoggingData method}}

\begin{fulllineitems}
\phantomsection\label{\detokenize{app.domain.helpers:app.domain.helpers.smart_dataclasses.LoggingData.log_fail}}\pysiglinewithargsret{\sphinxbfcode{\sphinxupquote{log\_fail}}}{\emph{\DUrole{n}{epoch}}, \emph{\DUrole{n}{message}\DUrole{o}{=}\DUrole{default_value}{\textquotesingle{}\textquotesingle{}}}}{}
Logs the epoch at which a simulation terminated due to a failure.

\begin{sphinxadmonition}{note}{Note:}
This method should only be called when simulation terminates due
to a failure such as a the loss of all blocks of a file block
or the simultaneous disconnection of all network nodes in the cluster.
\end{sphinxadmonition}
\begin{quote}\begin{description}
\item[{Parameters}] \leavevmode\begin{itemize}
\item {} 
\sphinxstyleliteralstrong{\sphinxupquote{message}} (\sphinxhref{https://docs.python.org/3.7/library/stdtypes.html\#str}{\sphinxstyleliteralemphasis{\sphinxupquote{str}}}) \textendash{} A log error message.

\item {} 
\sphinxstyleliteralstrong{\sphinxupquote{epoch}} (\sphinxhref{https://docs.python.org/3.7/library/functions.html\#int}{\sphinxstyleliteralemphasis{\sphinxupquote{int}}}) \textendash{} A simulation epoch at which termination occurred.

\end{itemize}

\item[{Return type}] \leavevmode
\sphinxhref{https://docs.python.org/3.7/library/constants.html\#None}{None}

\end{description}\end{quote}

\end{fulllineitems}

\index{log\_lost\_file\_blocks() (LoggingData method)@\spxentry{log\_lost\_file\_blocks()}\spxextra{LoggingData method}}

\begin{fulllineitems}
\phantomsection\label{\detokenize{app.domain.helpers:app.domain.helpers.smart_dataclasses.LoggingData.log_lost_file_blocks}}\pysiglinewithargsret{\sphinxbfcode{\sphinxupquote{log\_lost\_file\_blocks}}}{\emph{\DUrole{n}{n}}, \emph{\DUrole{n}{epoch}}}{}
Logs the amount of permanently lost file block blocks at an epoch.
\begin{quote}\begin{description}
\item[{Parameters}] \leavevmode\begin{itemize}
\item {} 
\sphinxstyleliteralstrong{\sphinxupquote{n}} (\sphinxhref{https://docs.python.org/3.7/library/functions.html\#int}{\sphinxstyleliteralemphasis{\sphinxupquote{int}}}) \textendash{} Number of blocks that were lost.

\item {} 
\sphinxstyleliteralstrong{\sphinxupquote{epoch}} (\sphinxhref{https://docs.python.org/3.7/library/functions.html\#int}{\sphinxstyleliteralemphasis{\sphinxupquote{int}}}) \textendash{} A simulation epoch index.

\end{itemize}

\item[{Return type}] \leavevmode
\sphinxhref{https://docs.python.org/3.7/library/constants.html\#None}{None}

\end{description}\end{quote}

\end{fulllineitems}

\index{log\_lost\_messages() (LoggingData method)@\spxentry{log\_lost\_messages()}\spxextra{LoggingData method}}

\begin{fulllineitems}
\phantomsection\label{\detokenize{app.domain.helpers:app.domain.helpers.smart_dataclasses.LoggingData.log_lost_messages}}\pysiglinewithargsret{\sphinxbfcode{\sphinxupquote{log\_lost\_messages}}}{\emph{\DUrole{n}{n}}, \emph{\DUrole{n}{epoch}}}{}
Logs the amount of failed message transmissions at an epoch.
\begin{quote}\begin{description}
\item[{Parameters}] \leavevmode\begin{itemize}
\item {} 
\sphinxstyleliteralstrong{\sphinxupquote{n}} (\sphinxhref{https://docs.python.org/3.7/library/functions.html\#int}{\sphinxstyleliteralemphasis{\sphinxupquote{int}}}) \textendash{} Number of lost terminated\_messages.

\item {} 
\sphinxstyleliteralstrong{\sphinxupquote{epoch}} (\sphinxhref{https://docs.python.org/3.7/library/functions.html\#int}{\sphinxstyleliteralemphasis{\sphinxupquote{int}}}) \textendash{} A simulation epoch index.

\end{itemize}

\item[{Return type}] \leavevmode
\sphinxhref{https://docs.python.org/3.7/library/constants.html\#None}{None}

\end{description}\end{quote}

\end{fulllineitems}

\index{log\_maintenance() (LoggingData method)@\spxentry{log\_maintenance()}\spxextra{LoggingData method}}

\begin{fulllineitems}
\phantomsection\label{\detokenize{app.domain.helpers:app.domain.helpers.smart_dataclasses.LoggingData.log_maintenance}}\pysiglinewithargsret{\sphinxbfcode{\sphinxupquote{log\_maintenance}}}{\emph{\DUrole{n}{size\_bm}}, \emph{\DUrole{n}{size\_am}}, \emph{\DUrole{n}{status\_bm}}, \emph{\DUrole{n}{status\_am}}, \emph{\DUrole{n}{epoch}}}{}
Logs cluster membership status and size at an epoch.
\begin{quote}\begin{description}
\item[{Parameters}] \leavevmode\begin{itemize}
\item {} 
\sphinxstyleliteralstrong{\sphinxupquote{size\_bm}} (\sphinxhref{https://docs.python.org/3.7/library/functions.html\#int}{\sphinxstyleliteralemphasis{\sphinxupquote{int}}}) \textendash{} The number of network nodes in the cluster before maintenance.

\item {} 
\sphinxstyleliteralstrong{\sphinxupquote{size\_am}} (\sphinxhref{https://docs.python.org/3.7/library/functions.html\#int}{\sphinxstyleliteralemphasis{\sphinxupquote{int}}}) \textendash{} The number of network nodes in the cluster after maintenance.

\item {} 
\sphinxstyleliteralstrong{\sphinxupquote{status\_bm}} (\sphinxhref{https://docs.python.org/3.7/library/stdtypes.html\#str}{\sphinxstyleliteralemphasis{\sphinxupquote{str}}}) \textendash{} A string that describes the status of the cluster before
maintenance.

\item {} 
\sphinxstyleliteralstrong{\sphinxupquote{status\_am}} (\sphinxhref{https://docs.python.org/3.7/library/stdtypes.html\#str}{\sphinxstyleliteralemphasis{\sphinxupquote{str}}}) \textendash{} A string that describes the status of the cluster after
maintenance.

\item {} 
\sphinxstyleliteralstrong{\sphinxupquote{epoch}} (\sphinxhref{https://docs.python.org/3.7/library/functions.html\#int}{\sphinxstyleliteralemphasis{\sphinxupquote{int}}}) \textendash{} A simulation epoch at which termination occurred.

\end{itemize}

\item[{Return type}] \leavevmode
\sphinxhref{https://docs.python.org/3.7/library/constants.html\#None}{None}

\end{description}\end{quote}

\end{fulllineitems}

\index{log\_matrices\_degrees() (LoggingData method)@\spxentry{log\_matrices\_degrees()}\spxextra{LoggingData method}}

\begin{fulllineitems}
\phantomsection\label{\detokenize{app.domain.helpers:app.domain.helpers.smart_dataclasses.LoggingData.log_matrices_degrees}}\pysiglinewithargsret{\sphinxbfcode{\sphinxupquote{log\_matrices\_degrees}}}{\emph{\DUrole{n}{nodes\_degrees}}}{}
Logs the degree of all nodes in a Markov Matrix overlay, at the
time of its creation, before any faults on the overlay occurs.
\begin{quote}\begin{description}
\item[{Parameters}] \leavevmode
\sphinxstyleliteralstrong{\sphinxupquote{nodes\_degrees}} (\sphinxstyleliteralemphasis{\sphinxupquote{Dict}}\sphinxstyleliteralemphasis{\sphinxupquote{{[}}}\sphinxhref{https://docs.python.org/3.7/library/stdtypes.html\#str}{\sphinxstyleliteralemphasis{\sphinxupquote{str}}}\sphinxstyleliteralemphasis{\sphinxupquote{, }}\sphinxhref{https://docs.python.org/3.7/library/stdtypes.html\#str}{\sphinxstyleliteralemphasis{\sphinxupquote{str}}}\sphinxstyleliteralemphasis{\sphinxupquote{{]}}}) \textendash{} A dictionary mapping the {\hyperref[\detokenize{app.domain:app.domain.network_nodes.Node.id}]{\sphinxcrossref{\sphinxcode{\sphinxupquote{node identifiers}}}}} to their \sphinxcode{\sphinxupquote{in\sphinxhyphen{}degree}}
and \sphinxcode{\sphinxupquote{out\sphinxhyphen{}degree}} seperated by the delimiter \sphinxcode{\sphinxupquote{i\#o}}.

\end{description}\end{quote}

\end{fulllineitems}

\index{log\_off\_nodes() (LoggingData method)@\spxentry{log\_off\_nodes()}\spxextra{LoggingData method}}

\begin{fulllineitems}
\phantomsection\label{\detokenize{app.domain.helpers:app.domain.helpers.smart_dataclasses.LoggingData.log_off_nodes}}\pysiglinewithargsret{\sphinxbfcode{\sphinxupquote{log\_off\_nodes}}}{\emph{\DUrole{n}{n}}, \emph{\DUrole{n}{epoch}}}{}
Logs the amount of disconnected network\_nodes at an epoch.
\begin{quote}\begin{description}
\item[{Parameters}] \leavevmode\begin{itemize}
\item {} 
\sphinxstyleliteralstrong{\sphinxupquote{n}} (\sphinxhref{https://docs.python.org/3.7/library/functions.html\#int}{\sphinxstyleliteralemphasis{\sphinxupquote{int}}}) \textendash{} Number of disconnected network\_nodes in the system.

\item {} 
\sphinxstyleliteralstrong{\sphinxupquote{epoch}} (\sphinxhref{https://docs.python.org/3.7/library/functions.html\#int}{\sphinxstyleliteralemphasis{\sphinxupquote{int}}}) \textendash{} A simulation epoch index.

\end{itemize}

\item[{Return type}] \leavevmode
\sphinxhref{https://docs.python.org/3.7/library/constants.html\#None}{None}

\end{description}\end{quote}

\end{fulllineitems}

\index{log\_replication\_delay() (LoggingData method)@\spxentry{log\_replication\_delay()}\spxextra{LoggingData method}}

\begin{fulllineitems}
\phantomsection\label{\detokenize{app.domain.helpers:app.domain.helpers.smart_dataclasses.LoggingData.log_replication_delay}}\pysiglinewithargsret{\sphinxbfcode{\sphinxupquote{log\_replication\_delay}}}{\emph{\DUrole{n}{delay}}, \emph{\DUrole{n}{calls}}, \emph{\DUrole{n}{epoch}}}{}
Logs the expected delay\_replication at epoch at an epoch.
\begin{quote}\begin{description}
\item[{Parameters}] \leavevmode\begin{itemize}
\item {} 
\sphinxstyleliteralstrong{\sphinxupquote{delay}} (\sphinxhref{https://docs.python.org/3.7/library/functions.html\#int}{\sphinxstyleliteralemphasis{\sphinxupquote{int}}}) \textendash{} The delay sum.

\item {} 
\sphinxstyleliteralstrong{\sphinxupquote{calls}} (\sphinxhref{https://docs.python.org/3.7/library/functions.html\#int}{\sphinxstyleliteralemphasis{\sphinxupquote{int}}}) \textendash{} Number of times a delay\_replication was generated.

\item {} 
\sphinxstyleliteralstrong{\sphinxupquote{epoch}} (\sphinxhref{https://docs.python.org/3.7/library/functions.html\#int}{\sphinxstyleliteralemphasis{\sphinxupquote{int}}}) \textendash{} A simulation epoch index.

\end{itemize}

\item[{Return type}] \leavevmode
\sphinxhref{https://docs.python.org/3.7/library/constants.html\#None}{None}

\end{description}\end{quote}

\end{fulllineitems}

\index{log\_suspicous\_node\_detection\_delay() (LoggingData method)@\spxentry{log\_suspicous\_node\_detection\_delay()}\spxextra{LoggingData method}}

\begin{fulllineitems}
\phantomsection\label{\detokenize{app.domain.helpers:app.domain.helpers.smart_dataclasses.LoggingData.log_suspicous_node_detection_delay}}\pysiglinewithargsret{\sphinxbfcode{\sphinxupquote{log\_suspicous\_node\_detection\_delay}}}{\emph{\DUrole{n}{node\_id}}, \emph{\DUrole{n}{delay}}}{}
Logs the expected delay\_replication at epoch at an epoch.
\begin{quote}\begin{description}
\item[{Parameters}] \leavevmode\begin{itemize}
\item {} 
\sphinxstyleliteralstrong{\sphinxupquote{delay}} (\sphinxhref{https://docs.python.org/3.7/library/functions.html\#int}{\sphinxstyleliteralemphasis{\sphinxupquote{int}}}) \textendash{} The time it took until the specified node was evicted from a
{\hyperref[\detokenize{app.domain:module-app.domain.cluster_groups}]{\sphinxcrossref{\sphinxcode{\sphinxupquote{Cluster}}}}} after it was known
to be offline by the perfect failure detector.

\item {} 
\sphinxstyleliteralstrong{\sphinxupquote{node\_id}} (\sphinxhref{https://docs.python.org/3.7/library/stdtypes.html\#str}{\sphinxstyleliteralemphasis{\sphinxupquote{str}}}) \textendash{} A unique {\hyperref[\detokenize{app.domain:module-app.domain.network_nodes}]{\sphinxcrossref{\sphinxcode{\sphinxupquote{Network Node}}}}} identifier.

\end{itemize}

\item[{Return type}] \leavevmode
\sphinxhref{https://docs.python.org/3.7/library/constants.html\#None}{None}

\end{description}\end{quote}

\end{fulllineitems}

\index{log\_topology\_goal\_performance() (LoggingData method)@\spxentry{log\_topology\_goal\_performance()}\spxextra{LoggingData method}}

\begin{fulllineitems}
\phantomsection\label{\detokenize{app.domain.helpers:app.domain.helpers.smart_dataclasses.LoggingData.log_topology_goal_performance}}\pysiglinewithargsret{\sphinxbfcode{\sphinxupquote{log\_topology\_goal\_performance}}}{\emph{\DUrole{n}{achieved\_goal}}, \emph{\DUrole{n}{distance\_magnitude}}}{}
Logs wether or not the topology reached it’s goals, on average.
\begin{quote}\begin{description}
\item[{Parameters}] \leavevmode\begin{itemize}
\item {} 
\sphinxstyleliteralstrong{\sphinxupquote{achieved\_goal}} (\sphinxhref{https://docs.python.org/3.7/library/functions.html\#bool}{\sphinxstyleliteralemphasis{\sphinxupquote{bool}}}) \textendash{} Indicates wether or not the topology being registered
achieved it’s desired density distribution.

\item {} 
\sphinxstyleliteralstrong{\sphinxupquote{distance\_magnitude}} (\sphinxhref{https://docs.python.org/3.7/library/functions.html\#int}{\sphinxstyleliteralemphasis{\sphinxupquote{int}}}) \textendash{} The magnitude of the distance between the desired density
distribution and the topology’s average density distribution.

\end{itemize}

\item[{Return type}] \leavevmode
\sphinxhref{https://docs.python.org/3.7/library/constants.html\#None}{None}

\end{description}\end{quote}

\end{fulllineitems}

\index{register\_convergence() (LoggingData method)@\spxentry{register\_convergence()}\spxextra{LoggingData method}}

\begin{fulllineitems}
\phantomsection\label{\detokenize{app.domain.helpers:app.domain.helpers.smart_dataclasses.LoggingData.register_convergence}}\pysiglinewithargsret{\sphinxbfcode{\sphinxupquote{register\_convergence}}}{\emph{\DUrole{n}{epoch}}}{}
Increments {\hyperref[\detokenize{app.domain.helpers:app.domain.helpers.smart_dataclasses.LoggingData.cswc}]{\sphinxcrossref{\sphinxcode{\sphinxupquote{cswc}}}}} by one and tries to update the {\hyperref[\detokenize{app.domain.helpers:app.domain.helpers.smart_dataclasses.LoggingData.convergence_set}]{\sphinxcrossref{\sphinxcode{\sphinxupquote{convergence\_set}}}}}

Checks if the counter for consecutive epoch convergence is bigger
than {\hyperref[\detokenize{app:app.environment_settings.MIN_CONVERGENCE_THRESHOLD}]{\sphinxcrossref{\sphinxcode{\sphinxupquote{MIN\_CONVERGENCE\_THRESHOLD}}}}}
and if it is, it appends the \sphinxcode{\sphinxupquote{epoch}} to the most recent
{\hyperref[\detokenize{app.domain.helpers:app.domain.helpers.smart_dataclasses.LoggingData.convergence_set}]{\sphinxcrossref{\sphinxcode{\sphinxupquote{convergence\_set}}}}}.
\begin{quote}\begin{description}
\item[{Parameters}] \leavevmode
\sphinxstyleliteralstrong{\sphinxupquote{epoch}} (\sphinxhref{https://docs.python.org/3.7/library/functions.html\#int}{\sphinxstyleliteralemphasis{\sphinxupquote{int}}}) \textendash{} The simulation epoch at which the convergence was verified.

\item[{Return type}] \leavevmode
\sphinxhref{https://docs.python.org/3.7/library/constants.html\#None}{None}

\end{description}\end{quote}

\end{fulllineitems}

\index{save\_sets\_and\_reset() (LoggingData method)@\spxentry{save\_sets\_and\_reset()}\spxextra{LoggingData method}}

\begin{fulllineitems}
\phantomsection\label{\detokenize{app.domain.helpers:app.domain.helpers.smart_dataclasses.LoggingData.save_sets_and_reset}}\pysiglinewithargsret{\sphinxbfcode{\sphinxupquote{save\_sets\_and\_reset}}}{}{}
Resets all convergence variables

Tries to update {\hyperref[\detokenize{app.domain.helpers:app.domain.helpers.smart_dataclasses.LoggingData.largest_convergence_window}]{\sphinxcrossref{\sphinxcode{\sphinxupquote{largest\_convergence\_window}}}}} and
{\hyperref[\detokenize{app.domain.helpers:app.domain.helpers.smart_dataclasses.LoggingData.convergence_sets}]{\sphinxcrossref{\sphinxcode{\sphinxupquote{convergence\_sets}}}}} when {\hyperref[\detokenize{app.domain.helpers:app.domain.helpers.smart_dataclasses.LoggingData.convergence_set}]{\sphinxcrossref{\sphinxcode{\sphinxupquote{convergence\_set}}}}}
is not an empty list.
\begin{quote}\begin{description}
\item[{Return type}] \leavevmode
\sphinxhref{https://docs.python.org/3.7/library/constants.html\#None}{None}

\end{description}\end{quote}

\end{fulllineitems}


\end{fulllineitems}



\subsubsection{Submodules}
\label{\detokenize{app.domain:submodules}}

\subsubsection{app.domain.cluster\_groups}
\label{\detokenize{app.domain:module-app.domain.cluster_groups}}\label{\detokenize{app.domain:app-domain-cluster-groups}}\index{module@\spxentry{module}!app.domain.cluster\_groups@\spxentry{app.domain.cluster\_groups}}\index{app.domain.cluster\_groups@\spxentry{app.domain.cluster\_groups}!module@\spxentry{module}}
This module contains domain specific classes that represent groups of
{\hyperref[\detokenize{app.domain:module-app.domain.network_nodes}]{\sphinxcrossref{\sphinxcode{\sphinxupquote{storage nodes}}}}}.
\index{Cluster (class in app.domain.cluster\_groups)@\spxentry{Cluster}\spxextra{class in app.domain.cluster\_groups}}

\begin{fulllineitems}
\phantomsection\label{\detokenize{app.domain:app.domain.cluster_groups.Cluster}}\pysiglinewithargsret{\sphinxbfcode{\sphinxupquote{class }}\sphinxbfcode{\sphinxupquote{Cluster}}}{\emph{\DUrole{n}{master}}, \emph{\DUrole{n}{file\_name}}, \emph{\DUrole{n}{members}}, \emph{\DUrole{n}{sim\_id}\DUrole{o}{=}\DUrole{default_value}{0}}, \emph{\DUrole{n}{origin}\DUrole{o}{=}\DUrole{default_value}{\textquotesingle{}\textquotesingle{}}}}{}
Bases: \sphinxhref{https://docs.python.org/3.7/library/functions.html\#object}{\sphinxcode{\sphinxupquote{object}}}

Represents a group of network nodes ensuring the durability of a file.
\index{id (Cluster attribute)@\spxentry{id}\spxextra{Cluster attribute}}

\begin{fulllineitems}
\phantomsection\label{\detokenize{app.domain:app.domain.cluster_groups.Cluster.id}}\pysigline{\sphinxbfcode{\sphinxupquote{id}}}
A unique identifier of the \sphinxcode{\sphinxupquote{Cluster}} instance.
\begin{quote}\begin{description}
\item[{Type}] \leavevmode
\sphinxhref{https://docs.python.org/3.7/library/stdtypes.html\#str}{str}

\end{description}\end{quote}

\end{fulllineitems}

\index{current\_epoch (Cluster attribute)@\spxentry{current\_epoch}\spxextra{Cluster attribute}}

\begin{fulllineitems}
\phantomsection\label{\detokenize{app.domain:app.domain.cluster_groups.Cluster.current_epoch}}\pysigline{\sphinxbfcode{\sphinxupquote{current\_epoch}}}
The simulation’s current epoch.
\begin{quote}\begin{description}
\item[{Type}] \leavevmode
\sphinxhref{https://docs.python.org/3.7/library/functions.html\#int}{int}

\end{description}\end{quote}

\end{fulllineitems}

\index{corruption\_chances (Cluster attribute)@\spxentry{corruption\_chances}\spxextra{Cluster attribute}}

\begin{fulllineitems}
\phantomsection\label{\detokenize{app.domain:app.domain.cluster_groups.Cluster.corruption_chances}}\pysigline{\sphinxbfcode{\sphinxupquote{corruption\_chances}}}
A two\sphinxhyphen{}element list containing the probability of
{\hyperref[\detokenize{app.domain.helpers:app.domain.helpers.smart_dataclasses.FileBlockData}]{\sphinxcrossref{\sphinxcode{\sphinxupquote{FileBlockData}}}}}
being corrupted and not being corrupted, respectively. See
{\hyperref[\detokenize{app:app.environment_settings.get_disk_error_chances}]{\sphinxcrossref{\sphinxcode{\sphinxupquote{get\_disk\_error\_chances()}}}}}
for corruption chance configuration.
\begin{quote}\begin{description}
\item[{Type}] \leavevmode
List{[}\sphinxhref{https://docs.python.org/3.7/library/functions.html\#float}{float}{]}

\end{description}\end{quote}

\end{fulllineitems}

\index{master (Cluster attribute)@\spxentry{master}\spxextra{Cluster attribute}}

\begin{fulllineitems}
\phantomsection\label{\detokenize{app.domain:app.domain.cluster_groups.Cluster.master}}\pysigline{\sphinxbfcode{\sphinxupquote{master}}}
A reference to a server that coordinates or monitors the \sphinxcode{\sphinxupquote{Cluster}}.
\begin{quote}\begin{description}
\item[{Type}] \leavevmode
{\hyperref[\detokenize{app.domain:app.domain.master_servers.Master}]{\sphinxcrossref{\sphinxcode{\sphinxupquote{Master}}}}}

\end{description}\end{quote}

\end{fulllineitems}

\index{members (Cluster attribute)@\spxentry{members}\spxextra{Cluster attribute}}

\begin{fulllineitems}
\phantomsection\label{\detokenize{app.domain:app.domain.cluster_groups.Cluster.members}}\pysigline{\sphinxbfcode{\sphinxupquote{members}}}
A collection of network nodes that belong to the \sphinxcode{\sphinxupquote{Cluster}}.
\begin{quote}\begin{description}
\item[{Type}] \leavevmode
{\hyperref[\detokenize{app:app.type_hints.NodeDict}]{\sphinxcrossref{\sphinxcode{\sphinxupquote{NodeDict}}}}}

\end{description}\end{quote}

\end{fulllineitems}

\index{\_members\_view (Cluster attribute)@\spxentry{\_members\_view}\spxextra{Cluster attribute}}

\begin{fulllineitems}
\phantomsection\label{\detokenize{app.domain:app.domain.cluster_groups.Cluster._members_view}}\pysigline{\sphinxbfcode{\sphinxupquote{\_members\_view}}}
A list representation of the nodes in {\hyperref[\detokenize{app.domain:app.domain.cluster_groups.Cluster.members}]{\sphinxcrossref{\sphinxcode{\sphinxupquote{members}}}}}.
\begin{quote}\begin{description}
\item[{Type}] \leavevmode
List{[}{\hyperref[\detokenize{app:app.type_hints.NodeType}]{\sphinxcrossref{\sphinxcode{\sphinxupquote{NodeType}}}}}{]}

\end{description}\end{quote}

\end{fulllineitems}

\index{file (Cluster attribute)@\spxentry{file}\spxextra{Cluster attribute}}

\begin{fulllineitems}
\phantomsection\label{\detokenize{app.domain:app.domain.cluster_groups.Cluster.file}}\pysigline{\sphinxbfcode{\sphinxupquote{file}}}
A reference to
{\hyperref[\detokenize{app.domain.helpers:app.domain.helpers.smart_dataclasses.FileData}]{\sphinxcrossref{\sphinxcode{\sphinxupquote{FileData}}}}}
object that represents the file being persisted by the Cluster
instance.
\begin{quote}\begin{description}
\item[{Type}] \leavevmode
{\hyperref[\detokenize{app.domain.helpers:app.domain.helpers.smart_dataclasses.FileData}]{\sphinxcrossref{\sphinxcode{\sphinxupquote{FileData}}}}}

\end{description}\end{quote}

\end{fulllineitems}

\index{critical\_size (Cluster attribute)@\spxentry{critical\_size}\spxextra{Cluster attribute}}

\begin{fulllineitems}
\phantomsection\label{\detokenize{app.domain:app.domain.cluster_groups.Cluster.critical_size}}\pysigline{\sphinxbfcode{\sphinxupquote{critical\_size}}}
Minimum number of network nodes plus required to exist in the
Cluster to assure the target replication level.
\begin{quote}\begin{description}
\item[{Type}] \leavevmode
\sphinxhref{https://docs.python.org/3.7/library/functions.html\#int}{int}

\end{description}\end{quote}

\end{fulllineitems}

\index{sufficient\_size (Cluster attribute)@\spxentry{sufficient\_size}\spxextra{Cluster attribute}}

\begin{fulllineitems}
\phantomsection\label{\detokenize{app.domain:app.domain.cluster_groups.Cluster.sufficient_size}}\pysigline{\sphinxbfcode{\sphinxupquote{sufficient\_size}}}
Sum of {\hyperref[\detokenize{app.domain:app.domain.cluster_groups.Cluster.critical_size}]{\sphinxcrossref{\sphinxcode{\sphinxupquote{critical\_size}}}}}
and the number of nodes expected to fail between two successive
recovery phases.
\begin{quote}\begin{description}
\item[{Type}] \leavevmode
\sphinxhref{https://docs.python.org/3.7/library/functions.html\#int}{int}

\end{description}\end{quote}

\end{fulllineitems}

\index{original\_size (Cluster attribute)@\spxentry{original\_size}\spxextra{Cluster attribute}}

\begin{fulllineitems}
\phantomsection\label{\detokenize{app.domain:app.domain.cluster_groups.Cluster.original_size}}\pysigline{\sphinxbfcode{\sphinxupquote{original\_size}}}
The initial and theoretically optimal
{\hyperref[\detokenize{app.domain:app.domain.cluster_groups.Cluster}]{\sphinxcrossref{\sphinxcode{\sphinxupquote{Cluster}}}}} size.
\begin{quote}\begin{description}
\item[{Type}] \leavevmode
\sphinxhref{https://docs.python.org/3.7/library/functions.html\#int}{int}

\end{description}\end{quote}

\end{fulllineitems}

\index{redundant\_size (Cluster attribute)@\spxentry{redundant\_size}\spxextra{Cluster attribute}}

\begin{fulllineitems}
\phantomsection\label{\detokenize{app.domain:app.domain.cluster_groups.Cluster.redundant_size}}\pysigline{\sphinxbfcode{\sphinxupquote{redundant\_size}}}
Application\sphinxhyphen{}specific parameter, which indicates that membership
of the Cluster must be pruned.
\begin{quote}\begin{description}
\item[{Type}] \leavevmode
\sphinxhref{https://docs.python.org/3.7/library/functions.html\#int}{int}

\end{description}\end{quote}

\end{fulllineitems}

\index{running (Cluster attribute)@\spxentry{running}\spxextra{Cluster attribute}}

\begin{fulllineitems}
\phantomsection\label{\detokenize{app.domain:app.domain.cluster_groups.Cluster.running}}\pysigline{\sphinxbfcode{\sphinxupquote{running}}}
Indicates if the Cluster instance is active. Used by
{\hyperref[\detokenize{app.domain:app.domain.master_servers.Master}]{\sphinxcrossref{\sphinxcode{\sphinxupquote{Master}}}}} to manage the
simulation processes.
\begin{quote}\begin{description}
\item[{Type}] \leavevmode
\sphinxhref{https://docs.python.org/3.7/library/functions.html\#bool}{bool}

\end{description}\end{quote}

\end{fulllineitems}

\index{\_membership\_changed (Cluster attribute)@\spxentry{\_membership\_changed}\spxextra{Cluster attribute}}

\begin{fulllineitems}
\phantomsection\label{\detokenize{app.domain:app.domain.cluster_groups.Cluster._membership_changed}}\pysigline{\sphinxbfcode{\sphinxupquote{\_membership\_changed}}}
Flag indicates wether or not {\hyperref[\detokenize{app.domain:app.domain.cluster_groups.Cluster._members_view}]{\sphinxcrossref{\sphinxcode{\sphinxupquote{\_members\_view}}}}} needs
to be updated during {\hyperref[\detokenize{app.domain:app.domain.cluster_groups.Cluster.membership_maintenance}]{\sphinxcrossref{\sphinxcode{\sphinxupquote{membership\_maintenance()}}}}}. The
variable is set to false at the beggining of every epoch and set
to true if the length of \sphinxcode{\sphinxupquote{off\_nodes}} list return by
{\hyperref[\detokenize{app.domain:app.domain.cluster_groups.Cluster.nodes_execute}]{\sphinxcrossref{\sphinxcode{\sphinxupquote{nodes\_execute()}}}}} is bigger than zero.
\begin{quote}\begin{description}
\item[{Type}] \leavevmode
\sphinxhref{https://docs.python.org/3.7/library/functions.html\#bool}{bool}

\end{description}\end{quote}

\end{fulllineitems}

\index{\_recovery\_epoch\_sum (Cluster attribute)@\spxentry{\_recovery\_epoch\_sum}\spxextra{Cluster attribute}}

\begin{fulllineitems}
\phantomsection\label{\detokenize{app.domain:app.domain.cluster_groups.Cluster._recovery_epoch_sum}}\pysigline{\sphinxbfcode{\sphinxupquote{\_recovery\_epoch\_sum}}}
Helper attribute that facilitates the storage of the sum of the
values returned by all
\sphinxcode{\sphinxupquote{set\_recovery\_epoch()}}
method calls. Important for logging purposes.
\begin{quote}\begin{description}
\item[{Type}] \leavevmode
\sphinxhref{https://docs.python.org/3.7/library/functions.html\#int}{int}

\end{description}\end{quote}

\end{fulllineitems}

\index{\_recovery\_epoch\_calls (Cluster attribute)@\spxentry{\_recovery\_epoch\_calls}\spxextra{Cluster attribute}}

\begin{fulllineitems}
\phantomsection\label{\detokenize{app.domain:app.domain.cluster_groups.Cluster._recovery_epoch_calls}}\pysigline{\sphinxbfcode{\sphinxupquote{\_recovery\_epoch\_calls}}}
Helper attribute that facilitates the storage of the sum of the
values returned by all
\sphinxcode{\sphinxupquote{set\_recovery\_epoch()}}
method calls throughout the {\hyperref[\detokenize{app.domain:app.domain.cluster_groups.Cluster.current_epoch}]{\sphinxcrossref{\sphinxcode{\sphinxupquote{current\_epoch}}}}}.
\begin{quote}\begin{description}
\item[{Type}] \leavevmode
\sphinxhref{https://docs.python.org/3.7/library/functions.html\#int}{int}

\end{description}\end{quote}

\end{fulllineitems}

\index{\_\_init\_\_() (Cluster method)@\spxentry{\_\_init\_\_()}\spxextra{Cluster method}}

\begin{fulllineitems}
\phantomsection\label{\detokenize{app.domain:app.domain.cluster_groups.Cluster.__init__}}\pysiglinewithargsret{\sphinxbfcode{\sphinxupquote{\_\_init\_\_}}}{\emph{\DUrole{n}{master}}, \emph{\DUrole{n}{file\_name}}, \emph{\DUrole{n}{members}}, \emph{\DUrole{n}{sim\_id}\DUrole{o}{=}\DUrole{default_value}{0}}, \emph{\DUrole{n}{origin}\DUrole{o}{=}\DUrole{default_value}{\textquotesingle{}\textquotesingle{}}}}{}
Instantiates an \sphinxcode{\sphinxupquote{Cluster}} object
\begin{quote}\begin{description}
\item[{Parameters}] \leavevmode\begin{itemize}
\item {} 
\sphinxstyleliteralstrong{\sphinxupquote{master}} ({\hyperref[\detokenize{app:app.type_hints.MasterType}]{\sphinxcrossref{\sphinxcode{\sphinxupquote{MasterType}}}}}) \textendash{} A reference to an {\hyperref[\detokenize{app.domain:app.domain.master_servers.Master}]{\sphinxcrossref{\sphinxcode{\sphinxupquote{Master}}}}}
object that manages the \sphinxcode{\sphinxupquote{Cluster}} being initialized.

\item {} 
\sphinxstyleliteralstrong{\sphinxupquote{file\_name}} (\sphinxhref{https://docs.python.org/3.7/library/stdtypes.html\#str}{\sphinxstyleliteralemphasis{\sphinxupquote{str}}}) \textendash{} The name of the file the \sphinxcode{\sphinxupquote{Cluster}} is responsible for
persisting.

\item {} 
\sphinxstyleliteralstrong{\sphinxupquote{members}} ({\hyperref[\detokenize{app:app.type_hints.NodeDict}]{\sphinxcrossref{\sphinxcode{\sphinxupquote{NodeDict}}}}}) \textendash{} A dictionary where keys are {\hyperref[\detokenize{app.domain:app.domain.network_nodes.Node.id}]{\sphinxcrossref{\sphinxcode{\sphinxupquote{node identifiers}}}}} and values are their
{\hyperref[\detokenize{app.domain:app.domain.network_nodes.Node}]{\sphinxcrossref{\sphinxcode{\sphinxupquote{instance objects}}}}}.

\item {} 
\sphinxstyleliteralstrong{\sphinxupquote{sim\_id}} (\sphinxhref{https://docs.python.org/3.7/library/functions.html\#int}{\sphinxstyleliteralemphasis{\sphinxupquote{int}}}) \textendash{} Identifier that generates unique output file names,
thus guaranteeing that different simulation instances do not
overwrite previous out files.

\item {} 
\sphinxstyleliteralstrong{\sphinxupquote{origin}} (\sphinxhref{https://docs.python.org/3.7/library/stdtypes.html\#str}{\sphinxstyleliteralemphasis{\sphinxupquote{str}}}) \textendash{} The name of the simulation file name that started
the simulation process.

\end{itemize}

\item[{Return type}] \leavevmode
\sphinxhref{https://docs.python.org/3.7/library/constants.html\#None}{None}

\end{description}\end{quote}

\end{fulllineitems}

\index{\_get\_new\_members() (Cluster method)@\spxentry{\_get\_new\_members()}\spxextra{Cluster method}}

\begin{fulllineitems}
\phantomsection\label{\detokenize{app.domain:app.domain.cluster_groups.Cluster._get_new_members}}\pysiglinewithargsret{\sphinxbfcode{\sphinxupquote{\_get\_new\_members}}}{}{}
Helper method that searches for possible
{\hyperref[\detokenize{app.domain:app.domain.network_nodes.Node}]{\sphinxcrossref{\sphinxcode{\sphinxupquote{network node}}}}} by querying
the {\hyperref[\detokenize{app.domain:app.domain.cluster_groups.Cluster.master}]{\sphinxcrossref{\sphinxcode{\sphinxupquote{master}}}}} of the \sphinxcode{\sphinxupquote{Cluster}}.
\begin{quote}\begin{description}
\item[{Returns}] \leavevmode
A dictionary mapping where keys are
{\hyperref[\detokenize{app.domain:app.domain.network_nodes.Node.id}]{\sphinxcrossref{\sphinxcode{\sphinxupquote{node identifiers}}}}}
and values are
{\hyperref[\detokenize{app.domain:app.domain.network_nodes.Node}]{\sphinxcrossref{\sphinxcode{\sphinxupquote{node instances}}}}}.

\item[{Return type}] \leavevmode
{\hyperref[\detokenize{app:app.type_hints.NodeDict}]{\sphinxcrossref{\sphinxcode{\sphinxupquote{NodeDict}}}}}

\end{description}\end{quote}

\end{fulllineitems}

\index{\_log\_evaluation() (Cluster method)@\spxentry{\_log\_evaluation()}\spxextra{Cluster method}}

\begin{fulllineitems}
\phantomsection\label{\detokenize{app.domain:app.domain.cluster_groups.Cluster._log_evaluation}}\pysiglinewithargsret{\sphinxbfcode{\sphinxupquote{\_log\_evaluation}}}{\emph{\DUrole{n}{plive}}, \emph{\DUrole{n}{ptotal}\DUrole{o}{=}\DUrole{default_value}{\sphinxhyphen{} 1}}}{}
Helper that collects \sphinxcode{\sphinxupquote{Cluster}} data and registers it on a
{\hyperref[\detokenize{app.domain.helpers:app.domain.helpers.smart_dataclasses.LoggingData}]{\sphinxcrossref{\sphinxcode{\sphinxupquote{logger}}}}}
object.
\begin{quote}\begin{description}
\item[{Parameters}] \leavevmode\begin{itemize}
\item {} 
\sphinxstyleliteralstrong{\sphinxupquote{plive}} (\sphinxhref{https://docs.python.org/3.7/library/functions.html\#int}{\sphinxstyleliteralemphasis{\sphinxupquote{int}}}) \textendash{} The number of existing parts in the cluster at the
simulation’s current epoch at online or suspect nodes.

\item {} 
\sphinxstyleliteralstrong{\sphinxupquote{ptotal}} (\sphinxhref{https://docs.python.org/3.7/library/functions.html\#int}{\sphinxstyleliteralemphasis{\sphinxupquote{int}}}) \textendash{} The number of existing parts in the cluster at the
simulation’s current epoch. This parameter is optional and
may be used or not depending on the intent of the system.
As a rule of thumb \sphinxcode{\sphinxupquote{plive}} tracks the number of parts that
are alive in the system for logging purposes, where as
\sphinxcode{\sphinxupquote{ptotal}} is used for comparisons and averages, e.g.,
{\hyperref[\detokenize{app.domain:app.domain.cluster_groups.SGCluster.evaluate}]{\sphinxcrossref{\sphinxcode{\sphinxupquote{SGCluster evaluate}}}}}.

\end{itemize}

\item[{Return type}] \leavevmode
\sphinxhref{https://docs.python.org/3.7/library/constants.html\#None}{None}

\end{description}\end{quote}

\end{fulllineitems}

\index{\_set\_fail() (Cluster method)@\spxentry{\_set\_fail()}\spxextra{Cluster method}}

\begin{fulllineitems}
\phantomsection\label{\detokenize{app.domain:app.domain.cluster_groups.Cluster._set_fail}}\pysiglinewithargsret{\sphinxbfcode{\sphinxupquote{\_set\_fail}}}{\emph{\DUrole{n}{message}}}{}
Ends the Cluster instance simulation.

Sets {\hyperref[\detokenize{app.domain:app.domain.cluster_groups.Cluster.running}]{\sphinxcrossref{\sphinxcode{\sphinxupquote{running}}}}} to \sphinxcode{\sphinxupquote{False}} and orders
{\hyperref[\detokenize{app.domain.helpers:app.domain.helpers.smart_dataclasses.FileData}]{\sphinxcrossref{\sphinxcode{\sphinxupquote{FileData}}}}} to write
{\hyperref[\detokenize{app.domain.helpers:app.domain.helpers.smart_dataclasses.LoggingData}]{\sphinxcrossref{\sphinxcode{\sphinxupquote{collected logs}}}}}
to disk and close it’s
{\hyperref[\detokenize{app.domain.helpers:app.domain.helpers.smart_dataclasses.FileData.out_file}]{\sphinxcrossref{\sphinxcode{\sphinxupquote{out\_file}}}}}
stream.
\begin{quote}\begin{description}
\item[{Parameters}] \leavevmode
\sphinxstyleliteralstrong{\sphinxupquote{message}} (\sphinxhref{https://docs.python.org/3.7/library/stdtypes.html\#str}{\sphinxstyleliteralemphasis{\sphinxupquote{str}}}) \textendash{} A short explanation of why the \sphinxcode{\sphinxupquote{Cluster}} terminated early.

\item[{Return type}] \leavevmode
\sphinxhref{https://docs.python.org/3.7/library/constants.html\#None}{None}

\end{description}\end{quote}

\end{fulllineitems}

\index{\_setup\_epoch() (Cluster method)@\spxentry{\_setup\_epoch()}\spxextra{Cluster method}}

\begin{fulllineitems}
\phantomsection\label{\detokenize{app.domain:app.domain.cluster_groups.Cluster._setup_epoch}}\pysiglinewithargsret{\sphinxbfcode{\sphinxupquote{\_setup\_epoch}}}{\emph{\DUrole{n}{epoch}}}{}
Initializes some attributes cluster attributes at the start of an
epoch.

This method also forces all of the \sphinxcode{\sphinxupquote{Clusters}} members to update
their connectivity status before any node is instructed to execute.
\begin{quote}\begin{description}
\item[{Parameters}] \leavevmode
\sphinxstyleliteralstrong{\sphinxupquote{epoch}} (\sphinxhref{https://docs.python.org/3.7/library/functions.html\#int}{\sphinxstyleliteralemphasis{\sphinxupquote{int}}}) \textendash{} The simulation’s current epoch.

\item[{Return type}] \leavevmode
\sphinxhref{https://docs.python.org/3.7/library/constants.html\#None}{None}

\end{description}\end{quote}

\end{fulllineitems}

\index{complain() (Cluster method)@\spxentry{complain()}\spxextra{Cluster method}}

\begin{fulllineitems}
\phantomsection\label{\detokenize{app.domain:app.domain.cluster_groups.Cluster.complain}}\pysiglinewithargsret{\sphinxbfcode{\sphinxupquote{complain}}}{\emph{\DUrole{n}{complainter}}, \emph{\DUrole{n}{complainee}}, \emph{\DUrole{n}{reason}}}{}
Registers a complaint against a possibly offline node.

\begin{sphinxadmonition}{note}{Note:}
This method provides no default functionality and should be
overridden in sub classes if required.
\end{sphinxadmonition}
\begin{quote}\begin{description}
\item[{Parameters}] \leavevmode\begin{itemize}
\item {} 
\sphinxstyleliteralstrong{\sphinxupquote{complainter}} (\sphinxhref{https://docs.python.org/3.7/library/stdtypes.html\#str}{\sphinxstyleliteralemphasis{\sphinxupquote{str}}}) \textendash{} The identifier of the complaining {\hyperref[\detokenize{app.domain:app.domain.network_nodes.Node}]{\sphinxcrossref{\sphinxcode{\sphinxupquote{network node}}}}}.

\item {} 
\sphinxstyleliteralstrong{\sphinxupquote{complainee}} (\sphinxhref{https://docs.python.org/3.7/library/stdtypes.html\#str}{\sphinxstyleliteralemphasis{\sphinxupquote{str}}}) \textendash{} The identifier of the {\hyperref[\detokenize{app.domain:app.domain.network_nodes.Node}]{\sphinxcrossref{\sphinxcode{\sphinxupquote{network node}}}}} being complained about.

\item {} 
\sphinxstyleliteralstrong{\sphinxupquote{reason}} ({\hyperref[\detokenize{app:app.type_hints.HttpResponse}]{\sphinxcrossref{\sphinxcode{\sphinxupquote{app.type\_hints.HttpResponse}}}}}) \textendash{} The {\hyperref[\detokenize{app.domain.helpers:app.domain.helpers.enums.HttpCodes}]{\sphinxcrossref{\sphinxcode{\sphinxupquote{http code}}}}}
that led to the complaint.

\end{itemize}

\item[{Return type}] \leavevmode
\sphinxhref{https://docs.python.org/3.7/library/constants.html\#None}{None}

\end{description}\end{quote}

\end{fulllineitems}

\index{evaluate() (Cluster method)@\spxentry{evaluate()}\spxextra{Cluster method}}

\begin{fulllineitems}
\phantomsection\label{\detokenize{app.domain:app.domain.cluster_groups.Cluster.evaluate}}\pysiglinewithargsret{\sphinxbfcode{\sphinxupquote{evaluate}}}{}{}
Evaluates and logs the health, possibly other parameters, of the
\sphinxcode{\sphinxupquote{Cluster}} at every epoch.
\begin{quote}\begin{description}
\item[{Return type}] \leavevmode
\sphinxhref{https://docs.python.org/3.7/library/constants.html\#None}{None}

\end{description}\end{quote}

\end{fulllineitems}

\index{execute\_epoch() (Cluster method)@\spxentry{execute\_epoch()}\spxextra{Cluster method}}

\begin{fulllineitems}
\phantomsection\label{\detokenize{app.domain:app.domain.cluster_groups.Cluster.execute_epoch}}\pysiglinewithargsret{\sphinxbfcode{\sphinxupquote{execute\_epoch}}}{\emph{\DUrole{n}{epoch}}}{}
Orders all {\hyperref[\detokenize{app.domain:app.domain.cluster_groups.Cluster.members}]{\sphinxcrossref{\sphinxcode{\sphinxupquote{members}}}}} to execute their epoch.

\begin{sphinxadmonition}{note}{Note:}
If the \sphinxcode{\sphinxupquote{Cluster}} terminates early, before it reaches
{\hyperref[\detokenize{app.domain:app.domain.master_servers.Master.MAX_EPOCHS}]{\sphinxcrossref{\sphinxcode{\sphinxupquote{MAX\_EPOCHS}}}}},
nothing should be logged in
{\hyperref[\detokenize{app.domain.helpers:app.domain.helpers.smart_dataclasses.LoggingData}]{\sphinxcrossref{\sphinxcode{\sphinxupquote{LoggingData}}}}}
at the specified \sphinxcode{\sphinxupquote{epoch}} to avoid skewing previously
collected results.
\end{sphinxadmonition}
\begin{quote}\begin{description}
\item[{Parameters}] \leavevmode
\sphinxstyleliteralstrong{\sphinxupquote{epoch}} (\sphinxhref{https://docs.python.org/3.7/library/functions.html\#int}{\sphinxstyleliteralemphasis{\sphinxupquote{int}}}) \textendash{} The epoch the \sphinxcode{\sphinxupquote{Cluster}} should currently be in, according
to it’s managing {\hyperref[\detokenize{app.domain:app.domain.cluster_groups.Cluster.master}]{\sphinxcrossref{\sphinxcode{\sphinxupquote{master}}}}} entity.

\item[{Returns}] \leavevmode
\sphinxcode{\sphinxupquote{False}} if \sphinxcode{\sphinxupquote{Cluster}} failed to persist the {\hyperref[\detokenize{app.domain:app.domain.cluster_groups.Cluster.file}]{\sphinxcrossref{\sphinxcode{\sphinxupquote{file}}}}} it
was responsible for, otherwise \sphinxcode{\sphinxupquote{True}}.

\item[{Return type}] \leavevmode
\sphinxhref{https://docs.python.org/3.7/library/constants.html\#None}{None}

\end{description}\end{quote}

\end{fulllineitems}

\index{get\_cluster\_status() (Cluster method)@\spxentry{get\_cluster\_status()}\spxextra{Cluster method}}

\begin{fulllineitems}
\phantomsection\label{\detokenize{app.domain:app.domain.cluster_groups.Cluster.get_cluster_status}}\pysiglinewithargsret{\sphinxbfcode{\sphinxupquote{get\_cluster\_status}}}{}{}
Determines the \sphinxcode{\sphinxupquote{Cluster}}’s status based on the length of the
current {\hyperref[\detokenize{app.domain:app.domain.cluster_groups.Cluster.members}]{\sphinxcrossref{\sphinxcode{\sphinxupquote{members}}}}} list.
\begin{quote}\begin{description}
\item[{Returns}] \leavevmode
The status of the \sphinxcode{\sphinxupquote{Cluster}} as a string.

\item[{Return type}] \leavevmode
\sphinxhref{https://docs.python.org/3.7/library/stdtypes.html\#str}{str}

\end{description}\end{quote}

\end{fulllineitems}

\index{get\_node() (Cluster method)@\spxentry{get\_node()}\spxextra{Cluster method}}

\begin{fulllineitems}
\phantomsection\label{\detokenize{app.domain:app.domain.cluster_groups.Cluster.get_node}}\pysiglinewithargsret{\sphinxbfcode{\sphinxupquote{get\_node}}}{}{}
Retrives a random node from the members of the cluster group,
whose status is likely to be online.
\begin{quote}\begin{description}
\item[{Returns}] \leavevmode
A random network node from {\hyperref[\detokenize{app.domain:app.domain.cluster_groups.Cluster.members}]{\sphinxcrossref{\sphinxcode{\sphinxupquote{members}}}}}.

\item[{Return type}] \leavevmode
{\hyperref[\detokenize{app:app.type_hints.NodeType}]{\sphinxcrossref{\sphinxcode{\sphinxupquote{NodeType}}}}}

\end{description}\end{quote}

\end{fulllineitems}

\index{maintain() (Cluster method)@\spxentry{maintain()}\spxextra{Cluster method}}

\begin{fulllineitems}
\phantomsection\label{\detokenize{app.domain:app.domain.cluster_groups.Cluster.maintain}}\pysiglinewithargsret{\sphinxbfcode{\sphinxupquote{maintain}}}{\emph{\DUrole{n}{off\_nodes}}}{}
Offers basic maintenance functionality for Cluster types.

If \sphinxcode{\sphinxupquote{off\_nodes}} list param as at least one node reference,
{\hyperref[\detokenize{app.domain:app.domain.cluster_groups.Cluster._membership_changed}]{\sphinxcrossref{\sphinxcode{\sphinxupquote{\_membership\_changed}}}}} is set to \sphinxcode{\sphinxupquote{True}}.
\begin{quote}\begin{description}
\item[{Parameters}] \leavevmode
\sphinxstyleliteralstrong{\sphinxupquote{off\_nodes}} (\sphinxstyleliteralemphasis{\sphinxupquote{List}}\sphinxstyleliteralemphasis{\sphinxupquote{{[}}}\sphinxstyleliteralemphasis{\sphinxupquote{th.NodeType}}\sphinxstyleliteralemphasis{\sphinxupquote{{]}}}) \textendash{} A possibly empty of offline nodes.

\item[{Return type}] \leavevmode
\sphinxhref{https://docs.python.org/3.7/library/constants.html\#None}{None}

\end{description}\end{quote}

\end{fulllineitems}

\index{membership\_maintenance() (Cluster method)@\spxentry{membership\_maintenance()}\spxextra{Cluster method}}

\begin{fulllineitems}
\phantomsection\label{\detokenize{app.domain:app.domain.cluster_groups.Cluster.membership_maintenance}}\pysiglinewithargsret{\sphinxbfcode{\sphinxupquote{membership\_maintenance}}}{}{}
Attempts to recruits new network nodes to be members of the cluster.

The method updates both {\hyperref[\detokenize{app.domain:app.domain.cluster_groups.Cluster.members}]{\sphinxcrossref{\sphinxcode{\sphinxupquote{members}}}}} and {\hyperref[\detokenize{app.domain:app.domain.cluster_groups.Cluster._members_view}]{\sphinxcrossref{\sphinxcode{\sphinxupquote{\_members\_view}}}}}.
\begin{quote}\begin{description}
\item[{Returns}] \leavevmode
A dictionary that is empty if membership did not change.

\item[{Return type}] \leavevmode
{\hyperref[\detokenize{app:app.type_hints.NodeDict}]{\sphinxcrossref{\sphinxcode{\sphinxupquote{NodeDict}}}}}

\end{description}\end{quote}

\end{fulllineitems}

\index{nodes\_execute() (Cluster method)@\spxentry{nodes\_execute()}\spxextra{Cluster method}}

\begin{fulllineitems}
\phantomsection\label{\detokenize{app.domain:app.domain.cluster_groups.Cluster.nodes_execute}}\pysiglinewithargsret{\sphinxbfcode{\sphinxupquote{nodes\_execute}}}{}{}
Queries all {\hyperref[\detokenize{app.domain:app.domain.cluster_groups.Cluster.members}]{\sphinxcrossref{\sphinxcode{\sphinxupquote{members}}}}} to execute the epoch.

This method logs the amount of lost replicas throughout
{\hyperref[\detokenize{app.domain:app.domain.cluster_groups.Cluster.current_epoch}]{\sphinxcrossref{\sphinxcode{\sphinxupquote{current\_epoch}}}}} according to the {\hyperref[\detokenize{app.domain:app.domain.cluster_groups.Cluster.members}]{\sphinxcrossref{\sphinxcode{\sphinxupquote{members}}}}} who went
offline and the
{\hyperref[\detokenize{app.domain.helpers:app.domain.helpers.smart_dataclasses.FileBlockData}]{\sphinxcrossref{\sphinxcode{\sphinxupquote{FileBlockData}}}}}
replicas they posssed and is responsible for
{\hyperref[\detokenize{app.domain:app.domain.cluster_groups.Cluster.set_replication_epoch}]{\sphinxcrossref{\sphinxcode{\sphinxupquote{setting a replication epoch}}}}}.
Similarly it logs the number of members who disconnected.
\begin{quote}\begin{description}
\item[{Returns}] \leavevmode
List of {\hyperref[\detokenize{app.domain:app.domain.cluster_groups.Cluster.members}]{\sphinxcrossref{\sphinxcode{\sphinxupquote{members}}}}} that disconnected during the
{\hyperref[\detokenize{app.domain:app.domain.cluster_groups.Cluster.current_epoch}]{\sphinxcrossref{\sphinxcode{\sphinxupquote{current\_epoch}}}}}. See
{\hyperref[\detokenize{app.domain:app.domain.network_nodes.Node.update_status}]{\sphinxcrossref{\sphinxcode{\sphinxupquote{app.domain.network\_nodes.Node.update\_status()}}}}}.

\item[{Return type}] \leavevmode
List{[}{\hyperref[\detokenize{app:app.type_hints.NodeType}]{\sphinxcrossref{\sphinxcode{\sphinxupquote{NodeType}}}}}{]}

\end{description}\end{quote}

\end{fulllineitems}

\index{route\_part() (Cluster method)@\spxentry{route\_part()}\spxextra{Cluster method}}

\begin{fulllineitems}
\phantomsection\label{\detokenize{app.domain:app.domain.cluster_groups.Cluster.route_part}}\pysiglinewithargsret{\sphinxbfcode{\sphinxupquote{route\_part}}}{\emph{\DUrole{n}{sender}}, \emph{\DUrole{n}{receiver}}, \emph{\DUrole{n}{replica}}, \emph{\DUrole{n}{is\_fresh}\DUrole{o}{=}\DUrole{default_value}{False}}}{}
Sends a {\hyperref[\detokenize{app.domain.helpers:app.domain.helpers.smart_dataclasses.FileBlockData}]{\sphinxcrossref{\sphinxcode{\sphinxupquote{file block replica}}}}} to some other
{\hyperref[\detokenize{app.domain:app.domain.network_nodes.Node}]{\sphinxcrossref{\sphinxcode{\sphinxupquote{network node}}}}} in
{\hyperref[\detokenize{app.domain:app.domain.cluster_groups.Cluster.members}]{\sphinxcrossref{\sphinxcode{\sphinxupquote{members}}}}}.
\begin{quote}\begin{description}
\item[{Parameters}] \leavevmode\begin{itemize}
\item {} 
\sphinxstyleliteralstrong{\sphinxupquote{sender}} (\sphinxhref{https://docs.python.org/3.7/library/stdtypes.html\#str}{\sphinxstyleliteralemphasis{\sphinxupquote{str}}}) \textendash{} An identifier of the
{\hyperref[\detokenize{app.domain:app.domain.network_nodes.Node}]{\sphinxcrossref{\sphinxcode{\sphinxupquote{network node}}}}}
who is sending the message.

\item {} 
\sphinxstyleliteralstrong{\sphinxupquote{receiver}} (\sphinxhref{https://docs.python.org/3.7/library/stdtypes.html\#str}{\sphinxstyleliteralemphasis{\sphinxupquote{str}}}) \textendash{} The destination
{\hyperref[\detokenize{app.domain:app.domain.network_nodes.Node}]{\sphinxcrossref{\sphinxcode{\sphinxupquote{network node}}}}}
identifier.

\item {} 
\sphinxstyleliteralstrong{\sphinxupquote{replica}} ({\hyperref[\detokenize{app.domain.helpers:app.domain.helpers.smart_dataclasses.FileBlockData}]{\sphinxcrossref{\sphinxcode{\sphinxupquote{FileBlockData}}}}}) \textendash{} The {\hyperref[\detokenize{app.domain.helpers:app.domain.helpers.smart_dataclasses.FileBlockData}]{\sphinxcrossref{\sphinxcode{\sphinxupquote{file block replica}}}}}
to be sent specified destination: \sphinxcode{\sphinxupquote{receiver}}.

\item {} 
\sphinxstyleliteralstrong{\sphinxupquote{is\_fresh}} (\sphinxhref{https://docs.python.org/3.7/library/functions.html\#bool}{\sphinxstyleliteralemphasis{\sphinxupquote{bool}}}) \textendash{} Prevents recently created replicas from being
corrupted, since they are not likely to be corrupted in disk.
This argument facilitates simulation.

\end{itemize}

\item[{Returns}] \leavevmode
An http code sent by the \sphinxcode{\sphinxupquote{receiver}}.

\item[{Return type}] \leavevmode
\sphinxhref{https://docs.python.org/3.7/library/functions.html\#int}{int}

\end{description}\end{quote}

\end{fulllineitems}

\index{set\_replication\_epoch() (Cluster method)@\spxentry{set\_replication\_epoch()}\spxextra{Cluster method}}

\begin{fulllineitems}
\phantomsection\label{\detokenize{app.domain:app.domain.cluster_groups.Cluster.set_replication_epoch}}\pysiglinewithargsret{\sphinxbfcode{\sphinxupquote{set\_replication\_epoch}}}{\emph{\DUrole{n}{replica}}}{}
Delegates to {\hyperref[\detokenize{app.domain.helpers:app.domain.helpers.smart_dataclasses.FileBlockData.set_replication_epoch}]{\sphinxcrossref{\sphinxcode{\sphinxupquote{set\_replication\_epoch()}}}}}.
\begin{quote}\begin{description}
\item[{Parameters}] \leavevmode
\sphinxstyleliteralstrong{\sphinxupquote{replica}} ({\hyperref[\detokenize{app.domain.helpers:app.domain.helpers.smart_dataclasses.FileBlockData}]{\sphinxcrossref{\sphinxstyleliteralemphasis{\sphinxupquote{domain.helpers.smart\_dataclasses.FileBlockData}}}}}) \textendash{} The {\hyperref[\detokenize{app.domain.helpers:app.domain.helpers.smart_dataclasses.FileBlockData}]{\sphinxcrossref{\sphinxcode{\sphinxupquote{file block replica}}}}} that
was lost.

\item[{Return type}] \leavevmode
\sphinxhref{https://docs.python.org/3.7/library/constants.html\#None}{None}

\end{description}\end{quote}

\end{fulllineitems}

\index{spread\_files() (Cluster method)@\spxentry{spread\_files()}\spxextra{Cluster method}}

\begin{fulllineitems}
\phantomsection\label{\detokenize{app.domain:app.domain.cluster_groups.Cluster.spread_files}}\pysiglinewithargsret{\sphinxbfcode{\sphinxupquote{spread\_files}}}{\emph{\DUrole{n}{replicas}}, \emph{\DUrole{n}{strat}\DUrole{o}{=}\DUrole{default_value}{\textquotesingle{}i\textquotesingle{}}}}{}
Distributes a collection of {\hyperref[\detokenize{app.domain.helpers:app.domain.helpers.smart_dataclasses.FileBlockData}]{\sphinxcrossref{\sphinxcode{\sphinxupquote{file block replicas}}}}} among the
{\hyperref[\detokenize{app.domain:app.domain.cluster_groups.Cluster.members}]{\sphinxcrossref{\sphinxcode{\sphinxupquote{members}}}}} of the cluster group.
\begin{quote}\begin{description}
\item[{Parameters}] \leavevmode\begin{itemize}
\item {} 
\sphinxstyleliteralstrong{\sphinxupquote{replicas}} ({\hyperref[\detokenize{app:app.type_hints.ReplicasDict}]{\sphinxcrossref{\sphinxcode{\sphinxupquote{ReplicasDict}}}}}) \textendash{} The {\hyperref[\detokenize{app.domain.helpers:app.domain.helpers.smart_dataclasses.FileBlockData}]{\sphinxcrossref{\sphinxcode{\sphinxupquote{FileBlockData}}}}}
replicas, without replication.

\item {} 
\sphinxstyleliteralstrong{\sphinxupquote{strat}} (\sphinxhref{https://docs.python.org/3.7/library/stdtypes.html\#str}{\sphinxstyleliteralemphasis{\sphinxupquote{str}}}) \textendash{} 
Defines how \sphinxcode{\sphinxupquote{replicas}} will be initially distributed in
the \sphinxcode{\sphinxupquote{Cluster}}. Unless overridden in children of this class the
received value of \sphinxcode{\sphinxupquote{strat}} will be ignored and will always
be set to the default value \sphinxcode{\sphinxupquote{i}}.
\begin{description}
\item[{i}] \leavevmode
This strategy creates a probability vector
containing the normalization of {\hyperref[\detokenize{app.domain:app.domain.network_nodes.Node.uptime}]{\sphinxcrossref{\sphinxcode{\sphinxupquote{network nodes
uptimes\textquotesingle{}}}}}} and uses
that vector to randomly select which
{\hyperref[\detokenize{app.domain:app.domain.network_nodes.Node}]{\sphinxcrossref{\sphinxcode{\sphinxupquote{node}}}}} will
receive each replica. There is a bias to give more
replicas to the most resillent {\hyperref[\detokenize{app.domain:app.domain.network_nodes.Node}]{\sphinxcrossref{\sphinxcode{\sphinxupquote{nodes}}}}} which results from
using the created probability vector.

\end{description}


\end{itemize}

\item[{Return type}] \leavevmode
\sphinxhref{https://docs.python.org/3.7/library/constants.html\#None}{None}

\end{description}\end{quote}

\end{fulllineitems}


\end{fulllineitems}

\index{HDFSCluster (class in app.domain.cluster\_groups)@\spxentry{HDFSCluster}\spxextra{class in app.domain.cluster\_groups}}

\begin{fulllineitems}
\phantomsection\label{\detokenize{app.domain:app.domain.cluster_groups.HDFSCluster}}\pysiglinewithargsret{\sphinxbfcode{\sphinxupquote{class }}\sphinxbfcode{\sphinxupquote{HDFSCluster}}}{\emph{\DUrole{n}{master}}, \emph{\DUrole{n}{file\_name}}, \emph{\DUrole{n}{members}}, \emph{\DUrole{n}{sim\_id}\DUrole{o}{=}\DUrole{default_value}{0}}, \emph{\DUrole{n}{origin}\DUrole{o}{=}\DUrole{default_value}{\textquotesingle{}\textquotesingle{}}}}{}
Bases: {\hyperref[\detokenize{app.domain:app.domain.cluster_groups.Cluster}]{\sphinxcrossref{\sphinxcode{\sphinxupquote{app.domain.cluster\_groups.Cluster}}}}}

Represents a group of network nodes ensuring the durability of a file
in a Hadoop Distributed File System scenario.

\begin{sphinxadmonition}{note}{Note:}
Members of \sphinxcode{\sphinxupquote{HDFSCluster}} are of type
{\hyperref[\detokenize{app.domain:app.domain.network_nodes.HDFSNode}]{\sphinxcrossref{\sphinxcode{\sphinxupquote{HDFSNode}}}}}, they do not
perform swarm guidance behaviors and instead report with regular
heartbeats to their {\hyperref[\detokenize{app.domain:app.domain.cluster_groups.HDFSCluster}]{\sphinxcrossref{\sphinxcode{\sphinxupquote{monitors}}}}}. This class could be a
\sphinxstyleemphasis{NameNode Server} in HDFS or a \sphinxstyleemphasis{master server} in GFS.
\end{sphinxadmonition}
\index{suspicious\_nodes (HDFSCluster attribute)@\spxentry{suspicious\_nodes}\spxextra{HDFSCluster attribute}}

\begin{fulllineitems}
\phantomsection\label{\detokenize{app.domain:app.domain.cluster_groups.HDFSCluster.suspicious_nodes}}\pysigline{\sphinxbfcode{\sphinxupquote{suspicious\_nodes}}}
A set containing the identifiers of suspicious
{\hyperref[\detokenize{app.domain:app.domain.network_nodes.HDFSNode}]{\sphinxcrossref{\sphinxcode{\sphinxupquote{network nodes}}}}}.
\begin{quote}\begin{description}
\item[{Type}] \leavevmode
\sphinxhref{https://docs.python.org/3.7/library/stdtypes.html\#set}{set}

\end{description}\end{quote}

\end{fulllineitems}

\index{data\_node\_heartbeats (HDFSCluster attribute)@\spxentry{data\_node\_heartbeats}\spxextra{HDFSCluster attribute}}

\begin{fulllineitems}
\phantomsection\label{\detokenize{app.domain:app.domain.cluster_groups.HDFSCluster.data_node_heartbeats}}\pysigline{\sphinxbfcode{\sphinxupquote{data\_node\_heartbeats}}}
A dictionary mapping {\hyperref[\detokenize{app.domain:app.domain.network_nodes.Node.id}]{\sphinxcrossref{\sphinxcode{\sphinxupquote{node identifiers}}}}} to the number of
complaints made against them. Each node has five lives. When they
miss five beats in a row, i.e., when the dictionary value count
is zero, they are evicted from the cluster.
\begin{quote}\begin{description}
\item[{Type}] \leavevmode
Dict{[}\sphinxhref{https://docs.python.org/3.7/library/stdtypes.html\#str}{str}, \sphinxhref{https://docs.python.org/3.7/library/functions.html\#int}{int}{]}

\end{description}\end{quote}

\end{fulllineitems}

\index{\_\_init\_\_() (HDFSCluster method)@\spxentry{\_\_init\_\_()}\spxextra{HDFSCluster method}}

\begin{fulllineitems}
\phantomsection\label{\detokenize{app.domain:app.domain.cluster_groups.HDFSCluster.__init__}}\pysiglinewithargsret{\sphinxbfcode{\sphinxupquote{\_\_init\_\_}}}{\emph{\DUrole{n}{master}}, \emph{\DUrole{n}{file\_name}}, \emph{\DUrole{n}{members}}, \emph{\DUrole{n}{sim\_id}\DUrole{o}{=}\DUrole{default_value}{0}}, \emph{\DUrole{n}{origin}\DUrole{o}{=}\DUrole{default_value}{\textquotesingle{}\textquotesingle{}}}}{}
Instantiates an \sphinxcode{\sphinxupquote{Cluster}} object
\begin{quote}\begin{description}
\item[{Parameters}] \leavevmode\begin{itemize}
\item {} 
\sphinxstyleliteralstrong{\sphinxupquote{master}} ({\hyperref[\detokenize{app:app.type_hints.MasterType}]{\sphinxcrossref{\sphinxcode{\sphinxupquote{MasterType}}}}}) \textendash{} A reference to an {\hyperref[\detokenize{app.domain:app.domain.master_servers.Master}]{\sphinxcrossref{\sphinxcode{\sphinxupquote{Master}}}}}
object that manages the \sphinxcode{\sphinxupquote{Cluster}} being initialized.

\item {} 
\sphinxstyleliteralstrong{\sphinxupquote{file\_name}} (\sphinxhref{https://docs.python.org/3.7/library/stdtypes.html\#str}{\sphinxstyleliteralemphasis{\sphinxupquote{str}}}) \textendash{} The name of the file the \sphinxcode{\sphinxupquote{Cluster}} is responsible for
persisting.

\item {} 
\sphinxstyleliteralstrong{\sphinxupquote{members}} ({\hyperref[\detokenize{app:app.type_hints.NodeDict}]{\sphinxcrossref{\sphinxcode{\sphinxupquote{NodeDict}}}}}) \textendash{} A dictionary where keys are {\hyperref[\detokenize{app.domain:app.domain.network_nodes.Node.id}]{\sphinxcrossref{\sphinxcode{\sphinxupquote{node identifiers}}}}} and values are their
{\hyperref[\detokenize{app.domain:app.domain.network_nodes.Node}]{\sphinxcrossref{\sphinxcode{\sphinxupquote{instance objects}}}}}.

\item {} 
\sphinxstyleliteralstrong{\sphinxupquote{sim\_id}} (\sphinxhref{https://docs.python.org/3.7/library/functions.html\#int}{\sphinxstyleliteralemphasis{\sphinxupquote{int}}}) \textendash{} Identifier that generates unique output file names,
thus guaranteeing that different simulation instances do not
overwrite previous out files.

\item {} 
\sphinxstyleliteralstrong{\sphinxupquote{origin}} (\sphinxhref{https://docs.python.org/3.7/library/stdtypes.html\#str}{\sphinxstyleliteralemphasis{\sphinxupquote{str}}}) \textendash{} The name of the simulation file name that started
the simulation process.

\end{itemize}

\item[{Return type}] \leavevmode
\sphinxhref{https://docs.python.org/3.7/library/constants.html\#None}{None}

\end{description}\end{quote}

\end{fulllineitems}

\index{evaluate() (HDFSCluster method)@\spxentry{evaluate()}\spxextra{HDFSCluster method}}

\begin{fulllineitems}
\phantomsection\label{\detokenize{app.domain:app.domain.cluster_groups.HDFSCluster.evaluate}}\pysiglinewithargsret{\sphinxbfcode{\sphinxupquote{evaluate}}}{}{}
Logs the number of existing replicas in the \sphinxcode{\sphinxupquote{HDFSCluster}}.
\begin{description}
\item[{Overrides:}] \leavevmode
{\hyperref[\detokenize{app.domain:app.domain.cluster_groups.Cluster.evaluate}]{\sphinxcrossref{\sphinxcode{\sphinxupquote{app.domain.cluster\_groups.Cluster.evaluate()}}}}}.

\end{description}
\begin{quote}\begin{description}
\item[{Return type}] \leavevmode
\sphinxhref{https://docs.python.org/3.7/library/constants.html\#None}{None}

\end{description}\end{quote}

\end{fulllineitems}

\index{maintain() (HDFSCluster method)@\spxentry{maintain()}\spxextra{HDFSCluster method}}

\begin{fulllineitems}
\phantomsection\label{\detokenize{app.domain:app.domain.cluster_groups.HDFSCluster.maintain}}\pysiglinewithargsret{\sphinxbfcode{\sphinxupquote{maintain}}}{\emph{\DUrole{n}{off\_nodes}}}{}
Evicts any {\hyperref[\detokenize{app.domain:app.domain.network_nodes.HDFSNode}]{\sphinxcrossref{\sphinxcode{\sphinxupquote{network node}}}}}
whose heartbeats in {\hyperref[\detokenize{app.domain:app.domain.cluster_groups.HDFSCluster.data_node_heartbeats}]{\sphinxcrossref{\sphinxcode{\sphinxupquote{data\_node\_heartbeats}}}}} reached zero.
\begin{description}
\item[{Extends:}] \leavevmode
{\hyperref[\detokenize{app.domain:app.domain.cluster_groups.Cluster.execute_epoch}]{\sphinxcrossref{\sphinxcode{\sphinxupquote{app.domain.cluster\_groups.Cluster.execute\_epoch()}}}}}.

\end{description}
\begin{quote}\begin{description}
\item[{Parameters}] \leavevmode
\sphinxstyleliteralstrong{\sphinxupquote{off\_nodes}} (List{[}{\hyperref[\detokenize{app:app.type_hints.NodeType}]{\sphinxcrossref{\sphinxcode{\sphinxupquote{NodeType}}}}}{]}) \textendash{} The subset of {\hyperref[\detokenize{app.domain:app.domain.cluster_groups.Cluster.members}]{\sphinxcrossref{\sphinxcode{\sphinxupquote{members}}}}} who disconnected
during the current epoch.

\item[{Return type}] \leavevmode
\sphinxhref{https://docs.python.org/3.7/library/constants.html\#None}{None}

\end{description}\end{quote}

\end{fulllineitems}

\index{membership\_maintenance() (HDFSCluster method)@\spxentry{membership\_maintenance()}\spxextra{HDFSCluster method}}

\begin{fulllineitems}
\phantomsection\label{\detokenize{app.domain:app.domain.cluster_groups.HDFSCluster.membership_maintenance}}\pysiglinewithargsret{\sphinxbfcode{\sphinxupquote{membership\_maintenance}}}{}{}
Attempts to recruits new network nodes to be members of the cluster.

The method updates both \sphinxcode{\sphinxupquote{members}} and \sphinxcode{\sphinxupquote{\_members\_view}}.
\begin{quote}\begin{description}
\item[{Returns}] \leavevmode
A dictionary that is empty if membership did not change.

\item[{Return type}] \leavevmode
{\hyperref[\detokenize{app:app.type_hints.NodeDict}]{\sphinxcrossref{\sphinxcode{\sphinxupquote{NodeDict}}}}}

\end{description}\end{quote}

\end{fulllineitems}

\index{nodes\_execute() (HDFSCluster method)@\spxentry{nodes\_execute()}\spxextra{HDFSCluster method}}

\begin{fulllineitems}
\phantomsection\label{\detokenize{app.domain:app.domain.cluster_groups.HDFSCluster.nodes_execute}}\pysiglinewithargsret{\sphinxbfcode{\sphinxupquote{nodes\_execute}}}{}{}
Queries all {\hyperref[\detokenize{app.domain:app.domain.cluster_groups.Cluster.members}]{\sphinxcrossref{\sphinxcode{\sphinxupquote{members}}}}} to execute the epoch.
\begin{description}
\item[{Overrides:}] \leavevmode
{\hyperref[\detokenize{app.domain:app.domain.cluster_groups.Cluster.nodes_execute}]{\sphinxcrossref{\sphinxcode{\sphinxupquote{app.domain.cluster\_groups.Cluster.nodes\_execute()}}}}}

\end{description}
\begin{quote}\begin{description}
\item[{Returns}] \leavevmode
A collection of {\hyperref[\detokenize{app.domain:app.domain.cluster_groups.Cluster.members}]{\sphinxcrossref{\sphinxcode{\sphinxupquote{members}}}}} who disconnected
during the current epoch. See
{\hyperref[\detokenize{app.domain:app.domain.network_nodes.HDFSNode.update_status}]{\sphinxcrossref{\sphinxcode{\sphinxupquote{app.domain.network\_nodes.HDFSNode.update\_status()}}}}}.

\item[{Return type}] \leavevmode
List{[}{\hyperref[\detokenize{app:app.type_hints.NodeType}]{\sphinxcrossref{\sphinxcode{\sphinxupquote{NodeType}}}}}{]}

\end{description}\end{quote}

\end{fulllineitems}


\end{fulllineitems}

\index{NewscastCluster (class in app.domain.cluster\_groups)@\spxentry{NewscastCluster}\spxextra{class in app.domain.cluster\_groups}}

\begin{fulllineitems}
\phantomsection\label{\detokenize{app.domain:app.domain.cluster_groups.NewscastCluster}}\pysiglinewithargsret{\sphinxbfcode{\sphinxupquote{class }}\sphinxbfcode{\sphinxupquote{NewscastCluster}}}{\emph{\DUrole{n}{master}}, \emph{\DUrole{n}{file\_name}}, \emph{\DUrole{n}{members}}, \emph{\DUrole{n}{sim\_id}\DUrole{o}{=}\DUrole{default_value}{0}}, \emph{\DUrole{n}{origin}\DUrole{o}{=}\DUrole{default_value}{\textquotesingle{}\textquotesingle{}}}}{}
Bases: {\hyperref[\detokenize{app.domain:app.domain.cluster_groups.Cluster}]{\sphinxcrossref{\sphinxcode{\sphinxupquote{app.domain.cluster\_groups.Cluster}}}}}

Represents a P2P network of nodes performing mean degree aggregation,
while simultaneously using Newscast for \sphinxcode{\sphinxupquote{view shuffling}}.
\index{\_\_init\_\_() (NewscastCluster method)@\spxentry{\_\_init\_\_()}\spxextra{NewscastCluster method}}

\begin{fulllineitems}
\phantomsection\label{\detokenize{app.domain:app.domain.cluster_groups.NewscastCluster.__init__}}\pysiglinewithargsret{\sphinxbfcode{\sphinxupquote{\_\_init\_\_}}}{\emph{\DUrole{n}{master}}, \emph{\DUrole{n}{file\_name}}, \emph{\DUrole{n}{members}}, \emph{\DUrole{n}{sim\_id}\DUrole{o}{=}\DUrole{default_value}{0}}, \emph{\DUrole{n}{origin}\DUrole{o}{=}\DUrole{default_value}{\textquotesingle{}\textquotesingle{}}}}{}
Instantiates an \sphinxcode{\sphinxupquote{Cluster}} object
\begin{quote}\begin{description}
\item[{Parameters}] \leavevmode\begin{itemize}
\item {} 
\sphinxstyleliteralstrong{\sphinxupquote{master}} ({\hyperref[\detokenize{app:app.type_hints.MasterType}]{\sphinxcrossref{\sphinxcode{\sphinxupquote{MasterType}}}}}) \textendash{} A reference to an {\hyperref[\detokenize{app.domain:app.domain.master_servers.Master}]{\sphinxcrossref{\sphinxcode{\sphinxupquote{Master}}}}}
object that manages the \sphinxcode{\sphinxupquote{Cluster}} being initialized.

\item {} 
\sphinxstyleliteralstrong{\sphinxupquote{file\_name}} (\sphinxhref{https://docs.python.org/3.7/library/stdtypes.html\#str}{\sphinxstyleliteralemphasis{\sphinxupquote{str}}}) \textendash{} The name of the file the \sphinxcode{\sphinxupquote{Cluster}} is responsible for
persisting.

\item {} 
\sphinxstyleliteralstrong{\sphinxupquote{members}} ({\hyperref[\detokenize{app:app.type_hints.NodeDict}]{\sphinxcrossref{\sphinxcode{\sphinxupquote{NodeDict}}}}}) \textendash{} A dictionary where keys are {\hyperref[\detokenize{app.domain:app.domain.network_nodes.Node.id}]{\sphinxcrossref{\sphinxcode{\sphinxupquote{node identifiers}}}}} and values are their
{\hyperref[\detokenize{app.domain:app.domain.network_nodes.Node}]{\sphinxcrossref{\sphinxcode{\sphinxupquote{instance objects}}}}}.

\item {} 
\sphinxstyleliteralstrong{\sphinxupquote{sim\_id}} (\sphinxhref{https://docs.python.org/3.7/library/functions.html\#int}{\sphinxstyleliteralemphasis{\sphinxupquote{int}}}) \textendash{} Identifier that generates unique output file names,
thus guaranteeing that different simulation instances do not
overwrite previous out files.

\item {} 
\sphinxstyleliteralstrong{\sphinxupquote{origin}} (\sphinxhref{https://docs.python.org/3.7/library/stdtypes.html\#str}{\sphinxstyleliteralemphasis{\sphinxupquote{str}}}) \textendash{} The name of the simulation file name that started
the simulation process.

\end{itemize}

\item[{Return type}] \leavevmode
\sphinxhref{https://docs.python.org/3.7/library/constants.html\#None}{None}

\end{description}\end{quote}

\end{fulllineitems}

\index{\_setup\_epoch() (NewscastCluster method)@\spxentry{\_setup\_epoch()}\spxextra{NewscastCluster method}}

\begin{fulllineitems}
\phantomsection\label{\detokenize{app.domain:app.domain.cluster_groups.NewscastCluster._setup_epoch}}\pysiglinewithargsret{\sphinxbfcode{\sphinxupquote{\_setup\_epoch}}}{\emph{\DUrole{n}{epoch}}}{}
Initializes some attributes cluster attributes at the start of an
epoch.
\begin{description}
\item[{Extends:}] \leavevmode
{\hyperref[\detokenize{app.domain:app.domain.cluster_groups.Cluster._setup_epoch}]{\sphinxcrossref{\sphinxcode{\sphinxupquote{app.domain.cluster\_groups.Cluster.\_setup\_epoch()}}}}}

\end{description}
\begin{quote}\begin{description}
\item[{Parameters}] \leavevmode
\sphinxstyleliteralstrong{\sphinxupquote{epoch}} (\sphinxhref{https://docs.python.org/3.7/library/functions.html\#int}{\sphinxstyleliteralemphasis{\sphinxupquote{int}}}) \textendash{} The simulation’s current epoch.

\item[{Return type}] \leavevmode
\sphinxhref{https://docs.python.org/3.7/library/constants.html\#None}{None}

\end{description}\end{quote}

\end{fulllineitems}

\index{evaluate() (NewscastCluster method)@\spxentry{evaluate()}\spxextra{NewscastCluster method}}

\begin{fulllineitems}
\phantomsection\label{\detokenize{app.domain:app.domain.cluster_groups.NewscastCluster.evaluate}}\pysiglinewithargsret{\sphinxbfcode{\sphinxupquote{evaluate}}}{}{}
Prints the epoch’s aggregated peer degree, to the command\sphinxhyphen{}line
interface.
\begin{quote}\begin{description}
\item[{Return type}] \leavevmode
\sphinxhref{https://docs.python.org/3.7/library/constants.html\#None}{None}

\end{description}\end{quote}

\end{fulllineitems}

\index{execute\_epoch() (NewscastCluster method)@\spxentry{execute\_epoch()}\spxextra{NewscastCluster method}}

\begin{fulllineitems}
\phantomsection\label{\detokenize{app.domain:app.domain.cluster_groups.NewscastCluster.execute_epoch}}\pysiglinewithargsret{\sphinxbfcode{\sphinxupquote{execute\_epoch}}}{\emph{\DUrole{n}{epoch}}}{}
Orders all \sphinxcode{\sphinxupquote{members}} to execute their epoch.

\begin{sphinxadmonition}{note}{Note:}
If the \sphinxcode{\sphinxupquote{Cluster}} terminates early, before it reaches
{\hyperref[\detokenize{app.domain:app.domain.master_servers.Master.MAX_EPOCHS}]{\sphinxcrossref{\sphinxcode{\sphinxupquote{MAX\_EPOCHS}}}}},
nothing should be logged in
{\hyperref[\detokenize{app.domain.helpers:app.domain.helpers.smart_dataclasses.LoggingData}]{\sphinxcrossref{\sphinxcode{\sphinxupquote{LoggingData}}}}}
at the specified \sphinxcode{\sphinxupquote{epoch}} to avoid skewing previously
collected results.
\end{sphinxadmonition}
\begin{quote}\begin{description}
\item[{Parameters}] \leavevmode
\sphinxstyleliteralstrong{\sphinxupquote{epoch}} (\sphinxhref{https://docs.python.org/3.7/library/functions.html\#int}{\sphinxstyleliteralemphasis{\sphinxupquote{int}}}) \textendash{} The epoch the \sphinxcode{\sphinxupquote{Cluster}} should currently be in, according
to it’s managing \sphinxcode{\sphinxupquote{master}} entity.

\item[{Returns}] \leavevmode
\sphinxcode{\sphinxupquote{False}} if \sphinxcode{\sphinxupquote{Cluster}} failed to persist the \sphinxcode{\sphinxupquote{file}} it
was responsible for, otherwise \sphinxcode{\sphinxupquote{True}}.

\item[{Return type}] \leavevmode
\sphinxhref{https://docs.python.org/3.7/library/constants.html\#None}{None}

\end{description}\end{quote}

\end{fulllineitems}

\index{log\_aggregation() (NewscastCluster method)@\spxentry{log\_aggregation()}\spxextra{NewscastCluster method}}

\begin{fulllineitems}
\phantomsection\label{\detokenize{app.domain:app.domain.cluster_groups.NewscastCluster.log_aggregation}}\pysiglinewithargsret{\sphinxbfcode{\sphinxupquote{log\_aggregation}}}{\emph{\DUrole{n}{value}}}{}~\begin{quote}\begin{description}
\item[{Parameters}] \leavevmode
\sphinxstyleliteralstrong{\sphinxupquote{value}} (\sphinxhref{https://docs.python.org/3.7/library/functions.html\#float}{\sphinxstyleliteralemphasis{\sphinxupquote{float}}}) \textendash{} 

\end{description}\end{quote}

\end{fulllineitems}

\index{nodes\_execute() (NewscastCluster method)@\spxentry{nodes\_execute()}\spxextra{NewscastCluster method}}

\begin{fulllineitems}
\phantomsection\label{\detokenize{app.domain:app.domain.cluster_groups.NewscastCluster.nodes_execute}}\pysiglinewithargsret{\sphinxbfcode{\sphinxupquote{nodes\_execute}}}{}{}
Queries all network node members execute the epoch.
\begin{description}
\item[{Overrides:}] \leavevmode
{\hyperref[\detokenize{app.domain:app.domain.cluster_groups.Cluster.nodes_execute}]{\sphinxcrossref{\sphinxcode{\sphinxupquote{app.domain.cluster\_groups.Cluster.nodes\_execute()}}}}}.
\begin{description}
\item[{Note:}] \leavevmode
\sphinxcode{\sphinxupquote{NewscasterCluster.nodes\_execute}}
always returns None.

\end{description}

\end{description}
\begin{quote}\begin{description}
\item[{Returns}] \leavevmode
A collection of members who disconnected during the current
epoch. See
{\hyperref[\detokenize{app.domain:app.domain.network_nodes.NewscastNode.update_status}]{\sphinxcrossref{\sphinxcode{\sphinxupquote{app.domain.network\_nodes.NewscastNode.update\_status()}}}}}.

\item[{Return type}] \leavevmode
List{[}{\hyperref[\detokenize{app:app.type_hints.NodeType}]{\sphinxcrossref{\sphinxcode{\sphinxupquote{NodeType}}}}}{]}

\end{description}\end{quote}

\end{fulllineitems}

\index{spread\_files() (NewscastCluster method)@\spxentry{spread\_files()}\spxextra{NewscastCluster method}}

\begin{fulllineitems}
\phantomsection\label{\detokenize{app.domain:app.domain.cluster_groups.NewscastCluster.spread_files}}\pysiglinewithargsret{\sphinxbfcode{\sphinxupquote{spread\_files}}}{\emph{\DUrole{n}{replicas}}, \emph{\DUrole{n}{strat}\DUrole{o}{=}\DUrole{default_value}{\textquotesingle{}o\textquotesingle{}}}}{}
Distributes a collection of {\hyperref[\detokenize{app.domain.helpers:app.domain.helpers.smart_dataclasses.FileBlockData}]{\sphinxcrossref{\sphinxcode{\sphinxupquote{file block replicas}}}}} among the
\sphinxcode{\sphinxupquote{members}} of the cluster group.
\begin{description}
\item[{Overrides:}] \leavevmode
\sphinxcode{\sphinxupquote{app.dommain.cluster\_groups.Cluster.spread\_files()}}

\end{description}
\begin{quote}\begin{description}
\item[{Parameters}] \leavevmode\begin{itemize}
\item {} 
\sphinxstyleliteralstrong{\sphinxupquote{replicas}} ({\hyperref[\detokenize{app:app.type_hints.ReplicasDict}]{\sphinxcrossref{\sphinxcode{\sphinxupquote{ReplicasDict}}}}}) \textendash{} The {\hyperref[\detokenize{app.domain.helpers:app.domain.helpers.smart_dataclasses.FileBlockData}]{\sphinxcrossref{\sphinxcode{\sphinxupquote{FileBlockData}}}}}
replicas, without replication.

\item {} 
\sphinxstyleliteralstrong{\sphinxupquote{strat}} (\sphinxhref{https://docs.python.org/3.7/library/stdtypes.html\#str}{\sphinxstyleliteralemphasis{\sphinxupquote{str}}}) \textendash{} 
Defines how \sphinxcode{\sphinxupquote{replicas}} will be initially distributed in
the \sphinxcode{\sphinxupquote{Cluster}}. Unless overridden in children of this class the
received value of \sphinxcode{\sphinxupquote{strat}} will be ignored and will always
be set to the default value \sphinxcode{\sphinxupquote{o}}.
\begin{description}
\item[{o}] \leavevmode
This strategy assumes erasure\sphinxhyphen{}coding is being used and
that each {\hyperref[\detokenize{app.domain:app.domain.network_nodes.Node}]{\sphinxcrossref{\sphinxcode{\sphinxupquote{network node}}}}} will have no more than
one encoded block, i.e., replication level is always
equal to one. Note however, that if there are more encoded
blocks than there are {\hyperref[\detokenize{app.domain:app.domain.network_nodes.Node}]{\sphinxcrossref{\sphinxcode{\sphinxupquote{network nodes}}}}}, some of these \sphinxcode{\sphinxupquote{nodes}}
might end up possessing an excessive amount of blocks.

\end{description}


\end{itemize}

\item[{Return type}] \leavevmode
\sphinxhref{https://docs.python.org/3.7/library/constants.html\#None}{None}

\end{description}\end{quote}

\end{fulllineitems}

\index{wire\_k\_out() (NewscastCluster method)@\spxentry{wire\_k\_out()}\spxextra{NewscastCluster method}}

\begin{fulllineitems}
\phantomsection\label{\detokenize{app.domain:app.domain.cluster_groups.NewscastCluster.wire_k_out}}\pysiglinewithargsret{\sphinxbfcode{\sphinxupquote{wire\_k\_out}}}{}{}
Creates a random directed P2P topology.

The initial cache size of each {\hyperref[\detokenize{app.domain:app.domain.network_nodes.NewscastNode}]{\sphinxcrossref{\sphinxcode{\sphinxupquote{network node}}}}}, is at most as big as
{\hyperref[\detokenize{app:app.environment_settings.NEWSCAST_CACHE_SIZE}]{\sphinxcrossref{\sphinxcode{\sphinxupquote{NEWSCAST\_CACHE\_SIZE}}}}}.

\begin{sphinxadmonition}{note}{Note:}
The topology does not have self loops, because
{\hyperref[\detokenize{app.domain:app.domain.network_nodes.NewscastNode.add_neighbor}]{\sphinxcrossref{\sphinxcode{\sphinxupquote{add\_neighbor()}}}}}
does not accept node self addition to
{\hyperref[\detokenize{app.domain:app.domain.network_nodes.NewscastNode.view}]{\sphinxcrossref{\sphinxcode{\sphinxupquote{view}}}}}. In rare
occasions, the selected node out\sphinxhyphen{}going edges might all be
invalid, this should be a non\sphinxhyphen{}issue, as the nodes will eventually
join the overaly throughout the simulation.
\end{sphinxadmonition}

\end{fulllineitems}


\end{fulllineitems}

\index{SGCluster (class in app.domain.cluster\_groups)@\spxentry{SGCluster}\spxextra{class in app.domain.cluster\_groups}}

\begin{fulllineitems}
\phantomsection\label{\detokenize{app.domain:app.domain.cluster_groups.SGCluster}}\pysiglinewithargsret{\sphinxbfcode{\sphinxupquote{class }}\sphinxbfcode{\sphinxupquote{SGCluster}}}{\emph{\DUrole{n}{master}}, \emph{\DUrole{n}{file\_name}}, \emph{\DUrole{n}{members}}, \emph{\DUrole{n}{sim\_id}\DUrole{o}{=}\DUrole{default_value}{0}}, \emph{\DUrole{n}{origin}\DUrole{o}{=}\DUrole{default_value}{\textquotesingle{}\textquotesingle{}}}}{}
Bases: {\hyperref[\detokenize{app.domain:app.domain.cluster_groups.Cluster}]{\sphinxcrossref{\sphinxcode{\sphinxupquote{app.domain.cluster\_groups.Cluster}}}}}

Represents a group of network nodes persisting a file using swarm
guidance algorithm.
\index{v\_ (SGCluster attribute)@\spxentry{v\_}\spxextra{SGCluster attribute}}

\begin{fulllineitems}
\phantomsection\label{\detokenize{app.domain:app.domain.cluster_groups.SGCluster.v_}}\pysigline{\sphinxbfcode{\sphinxupquote{v\_}}}
Density distribution cluster members must achieve with independent
realizations for ideal persistence of the file.
\begin{quote}\begin{description}
\item[{Type}] \leavevmode
\sphinxhref{https://pandas.pydata.org/docs/reference/api/pandas.DataFrame.html\#pandas.DataFrame}{\sphinxcode{\sphinxupquote{DataFrame}}}

\end{description}\end{quote}

\end{fulllineitems}

\index{cv\_ (SGCluster attribute)@\spxentry{cv\_}\spxextra{SGCluster attribute}}

\begin{fulllineitems}
\phantomsection\label{\detokenize{app.domain:app.domain.cluster_groups.SGCluster.cv_}}\pysigline{\sphinxbfcode{\sphinxupquote{cv\_}}}
Tracks the file current density distribution, updated at each epoch.
\begin{quote}\begin{description}
\item[{Type}] \leavevmode
\sphinxhref{https://pandas.pydata.org/docs/reference/api/pandas.DataFrame.html\#pandas.DataFrame}{\sphinxcode{\sphinxupquote{DataFrame}}}

\end{description}\end{quote}

\end{fulllineitems}

\index{avg\_ (SGCluster attribute)@\spxentry{avg\_}\spxextra{SGCluster attribute}}

\begin{fulllineitems}
\phantomsection\label{\detokenize{app.domain:app.domain.cluster_groups.SGCluster.avg_}}\pysigline{\sphinxbfcode{\sphinxupquote{avg\_}}}
Tracks the file average density distribution. Used to assert if
throughout the life time of a cluster, the desired density
distribution {\hyperref[\detokenize{app.domain:app.domain.cluster_groups.SGCluster.v_}]{\sphinxcrossref{\sphinxcode{\sphinxupquote{v\_}}}}} was achieved on average. Differs from
{\hyperref[\detokenize{app.domain:app.domain.cluster_groups.SGCluster.cv_}]{\sphinxcrossref{\sphinxcode{\sphinxupquote{cv\_}}}}} because \sphinxtitleref{cv\_} is used for instantaneous
convergence comparison.
\begin{quote}\begin{description}
\item[{Type}] \leavevmode
\sphinxhref{https://pandas.pydata.org/docs/reference/api/pandas.DataFrame.html\#pandas.DataFrame}{\sphinxcode{\sphinxupquote{DataFrame}}}

\end{description}\end{quote}

\end{fulllineitems}

\index{\_timer (SGCluster attribute)@\spxentry{\_timer}\spxextra{SGCluster attribute}}

\begin{fulllineitems}
\phantomsection\label{\detokenize{app.domain:app.domain.cluster_groups.SGCluster._timer}}\pysigline{\sphinxbfcode{\sphinxupquote{\_timer}}}
Used as a logical clock to divide the entries of {\hyperref[\detokenize{app.domain:app.domain.cluster_groups.SGCluster.avg_}]{\sphinxcrossref{\sphinxcode{\sphinxupquote{avg\_}}}}}
when a topology changes.
\begin{quote}\begin{description}
\item[{Type}] \leavevmode
\sphinxhref{https://docs.python.org/3.7/library/functions.html\#int}{int}

\end{description}\end{quote}

\end{fulllineitems}

\index{\_\_init\_\_() (SGCluster method)@\spxentry{\_\_init\_\_()}\spxextra{SGCluster method}}

\begin{fulllineitems}
\phantomsection\label{\detokenize{app.domain:app.domain.cluster_groups.SGCluster.__init__}}\pysiglinewithargsret{\sphinxbfcode{\sphinxupquote{\_\_init\_\_}}}{\emph{\DUrole{n}{master}}, \emph{\DUrole{n}{file\_name}}, \emph{\DUrole{n}{members}}, \emph{\DUrole{n}{sim\_id}\DUrole{o}{=}\DUrole{default_value}{0}}, \emph{\DUrole{n}{origin}\DUrole{o}{=}\DUrole{default_value}{\textquotesingle{}\textquotesingle{}}}}{}
Instantiates an \sphinxcode{\sphinxupquote{Cluster}} object
\begin{quote}\begin{description}
\item[{Parameters}] \leavevmode\begin{itemize}
\item {} 
\sphinxstyleliteralstrong{\sphinxupquote{master}} ({\hyperref[\detokenize{app:app.type_hints.MasterType}]{\sphinxcrossref{\sphinxcode{\sphinxupquote{MasterType}}}}}) \textendash{} A reference to an {\hyperref[\detokenize{app.domain:app.domain.master_servers.Master}]{\sphinxcrossref{\sphinxcode{\sphinxupquote{Master}}}}}
object that manages the \sphinxcode{\sphinxupquote{Cluster}} being initialized.

\item {} 
\sphinxstyleliteralstrong{\sphinxupquote{file\_name}} (\sphinxhref{https://docs.python.org/3.7/library/stdtypes.html\#str}{\sphinxstyleliteralemphasis{\sphinxupquote{str}}}) \textendash{} The name of the file the \sphinxcode{\sphinxupquote{Cluster}} is responsible for
persisting.

\item {} 
\sphinxstyleliteralstrong{\sphinxupquote{members}} ({\hyperref[\detokenize{app:app.type_hints.NodeDict}]{\sphinxcrossref{\sphinxcode{\sphinxupquote{NodeDict}}}}}) \textendash{} A dictionary where keys are {\hyperref[\detokenize{app.domain:app.domain.network_nodes.Node.id}]{\sphinxcrossref{\sphinxcode{\sphinxupquote{node identifiers}}}}} and values are their
{\hyperref[\detokenize{app.domain:app.domain.network_nodes.Node}]{\sphinxcrossref{\sphinxcode{\sphinxupquote{instance objects}}}}}.

\item {} 
\sphinxstyleliteralstrong{\sphinxupquote{sim\_id}} (\sphinxhref{https://docs.python.org/3.7/library/functions.html\#int}{\sphinxstyleliteralemphasis{\sphinxupquote{int}}}) \textendash{} Identifier that generates unique output file names,
thus guaranteeing that different simulation instances do not
overwrite previous out files.

\item {} 
\sphinxstyleliteralstrong{\sphinxupquote{origin}} (\sphinxhref{https://docs.python.org/3.7/library/stdtypes.html\#str}{\sphinxstyleliteralemphasis{\sphinxupquote{str}}}) \textendash{} The name of the simulation file name that started
the simulation process.

\end{itemize}

\item[{Return type}] \leavevmode
\sphinxhref{https://docs.python.org/3.7/library/constants.html\#None}{None}

\end{description}\end{quote}

\end{fulllineitems}

\index{\_log\_evaluation() (SGCluster method)@\spxentry{\_log\_evaluation()}\spxextra{SGCluster method}}

\begin{fulllineitems}
\phantomsection\label{\detokenize{app.domain:app.domain.cluster_groups.SGCluster._log_evaluation}}\pysiglinewithargsret{\sphinxbfcode{\sphinxupquote{\_log\_evaluation}}}{\emph{\DUrole{n}{pcount}}, \emph{\DUrole{n}{ptotal}\DUrole{o}{=}\DUrole{default_value}{\sphinxhyphen{} 1}}}{}
Helper that collects \sphinxcode{\sphinxupquote{Cluster}} data and registers it on a
{\hyperref[\detokenize{app.domain.helpers:app.domain.helpers.smart_dataclasses.LoggingData}]{\sphinxcrossref{\sphinxcode{\sphinxupquote{logger}}}}}
object.
\begin{quote}\begin{description}
\item[{Parameters}] \leavevmode\begin{itemize}
\item {} 
\sphinxstyleliteralstrong{\sphinxupquote{plive}} \textendash{} The number of existing parts in the cluster at the
simulation’s current epoch at online or suspect nodes.

\item {} 
\sphinxstyleliteralstrong{\sphinxupquote{ptotal}} (\sphinxhref{https://docs.python.org/3.7/library/functions.html\#int}{\sphinxstyleliteralemphasis{\sphinxupquote{int}}}) \textendash{} The number of existing parts in the cluster at the
simulation’s current epoch. This parameter is optional and
may be used or not depending on the intent of the system.
As a rule of thumb \sphinxcode{\sphinxupquote{plive}} tracks the number of parts that
are alive in the system for logging purposes, where as
\sphinxcode{\sphinxupquote{ptotal}} is used for comparisons and averages, e.g.,
{\hyperref[\detokenize{app.domain:app.domain.cluster_groups.SGCluster.evaluate}]{\sphinxcrossref{\sphinxcode{\sphinxupquote{SGCluster evaluate}}}}}.

\item {} 
\sphinxstyleliteralstrong{\sphinxupquote{pcount}} (\sphinxhref{https://docs.python.org/3.7/library/functions.html\#int}{\sphinxstyleliteralemphasis{\sphinxupquote{int}}}) \textendash{} 

\end{itemize}

\item[{Return type}] \leavevmode
\sphinxhref{https://docs.python.org/3.7/library/constants.html\#None}{None}

\end{description}\end{quote}

\end{fulllineitems}

\index{\_normalize\_avg\_() (SGCluster method)@\spxentry{\_normalize\_avg\_()}\spxextra{SGCluster method}}

\begin{fulllineitems}
\phantomsection\label{\detokenize{app.domain:app.domain.cluster_groups.SGCluster._normalize_avg_}}\pysiglinewithargsret{\sphinxbfcode{\sphinxupquote{\_normalize\_avg\_}}}{}{}
\end{fulllineitems}

\index{\_pretty\_print\_eq\_distr\_table() (SGCluster method)@\spxentry{\_pretty\_print\_eq\_distr\_table()}\spxextra{SGCluster method}}

\begin{fulllineitems}
\phantomsection\label{\detokenize{app.domain:app.domain.cluster_groups.SGCluster._pretty_print_eq_distr_table}}\pysiglinewithargsret{\sphinxbfcode{\sphinxupquote{\_pretty\_print\_eq\_distr\_table}}}{\emph{\DUrole{n}{target}}, \emph{\DUrole{n}{rtol}}, \emph{\DUrole{n}{atol}}}{}
Pretty prints a PSQL formatted table for visual vector comparison.
\begin{quote}\begin{description}
\item[{Parameters}] \leavevmode\begin{itemize}
\item {} 
\sphinxstyleliteralstrong{\sphinxupquote{target}} (\sphinxhref{https://pandas.pydata.org/docs/reference/api/pandas.DataFrame.html\#pandas.DataFrame}{\sphinxcode{\sphinxupquote{DataFrame}}}) \textendash{} The \sphinxhref{https://pandas.pydata.org/docs/reference/api/pandas.DataFrame.html\#pandas.DataFrame}{\sphinxcode{\sphinxupquote{DataFrame}}} object to be formatted
as PSQL table.

\item {} 
\sphinxstyleliteralstrong{\sphinxupquote{atol}} (\sphinxhref{https://docs.python.org/3.7/library/functions.html\#float}{\sphinxstyleliteralemphasis{\sphinxupquote{float}}}) \textendash{} The allowed absolute tolerance.

\item {} 
\sphinxstyleliteralstrong{\sphinxupquote{rtol}} (\sphinxhref{https://docs.python.org/3.7/library/functions.html\#float}{\sphinxstyleliteralemphasis{\sphinxupquote{float}}}) \textendash{} The allowed relative tolerance.

\end{itemize}

\item[{Return type}] \leavevmode
Any

\end{description}\end{quote}

\end{fulllineitems}

\index{\_validate\_transition\_matrix() (SGCluster method)@\spxentry{\_validate\_transition\_matrix()}\spxextra{SGCluster method}}

\begin{fulllineitems}
\phantomsection\label{\detokenize{app.domain:app.domain.cluster_groups.SGCluster._validate_transition_matrix}}\pysiglinewithargsret{\sphinxbfcode{\sphinxupquote{\_validate\_transition\_matrix}}}{\emph{\DUrole{n}{m}}, \emph{\DUrole{n}{v\_}}}{}
Asserts if \sphinxcode{\sphinxupquote{m}} is a Markov Matrix.

Verification is done by raising the \sphinxcode{\sphinxupquote{m}} to the power
of \sphinxcode{\sphinxupquote{4096}} (just a large number) and checking if all columns of the
powered matrix are element\sphinxhyphen{}wise equal to the
entries of \sphinxcode{\sphinxupquote{target\_distribution}}.
\begin{quote}\begin{description}
\item[{Parameters}] \leavevmode\begin{itemize}
\item {} 
\sphinxstyleliteralstrong{\sphinxupquote{m}} (\sphinxhref{https://pandas.pydata.org/docs/reference/api/pandas.DataFrame.html\#pandas.DataFrame}{\sphinxcode{\sphinxupquote{DataFrame}}}) \textendash{} The matrix to be verified.

\item {} 
\sphinxstyleliteralstrong{\sphinxupquote{v\_}} (\sphinxhref{https://pandas.pydata.org/docs/reference/api/pandas.DataFrame.html\#pandas.DataFrame}{\sphinxcode{\sphinxupquote{DataFrame}}}) \textendash{} The steady state the \sphinxcode{\sphinxupquote{m}} is expected to have.

\end{itemize}

\item[{Returns}] \leavevmode
\sphinxcode{\sphinxupquote{True}} if the matrix converges to the \sphinxcode{\sphinxupquote{target\_distribution}},
otherwise \sphinxcode{\sphinxupquote{False}}. I.e., if \sphinxcode{\sphinxupquote{m}} is a
markov matrix.

\item[{Return type}] \leavevmode
\sphinxhref{https://docs.python.org/3.7/library/functions.html\#bool}{bool}

\end{description}\end{quote}

\end{fulllineitems}

\index{add\_cloud\_reference() (SGCluster method)@\spxentry{add\_cloud\_reference()}\spxextra{SGCluster method}}

\begin{fulllineitems}
\phantomsection\label{\detokenize{app.domain:app.domain.cluster_groups.SGCluster.add_cloud_reference}}\pysiglinewithargsret{\sphinxbfcode{\sphinxupquote{add\_cloud\_reference}}}{}{}
Adds a cloud server to the {\hyperref[\detokenize{app.domain:app.domain.cluster_groups.Cluster.members}]{\sphinxcrossref{\sphinxcode{\sphinxupquote{members}}}}} of
the \sphinxcode{\sphinxupquote{SGCluster}}.

This method is used when \sphinxcode{\sphinxupquote{SGCluster}} membership size becomes
compromised and a backup solution using cloud approaches is desired.
The idea is that surviving members upload their replicas to the cloud
server, e.g., an Amazon S3 instance. See Master method
{\hyperref[\detokenize{app.domain:app.domain.master_servers.SGMaster.get_cloud_reference}]{\sphinxcrossref{\sphinxcode{\sphinxupquote{get\_cloud\_reference()}}}}}
for more details.

\begin{sphinxadmonition}{note}{Note:}
This method is virtual.
\end{sphinxadmonition}
\begin{quote}\begin{description}
\item[{Return type}] \leavevmode
\sphinxhref{https://docs.python.org/3.7/library/constants.html\#None}{None}

\end{description}\end{quote}

\end{fulllineitems}

\index{broadcast\_transition\_matrix() (SGCluster method)@\spxentry{broadcast\_transition\_matrix()}\spxextra{SGCluster method}}

\begin{fulllineitems}
\phantomsection\label{\detokenize{app.domain:app.domain.cluster_groups.SGCluster.broadcast_transition_matrix}}\pysiglinewithargsret{\sphinxbfcode{\sphinxupquote{broadcast\_transition\_matrix}}}{\emph{\DUrole{n}{m}}}{}
Slices a  matrix and delivers columns to the respective
{\hyperref[\detokenize{app.domain:app.domain.network_nodes.SGNode}]{\sphinxcrossref{\sphinxcode{\sphinxupquote{network nodes}}}}}.
\begin{quote}\begin{description}
\item[{Parameters}] \leavevmode
\sphinxstyleliteralstrong{\sphinxupquote{m}} (\sphinxhref{https://pandas.pydata.org/docs/reference/api/pandas.DataFrame.html\#pandas.DataFrame}{\sphinxcode{\sphinxupquote{DataFrame}}}) \textendash{} A matrix to be broadcasted to the network nodes
belonging who are currently members of the Cluster instance.

\item[{Return type}] \leavevmode
\sphinxhref{https://docs.python.org/3.7/library/constants.html\#None}{None}

\end{description}\end{quote}

\begin{sphinxadmonition}{note}{Note:}
An optimization could be made that configures a transition matrix
for the cluster, independent of of file names, i.e., turn cluster
groups into groups persisting multiple files instead of only one,
thus reducing simulation spaceoverheads and in real\sphinxhyphen{}life
scenarios, decreasing the load done to metadata servers, through
queries and matrix calculations. For simplicity of implementation
each cluster only manages one file.
\end{sphinxadmonition}

\end{fulllineitems}

\index{create\_and\_bcast\_new\_transition\_matrix() (SGCluster method)@\spxentry{create\_and\_bcast\_new\_transition\_matrix()}\spxextra{SGCluster method}}

\begin{fulllineitems}
\phantomsection\label{\detokenize{app.domain:app.domain.cluster_groups.SGCluster.create_and_bcast_new_transition_matrix}}\pysiglinewithargsret{\sphinxbfcode{\sphinxupquote{create\_and\_bcast\_new\_transition\_matrix}}}{}{}
Helper method that attempts to generate a markov matrix to be
sliced and distributed to the \sphinxcode{\sphinxupquote{SGCluster}}
{\hyperref[\detokenize{app.domain:app.domain.cluster_groups.Cluster.members}]{\sphinxcrossref{\sphinxcode{\sphinxupquote{members}}}}}.

At most three transition matrices will be generated. The first to be
successfully {\hyperref[\detokenize{app.domain:app.domain.cluster_groups.SGCluster._validate_transition_matrix}]{\sphinxcrossref{\sphinxcode{\sphinxupquote{validated}}}}} is
distributed to the {\hyperref[\detokenize{app.domain:app.domain.network_nodes.SGNode}]{\sphinxcrossref{\sphinxcode{\sphinxupquote{network nodes}}}}}. If all matrices are invalid,
the last matrix will be used to prevent infinite loops in the
simulation. This is not an issue as eventually the membership of the
\sphinxcode{\sphinxupquote{SGCluster}} will change, thus, more opportunities to perform a
correct swarm guidance behavior will be possible.
\begin{quote}\begin{description}
\item[{Return type}] \leavevmode
\sphinxhref{https://docs.python.org/3.7/library/constants.html\#None}{None}

\end{description}\end{quote}

\end{fulllineitems}

\index{equal\_distributions() (SGCluster method)@\spxentry{equal\_distributions()}\spxextra{SGCluster method}}

\begin{fulllineitems}
\phantomsection\label{\detokenize{app.domain:app.domain.cluster_groups.SGCluster.equal_distributions}}\pysiglinewithargsret{\sphinxbfcode{\sphinxupquote{equal\_distributions}}}{}{}
Asserts if the {\hyperref[\detokenize{app.domain:app.domain.cluster_groups.SGCluster.v_}]{\sphinxcrossref{\sphinxcode{\sphinxupquote{desired distribution}}}}} and
{\hyperref[\detokenize{app.domain:app.domain.cluster_groups.SGCluster.cv_}]{\sphinxcrossref{\sphinxcode{\sphinxupquote{current distribution}}}}} are equal.

Equalility is calculated using numpy allclose function which has the
following formula:

\begin{sphinxVerbatim}[commandchars=\\\{\}]
absolute(`a` \PYGZhy{} `b`) \PYGZlt{}= (`atol` + `rtol` * absolute(`b`))
\end{sphinxVerbatim}
\begin{quote}\begin{description}
\item[{Returns}] \leavevmode
\sphinxcode{\sphinxupquote{True}} if distributions are close enough to be considered equal,
otherwise, it returns \sphinxcode{\sphinxupquote{False}}.

\item[{Return type}] \leavevmode
\sphinxhref{https://docs.python.org/3.7/library/functions.html\#bool}{bool}

\end{description}\end{quote}

\end{fulllineitems}

\index{evaluate() (SGCluster method)@\spxentry{evaluate()}\spxextra{SGCluster method}}

\begin{fulllineitems}
\phantomsection\label{\detokenize{app.domain:app.domain.cluster_groups.SGCluster.evaluate}}\pysiglinewithargsret{\sphinxbfcode{\sphinxupquote{evaluate}}}{}{}
Evaluates and logs the health, possibly other parameters, of the
\sphinxcode{\sphinxupquote{Cluster}} at every epoch.
\begin{quote}\begin{description}
\item[{Return type}] \leavevmode
\sphinxhref{https://docs.python.org/3.7/library/constants.html\#None}{None}

\end{description}\end{quote}

\end{fulllineitems}

\index{execute\_epoch() (SGCluster method)@\spxentry{execute\_epoch()}\spxextra{SGCluster method}}

\begin{fulllineitems}
\phantomsection\label{\detokenize{app.domain:app.domain.cluster_groups.SGCluster.execute_epoch}}\pysiglinewithargsret{\sphinxbfcode{\sphinxupquote{execute\_epoch}}}{\emph{\DUrole{n}{epoch}}}{}
Orders all \sphinxcode{\sphinxupquote{members}} to execute their epoch.

\begin{sphinxadmonition}{note}{Note:}
If the \sphinxcode{\sphinxupquote{Cluster}} terminates early, before it reaches
{\hyperref[\detokenize{app.domain:app.domain.master_servers.Master.MAX_EPOCHS}]{\sphinxcrossref{\sphinxcode{\sphinxupquote{MAX\_EPOCHS}}}}},
nothing should be logged in
{\hyperref[\detokenize{app.domain.helpers:app.domain.helpers.smart_dataclasses.LoggingData}]{\sphinxcrossref{\sphinxcode{\sphinxupquote{LoggingData}}}}}
at the specified \sphinxcode{\sphinxupquote{epoch}} to avoid skewing previously
collected results.
\end{sphinxadmonition}
\begin{quote}\begin{description}
\item[{Parameters}] \leavevmode
\sphinxstyleliteralstrong{\sphinxupquote{epoch}} (\sphinxhref{https://docs.python.org/3.7/library/functions.html\#int}{\sphinxstyleliteralemphasis{\sphinxupquote{int}}}) \textendash{} The epoch the \sphinxcode{\sphinxupquote{Cluster}} should currently be in, according
to it’s managing \sphinxcode{\sphinxupquote{master}} entity.

\item[{Returns}] \leavevmode
\sphinxcode{\sphinxupquote{False}} if \sphinxcode{\sphinxupquote{Cluster}} failed to persist the \sphinxcode{\sphinxupquote{file}} it
was responsible for, otherwise \sphinxcode{\sphinxupquote{True}}.

\item[{Return type}] \leavevmode
\sphinxhref{https://docs.python.org/3.7/library/constants.html\#None}{None}

\end{description}\end{quote}

\end{fulllineitems}

\index{maintain() (SGCluster method)@\spxentry{maintain()}\spxextra{SGCluster method}}

\begin{fulllineitems}
\phantomsection\label{\detokenize{app.domain:app.domain.cluster_groups.SGCluster.maintain}}\pysiglinewithargsret{\sphinxbfcode{\sphinxupquote{maintain}}}{\emph{\DUrole{n}{off\_nodes}}}{}
Evicts any node who is referenced in off\_nodes list.
\begin{description}
\item[{Extends:}] \leavevmode
{\hyperref[\detokenize{app.domain:app.domain.cluster_groups.Cluster.maintain}]{\sphinxcrossref{\sphinxcode{\sphinxupquote{app.domain.cluster\_groups.Cluster.maintain()}}}}}.

\end{description}
\begin{quote}\begin{description}
\item[{Parameters}] \leavevmode
\sphinxstyleliteralstrong{\sphinxupquote{off\_nodes}} (List{[}{\hyperref[\detokenize{app:app.type_hints.NodeType}]{\sphinxcrossref{\sphinxcode{\sphinxupquote{NodeType}}}}}{]}) \textendash{} The subset of {\hyperref[\detokenize{app.domain:app.domain.cluster_groups.Cluster.members}]{\sphinxcrossref{\sphinxcode{\sphinxupquote{members}}}}} who disconnected
during the current epoch.

\item[{Return type}] \leavevmode
\sphinxhref{https://docs.python.org/3.7/library/constants.html\#None}{None}

\end{description}\end{quote}

\end{fulllineitems}

\index{membership\_maintenance() (SGCluster method)@\spxentry{membership\_maintenance()}\spxextra{SGCluster method}}

\begin{fulllineitems}
\phantomsection\label{\detokenize{app.domain:app.domain.cluster_groups.SGCluster.membership_maintenance}}\pysiglinewithargsret{\sphinxbfcode{\sphinxupquote{membership\_maintenance}}}{}{}
Attempts to recruits new network nodes to be members of the cluster.

The method updates both \sphinxcode{\sphinxupquote{members}} and \sphinxcode{\sphinxupquote{\_members\_view}}.
\begin{description}
\item[{Extends:}] \leavevmode
{\hyperref[\detokenize{app.domain:app.domain.cluster_groups.Cluster.membership_maintenance}]{\sphinxcrossref{\sphinxcode{\sphinxupquote{app.domain.cluster\_groups.Cluster.membership\_maintenance()}}}}}.

\sphinxcode{\sphinxupquote{SGCluster.membership\_maintenance}} adds and removes cloud
references depending depending on the length of {\hyperref[\detokenize{app.domain:app.domain.cluster_groups.Cluster.members}]{\sphinxcrossref{\sphinxcode{\sphinxupquote{members}}}}}
before maintenance is performed.

\end{description}
\begin{quote}\begin{description}
\item[{Returns}] \leavevmode
A dictionary that is empty if membership did not change.

\item[{Return type}] \leavevmode
{\hyperref[\detokenize{app:app.type_hints.NodeDict}]{\sphinxcrossref{\sphinxcode{\sphinxupquote{NodeDict}}}}}

\end{description}\end{quote}

\end{fulllineitems}

\index{new\_desired\_distribution() (SGCluster method)@\spxentry{new\_desired\_distribution()}\spxextra{SGCluster method}}

\begin{fulllineitems}
\phantomsection\label{\detokenize{app.domain:app.domain.cluster_groups.SGCluster.new_desired_distribution}}\pysiglinewithargsret{\sphinxbfcode{\sphinxupquote{new\_desired\_distribution}}}{\emph{\DUrole{n}{member\_ids}}, \emph{\DUrole{n}{member\_uptimes}}}{}
Sets a new {\hyperref[\detokenize{app.domain:app.domain.cluster_groups.SGCluster.v_}]{\sphinxcrossref{\sphinxcode{\sphinxupquote{desired distribution}}}}} for the
\sphinxcode{\sphinxupquote{SGCluster}}.

Received \sphinxcode{\sphinxupquote{member\_uptimes}} are normalized to create a stochastic
representation of the desired distribution, which can be used by the
different transition matrix generation strategies.
\begin{quote}\begin{description}
\item[{Parameters}] \leavevmode\begin{itemize}
\item {} 
\sphinxstyleliteralstrong{\sphinxupquote{member\_ids}} (\sphinxstyleliteralemphasis{\sphinxupquote{List}}\sphinxstyleliteralemphasis{\sphinxupquote{{[}}}\sphinxhref{https://docs.python.org/3.7/library/stdtypes.html\#str}{\sphinxstyleliteralemphasis{\sphinxupquote{str}}}\sphinxstyleliteralemphasis{\sphinxupquote{{]}}}) \textendash{} A list of {\hyperref[\detokenize{app.domain:app.domain.network_nodes.Node.id}]{\sphinxcrossref{\sphinxcode{\sphinxupquote{node identifiers}}}}} who are
{\hyperref[\detokenize{app.domain:app.domain.cluster_groups.Cluster.members}]{\sphinxcrossref{\sphinxcode{\sphinxupquote{members}}}}} of the \sphinxcode{\sphinxupquote{SGCluster}}.

\item {} 
\sphinxstyleliteralstrong{\sphinxupquote{member\_uptimes}} (\sphinxstyleliteralemphasis{\sphinxupquote{List}}\sphinxstyleliteralemphasis{\sphinxupquote{{[}}}\sphinxhref{https://docs.python.org/3.7/library/functions.html\#float}{\sphinxstyleliteralemphasis{\sphinxupquote{float}}}\sphinxstyleliteralemphasis{\sphinxupquote{{]}}}) \textendash{} A list of {\hyperref[\detokenize{app.domain:app.domain.network_nodes.Node.uptime}]{\sphinxcrossref{\sphinxcode{\sphinxupquote{node identifiers}}}}}.

\end{itemize}

\item[{Return type}] \leavevmode
List{[}\sphinxhref{https://docs.python.org/3.7/library/functions.html\#float}{float}{]}

\end{description}\end{quote}

\begin{sphinxadmonition}{note}{Note:}
\sphinxcode{\sphinxupquote{member\_ids}} and \sphinxcode{\sphinxupquote{member\_uptimes}} elements at each index should
belong to each other, i.e., they should originate from from the
same {\hyperref[\detokenize{app.domain:app.domain.network_nodes.SGNode}]{\sphinxcrossref{\sphinxcode{\sphinxupquote{network node}}}}}.
\end{sphinxadmonition}
\begin{quote}\begin{description}
\item[{Returns}] \leavevmode
A list of floats with normalized uptimes which represent the
“reliability” of network nodes.

\item[{Parameters}] \leavevmode\begin{itemize}
\item {} 
\sphinxstyleliteralstrong{\sphinxupquote{member\_ids}} (\sphinxstyleliteralemphasis{\sphinxupquote{List}}\sphinxstyleliteralemphasis{\sphinxupquote{{[}}}\sphinxhref{https://docs.python.org/3.7/library/stdtypes.html\#str}{\sphinxstyleliteralemphasis{\sphinxupquote{str}}}\sphinxstyleliteralemphasis{\sphinxupquote{{]}}}) \textendash{} 

\item {} 
\sphinxstyleliteralstrong{\sphinxupquote{member\_uptimes}} (\sphinxstyleliteralemphasis{\sphinxupquote{List}}\sphinxstyleliteralemphasis{\sphinxupquote{{[}}}\sphinxhref{https://docs.python.org/3.7/library/functions.html\#float}{\sphinxstyleliteralemphasis{\sphinxupquote{float}}}\sphinxstyleliteralemphasis{\sphinxupquote{{]}}}) \textendash{} 

\end{itemize}

\item[{Return type}] \leavevmode
List{[}\sphinxhref{https://docs.python.org/3.7/library/functions.html\#float}{float}{]}

\end{description}\end{quote}

\end{fulllineitems}

\index{new\_transition\_matrix() (SGCluster method)@\spxentry{new\_transition\_matrix()}\spxextra{SGCluster method}}

\begin{fulllineitems}
\phantomsection\label{\detokenize{app.domain:app.domain.cluster_groups.SGCluster.new_transition_matrix}}\pysiglinewithargsret{\sphinxbfcode{\sphinxupquote{new\_transition\_matrix}}}{}{}
Creates a new transition matrix that is likely to be a Markov Matrix.
\begin{quote}\begin{description}
\item[{Returns}] \leavevmode
The labeled matrix that has the fastests mixing rate from all
the pondered strategies.

\item[{Return type}] \leavevmode
\sphinxhref{https://pandas.pydata.org/docs/reference/api/pandas.DataFrame.html\#pandas.DataFrame}{\sphinxcode{\sphinxupquote{DataFrame}}}

\end{description}\end{quote}

\end{fulllineitems}

\index{nodes\_execute() (SGCluster method)@\spxentry{nodes\_execute()}\spxextra{SGCluster method}}

\begin{fulllineitems}
\phantomsection\label{\detokenize{app.domain:app.domain.cluster_groups.SGCluster.nodes_execute}}\pysiglinewithargsret{\sphinxbfcode{\sphinxupquote{nodes\_execute}}}{}{}
Queries all network node members execute the epoch.
\begin{description}
\item[{Overrides:}] \leavevmode
{\hyperref[\detokenize{app.domain:app.domain.cluster_groups.Cluster.nodes_execute}]{\sphinxcrossref{\sphinxcode{\sphinxupquote{app.domain.cluster\_groups.Cluster.nodes\_execute()}}}}}.

\end{description}
\begin{quote}\begin{description}
\item[{Returns}] \leavevmode
A collection of members who disconnected during the current
epoch. See
{\hyperref[\detokenize{app.domain:app.domain.network_nodes.Node.update_status}]{\sphinxcrossref{\sphinxcode{\sphinxupquote{app.domain.network\_nodes.Node.update\_status()}}}}}.

\item[{Return type}] \leavevmode
List{[}{\hyperref[\detokenize{app:app.type_hints.NodeType}]{\sphinxcrossref{\sphinxcode{\sphinxupquote{NodeType}}}}}{]}

\end{description}\end{quote}

\end{fulllineitems}

\index{remove\_cloud\_reference() (SGCluster method)@\spxentry{remove\_cloud\_reference()}\spxextra{SGCluster method}}

\begin{fulllineitems}
\phantomsection\label{\detokenize{app.domain:app.domain.cluster_groups.SGCluster.remove_cloud_reference}}\pysiglinewithargsret{\sphinxbfcode{\sphinxupquote{remove\_cloud\_reference}}}{}{}
Remove cloud references and delete files within it

\begin{sphinxadmonition}{note}{Note:}
This method is virtual.
\end{sphinxadmonition}
\begin{quote}\begin{description}
\item[{Return type}] \leavevmode
\sphinxhref{https://docs.python.org/3.7/library/constants.html\#None}{None}

\end{description}\end{quote}

\end{fulllineitems}

\index{select\_fastest\_topology() (SGCluster method)@\spxentry{select\_fastest\_topology()}\spxextra{SGCluster method}}

\begin{fulllineitems}
\phantomsection\label{\detokenize{app.domain:app.domain.cluster_groups.SGCluster.select_fastest_topology}}\pysiglinewithargsret{\sphinxbfcode{\sphinxupquote{select\_fastest\_topology}}}{\emph{\DUrole{n}{a}}, \emph{\DUrole{n}{v\_}}}{}
Creates multiple transition matrices and selects the fastest.

The fastest of the created transition matrices corresponds to the one
with a faster mixing rate.
\begin{quote}\begin{description}
\item[{Parameters}] \leavevmode\begin{itemize}
\item {} 
\sphinxstyleliteralstrong{\sphinxupquote{a}} (\sphinxhref{https://numpy.org/doc/stable/reference/generated/numpy.ndarray.html\#numpy.ndarray}{\sphinxcode{\sphinxupquote{ndarray}}}) \textendash{} An adjacency matrix that represents the network topology.

\item {} 
\sphinxstyleliteralstrong{\sphinxupquote{v\_}} (\sphinxhref{https://numpy.org/doc/stable/reference/generated/numpy.ndarray.html\#numpy.ndarray}{\sphinxcode{\sphinxupquote{ndarray}}}) \textendash{} A desired distribution vector that defines the returned
matrix steady state property.

\end{itemize}

\item[{Returns}] \leavevmode
A transition matrix that is likely to be a markov matrix whose
steady state is \sphinxcode{\sphinxupquote{v\_}}, but is not yet validated. See
{\hyperref[\detokenize{app.domain:app.domain.cluster_groups.SGCluster._validate_transition_matrix}]{\sphinxcrossref{\sphinxcode{\sphinxupquote{\_validate\_transition\_matrix()}}}}}.

\item[{Return type}] \leavevmode
\sphinxhref{https://numpy.org/doc/stable/reference/generated/numpy.ndarray.html\#numpy.ndarray}{\sphinxcode{\sphinxupquote{ndarray}}}

\end{description}\end{quote}

\end{fulllineitems}

\index{spread\_files() (SGCluster method)@\spxentry{spread\_files()}\spxextra{SGCluster method}}

\begin{fulllineitems}
\phantomsection\label{\detokenize{app.domain:app.domain.cluster_groups.SGCluster.spread_files}}\pysiglinewithargsret{\sphinxbfcode{\sphinxupquote{spread\_files}}}{\emph{\DUrole{n}{replicas}}, \emph{\DUrole{n}{strat}\DUrole{o}{=}\DUrole{default_value}{\textquotesingle{}i\textquotesingle{}}}}{}
Distributes a collection of
{\hyperref[\detokenize{app.domain.helpers:app.domain.helpers.smart_dataclasses.FileBlockData}]{\sphinxcrossref{\sphinxcode{\sphinxupquote{FileBlockData}}}}}
objects among the {\hyperref[\detokenize{app.domain:app.domain.cluster_groups.Cluster.members}]{\sphinxcrossref{\sphinxcode{\sphinxupquote{members}}}}} of the \sphinxcode{\sphinxupquote{SGCluster}}.
\begin{description}
\item[{Overrides:}] \leavevmode
{\hyperref[\detokenize{app.domain:app.domain.cluster_groups.Cluster.spread_files}]{\sphinxcrossref{\sphinxcode{\sphinxupquote{app.domain.cluster\_groups.Cluster.spread\_files()}}}}}.

\end{description}
\begin{quote}\begin{description}
\item[{Parameters}] \leavevmode\begin{itemize}
\item {} 
\sphinxstyleliteralstrong{\sphinxupquote{replicas}} ({\hyperref[\detokenize{app:app.type_hints.ReplicasDict}]{\sphinxcrossref{\sphinxcode{\sphinxupquote{ReplicasDict}}}}}) \textendash{} The {\hyperref[\detokenize{app.domain.helpers:app.domain.helpers.smart_dataclasses.FileBlockData}]{\sphinxcrossref{\sphinxcode{\sphinxupquote{FileBlockData}}}}}
replicas, without replication.

\item {} 
\sphinxstyleliteralstrong{\sphinxupquote{strat}} (\sphinxhref{https://docs.python.org/3.7/library/stdtypes.html\#str}{\sphinxstyleliteralemphasis{\sphinxupquote{str}}}) \textendash{} 
Defines how \sphinxcode{\sphinxupquote{replicas}} will be initially distributed in
the \sphinxcode{\sphinxupquote{Cluster}}.
\begin{description}
\item[{u}] \leavevmode
Each {\hyperref[\detokenize{app.domain.helpers:app.domain.helpers.smart_dataclasses.FileBlockData}]{\sphinxcrossref{\sphinxcode{\sphinxupquote{file block replica}}}}} in
\sphinxcode{\sphinxupquote{replicas}} is distributed following a
uniform probability vector among \sphinxcode{\sphinxupquote{members}} of
the cluster group.

\item[{a}] \leavevmode
Each {\hyperref[\detokenize{app.domain.helpers:app.domain.helpers.smart_dataclasses.FileBlockData}]{\sphinxcrossref{\sphinxcode{\sphinxupquote{file block replica}}}}}
in \sphinxcode{\sphinxupquote{replicas}} is given up to \sphinxcode{\sphinxupquote{N}} different
\sphinxcode{\sphinxupquote{members}} where \sphinxcode{\sphinxupquote{N}} is equal to
{\hyperref[\detokenize{app:app.environment_settings.REPLICATION_LEVEL}]{\sphinxcrossref{\sphinxcode{\sphinxupquote{REPLICATION\_LEVEL}}}}}.

\item[{i}] \leavevmode
Each {\hyperref[\detokenize{app.domain.helpers:app.domain.helpers.smart_dataclasses.FileBlockData}]{\sphinxcrossref{\sphinxcode{\sphinxupquote{file block replica}}}}}
in \sphinxcode{\sphinxupquote{replicas}} with bias towards the
ideal steady state distribution. This implementation of
differs from
{\hyperref[\detokenize{app.domain:app.domain.cluster_groups.Cluster.spread_files}]{\sphinxcrossref{\sphinxcode{\sphinxupquote{app.domain.cluster\_groups.Cluster.spread\_files()}}}}},
because it is not necessarely based on
{\hyperref[\detokenize{app.domain:app.domain.network_nodes.Node}]{\sphinxcrossref{\sphinxcode{\sphinxupquote{node}}}}} uptime.

\end{description}


\end{itemize}

\item[{Return type}] \leavevmode
\sphinxhref{https://docs.python.org/3.7/library/constants.html\#None}{None}

\end{description}\end{quote}

\end{fulllineitems}


\end{fulllineitems}

\index{SGClusterExt (class in app.domain.cluster\_groups)@\spxentry{SGClusterExt}\spxextra{class in app.domain.cluster\_groups}}

\begin{fulllineitems}
\phantomsection\label{\detokenize{app.domain:app.domain.cluster_groups.SGClusterExt}}\pysiglinewithargsret{\sphinxbfcode{\sphinxupquote{class }}\sphinxbfcode{\sphinxupquote{SGClusterExt}}}{\emph{\DUrole{n}{master}}, \emph{\DUrole{n}{file\_name}}, \emph{\DUrole{n}{members}}, \emph{\DUrole{n}{sim\_id}\DUrole{o}{=}\DUrole{default_value}{0}}, \emph{\DUrole{n}{origin}\DUrole{o}{=}\DUrole{default_value}{\textquotesingle{}\textquotesingle{}}}}{}
Bases: {\hyperref[\detokenize{app.domain:app.domain.cluster_groups.SGCluster}]{\sphinxcrossref{\sphinxcode{\sphinxupquote{app.domain.cluster\_groups.SGCluster}}}}}

Represents a group of network nodes persisting a file.

\sphinxcode{\sphinxupquote{SGClusterExt}} instances differ from
{\hyperref[\detokenize{app.domain:app.domain.cluster_groups.SGCluster}]{\sphinxcrossref{\sphinxcode{\sphinxupquote{SGCluster}}}}} because their members are
of type {\hyperref[\detokenize{app.domain:app.domain.network_nodes.SGNodeExt}]{\sphinxcrossref{\sphinxcode{\sphinxupquote{SGNodeExt}}}}}. When combined
these classes give nodes the responsibility of collaborating in the
detection of faulty members of the \sphinxcode{\sphinxupquote{SGClusterExt}} and eventually
kicking them out of the group.
\index{complaint\_threshold (SGClusterExt attribute)@\spxentry{complaint\_threshold}\spxextra{SGClusterExt attribute}}

\begin{fulllineitems}
\phantomsection\label{\detokenize{app.domain:app.domain.cluster_groups.SGClusterExt.complaint_threshold}}\pysigline{\sphinxbfcode{\sphinxupquote{complaint\_threshold}}}
Reference value that defines the maximum number of complaints a
{\hyperref[\detokenize{app.domain:app.domain.network_nodes.SGNodeExt}]{\sphinxcrossref{\sphinxcode{\sphinxupquote{network node}}}}}
can receive before it is evicted from the \sphinxcode{\sphinxupquote{SGClusterExt}}.
\begin{quote}\begin{description}
\item[{Type}] \leavevmode
\sphinxhref{https://docs.python.org/3.7/library/functions.html\#int}{int}

\end{description}\end{quote}

\end{fulllineitems}

\index{nodes\_complaints (SGClusterExt attribute)@\spxentry{nodes\_complaints}\spxextra{SGClusterExt attribute}}

\begin{fulllineitems}
\phantomsection\label{\detokenize{app.domain:app.domain.cluster_groups.SGClusterExt.nodes_complaints}}\pysigline{\sphinxbfcode{\sphinxupquote{nodes\_complaints}}}
A dictionary mapping {\hyperref[\detokenize{app.domain:app.domain.network_nodes.Node.id}]{\sphinxcrossref{\sphinxcode{\sphinxupquote{network node identifiers\textquotesingle{}}}}}} to the number of complaints
made against them by other {\hyperref[\detokenize{app.domain:app.domain.cluster_groups.Cluster.members}]{\sphinxcrossref{\sphinxcode{\sphinxupquote{members}}}}}. When
complaints becomes bigger than py:py:attr:\sphinxtitleref{complaint\_threshold}
the complaintee is evicted from the group.
\begin{quote}\begin{description}
\item[{Type}] \leavevmode
Dict{[}\sphinxhref{https://docs.python.org/3.7/library/stdtypes.html\#str}{str}, \sphinxhref{https://docs.python.org/3.7/library/functions.html\#int}{int}{]}

\end{description}\end{quote}

\end{fulllineitems}

\index{suspicious\_nodes (SGClusterExt attribute)@\spxentry{suspicious\_nodes}\spxextra{SGClusterExt attribute}}

\begin{fulllineitems}
\phantomsection\label{\detokenize{app.domain:app.domain.cluster_groups.SGClusterExt.suspicious_nodes}}\pysigline{\sphinxbfcode{\sphinxupquote{suspicious\_nodes}}}
A dictionary containing the unique {\hyperref[\detokenize{app.domain:app.domain.network_nodes.Node.id}]{\sphinxcrossref{\sphinxcode{\sphinxupquote{node identifiers}}}}} of known suspicious
members and how many epochs have passed since they changed to such
status.
\begin{quote}\begin{description}
\item[{Type}] \leavevmode
Dict{[}\sphinxhref{https://docs.python.org/3.7/library/stdtypes.html\#str}{str}, \sphinxhref{https://docs.python.org/3.7/library/functions.html\#int}{int}{]}

\end{description}\end{quote}

\end{fulllineitems}

\index{\_epoch\_complaints (SGClusterExt attribute)@\spxentry{\_epoch\_complaints}\spxextra{SGClusterExt attribute}}

\begin{fulllineitems}
\phantomsection\label{\detokenize{app.domain:app.domain.cluster_groups.SGClusterExt._epoch_complaints}}\pysigline{\sphinxbfcode{\sphinxupquote{\_epoch\_complaints}}}
A set of unique identifiers formed from the concatenation of
{\hyperref[\detokenize{app.domain:app.domain.network_nodes.Node.id}]{\sphinxcrossref{\sphinxcode{\sphinxupquote{node identifiers}}}}},
to avoid multiple complaint registrations on the same epoch,
done by the same source towards the same target. The set is
reset every epoch.
\begin{quote}\begin{description}
\item[{Type}] \leavevmode
\sphinxhref{https://docs.python.org/3.7/library/stdtypes.html\#set}{set}

\end{description}\end{quote}

\end{fulllineitems}

\index{\_\_init\_\_() (SGClusterExt method)@\spxentry{\_\_init\_\_()}\spxextra{SGClusterExt method}}

\begin{fulllineitems}
\phantomsection\label{\detokenize{app.domain:app.domain.cluster_groups.SGClusterExt.__init__}}\pysiglinewithargsret{\sphinxbfcode{\sphinxupquote{\_\_init\_\_}}}{\emph{\DUrole{n}{master}}, \emph{\DUrole{n}{file\_name}}, \emph{\DUrole{n}{members}}, \emph{\DUrole{n}{sim\_id}\DUrole{o}{=}\DUrole{default_value}{0}}, \emph{\DUrole{n}{origin}\DUrole{o}{=}\DUrole{default_value}{\textquotesingle{}\textquotesingle{}}}}{}
Instantiates an \sphinxcode{\sphinxupquote{Cluster}} object
\begin{quote}\begin{description}
\item[{Parameters}] \leavevmode\begin{itemize}
\item {} 
\sphinxstyleliteralstrong{\sphinxupquote{master}} ({\hyperref[\detokenize{app:app.type_hints.MasterType}]{\sphinxcrossref{\sphinxcode{\sphinxupquote{MasterType}}}}}) \textendash{} A reference to an {\hyperref[\detokenize{app.domain:app.domain.master_servers.Master}]{\sphinxcrossref{\sphinxcode{\sphinxupquote{Master}}}}}
object that manages the \sphinxcode{\sphinxupquote{Cluster}} being initialized.

\item {} 
\sphinxstyleliteralstrong{\sphinxupquote{file\_name}} (\sphinxhref{https://docs.python.org/3.7/library/stdtypes.html\#str}{\sphinxstyleliteralemphasis{\sphinxupquote{str}}}) \textendash{} The name of the file the \sphinxcode{\sphinxupquote{Cluster}} is responsible for
persisting.

\item {} 
\sphinxstyleliteralstrong{\sphinxupquote{members}} ({\hyperref[\detokenize{app:app.type_hints.NodeDict}]{\sphinxcrossref{\sphinxcode{\sphinxupquote{NodeDict}}}}}) \textendash{} A dictionary where keys are {\hyperref[\detokenize{app.domain:app.domain.network_nodes.Node.id}]{\sphinxcrossref{\sphinxcode{\sphinxupquote{node identifiers}}}}} and values are their
{\hyperref[\detokenize{app.domain:app.domain.network_nodes.Node}]{\sphinxcrossref{\sphinxcode{\sphinxupquote{instance objects}}}}}.

\item {} 
\sphinxstyleliteralstrong{\sphinxupquote{sim\_id}} (\sphinxhref{https://docs.python.org/3.7/library/functions.html\#int}{\sphinxstyleliteralemphasis{\sphinxupquote{int}}}) \textendash{} Identifier that generates unique output file names,
thus guaranteeing that different simulation instances do not
overwrite previous out files.

\item {} 
\sphinxstyleliteralstrong{\sphinxupquote{origin}} (\sphinxhref{https://docs.python.org/3.7/library/stdtypes.html\#str}{\sphinxstyleliteralemphasis{\sphinxupquote{str}}}) \textendash{} The name of the simulation file name that started
the simulation process.

\end{itemize}

\item[{Return type}] \leavevmode
\sphinxhref{https://docs.python.org/3.7/library/constants.html\#None}{None}

\end{description}\end{quote}

\end{fulllineitems}

\index{complain() (SGClusterExt method)@\spxentry{complain()}\spxextra{SGClusterExt method}}

\begin{fulllineitems}
\phantomsection\label{\detokenize{app.domain:app.domain.cluster_groups.SGClusterExt.complain}}\pysiglinewithargsret{\sphinxbfcode{\sphinxupquote{complain}}}{\emph{\DUrole{n}{complainter}}, \emph{\DUrole{n}{complainee}}, \emph{\DUrole{n}{reason}}}{}
Registers a complaint against a possibly offline node.

A unique identifier for the complaint is generated by concatenation
of the complainter and the complainee unique identifiers.
\begin{description}
\item[{Overrides:}] \leavevmode
{\hyperref[\detokenize{app.domain:app.domain.cluster_groups.Cluster.complain}]{\sphinxcrossref{\sphinxcode{\sphinxupquote{app.domain.cluster\_groups.Cluster.complain()}}}}}

\end{description}
\begin{quote}\begin{description}
\item[{Parameters}] \leavevmode\begin{itemize}
\item {} 
\sphinxstyleliteralstrong{\sphinxupquote{complainter}} (\sphinxhref{https://docs.python.org/3.7/library/stdtypes.html\#str}{\sphinxstyleliteralemphasis{\sphinxupquote{str}}}) \textendash{} The identifier of the complaining
{\hyperref[\detokenize{app.domain:app.domain.network_nodes.SGNodeExt}]{\sphinxcrossref{\sphinxcode{\sphinxupquote{SGNodeExt}}}}}.

\item {} 
\sphinxstyleliteralstrong{\sphinxupquote{complainee}} (\sphinxhref{https://docs.python.org/3.7/library/stdtypes.html\#str}{\sphinxstyleliteralemphasis{\sphinxupquote{str}}}) \textendash{} The identifier of the
{\hyperref[\detokenize{app.domain:app.domain.network_nodes.SGNodeExt}]{\sphinxcrossref{\sphinxcode{\sphinxupquote{SGNodeExt}}}}}
being complained about.

\item {} 
\sphinxstyleliteralstrong{\sphinxupquote{reason}} ({\hyperref[\detokenize{app:app.type_hints.HttpResponse}]{\sphinxcrossref{\sphinxcode{\sphinxupquote{app.type\_hints.HttpResponse}}}}}) \textendash{} The {\hyperref[\detokenize{app.domain.helpers:app.domain.helpers.enums.HttpCodes}]{\sphinxcrossref{\sphinxcode{\sphinxupquote{http code}}}}}
that led to the complaint.

\end{itemize}

\item[{Return type}] \leavevmode
\sphinxhref{https://docs.python.org/3.7/library/constants.html\#None}{None}

\end{description}\end{quote}

\end{fulllineitems}

\index{execute\_epoch() (SGClusterExt method)@\spxentry{execute\_epoch()}\spxextra{SGClusterExt method}}

\begin{fulllineitems}
\phantomsection\label{\detokenize{app.domain:app.domain.cluster_groups.SGClusterExt.execute_epoch}}\pysiglinewithargsret{\sphinxbfcode{\sphinxupquote{execute\_epoch}}}{\emph{\DUrole{n}{epoch}}}{}
Orders all \sphinxcode{\sphinxupquote{members}} to execute their epoch.

\begin{sphinxadmonition}{note}{Note:}
If the \sphinxcode{\sphinxupquote{Cluster}} terminates early, before it reaches
{\hyperref[\detokenize{app.domain:app.domain.master_servers.Master.MAX_EPOCHS}]{\sphinxcrossref{\sphinxcode{\sphinxupquote{MAX\_EPOCHS}}}}},
nothing should be logged in
{\hyperref[\detokenize{app.domain.helpers:app.domain.helpers.smart_dataclasses.LoggingData}]{\sphinxcrossref{\sphinxcode{\sphinxupquote{LoggingData}}}}}
at the specified \sphinxcode{\sphinxupquote{epoch}} to avoid skewing previously
collected results.
\end{sphinxadmonition}
\begin{quote}\begin{description}
\item[{Parameters}] \leavevmode
\sphinxstyleliteralstrong{\sphinxupquote{epoch}} (\sphinxhref{https://docs.python.org/3.7/library/functions.html\#int}{\sphinxstyleliteralemphasis{\sphinxupquote{int}}}) \textendash{} The epoch the \sphinxcode{\sphinxupquote{Cluster}} should currently be in, according
to it’s managing \sphinxcode{\sphinxupquote{master}} entity.

\item[{Returns}] \leavevmode
\sphinxcode{\sphinxupquote{False}} if \sphinxcode{\sphinxupquote{Cluster}} failed to persist the \sphinxcode{\sphinxupquote{file}} it
was responsible for, otherwise \sphinxcode{\sphinxupquote{True}}.

\item[{Return type}] \leavevmode
\sphinxhref{https://docs.python.org/3.7/library/constants.html\#None}{None}

\end{description}\end{quote}

\end{fulllineitems}

\index{maintain() (SGClusterExt method)@\spxentry{maintain()}\spxextra{SGClusterExt method}}

\begin{fulllineitems}
\phantomsection\label{\detokenize{app.domain:app.domain.cluster_groups.SGClusterExt.maintain}}\pysiglinewithargsret{\sphinxbfcode{\sphinxupquote{maintain}}}{\emph{\DUrole{n}{off\_nodes}}}{}
Evicts any {\hyperref[\detokenize{app.domain:app.domain.network_nodes.SGNodeExt}]{\sphinxcrossref{\sphinxcode{\sphinxupquote{network node}}}}} who has
been complained about more than {\hyperref[\detokenize{app.domain:app.domain.cluster_groups.SGClusterExt.complaint_threshold}]{\sphinxcrossref{\sphinxcode{\sphinxupquote{complaint\_threshold}}}}} times.
\begin{description}
\item[{Overrides:}] \leavevmode
{\hyperref[\detokenize{app.domain:app.domain.cluster_groups.Cluster.maintain}]{\sphinxcrossref{\sphinxcode{\sphinxupquote{app.domain.cluster\_groups.Cluster.maintain()}}}}}.

\end{description}
\begin{quote}\begin{description}
\item[{Parameters}] \leavevmode
\sphinxstyleliteralstrong{\sphinxupquote{off\_nodes}} (List{[}{\hyperref[\detokenize{app:app.type_hints.NodeType}]{\sphinxcrossref{\sphinxcode{\sphinxupquote{NodeType}}}}}{]}) \textendash{} The subset of {\hyperref[\detokenize{app.domain:app.domain.cluster_groups.Cluster.members}]{\sphinxcrossref{\sphinxcode{\sphinxupquote{members}}}}} who disconnected
during the current epoch.

\item[{Return type}] \leavevmode
\sphinxhref{https://docs.python.org/3.7/library/constants.html\#None}{None}

\end{description}\end{quote}

\end{fulllineitems}

\index{nodes\_execute() (SGClusterExt method)@\spxentry{nodes\_execute()}\spxextra{SGClusterExt method}}

\begin{fulllineitems}
\phantomsection\label{\detokenize{app.domain:app.domain.cluster_groups.SGClusterExt.nodes_execute}}\pysiglinewithargsret{\sphinxbfcode{\sphinxupquote{nodes\_execute}}}{}{}
Queries all network node members execute the epoch.
\begin{description}
\item[{Overrides:}] \leavevmode
{\hyperref[\detokenize{app.domain:app.domain.cluster_groups.SGCluster.nodes_execute}]{\sphinxcrossref{\sphinxcode{\sphinxupquote{app.domain.cluster\_groups.SGCluster.nodes\_execute()}}}}}.

Offline {\hyperref[\detokenize{app.domain:app.domain.network_nodes.SGNodeExt}]{\sphinxcrossref{\sphinxcode{\sphinxupquote{network nodes}}}}}
are considered suspects until enough complaints
from other \sphinxcode{\sphinxupquote{SGNodeExt}} {\hyperref[\detokenize{app.domain:app.domain.cluster_groups.Cluster.members}]{\sphinxcrossref{\sphinxcode{\sphinxupquote{members}}}}} are received.
This is important because lost parts can not be logged multiple
times. Yet suspected {\hyperref[\detokenize{app.domain:app.domain.network_nodes.SGNodeExt}]{\sphinxcrossref{\sphinxcode{\sphinxupquote{network nodes}}}}} need to be contabilized
as offline for simulation purposes without being evicted from the
group until they are detected by their peers as being offline.

\end{description}
\begin{quote}\begin{description}
\item[{Returns}] \leavevmode
A collection of {\hyperref[\detokenize{app.domain:app.domain.cluster_groups.Cluster.members}]{\sphinxcrossref{\sphinxcode{\sphinxupquote{members}}}}} who disconnected
during the current epoch.
See {\hyperref[\detokenize{app.domain:app.domain.network_nodes.SGNodeExt.update_status}]{\sphinxcrossref{\sphinxcode{\sphinxupquote{app.domain.network\_nodes.SGNodeExt.update\_status()}}}}}.

\item[{Return type}] \leavevmode
List{[}{\hyperref[\detokenize{app:app.type_hints.NodeType}]{\sphinxcrossref{\sphinxcode{\sphinxupquote{NodeType}}}}}{]}

\end{description}\end{quote}

\end{fulllineitems}


\end{fulllineitems}

\index{SGClusterPerfect (class in app.domain.cluster\_groups)@\spxentry{SGClusterPerfect}\spxextra{class in app.domain.cluster\_groups}}

\begin{fulllineitems}
\phantomsection\label{\detokenize{app.domain:app.domain.cluster_groups.SGClusterPerfect}}\pysiglinewithargsret{\sphinxbfcode{\sphinxupquote{class }}\sphinxbfcode{\sphinxupquote{SGClusterPerfect}}}{\emph{\DUrole{n}{master}}, \emph{\DUrole{n}{file\_name}}, \emph{\DUrole{n}{members}}, \emph{\DUrole{n}{sim\_id}\DUrole{o}{=}\DUrole{default_value}{0}}, \emph{\DUrole{n}{origin}\DUrole{o}{=}\DUrole{default_value}{\textquotesingle{}\textquotesingle{}}}}{}
Bases: {\hyperref[\detokenize{app.domain:app.domain.cluster_groups.SGCluster}]{\sphinxcrossref{\sphinxcode{\sphinxupquote{app.domain.cluster\_groups.SGCluster}}}}}

Represents a group of network nodes persisting a file using swarm
guidance algorithm.

This implementation assumes nodes never disconnect, there are no disk
errors and there is no link loss, i.e., it is used to study properties of
the system independently of computing environment.
\index{\_\_init\_\_() (SGClusterPerfect method)@\spxentry{\_\_init\_\_()}\spxextra{SGClusterPerfect method}}

\begin{fulllineitems}
\phantomsection\label{\detokenize{app.domain:app.domain.cluster_groups.SGClusterPerfect.__init__}}\pysiglinewithargsret{\sphinxbfcode{\sphinxupquote{\_\_init\_\_}}}{\emph{\DUrole{n}{master}}, \emph{\DUrole{n}{file\_name}}, \emph{\DUrole{n}{members}}, \emph{\DUrole{n}{sim\_id}\DUrole{o}{=}\DUrole{default_value}{0}}, \emph{\DUrole{n}{origin}\DUrole{o}{=}\DUrole{default_value}{\textquotesingle{}\textquotesingle{}}}}{}
Instantiates an \sphinxcode{\sphinxupquote{Cluster}} object
\begin{quote}\begin{description}
\item[{Parameters}] \leavevmode\begin{itemize}
\item {} 
\sphinxstyleliteralstrong{\sphinxupquote{master}} ({\hyperref[\detokenize{app:app.type_hints.MasterType}]{\sphinxcrossref{\sphinxcode{\sphinxupquote{MasterType}}}}}) \textendash{} A reference to an {\hyperref[\detokenize{app.domain:app.domain.master_servers.Master}]{\sphinxcrossref{\sphinxcode{\sphinxupquote{Master}}}}}
object that manages the \sphinxcode{\sphinxupquote{Cluster}} being initialized.

\item {} 
\sphinxstyleliteralstrong{\sphinxupquote{file\_name}} (\sphinxhref{https://docs.python.org/3.7/library/stdtypes.html\#str}{\sphinxstyleliteralemphasis{\sphinxupquote{str}}}) \textendash{} The name of the file the \sphinxcode{\sphinxupquote{Cluster}} is responsible for
persisting.

\item {} 
\sphinxstyleliteralstrong{\sphinxupquote{members}} ({\hyperref[\detokenize{app:app.type_hints.NodeDict}]{\sphinxcrossref{\sphinxcode{\sphinxupquote{NodeDict}}}}}) \textendash{} A dictionary where keys are {\hyperref[\detokenize{app.domain:app.domain.network_nodes.Node.id}]{\sphinxcrossref{\sphinxcode{\sphinxupquote{node identifiers}}}}} and values are their
{\hyperref[\detokenize{app.domain:app.domain.network_nodes.Node}]{\sphinxcrossref{\sphinxcode{\sphinxupquote{instance objects}}}}}.

\item {} 
\sphinxstyleliteralstrong{\sphinxupquote{sim\_id}} (\sphinxhref{https://docs.python.org/3.7/library/functions.html\#int}{\sphinxstyleliteralemphasis{\sphinxupquote{int}}}) \textendash{} Identifier that generates unique output file names,
thus guaranteeing that different simulation instances do not
overwrite previous out files.

\item {} 
\sphinxstyleliteralstrong{\sphinxupquote{origin}} (\sphinxhref{https://docs.python.org/3.7/library/stdtypes.html\#str}{\sphinxstyleliteralemphasis{\sphinxupquote{str}}}) \textendash{} The name of the simulation file name that started
the simulation process.

\end{itemize}

\item[{Return type}] \leavevmode
\sphinxhref{https://docs.python.org/3.7/library/constants.html\#None}{None}

\end{description}\end{quote}

\end{fulllineitems}

\index{execute\_epoch() (SGClusterPerfect method)@\spxentry{execute\_epoch()}\spxextra{SGClusterPerfect method}}

\begin{fulllineitems}
\phantomsection\label{\detokenize{app.domain:app.domain.cluster_groups.SGClusterPerfect.execute_epoch}}\pysiglinewithargsret{\sphinxbfcode{\sphinxupquote{execute\_epoch}}}{\emph{\DUrole{n}{epoch}}}{}
Orders all \sphinxcode{\sphinxupquote{members}} to execute their epoch.

\begin{sphinxadmonition}{note}{Note:}
If the \sphinxcode{\sphinxupquote{Cluster}} terminates early, before it reaches
{\hyperref[\detokenize{app.domain:app.domain.master_servers.Master.MAX_EPOCHS}]{\sphinxcrossref{\sphinxcode{\sphinxupquote{MAX\_EPOCHS}}}}},
nothing should be logged in
{\hyperref[\detokenize{app.domain.helpers:app.domain.helpers.smart_dataclasses.LoggingData}]{\sphinxcrossref{\sphinxcode{\sphinxupquote{LoggingData}}}}}
at the specified \sphinxcode{\sphinxupquote{epoch}} to avoid skewing previously
collected results.
\end{sphinxadmonition}
\begin{quote}\begin{description}
\item[{Parameters}] \leavevmode
\sphinxstyleliteralstrong{\sphinxupquote{epoch}} (\sphinxhref{https://docs.python.org/3.7/library/functions.html\#int}{\sphinxstyleliteralemphasis{\sphinxupquote{int}}}) \textendash{} The epoch the \sphinxcode{\sphinxupquote{Cluster}} should currently be in, according
to it’s managing \sphinxcode{\sphinxupquote{master}} entity.

\item[{Returns}] \leavevmode
\sphinxcode{\sphinxupquote{False}} if \sphinxcode{\sphinxupquote{Cluster}} failed to persist the \sphinxcode{\sphinxupquote{file}} it
was responsible for, otherwise \sphinxcode{\sphinxupquote{True}}.

\item[{Return type}] \leavevmode
\sphinxhref{https://docs.python.org/3.7/library/constants.html\#None}{None}

\end{description}\end{quote}

\end{fulllineitems}

\index{new\_transition\_matrix() (SGClusterPerfect method)@\spxentry{new\_transition\_matrix()}\spxextra{SGClusterPerfect method}}

\begin{fulllineitems}
\phantomsection\label{\detokenize{app.domain:app.domain.cluster_groups.SGClusterPerfect.new_transition_matrix}}\pysiglinewithargsret{\sphinxbfcode{\sphinxupquote{new\_transition\_matrix}}}{}{}
Creates a new transition matrix that is likely to be a Markov Matrix.
\begin{quote}\begin{description}
\item[{Returns}] \leavevmode
The labeled matrix that has the fastests mixing rate from all
the pondered strategies.

\item[{Return type}] \leavevmode
\sphinxhref{https://pandas.pydata.org/docs/reference/api/pandas.DataFrame.html\#pandas.DataFrame}{\sphinxcode{\sphinxupquote{DataFrame}}}

\end{description}\end{quote}

\end{fulllineitems}

\index{nodes\_execute() (SGClusterPerfect method)@\spxentry{nodes\_execute()}\spxextra{SGClusterPerfect method}}

\begin{fulllineitems}
\phantomsection\label{\detokenize{app.domain:app.domain.cluster_groups.SGClusterPerfect.nodes_execute}}\pysiglinewithargsret{\sphinxbfcode{\sphinxupquote{nodes\_execute}}}{}{}
Queries all network node members execute the epoch.
\begin{description}
\item[{Overrides:}] \leavevmode
{\hyperref[\detokenize{app.domain:app.domain.cluster_groups.Cluster.nodes_execute}]{\sphinxcrossref{\sphinxcode{\sphinxupquote{app.domain.cluster\_groups.Cluster.nodes\_execute()}}}}}.

\end{description}
\begin{quote}\begin{description}
\item[{Returns}] \leavevmode
A collection of members who disconnected during the current
epoch. See
{\hyperref[\detokenize{app.domain:app.domain.network_nodes.Node.update_status}]{\sphinxcrossref{\sphinxcode{\sphinxupquote{app.domain.network\_nodes.Node.update\_status()}}}}}.

\item[{Return type}] \leavevmode
List{[}{\hyperref[\detokenize{app:app.type_hints.NodeType}]{\sphinxcrossref{\sphinxcode{\sphinxupquote{NodeType}}}}}{]}

\end{description}\end{quote}

\end{fulllineitems}

\index{select\_fastest\_topology() (SGClusterPerfect method)@\spxentry{select\_fastest\_topology()}\spxextra{SGClusterPerfect method}}

\begin{fulllineitems}
\phantomsection\label{\detokenize{app.domain:app.domain.cluster_groups.SGClusterPerfect.select_fastest_topology}}\pysiglinewithargsret{\sphinxbfcode{\sphinxupquote{select\_fastest\_topology}}}{\emph{\DUrole{n}{a}}, \emph{\DUrole{n}{v\_}}}{}
Creates multiple transition matrices and selects the fastest.

The fastest of the created transition matrices corresponds to the one
with a faster mixing rate.
\begin{quote}\begin{description}
\item[{Parameters}] \leavevmode\begin{itemize}
\item {} 
\sphinxstyleliteralstrong{\sphinxupquote{a}} (\sphinxhref{https://numpy.org/doc/stable/reference/generated/numpy.ndarray.html\#numpy.ndarray}{\sphinxcode{\sphinxupquote{ndarray}}}) \textendash{} An adjacency matrix that represents the network topology.

\item {} 
\sphinxstyleliteralstrong{\sphinxupquote{v\_}} (\sphinxhref{https://numpy.org/doc/stable/reference/generated/numpy.ndarray.html\#numpy.ndarray}{\sphinxcode{\sphinxupquote{ndarray}}}) \textendash{} A desired distribution vector that defines the returned
matrix steady state property.

\end{itemize}

\item[{Returns}] \leavevmode
A transition matrix that is likely to be a markov matrix whose
steady state is \sphinxcode{\sphinxupquote{v\_}}, but is not yet validated. See
\sphinxcode{\sphinxupquote{\_validate\_transition\_matrix()}}.

\item[{Return type}] \leavevmode
\sphinxhref{https://numpy.org/doc/stable/reference/generated/numpy.ndarray.html\#numpy.ndarray}{\sphinxcode{\sphinxupquote{ndarray}}}

\end{description}\end{quote}

\end{fulllineitems}


\end{fulllineitems}



\subsubsection{app.domain.master\_servers}
\label{\detokenize{app.domain:module-app.domain.master_servers}}\label{\detokenize{app.domain:app-domain-master-servers}}\index{module@\spxentry{module}!app.domain.master\_servers@\spxentry{app.domain.master\_servers}}\index{app.domain.master\_servers@\spxentry{app.domain.master\_servers}!module@\spxentry{module}}
This module contains domain specific classes that coordinate all
{\hyperref[\detokenize{app.domain:module-app.domain.cluster_groups}]{\sphinxcrossref{\sphinxcode{\sphinxupquote{app.domain.cluster\_groups}}}}} of a simulation instance. These could
simulate centralized authentication servers, file localization or
file metadata servers or a bank of currently online and offline
{\hyperref[\detokenize{app.domain:module-app.domain.network_nodes}]{\sphinxcrossref{\sphinxcode{\sphinxupquote{storage nodes}}}}}.
\index{HDFSMaster (class in app.domain.master\_servers)@\spxentry{HDFSMaster}\spxextra{class in app.domain.master\_servers}}

\begin{fulllineitems}
\phantomsection\label{\detokenize{app.domain:app.domain.master_servers.HDFSMaster}}\pysiglinewithargsret{\sphinxbfcode{\sphinxupquote{class }}\sphinxbfcode{\sphinxupquote{HDFSMaster}}}{\emph{\DUrole{n}{simfile\_name}}, \emph{\DUrole{n}{sid}}, \emph{\DUrole{n}{epochs}}, \emph{\DUrole{n}{cluster\_class}}, \emph{\DUrole{n}{node\_class}}}{}
Bases: {\hyperref[\detokenize{app.domain:app.domain.master_servers.Master}]{\sphinxcrossref{\sphinxcode{\sphinxupquote{app.domain.master\_servers.Master}}}}}
\index{\_process\_simfile() (HDFSMaster method)@\spxentry{\_process\_simfile()}\spxextra{HDFSMaster method}}

\begin{fulllineitems}
\phantomsection\label{\detokenize{app.domain:app.domain.master_servers.HDFSMaster._process_simfile}}\pysiglinewithargsret{\sphinxbfcode{\sphinxupquote{\_process\_simfile}}}{\emph{\DUrole{n}{path}}, \emph{\DUrole{n}{cluster\_class}}, \emph{\DUrole{n}{node\_class}}}{}
Opens and processes the simulation filed referenced in \sphinxtitleref{path}.
\begin{description}
\item[{Overrides:}] \leavevmode
{\hyperref[\detokenize{app.domain:app.domain.master_servers.Master._process_simfile}]{\sphinxcrossref{\sphinxcode{\sphinxupquote{app.domain.master\_servers.Master.\_process\_simfile()}}}}}.

The method is exactly the same except for one instruction. The
{\hyperref[\detokenize{app.domain:app.domain.master_servers.Master._split_files}]{\sphinxcrossref{\sphinxcode{\sphinxupquote{\_split\_files()}}}}} is
invoked with fixed \sphinxtitleref{bsize} = 1MB. The reason for this is
two\sphinxhyphen{}fold:
\begin{itemize}
\item {} 
The default and, thus recommended, block size for the                 hadoop distributed file system is 128MB. The system is not                 designed to perform well with small file blocks, but SG                 requires many file blocks to  work, hence being more                 effective with small block sizes.

\item {} 
Hadoop limits the minimum block size to be 1MB,                 \sphinxhref{https://hadoop.apache.org/docs/r2.6.0/hadoop-project-dist/hadoop-hdfs/hdfs-default.xml\#dfs.namenode.fs-limits.min-block-size}{dfs.namenode.fs\sphinxhyphen{}limits.min\sphinxhyphen{}block\sphinxhyphen{}size}.                 For this reason, we make HDFSMaster split files into 1MB                 chunks, as that is the closest we would get to our Hive’s                 default block size in the real world.

\end{itemize}

The other difference is that the spread strategy is ignored.
We are not interested in knowing if the way the files are
initially spread affects the time it takes for clusters to
achieve a steady\sphinxhyphen{}state distribution since in HDFS
{\hyperref[\detokenize{app.domain.helpers:app.domain.helpers.smart_dataclasses.FileBlockData}]{\sphinxcrossref{\sphinxcode{\sphinxupquote{file block replicas}}}}} are
stationary on data nodes until they die.

\end{description}
\begin{quote}\begin{description}
\item[{Parameters}] \leavevmode\begin{itemize}
\item {} 
\sphinxstyleliteralstrong{\sphinxupquote{path}} (\sphinxhref{https://docs.python.org/3.7/library/stdtypes.html\#str}{\sphinxstyleliteralemphasis{\sphinxupquote{str}}}) \textendash{} The path to the simulation file. Including extension and
parent folders.

\item {} 
\sphinxstyleliteralstrong{\sphinxupquote{cluster\_class}} (\sphinxhref{https://docs.python.org/3.7/library/stdtypes.html\#str}{\sphinxstyleliteralemphasis{\sphinxupquote{str}}}) \textendash{} The name of the class used to instantiate cluster group
instances through reflection.
See {\hyperref[\detokenize{app.domain:module-app.domain.cluster_groups}]{\sphinxcrossref{\sphinxcode{\sphinxupquote{app.domain.cluster\_groups}}}}}.

\item {} 
\sphinxstyleliteralstrong{\sphinxupquote{node\_class}} (\sphinxhref{https://docs.python.org/3.7/library/stdtypes.html\#str}{\sphinxstyleliteralemphasis{\sphinxupquote{str}}}) \textendash{} The name of the class used to instantiate network node
instances through reflection.
See {\hyperref[\detokenize{app.domain:module-app.domain.network_nodes}]{\sphinxcrossref{\sphinxcode{\sphinxupquote{app.domain.network\_nodes}}}}}.

\end{itemize}

\item[{Return type}] \leavevmode
\sphinxhref{https://docs.python.org/3.7/library/constants.html\#None}{None}

\end{description}\end{quote}

\end{fulllineitems}


\end{fulllineitems}

\index{Master (class in app.domain.master\_servers)@\spxentry{Master}\spxextra{class in app.domain.master\_servers}}

\begin{fulllineitems}
\phantomsection\label{\detokenize{app.domain:app.domain.master_servers.Master}}\pysiglinewithargsret{\sphinxbfcode{\sphinxupquote{class }}\sphinxbfcode{\sphinxupquote{Master}}}{\emph{\DUrole{n}{simfile\_name}}, \emph{\DUrole{n}{sid}}, \emph{\DUrole{n}{epochs}}, \emph{\DUrole{n}{cluster\_class}}, \emph{\DUrole{n}{node\_class}}}{}
Bases: \sphinxhref{https://docs.python.org/3.7/library/functions.html\#object}{\sphinxcode{\sphinxupquote{object}}}

Simulation manager class, some kind of puppet\sphinxhyphen{}master. Could represent
an authentication server or a monitor that decides along with other
\sphinxcode{\sphinxupquote{Master}} entities what {\hyperref[\detokenize{app.domain:app.domain.network_nodes.Node}]{\sphinxcrossref{\sphinxcode{\sphinxupquote{network nodes}}}}} are online using consensus algorithms.
\index{origin (Master attribute)@\spxentry{origin}\spxextra{Master attribute}}

\begin{fulllineitems}
\phantomsection\label{\detokenize{app.domain:app.domain.master_servers.Master.origin}}\pysigline{\sphinxbfcode{\sphinxupquote{origin}}}
The name of the simulation file name that started the simulation
process.
\begin{quote}\begin{description}
\item[{Type}] \leavevmode
\sphinxhref{https://docs.python.org/3.7/library/stdtypes.html\#str}{str}

\end{description}\end{quote}

\end{fulllineitems}

\index{sid (Master attribute)@\spxentry{sid}\spxextra{Master attribute}}

\begin{fulllineitems}
\phantomsection\label{\detokenize{app.domain:app.domain.master_servers.Master.sid}}\pysigline{\sphinxbfcode{\sphinxupquote{sid}}}
Identifier that generates unique output file names,
thus guaranteeing that different simulation instances do not
overwrite previous out files.
\begin{quote}\begin{description}
\item[{Type}] \leavevmode
\sphinxhref{https://docs.python.org/3.7/library/functions.html\#int}{int}

\end{description}\end{quote}

\end{fulllineitems}

\index{epoch (Master attribute)@\spxentry{epoch}\spxextra{Master attribute}}

\begin{fulllineitems}
\phantomsection\label{\detokenize{app.domain:app.domain.master_servers.Master.epoch}}\pysigline{\sphinxbfcode{\sphinxupquote{epoch}}}
The simulation’s current epoch.
\begin{quote}\begin{description}
\item[{Type}] \leavevmode
\sphinxhref{https://docs.python.org/3.7/library/functions.html\#int}{int}

\end{description}\end{quote}

\end{fulllineitems}

\index{cluster\_groups (Master attribute)@\spxentry{cluster\_groups}\spxextra{Master attribute}}

\begin{fulllineitems}
\phantomsection\label{\detokenize{app.domain:app.domain.master_servers.Master.cluster_groups}}\pysigline{\sphinxbfcode{\sphinxupquote{cluster\_groups}}}
A collection of {\hyperref[\detokenize{app.domain:app.domain.cluster_groups.Cluster}]{\sphinxcrossref{\sphinxcode{\sphinxupquote{cluster groups}}}}} managed by the \sphinxcode{\sphinxupquote{Master}}.
Keys are {\hyperref[\detokenize{app.domain:app.domain.cluster_groups.Cluster.id}]{\sphinxcrossref{\sphinxcode{\sphinxupquote{cluster identifiers}}}}} and values are the
cluster instances.
\begin{quote}\begin{description}
\item[{Type}] \leavevmode
{\hyperref[\detokenize{app:app.type_hints.ClusterDict}]{\sphinxcrossref{\sphinxcode{\sphinxupquote{app.type\_hints.ClusterDict}}}}}

\end{description}\end{quote}

\end{fulllineitems}

\index{network\_nodes (Master attribute)@\spxentry{network\_nodes}\spxextra{Master attribute}}

\begin{fulllineitems}
\phantomsection\label{\detokenize{app.domain:app.domain.master_servers.Master.network_nodes}}\pysigline{\sphinxbfcode{\sphinxupquote{network\_nodes}}}
A dictionary mapping {\hyperref[\detokenize{app.domain:app.domain.network_nodes.Node.id}]{\sphinxcrossref{\sphinxcode{\sphinxupquote{node identifiers}}}}} to their instance objects.
This collection differs from
{\hyperref[\detokenize{app.domain:app.domain.cluster_groups.Cluster.members}]{\sphinxcrossref{\sphinxcode{\sphinxupquote{app.domain.cluster\_groups.Cluster.members}}}}} attribute
in the sense that the former \sphinxcode{\sphinxupquote{network\_nodes}} includes all
nodes, both online and offline, available on the entire
distributed backup storage system regardless of their
participation in any {\hyperref[\detokenize{app.domain:app.domain.cluster_groups.Cluster}]{\sphinxcrossref{\sphinxcode{\sphinxupquote{cluster group}}}}}.
\begin{quote}\begin{description}
\item[{Type}] \leavevmode
{\hyperref[\detokenize{app:app.type_hints.NodeDict}]{\sphinxcrossref{\sphinxcode{\sphinxupquote{app.type\_hints.NodeDict}}}}}

\end{description}\end{quote}

\end{fulllineitems}

\index{\_\_init\_\_() (Master method)@\spxentry{\_\_init\_\_()}\spxextra{Master method}}

\begin{fulllineitems}
\phantomsection\label{\detokenize{app.domain:app.domain.master_servers.Master.__init__}}\pysiglinewithargsret{\sphinxbfcode{\sphinxupquote{\_\_init\_\_}}}{\emph{\DUrole{n}{simfile\_name}}, \emph{\DUrole{n}{sid}}, \emph{\DUrole{n}{epochs}}, \emph{\DUrole{n}{cluster\_class}}, \emph{\DUrole{n}{node\_class}}}{}
Instantiates an Master object.
\begin{quote}\begin{description}
\item[{Parameters}] \leavevmode\begin{itemize}
\item {} 
\sphinxstyleliteralstrong{\sphinxupquote{simfile\_name}} (\sphinxhref{https://docs.python.org/3.7/library/stdtypes.html\#str}{\sphinxstyleliteralemphasis{\sphinxupquote{str}}}) \textendash{} A path to the simulation file to be run by the simulator.

\item {} 
\sphinxstyleliteralstrong{\sphinxupquote{sid}} (\sphinxhref{https://docs.python.org/3.7/library/functions.html\#int}{\sphinxstyleliteralemphasis{\sphinxupquote{int}}}) \textendash{} Identifier that generates unique output file names,
thus guaranteeing that different simulation instances do not
overwrite previous out files.

\item {} 
\sphinxstyleliteralstrong{\sphinxupquote{epochs}} (\sphinxhref{https://docs.python.org/3.7/library/functions.html\#int}{\sphinxstyleliteralemphasis{\sphinxupquote{int}}}) \textendash{} The number of discrete time steps the simulation lasts.

\item {} 
\sphinxstyleliteralstrong{\sphinxupquote{cluster\_class}} (\sphinxhref{https://docs.python.org/3.7/library/stdtypes.html\#str}{\sphinxstyleliteralemphasis{\sphinxupquote{str}}}) \textendash{} The name of the class used to instantiate cluster group
instances through reflection. See {\hyperref[\detokenize{app.domain:module-app.domain.cluster_groups}]{\sphinxcrossref{\sphinxcode{\sphinxupquote{cluster groups module}}}}}.

\item {} 
\sphinxstyleliteralstrong{\sphinxupquote{node\_class}} (\sphinxhref{https://docs.python.org/3.7/library/stdtypes.html\#str}{\sphinxstyleliteralemphasis{\sphinxupquote{str}}}) \textendash{} The name of the class used to instantiate network node
instances through reflection. See {\hyperref[\detokenize{app.domain:module-app.domain.network_nodes}]{\sphinxcrossref{\sphinxcode{\sphinxupquote{network nodes module}}}}}.

\end{itemize}

\item[{Return type}] \leavevmode
\sphinxhref{https://docs.python.org/3.7/library/constants.html\#None}{None}

\end{description}\end{quote}

\end{fulllineitems}

\index{\_create\_network\_nodes() (Master method)@\spxentry{\_create\_network\_nodes()}\spxextra{Master method}}

\begin{fulllineitems}
\phantomsection\label{\detokenize{app.domain:app.domain.master_servers.Master._create_network_nodes}}\pysiglinewithargsret{\sphinxbfcode{\sphinxupquote{\_create\_network\_nodes}}}{\emph{\DUrole{n}{json}}, \emph{\DUrole{n}{node\_class}}}{}
Helper method that instantiates all
{\hyperref[\detokenize{app.domain:app.domain.network_nodes.Node}]{\sphinxcrossref{\sphinxcode{\sphinxupquote{network nodes}}}}} that are
specified in the simulation file.
\begin{quote}\begin{description}
\item[{Parameters}] \leavevmode\begin{itemize}
\item {} 
\sphinxstyleliteralstrong{\sphinxupquote{json}} (\sphinxstyleliteralemphasis{\sphinxupquote{Dict}}\sphinxstyleliteralemphasis{\sphinxupquote{{[}}}\sphinxhref{https://docs.python.org/3.7/library/stdtypes.html\#str}{\sphinxstyleliteralemphasis{\sphinxupquote{str}}}\sphinxstyleliteralemphasis{\sphinxupquote{, }}\sphinxstyleliteralemphasis{\sphinxupquote{Any}}\sphinxstyleliteralemphasis{\sphinxupquote{{]}}}) \textendash{} The simulation file in JSON dictionary object format.

\item {} 
\sphinxstyleliteralstrong{\sphinxupquote{node\_class}} (\sphinxhref{https://docs.python.org/3.7/library/stdtypes.html\#str}{\sphinxstyleliteralemphasis{\sphinxupquote{str}}}) \textendash{} The type of network node to create.

\end{itemize}

\item[{Return type}] \leavevmode
\sphinxhref{https://docs.python.org/3.7/library/constants.html\#None}{None}

\end{description}\end{quote}

\end{fulllineitems}

\index{\_new\_cluster\_group() (Master method)@\spxentry{\_new\_cluster\_group()}\spxextra{Master method}}

\begin{fulllineitems}
\phantomsection\label{\detokenize{app.domain:app.domain.master_servers.Master._new_cluster_group}}\pysiglinewithargsret{\sphinxbfcode{\sphinxupquote{\_new\_cluster\_group}}}{\emph{\DUrole{n}{cluster\_class}}, \emph{\DUrole{n}{size}}, \emph{\DUrole{n}{fname}}}{}
Helper method that initializes a new Cluster group.
\begin{quote}\begin{description}
\item[{Parameters}] \leavevmode\begin{itemize}
\item {} 
\sphinxstyleliteralstrong{\sphinxupquote{cluster\_class}} (\sphinxhref{https://docs.python.org/3.7/library/stdtypes.html\#str}{\sphinxstyleliteralemphasis{\sphinxupquote{str}}}) \textendash{} The name of the class used to instantiate cluster group
instances through reflection. See {\hyperref[\detokenize{app.domain:module-app.domain.cluster_groups}]{\sphinxcrossref{\sphinxcode{\sphinxupquote{cluster groups module}}}}}.

\item {} 
\sphinxstyleliteralstrong{\sphinxupquote{size}} (\sphinxhref{https://docs.python.org/3.7/library/functions.html\#int}{\sphinxstyleliteralemphasis{\sphinxupquote{int}}}) \textendash{} The {\hyperref[\detokenize{app.domain:app.domain.cluster_groups.Cluster}]{\sphinxcrossref{\sphinxcode{\sphinxupquote{cluster\textquotesingle{}s}}}}}
initial memberhip size.

\item {} 
\sphinxstyleliteralstrong{\sphinxupquote{fname}} (\sphinxhref{https://docs.python.org/3.7/library/stdtypes.html\#str}{\sphinxstyleliteralemphasis{\sphinxupquote{str}}}) \textendash{} The name of the fille being stored in the cluster.

\end{itemize}

\item[{Returns}] \leavevmode
The {\hyperref[\detokenize{app.domain:app.domain.cluster_groups.Cluster}]{\sphinxcrossref{\sphinxcode{\sphinxupquote{Cluster}}}}} instance.

\item[{Return type}] \leavevmode
{\hyperref[\detokenize{app:app.type_hints.ClusterType}]{\sphinxcrossref{\sphinxcode{\sphinxupquote{ClusterType}}}}}

\end{description}\end{quote}

\end{fulllineitems}

\index{\_new\_network\_node() (Master method)@\spxentry{\_new\_network\_node()}\spxextra{Master method}}

\begin{fulllineitems}
\phantomsection\label{\detokenize{app.domain:app.domain.master_servers.Master._new_network_node}}\pysiglinewithargsret{\sphinxbfcode{\sphinxupquote{\_new\_network\_node}}}{\emph{\DUrole{n}{node\_class}}, \emph{\DUrole{n}{nid}}, \emph{\DUrole{n}{node\_uptime}}}{}
Helper method that initializes a new Node.
\begin{quote}\begin{description}
\item[{Parameters}] \leavevmode\begin{itemize}
\item {} 
\sphinxstyleliteralstrong{\sphinxupquote{node\_class}} (\sphinxhref{https://docs.python.org/3.7/library/stdtypes.html\#str}{\sphinxstyleliteralemphasis{\sphinxupquote{str}}}) \textendash{} The name of the class used to instantiate network node
instances through reflection. See {\hyperref[\detokenize{app.domain:module-app.domain.network_nodes}]{\sphinxcrossref{\sphinxcode{\sphinxupquote{network nodes module}}}}}.

\item {} 
\sphinxstyleliteralstrong{\sphinxupquote{nid}} (\sphinxhref{https://docs.python.org/3.7/library/stdtypes.html\#str}{\sphinxstyleliteralemphasis{\sphinxupquote{str}}}) \textendash{} An id that will uniquely identifies the
{\hyperref[\detokenize{app.domain:app.domain.network_nodes.Node}]{\sphinxcrossref{\sphinxcode{\sphinxupquote{network node}}}}}.

\item {} 
\sphinxstyleliteralstrong{\sphinxupquote{node\_uptime}} (\sphinxhref{https://docs.python.org/3.7/library/stdtypes.html\#str}{\sphinxstyleliteralemphasis{\sphinxupquote{str}}}) \textendash{} A float value in string representation that defines the
uptime of the network node.

\end{itemize}

\item[{Returns}] \leavevmode
The {\hyperref[\detokenize{app.domain:app.domain.network_nodes.Node}]{\sphinxcrossref{\sphinxcode{\sphinxupquote{Node}}}}} instance.

\item[{Return type}] \leavevmode
{\hyperref[\detokenize{app:app.type_hints.NodeType}]{\sphinxcrossref{\sphinxcode{\sphinxupquote{NodeType}}}}}

\end{description}\end{quote}

\end{fulllineitems}

\index{\_process\_simfile() (Master method)@\spxentry{\_process\_simfile()}\spxextra{Master method}}

\begin{fulllineitems}
\phantomsection\label{\detokenize{app.domain:app.domain.master_servers.Master._process_simfile}}\pysiglinewithargsret{\sphinxbfcode{\sphinxupquote{\_process\_simfile}}}{\emph{\DUrole{n}{path}}, \emph{\DUrole{n}{cluster\_class}}, \emph{\DUrole{n}{node\_class}}}{}
Opens and processes the simulation filed referenced in \sphinxcode{\sphinxupquote{path}}.

This method opens the file reads the json data inside it. Combined
with {\hyperref[\detokenize{app:module-app.environment_settings}]{\sphinxcrossref{\sphinxcode{\sphinxupquote{app.environment\_settings}}}}} it sets up the class
instances to be used during the simulation (e.g.,
{\hyperref[\detokenize{app.domain:app.domain.cluster_groups.Cluster}]{\sphinxcrossref{\sphinxcode{\sphinxupquote{cluster groups}}}}} and
{\hyperref[\detokenize{app.domain:app.domain.network_nodes.Node}]{\sphinxcrossref{\sphinxcode{\sphinxupquote{network nodes}}}}}). This
method also be splits the file to be persisted in the simulation into
multiple \sphinxcode{\sphinxupquote{blocks}} or \sphinxcode{\sphinxupquote{chunks}} and for triggering the initial
{\hyperref[\detokenize{app.domain:app.domain.cluster_groups.Cluster.spread_files}]{\sphinxcrossref{\sphinxcode{\sphinxupquote{file spreading}}}}} mechanism.
\begin{quote}\begin{description}
\item[{Parameters}] \leavevmode\begin{itemize}
\item {} 
\sphinxstyleliteralstrong{\sphinxupquote{path}} (\sphinxhref{https://docs.python.org/3.7/library/stdtypes.html\#str}{\sphinxstyleliteralemphasis{\sphinxupquote{str}}}) \textendash{} The path to the simulation file. Including extension and
parent folders.

\item {} 
\sphinxstyleliteralstrong{\sphinxupquote{cluster\_class}} (\sphinxhref{https://docs.python.org/3.7/library/stdtypes.html\#str}{\sphinxstyleliteralemphasis{\sphinxupquote{str}}}) \textendash{} The name of the class used to instantiate cluster group
instances through reflection.
See {\hyperref[\detokenize{app.domain:module-app.domain.cluster_groups}]{\sphinxcrossref{\sphinxcode{\sphinxupquote{app.domain.cluster\_groups}}}}}.

\item {} 
\sphinxstyleliteralstrong{\sphinxupquote{node\_class}} (\sphinxhref{https://docs.python.org/3.7/library/stdtypes.html\#str}{\sphinxstyleliteralemphasis{\sphinxupquote{str}}}) \textendash{} The name of the class used to instantiate network node
instances through reflection.
See {\hyperref[\detokenize{app.domain:module-app.domain.network_nodes}]{\sphinxcrossref{\sphinxcode{\sphinxupquote{app.domain.network\_nodes}}}}}.

\end{itemize}

\item[{Return type}] \leavevmode
\sphinxhref{https://docs.python.org/3.7/library/constants.html\#None}{None}

\end{description}\end{quote}

\end{fulllineitems}

\index{\_split\_files() (Master method)@\spxentry{\_split\_files()}\spxextra{Master method}}

\begin{fulllineitems}
\phantomsection\label{\detokenize{app.domain:app.domain.master_servers.Master._split_files}}\pysiglinewithargsret{\sphinxbfcode{\sphinxupquote{\_split\_files}}}{\emph{\DUrole{n}{fname}}, \emph{\DUrole{n}{cluster}}, \emph{\DUrole{n}{bsize}}}{}
Helper method that splits the files into multiple blocks to be
persisted in a {\hyperref[\detokenize{app.domain:app.domain.cluster_groups.Cluster}]{\sphinxcrossref{\sphinxcode{\sphinxupquote{cluster group}}}}}.
\begin{quote}\begin{description}
\item[{Parameters}] \leavevmode\begin{itemize}
\item {} 
\sphinxstyleliteralstrong{\sphinxupquote{fname}} (\sphinxhref{https://docs.python.org/3.7/library/stdtypes.html\#str}{\sphinxstyleliteralemphasis{\sphinxupquote{str}}}) \textendash{} The name of the file located in
{\hyperref[\detokenize{app:app.environment_settings.SHARED_ROOT}]{\sphinxcrossref{\sphinxcode{\sphinxupquote{SHARED\_ROOT}}}}} folder to be
read and splitted.

\item {} 
\sphinxstyleliteralstrong{\sphinxupquote{cluster}} ({\hyperref[\detokenize{app:app.type_hints.ClusterType}]{\sphinxcrossref{\sphinxcode{\sphinxupquote{ClusterType}}}}}) \textendash{} A reference to the {\hyperref[\detokenize{app.domain:app.domain.cluster_groups.Cluster}]{\sphinxcrossref{\sphinxcode{\sphinxupquote{cluster group}}}}} whose
{\hyperref[\detokenize{app.domain:app.domain.cluster_groups.Cluster.members}]{\sphinxcrossref{\sphinxcode{\sphinxupquote{members}}}}} will be
responsible for ensuring the file specified in \sphinxcode{\sphinxupquote{fname}}
becomes durable.

\item {} 
\sphinxstyleliteralstrong{\sphinxupquote{bsize}} (\sphinxhref{https://docs.python.org/3.7/library/functions.html\#int}{\sphinxstyleliteralemphasis{\sphinxupquote{int}}}) \textendash{} The maximum amount of bytes each file block can have.

\end{itemize}

\item[{Returns}] \leavevmode
A dictionary in which the keys are integers and values are
{\hyperref[\detokenize{app.domain.helpers:app.domain.helpers.smart_dataclasses.FileBlockData}]{\sphinxcrossref{\sphinxcode{\sphinxupquote{file blocks}}}}}, whose
attribute {\hyperref[\detokenize{app.domain.helpers:app.domain.helpers.smart_dataclasses.FileBlockData.number}]{\sphinxcrossref{\sphinxcode{\sphinxupquote{number}}}}}
is the key.

\item[{Return type}] \leavevmode
{\hyperref[\detokenize{app:app.type_hints.ReplicasDict}]{\sphinxcrossref{\sphinxcode{\sphinxupquote{ReplicasDict}}}}}

\end{description}\end{quote}

\end{fulllineitems}

\index{execute\_simulation() (Master method)@\spxentry{execute\_simulation()}\spxextra{Master method}}

\begin{fulllineitems}
\phantomsection\label{\detokenize{app.domain:app.domain.master_servers.Master.execute_simulation}}\pysiglinewithargsret{\sphinxbfcode{\sphinxupquote{execute\_simulation}}}{}{}
Starts the simulation processes.
\begin{quote}\begin{description}
\item[{Return type}] \leavevmode
\sphinxhref{https://docs.python.org/3.7/library/constants.html\#None}{None}

\end{description}\end{quote}

\end{fulllineitems}

\index{find\_online\_nodes() (Master method)@\spxentry{find\_online\_nodes()}\spxextra{Master method}}

\begin{fulllineitems}
\phantomsection\label{\detokenize{app.domain:app.domain.master_servers.Master.find_online_nodes}}\pysiglinewithargsret{\sphinxbfcode{\sphinxupquote{find\_online\_nodes}}}{\emph{\DUrole{n}{n}\DUrole{o}{=}\DUrole{default_value}{1}}, \emph{\DUrole{n}{blacklist}\DUrole{o}{=}\DUrole{default_value}{None}}}{}
Finds \sphinxcode{\sphinxupquote{n}} {\hyperref[\detokenize{app.domain:app.domain.network_nodes.Node}]{\sphinxcrossref{\sphinxcode{\sphinxupquote{network nodes}}}}} who are currently registered at the
\sphinxcode{\sphinxupquote{Master}} and whose status is online.
\begin{quote}\begin{description}
\item[{Parameters}] \leavevmode\begin{itemize}
\item {} 
\sphinxstyleliteralstrong{\sphinxupquote{n}} (\sphinxhref{https://docs.python.org/3.7/library/functions.html\#int}{\sphinxstyleliteralemphasis{\sphinxupquote{int}}}) \textendash{} How many {\hyperref[\detokenize{app.domain:app.domain.network_nodes.Node}]{\sphinxcrossref{\sphinxcode{\sphinxupquote{network node}}}}} references the requesting
entity wants to find.

\item {} 
\sphinxstyleliteralstrong{\sphinxupquote{blacklist}} ({\hyperref[\detokenize{app:app.type_hints.NodeDict}]{\sphinxcrossref{\sphinxcode{\sphinxupquote{NodeDict}}}}}) \textendash{} A collection of {\hyperref[\detokenize{app.domain:app.domain.network_nodes.Node.id}]{\sphinxcrossref{\sphinxcode{\sphinxupquote{nodes identifiers}}}}} and their object
instances, which specify nodes the requesting entity has
no interest in.

\end{itemize}

\item[{Returns}] \leavevmode
A collection of {\hyperref[\detokenize{app.domain:app.domain.network_nodes.Node}]{\sphinxcrossref{\sphinxcode{\sphinxupquote{network nodes}}}}}
which is at most as big as \sphinxcode{\sphinxupquote{n}}, which does not include any
node named in \sphinxcode{\sphinxupquote{blacklist}}.

\item[{Return type}] \leavevmode
{\hyperref[\detokenize{app:app.type_hints.NodeDict}]{\sphinxcrossref{\sphinxcode{\sphinxupquote{NodeDict}}}}}

\end{description}\end{quote}

\end{fulllineitems}

\index{MAX\_EPOCHS (Master attribute)@\spxentry{MAX\_EPOCHS}\spxextra{Master attribute}}

\begin{fulllineitems}
\phantomsection\label{\detokenize{app.domain:app.domain.master_servers.Master.MAX_EPOCHS}}\pysigline{\sphinxbfcode{\sphinxupquote{MAX\_EPOCHS}}\sphinxbfcode{\sphinxupquote{: Optional\DUrole{p}{{[}}\sphinxhref{https://docs.python.org/3.7/library/functions.html\#int}{int}\DUrole{p}{{]}}}}\sphinxbfcode{\sphinxupquote{ = None}}}
\end{fulllineitems}

\index{MAX\_EPOCHS\_PLUS\_ONE (Master attribute)@\spxentry{MAX\_EPOCHS\_PLUS\_ONE}\spxextra{Master attribute}}

\begin{fulllineitems}
\phantomsection\label{\detokenize{app.domain:app.domain.master_servers.Master.MAX_EPOCHS_PLUS_ONE}}\pysigline{\sphinxbfcode{\sphinxupquote{MAX\_EPOCHS\_PLUS\_ONE}}\sphinxbfcode{\sphinxupquote{: Optional\DUrole{p}{{[}}\sphinxhref{https://docs.python.org/3.7/library/functions.html\#int}{int}\DUrole{p}{{]}}}}\sphinxbfcode{\sphinxupquote{ = None}}}
\end{fulllineitems}


\end{fulllineitems}

\index{NewscastMaster (class in app.domain.master\_servers)@\spxentry{NewscastMaster}\spxextra{class in app.domain.master\_servers}}

\begin{fulllineitems}
\phantomsection\label{\detokenize{app.domain:app.domain.master_servers.NewscastMaster}}\pysiglinewithargsret{\sphinxbfcode{\sphinxupquote{class }}\sphinxbfcode{\sphinxupquote{NewscastMaster}}}{\emph{\DUrole{n}{simfile\_name}}, \emph{\DUrole{n}{sid}}, \emph{\DUrole{n}{epochs}}, \emph{\DUrole{n}{cluster\_class}}, \emph{\DUrole{n}{node\_class}}}{}
Bases: {\hyperref[\detokenize{app.domain:app.domain.master_servers.Master}]{\sphinxcrossref{\sphinxcode{\sphinxupquote{app.domain.master\_servers.Master}}}}}
\index{\_\_init\_\_() (NewscastMaster method)@\spxentry{\_\_init\_\_()}\spxextra{NewscastMaster method}}

\begin{fulllineitems}
\phantomsection\label{\detokenize{app.domain:app.domain.master_servers.NewscastMaster.__init__}}\pysiglinewithargsret{\sphinxbfcode{\sphinxupquote{\_\_init\_\_}}}{\emph{\DUrole{n}{simfile\_name}}, \emph{\DUrole{n}{sid}}, \emph{\DUrole{n}{epochs}}, \emph{\DUrole{n}{cluster\_class}}, \emph{\DUrole{n}{node\_class}}}{}
Instantiates an Master object.
\begin{quote}\begin{description}
\item[{Parameters}] \leavevmode\begin{itemize}
\item {} 
\sphinxstyleliteralstrong{\sphinxupquote{simfile\_name}} (\sphinxhref{https://docs.python.org/3.7/library/stdtypes.html\#str}{\sphinxstyleliteralemphasis{\sphinxupquote{str}}}) \textendash{} A path to the simulation file to be run by the simulator.

\item {} 
\sphinxstyleliteralstrong{\sphinxupquote{sid}} (\sphinxhref{https://docs.python.org/3.7/library/functions.html\#int}{\sphinxstyleliteralemphasis{\sphinxupquote{int}}}) \textendash{} Identifier that generates unique output file names,
thus guaranteeing that different simulation instances do not
overwrite previous out files.

\item {} 
\sphinxstyleliteralstrong{\sphinxupquote{epochs}} (\sphinxhref{https://docs.python.org/3.7/library/functions.html\#int}{\sphinxstyleliteralemphasis{\sphinxupquote{int}}}) \textendash{} The number of discrete time steps the simulation lasts.

\item {} 
\sphinxstyleliteralstrong{\sphinxupquote{cluster\_class}} (\sphinxhref{https://docs.python.org/3.7/library/stdtypes.html\#str}{\sphinxstyleliteralemphasis{\sphinxupquote{str}}}) \textendash{} The name of the class used to instantiate cluster group
instances through reflection. See {\hyperref[\detokenize{app.domain:module-app.domain.cluster_groups}]{\sphinxcrossref{\sphinxcode{\sphinxupquote{cluster groups module}}}}}.

\item {} 
\sphinxstyleliteralstrong{\sphinxupquote{node\_class}} (\sphinxhref{https://docs.python.org/3.7/library/stdtypes.html\#str}{\sphinxstyleliteralemphasis{\sphinxupquote{str}}}) \textendash{} The name of the class used to instantiate network node
instances through reflection. See {\hyperref[\detokenize{app.domain:module-app.domain.network_nodes}]{\sphinxcrossref{\sphinxcode{\sphinxupquote{network nodes module}}}}}.

\end{itemize}

\item[{Return type}] \leavevmode
\sphinxhref{https://docs.python.org/3.7/library/constants.html\#None}{None}

\end{description}\end{quote}

\end{fulllineitems}

\index{\_process\_simfile() (NewscastMaster method)@\spxentry{\_process\_simfile()}\spxextra{NewscastMaster method}}

\begin{fulllineitems}
\phantomsection\label{\detokenize{app.domain:app.domain.master_servers.NewscastMaster._process_simfile}}\pysiglinewithargsret{\sphinxbfcode{\sphinxupquote{\_process\_simfile}}}{\emph{\DUrole{n}{path}}, \emph{\DUrole{n}{cluster\_class}}, \emph{\DUrole{n}{node\_class}}}{}
Opens and processes the simulation filed referenced in \sphinxtitleref{path}.
\begin{description}
\item[{Overrides:}] \leavevmode
{\hyperref[\detokenize{app.domain:app.domain.master_servers.Master._process_simfile}]{\sphinxcrossref{\sphinxcode{\sphinxupquote{app.domain.master\_servers.Master.\_process\_simfile()}}}}}.

Newscast is a gossip\sphinxhyphen{}based P2P network. We assume erasure\sphinxhyphen{}coding
would be used in this scenario and thus, for simplicity,
we divide the specified file’s size into multiple \sphinxcode{\sphinxupquote{1/N}},
where \sphinxcode{\sphinxupquote{N}} is the number of {\hyperref[\detokenize{app.domain:app.domain.network_nodes.NewscastNode}]{\sphinxcrossref{\sphinxcode{\sphinxupquote{network nodes}}}}} in the system.

\end{description}

\begin{sphinxadmonition}{note}{Note:}
This class, {\hyperref[\detokenize{app.domain:app.domain.cluster_groups.NewscastCluster}]{\sphinxcrossref{\sphinxcode{\sphinxupquote{NewscastCluster}}}}}
and {\hyperref[\detokenize{app.domain:app.domain.network_nodes.NewscastNode}]{\sphinxcrossref{\sphinxcode{\sphinxupquote{NewscastNode}}}}} were
created to test our simulators performance, concerning the amount
of supported simultaneous network nodes in a simulation. We do
not actually care if the created file blocks are lost as the
{\hyperref[\detokenize{app.domain:app.domain.network_nodes.NewscastNode}]{\sphinxcrossref{\sphinxcode{\sphinxupquote{network nodes}}}}}
job in the simulation is to carry out the
protocol defined in \sphinxhref{http://peersim.sourceforge.net/doc/index.html}{PeerSim’s AverageFunction}. \sphinxhref{http://peersim.sourceforge.net/}{PeerSim} uses configuration \sphinxcode{\sphinxupquote{Example 2}}
provided in release 1.0.5, as a means of testing the simulator
performance, according to this \sphinxhref{https://www.gsd.inesc-id.pt/~lveiga/papers/msc-supervised-thesis-abstracts/jneto-FINAL.pdf}{Ms.C. dissertation by J. Neto}.
This configuration uses Newscast protocol with AverageFunction
and periodic monitoring of the system state. We implement our
version of \sphinxhref{https://dl.acm.org/doi/abs/10.1007/978-3-642-03869-3\_50}{Adaptaive Peer Sampling with Newscast} by
N. Tölgyesi and M. Jelasity, to avoid the effort of translating
PeerSim’s code.
\end{sphinxadmonition}
\begin{quote}\begin{description}
\item[{Parameters}] \leavevmode\begin{itemize}
\item {} 
\sphinxstyleliteralstrong{\sphinxupquote{path}} (\sphinxhref{https://docs.python.org/3.7/library/stdtypes.html\#str}{\sphinxstyleliteralemphasis{\sphinxupquote{str}}}) \textendash{} The path to the simulation file. Including extension and
parent folders.

\item {} 
\sphinxstyleliteralstrong{\sphinxupquote{cluster\_class}} (\sphinxhref{https://docs.python.org/3.7/library/stdtypes.html\#str}{\sphinxstyleliteralemphasis{\sphinxupquote{str}}}) \textendash{} The name of the class used to instantiate cluster group
instances through reflection.
See {\hyperref[\detokenize{app.domain:module-app.domain.cluster_groups}]{\sphinxcrossref{\sphinxcode{\sphinxupquote{app.domain.cluster\_groups}}}}}.

\item {} 
\sphinxstyleliteralstrong{\sphinxupquote{node\_class}} (\sphinxhref{https://docs.python.org/3.7/library/stdtypes.html\#str}{\sphinxstyleliteralemphasis{\sphinxupquote{str}}}) \textendash{} The name of the class used to instantiate network node
instances through reflection.
See {\hyperref[\detokenize{app.domain:module-app.domain.network_nodes}]{\sphinxcrossref{\sphinxcode{\sphinxupquote{app.domain.network\_nodes}}}}}.

\end{itemize}

\item[{Return type}] \leavevmode
\sphinxhref{https://docs.python.org/3.7/library/constants.html\#None}{None}

\end{description}\end{quote}

\end{fulllineitems}


\end{fulllineitems}

\index{SGMaster (class in app.domain.master\_servers)@\spxentry{SGMaster}\spxextra{class in app.domain.master\_servers}}

\begin{fulllineitems}
\phantomsection\label{\detokenize{app.domain:app.domain.master_servers.SGMaster}}\pysiglinewithargsret{\sphinxbfcode{\sphinxupquote{class }}\sphinxbfcode{\sphinxupquote{SGMaster}}}{\emph{\DUrole{n}{simfile\_name}}, \emph{\DUrole{n}{sid}}, \emph{\DUrole{n}{epochs}}, \emph{\DUrole{n}{cluster\_class}}, \emph{\DUrole{n}{node\_class}}}{}
Bases: {\hyperref[\detokenize{app.domain:app.domain.master_servers.Master}]{\sphinxcrossref{\sphinxcode{\sphinxupquote{app.domain.master\_servers.Master}}}}}
\index{get\_cloud\_reference() (SGMaster method)@\spxentry{get\_cloud\_reference()}\spxextra{SGMaster method}}

\begin{fulllineitems}
\phantomsection\label{\detokenize{app.domain:app.domain.master_servers.SGMaster.get_cloud_reference}}\pysiglinewithargsret{\sphinxbfcode{\sphinxupquote{get\_cloud\_reference}}}{}{}
Use to obtain a reference to 3rd party cloud storage provider

The cloud storage provider can be used to temporarely host files
belonging to \sphinxcode{\sphinxupquote{cluster clusters}} in bad
conditions that may compromise the file durability of the files they
are responsible for persisting.

\begin{sphinxadmonition}{note}{Note:}
This method is virtual.
\end{sphinxadmonition}
\begin{quote}\begin{description}
\item[{Returns}] \leavevmode
A pointer to the cloud server, e.g., an IP Address.

\item[{Return type}] \leavevmode
\sphinxhref{https://docs.python.org/3.7/library/stdtypes.html\#str}{str}

\end{description}\end{quote}

\end{fulllineitems}


\end{fulllineitems}

\index{\_PersistentingDict (in module app.domain.master\_servers)@\spxentry{\_PersistentingDict}\spxextra{in module app.domain.master\_servers}}

\begin{fulllineitems}
\phantomsection\label{\detokenize{app.domain:app.domain.master_servers._PersistentingDict}}\pysigline{\sphinxbfcode{\sphinxupquote{\_PersistentingDict}}\sphinxbfcode{\sphinxupquote{: Dict\DUrole{p}{{[}}\sphinxhref{https://docs.python.org/3.7/library/stdtypes.html\#str}{str}\DUrole{p}{, }Dict\DUrole{p}{{[}}\sphinxhref{https://docs.python.org/3.7/library/stdtypes.html\#str}{str}\DUrole{p}{, }Union\DUrole{p}{{[}}List\DUrole{p}{{[}}\sphinxhref{https://docs.python.org/3.7/library/stdtypes.html\#str}{str}\DUrole{p}{{]}}\DUrole{p}{, }\sphinxhref{https://docs.python.org/3.7/library/stdtypes.html\#str}{str}\DUrole{p}{{]}}\DUrole{p}{{]}}\DUrole{p}{{]}}}}}
\end{fulllineitems}



\subsubsection{app.domain.network\_nodes}
\label{\detokenize{app.domain:module-app.domain.network_nodes}}\label{\detokenize{app.domain:app-domain-network-nodes}}\index{module@\spxentry{module}!app.domain.network\_nodes@\spxentry{app.domain.network\_nodes}}\index{app.domain.network\_nodes@\spxentry{app.domain.network\_nodes}!module@\spxentry{module}}
This module contains domain specific classes that represent network nodes
responsible for the storage of {\hyperref[\detokenize{app.domain.helpers:app.domain.helpers.smart_dataclasses.FileBlockData}]{\sphinxcrossref{\sphinxcode{\sphinxupquote{file blocks}}}}}. These could be
reliable servers or P2P nodes.
\index{HDFSNode (class in app.domain.network\_nodes)@\spxentry{HDFSNode}\spxextra{class in app.domain.network\_nodes}}

\begin{fulllineitems}
\phantomsection\label{\detokenize{app.domain:app.domain.network_nodes.HDFSNode}}\pysiglinewithargsret{\sphinxbfcode{\sphinxupquote{class }}\sphinxbfcode{\sphinxupquote{HDFSNode}}}{\emph{\DUrole{n}{uid}}, \emph{\DUrole{n}{uptime}}}{}
Bases: {\hyperref[\detokenize{app.domain:app.domain.network_nodes.Node}]{\sphinxcrossref{\sphinxcode{\sphinxupquote{app.domain.network\_nodes.Node}}}}}

Represents a data node in the Hadoop Distribute File System.
\index{execute\_epoch() (HDFSNode method)@\spxentry{execute\_epoch()}\spxextra{HDFSNode method}}

\begin{fulllineitems}
\phantomsection\label{\detokenize{app.domain:app.domain.network_nodes.HDFSNode.execute_epoch}}\pysiglinewithargsret{\sphinxbfcode{\sphinxupquote{execute\_epoch}}}{\emph{\DUrole{n}{cluster}}, \emph{\DUrole{n}{fid}}}{}
Instructs the \sphinxcode{\sphinxupquote{HDFSNode}} instance to execute the epoch.

The method iterates \sphinxcode{\sphinxupquote{files}} held in disk and attempts to
corrupt them silently. In HDFS file blocks’ \sphinxcode{\sphinxupquote{sha256}} are only
verified when a user or client accesses the remote replica. Hence,
no replication epoch is set up when a corruption occurs. The
corruption is still logged in the output file.
\begin{description}
\item[{Overrides:}] \leavevmode
{\hyperref[\detokenize{app.domain:app.domain.network_nodes.Node.execute_epoch}]{\sphinxcrossref{\sphinxcode{\sphinxupquote{app.domain.network\_nodes.Node.execute\_epoch()}}}}}.

\end{description}
\begin{quote}\begin{description}
\item[{Parameters}] \leavevmode\begin{itemize}
\item {} 
\sphinxstyleliteralstrong{\sphinxupquote{cluster}} ({\hyperref[\detokenize{app:app.type_hints.ClusterType}]{\sphinxcrossref{\sphinxcode{\sphinxupquote{ClusterType}}}}}) \textendash{} A reference to the
{\hyperref[\detokenize{app.domain:app.domain.cluster_groups.Cluster}]{\sphinxcrossref{\sphinxcode{\sphinxupquote{Cluster}}}}} that invoked
the \sphinxcode{\sphinxupquote{Node}} method.

\item {} 
\sphinxstyleliteralstrong{\sphinxupquote{fid}} (\sphinxhref{https://docs.python.org/3.7/library/stdtypes.html\#str}{\sphinxstyleliteralemphasis{\sphinxupquote{str}}}) \textendash{} The {\hyperref[\detokenize{app.domain.helpers:app.domain.helpers.smart_dataclasses.FileData.name}]{\sphinxcrossref{\sphinxcode{\sphinxupquote{file name identifier}}}}}
of the file being simulated.

\end{itemize}

\item[{Return type}] \leavevmode
\sphinxhref{https://docs.python.org/3.7/library/constants.html\#None}{None}

\end{description}\end{quote}

\end{fulllineitems}

\index{replicate\_part() (HDFSNode method)@\spxentry{replicate\_part()}\spxextra{HDFSNode method}}

\begin{fulllineitems}
\phantomsection\label{\detokenize{app.domain:app.domain.network_nodes.HDFSNode.replicate_part}}\pysiglinewithargsret{\sphinxbfcode{\sphinxupquote{replicate\_part}}}{\emph{\DUrole{n}{cluster}}, \emph{\DUrole{n}{replica}}}{}
Attempts to restore the replication level of the specified file
block replica.

Replicas are sent selectively in descending order to the
most reliable Nodes in the \sphinxcode{\sphinxupquote{Cluster}} down to the least
reliable.
\begin{description}
\item[{Overrides:}] \leavevmode
{\hyperref[\detokenize{app.domain:app.domain.network_nodes.Node.replicate_part}]{\sphinxcrossref{\sphinxcode{\sphinxupquote{app.domain.network\_nodes.Node.replicate\_part()}}}}}.

\end{description}

\begin{sphinxadmonition}{note}{Note:}
There are no guarantees that
{\hyperref[\detokenize{app:app.environment_settings.REPLICATION_LEVEL}]{\sphinxcrossref{\sphinxcode{\sphinxupquote{REPLICATION\_LEVEL}}}}} will be
completely restored during the execution of this method.
\end{sphinxadmonition}
\begin{quote}\begin{description}
\item[{Parameters}] \leavevmode\begin{itemize}
\item {} 
\sphinxstyleliteralstrong{\sphinxupquote{cluster}} ({\hyperref[\detokenize{app:app.type_hints.ClusterType}]{\sphinxcrossref{\sphinxcode{\sphinxupquote{ClusterType}}}}}) \textendash{} A reference to the
{\hyperref[\detokenize{app.domain:app.domain.cluster_groups.Cluster}]{\sphinxcrossref{\sphinxcode{\sphinxupquote{Cluster}}}}} that will
deliver the new \sphinxcode{\sphinxupquote{replica}}.

\item {} 
\sphinxstyleliteralstrong{\sphinxupquote{replica}} ({\hyperref[\detokenize{app.domain.helpers:app.domain.helpers.smart_dataclasses.FileBlockData}]{\sphinxcrossref{\sphinxcode{\sphinxupquote{FileBlockData}}}}}) \textendash{} The {\hyperref[\detokenize{app.domain.helpers:app.domain.helpers.smart_dataclasses.FileBlockData}]{\sphinxcrossref{\sphinxcode{\sphinxupquote{file block replica}}}}}
to be delivered.

\end{itemize}

\item[{Return type}] \leavevmode
\sphinxhref{https://docs.python.org/3.7/library/constants.html\#None}{None}

\end{description}\end{quote}

\end{fulllineitems}

\index{update\_status() (HDFSNode method)@\spxentry{update\_status()}\spxextra{HDFSNode method}}

\begin{fulllineitems}
\phantomsection\label{\detokenize{app.domain:app.domain.network_nodes.HDFSNode.update_status}}\pysiglinewithargsret{\sphinxbfcode{\sphinxupquote{update\_status}}}{}{}
Used to update the time to live of the node instance.

When invoked, the network node decides if it should remain online or
change some other state.
\begin{description}
\item[{Overrides:}] \leavevmode
{\hyperref[\detokenize{app.domain:app.domain.network_nodes.Node.update_status}]{\sphinxcrossref{\sphinxcode{\sphinxupquote{app.domain.network\_nodes.Node.update\_status()}}}}}.

\end{description}
\begin{quote}\begin{description}
\item[{Returns}] \leavevmode
The the status of the \sphinxcode{\sphinxupquote{Node}}.

\item[{Return type}] \leavevmode
{\hyperref[\detokenize{app.domain.helpers:app.domain.helpers.enums.Status}]{\sphinxcrossref{\sphinxcode{\sphinxupquote{Status}}}}}

\end{description}\end{quote}

\end{fulllineitems}


\end{fulllineitems}

\index{NewscastNode (class in app.domain.network\_nodes)@\spxentry{NewscastNode}\spxextra{class in app.domain.network\_nodes}}

\begin{fulllineitems}
\phantomsection\label{\detokenize{app.domain:app.domain.network_nodes.NewscastNode}}\pysiglinewithargsret{\sphinxbfcode{\sphinxupquote{class }}\sphinxbfcode{\sphinxupquote{NewscastNode}}}{\emph{\DUrole{n}{uid}}, \emph{\DUrole{n}{uptime}}}{}
Bases: {\hyperref[\detokenize{app.domain:app.domain.network_nodes.Node}]{\sphinxcrossref{\sphinxcode{\sphinxupquote{app.domain.network\_nodes.Node}}}}}

Represents a Peer running Newscast protocol, using shuffling
techniques to exchange acquaintances with other network peers and
performing peer degree aggregation using AverageFunction.
\index{view (NewscastNode attribute)@\spxentry{view}\spxextra{NewscastNode attribute}}

\begin{fulllineitems}
\phantomsection\label{\detokenize{app.domain:app.domain.network_nodes.NewscastNode.view}}\pysigline{\sphinxbfcode{\sphinxupquote{view}}}
A partial view of the P2P network. \sphinxcode{\sphinxupquote{View}} is a collection of
{\hyperref[\detokenize{app.domain:app.domain.network_nodes.NewscastNode}]{\sphinxcrossref{\sphinxcode{\sphinxupquote{network nodes}}}}},
the \sphinxcode{\sphinxupquote{NewscastNode}} instance may contact other than himself. Keys
are {\hyperref[\detokenize{app.domain:app.domain.network_nodes.NewscastNode}]{\sphinxcrossref{\sphinxcode{\sphinxupquote{NewscastNode}}}}} instances, and values are their age
in the dictionary. A key\sphinxhyphen{}value pair is commonly referenced as a
\sphinxcode{\sphinxupquote{descriptor}}.

\end{fulllineitems}

\index{aggregation\_value (NewscastNode attribute)@\spxentry{aggregation\_value}\spxextra{NewscastNode attribute}}

\begin{fulllineitems}
\phantomsection\label{\detokenize{app.domain:app.domain.network_nodes.NewscastNode.aggregation_value}}\pysigline{\sphinxbfcode{\sphinxupquote{aggregation\_value}}}
Stores the aggregation value. The type of \sphinxcode{\sphinxupquote{aggregation\_value}}
is defined by the body of the {\hyperref[\detokenize{app.domain:app.domain.network_nodes.NewscastNode.aggregate}]{\sphinxcrossref{\sphinxcode{\sphinxupquote{aggregate()}}}}} method.

\end{fulllineitems}

\index{\_\_init\_\_() (NewscastNode method)@\spxentry{\_\_init\_\_()}\spxextra{NewscastNode method}}

\begin{fulllineitems}
\phantomsection\label{\detokenize{app.domain:app.domain.network_nodes.NewscastNode.__init__}}\pysiglinewithargsret{\sphinxbfcode{\sphinxupquote{\_\_init\_\_}}}{\emph{\DUrole{n}{uid}}, \emph{\DUrole{n}{uptime}}}{}
Instantiates a \sphinxcode{\sphinxupquote{Node}} object.

These are network nodes responsible for persisting
\sphinxcode{\sphinxupquote{file block replicas}}.
\begin{quote}\begin{description}
\item[{Parameters}] \leavevmode\begin{itemize}
\item {} 
\sphinxstyleliteralstrong{\sphinxupquote{uid}} (\sphinxhref{https://docs.python.org/3.7/library/stdtypes.html\#str}{\sphinxstyleliteralemphasis{\sphinxupquote{str}}}) \textendash{} An unique identifier for the \sphinxcode{\sphinxupquote{Node}} instance.

\item {} 
\sphinxstyleliteralstrong{\sphinxupquote{uptime}} (\sphinxhref{https://docs.python.org/3.7/library/functions.html\#float}{\sphinxstyleliteralemphasis{\sphinxupquote{float}}}) \textendash{} The availability of the \sphinxcode{\sphinxupquote{Node}} instance.

\end{itemize}

\item[{Return type}] \leavevmode
\sphinxhref{https://docs.python.org/3.7/library/constants.html\#None}{None}

\end{description}\end{quote}

\end{fulllineitems}

\index{\_merge() (NewscastNode method)@\spxentry{\_merge()}\spxextra{NewscastNode method}}

\begin{fulllineitems}
\phantomsection\label{\detokenize{app.domain:app.domain.network_nodes.NewscastNode._merge}}\pysiglinewithargsret{\sphinxbfcode{\sphinxupquote{\_merge}}}{\emph{\DUrole{n}{a}}, \emph{\DUrole{n}{b}}}{}
Merges two network views. If a node descriptor exists in both
views, the most recent descriptor is kept.
\begin{quote}\begin{description}
\item[{Parameters}] \leavevmode\begin{itemize}
\item {} 
\sphinxstyleliteralstrong{\sphinxupquote{a}} (\sphinxstyleliteralemphasis{\sphinxupquote{\_NetworkView}}) \textendash{} A dictionary where keys are {\hyperref[\detokenize{app.domain:app.domain.network_nodes.Node}]{\sphinxcrossref{\sphinxcode{\sphinxupquote{network nodes}}}}}
and values are their respective age in the view.

\item {} 
\sphinxstyleliteralstrong{\sphinxupquote{b}} (\sphinxstyleliteralemphasis{\sphinxupquote{\_NetworkView}}) \textendash{} A dictionary where keys are {\hyperref[\detokenize{app.domain:app.domain.network_nodes.Node}]{\sphinxcrossref{\sphinxcode{\sphinxupquote{network nodes}}}}}
and values are their respective age in the view.

\end{itemize}

\item[{Returns}] \leavevmode
The set union of both views with only the most up to date
descriptors.

\item[{Return type}] \leavevmode
\_NetworkView

\end{description}\end{quote}

\end{fulllineitems}

\index{\_select\_view() (NewscastNode method)@\spxentry{\_select\_view()}\spxextra{NewscastNode method}}

\begin{fulllineitems}
\phantomsection\label{\detokenize{app.domain:app.domain.network_nodes.NewscastNode._select_view}}\pysiglinewithargsret{\sphinxbfcode{\sphinxupquote{\_select\_view}}}{\emph{\DUrole{n}{view\_buffer}}}{}
Reduces the size of the view to a predefined maximum size.

:param A dictionary where keys are {\hyperref[\detokenize{app.domain:app.domain.network_nodes.Node}]{\sphinxcrossref{\sphinxcode{\sphinxupquote{network nodes}}}}}:
:param and values are their respective age in the view.:
\begin{quote}\begin{description}
\item[{Returns}] \leavevmode
The \sphinxcode{\sphinxupquote{view\_buffer}} with at most \sphinxcode{\sphinxupquote{max\_view\_size}} descriptors.

\item[{Parameters}] \leavevmode
\sphinxstyleliteralstrong{\sphinxupquote{view\_buffer}} (\sphinxstyleliteralemphasis{\sphinxupquote{\_NetworkView}}) \textendash{} 

\item[{Return type}] \leavevmode
\_NetworkView

\end{description}\end{quote}

\end{fulllineitems}

\index{add\_neighbor() (NewscastNode method)@\spxentry{add\_neighbor()}\spxextra{NewscastNode method}}

\begin{fulllineitems}
\phantomsection\label{\detokenize{app.domain:app.domain.network_nodes.NewscastNode.add_neighbor}}\pysiglinewithargsret{\sphinxbfcode{\sphinxupquote{add\_neighbor}}}{\emph{\DUrole{n}{node}}}{}
Adds a new network node to the node instance’s view.

If the view is full, the eldest node is replaced with the new node.
Otherwise, the new {\hyperref[\detokenize{app.domain:app.domain.network_nodes.NewscastNode}]{\sphinxcrossref{\sphinxcode{\sphinxupquote{NewscastNode}}}}} is added to the
instance’s view with age zero, unless the entry is already in
{\hyperref[\detokenize{app.domain:app.domain.network_nodes.NewscastNode.view}]{\sphinxcrossref{\sphinxcode{\sphinxupquote{view}}}}} or the \sphinxcode{\sphinxupquote{node}} is the current \sphinxcode{\sphinxupquote{NewscastNode}}
instance.
\begin{quote}\begin{description}
\item[{Returns}] \leavevmode
\sphinxcode{\sphinxupquote{True}} if \sphinxcode{\sphinxupquote{node}} was successfuly added, \sphinxcode{\sphinxupquote{False}} otherwise.

\item[{Parameters}] \leavevmode
\sphinxstyleliteralstrong{\sphinxupquote{node}} ({\hyperref[\detokenize{app.domain:app.domain.network_nodes.NewscastNode}]{\sphinxcrossref{\sphinxstyleliteralemphasis{\sphinxupquote{app.domain.network\_nodes.NewscastNode}}}}}) \textendash{} 

\item[{Return type}] \leavevmode
\sphinxhref{https://docs.python.org/3.7/library/functions.html\#bool}{bool}

\end{description}\end{quote}

\end{fulllineitems}

\index{aggregate() (NewscastNode method)@\spxentry{aggregate()}\spxextra{NewscastNode method}}

\begin{fulllineitems}
\phantomsection\label{\detokenize{app.domain:app.domain.network_nodes.NewscastNode.aggregate}}\pysiglinewithargsret{\sphinxbfcode{\sphinxupquote{aggregate}}}{\emph{\DUrole{n}{node}\DUrole{o}{=}\DUrole{default_value}{None}}}{}
The network node instance contacts another node from his view, then,
both nodes assign the mean of their degrees to
{\hyperref[\detokenize{app.domain:app.domain.network_nodes.NewscastNode.aggregation_value}]{\sphinxcrossref{\sphinxcode{\sphinxupquote{aggregation\_value}}}}}.
\begin{quote}\begin{description}
\item[{Parameters}] \leavevmode
\sphinxstyleliteralstrong{\sphinxupquote{node}} (\sphinxstyleliteralemphasis{\sphinxupquote{Optional}}\sphinxstyleliteralemphasis{\sphinxupquote{{[}}}{\hyperref[\detokenize{app.domain:app.domain.network_nodes.NewscastNode}]{\sphinxcrossref{\sphinxstyleliteralemphasis{\sphinxupquote{app.domain.network\_nodes.NewscastNode}}}}}\sphinxstyleliteralemphasis{\sphinxupquote{{]}}}) \textendash{} When \sphinxcode{\sphinxupquote{node}} is None a random \sphinxcode{\sphinxupquote{NewscastNode}} is selected
from {\hyperref[\detokenize{app.domain:app.domain.network_nodes.NewscastNode.view}]{\sphinxcrossref{\sphinxcode{\sphinxupquote{view}}}}}. When specified to be contacted is the
one referenced in the parameter.

\item[{Return type}] \leavevmode
\sphinxhref{https://docs.python.org/3.7/library/constants.html\#None}{None}

\end{description}\end{quote}

\end{fulllineitems}

\index{execute\_epoch() (NewscastNode method)@\spxentry{execute\_epoch()}\spxextra{NewscastNode method}}

\begin{fulllineitems}
\phantomsection\label{\detokenize{app.domain:app.domain.network_nodes.NewscastNode.execute_epoch}}\pysiglinewithargsret{\sphinxbfcode{\sphinxupquote{execute\_epoch}}}{\emph{\DUrole{n}{cluster}}, \emph{\DUrole{n}{fid}}}{}
Instructs the \sphinxcode{\sphinxupquote{NewscastNode}} instance to execute the epoch.

During the execution of the epoch, the \sphinxcode{\sphinxupquote{NewscastNode}}
instance randomly selects another \sphinxcode{\sphinxupquote{NewscastNode}} who belongs to his
{\hyperref[\detokenize{app.domain:app.domain.network_nodes.NewscastNode.view}]{\sphinxcrossref{\sphinxcode{\sphinxupquote{view}}}}} and aggregates their degree using the Average
Function. Sometimes, during the epoch, the \sphinxcode{\sphinxupquote{NewscastNode}} instance
will also perform shuffling with the selected target.
\begin{description}
\item[{Overrides:}] \leavevmode
{\hyperref[\detokenize{app.domain:app.domain.network_nodes.Node.execute_epoch}]{\sphinxcrossref{\sphinxcode{\sphinxupquote{app.domain.network\_nodes.Node.execute\_epoch()}}}}}.

\end{description}
\begin{quote}\begin{description}
\item[{Parameters}] \leavevmode\begin{itemize}
\item {} 
\sphinxstyleliteralstrong{\sphinxupquote{cluster}} ({\hyperref[\detokenize{app:app.type_hints.ClusterType}]{\sphinxcrossref{\sphinxcode{\sphinxupquote{ClusterType}}}}}) \textendash{} A reference to the
{\hyperref[\detokenize{app.domain:app.domain.cluster_groups.Cluster}]{\sphinxcrossref{\sphinxcode{\sphinxupquote{Cluster}}}}} that invoked
the \sphinxcode{\sphinxupquote{Node}} method.

\item {} 
\sphinxstyleliteralstrong{\sphinxupquote{fid}} (\sphinxhref{https://docs.python.org/3.7/library/stdtypes.html\#str}{\sphinxstyleliteralemphasis{\sphinxupquote{str}}}) \textendash{} The {\hyperref[\detokenize{app.domain.helpers:app.domain.helpers.smart_dataclasses.FileData.name}]{\sphinxcrossref{\sphinxcode{\sphinxupquote{file name identifier}}}}}
of the file being simulated.

\end{itemize}

\item[{Return type}] \leavevmode
\sphinxhref{https://docs.python.org/3.7/library/constants.html\#None}{None}

\end{description}\end{quote}

\end{fulllineitems}

\index{get\_degree() (NewscastNode method)@\spxentry{get\_degree()}\spxextra{NewscastNode method}}

\begin{fulllineitems}
\phantomsection\label{\detokenize{app.domain:app.domain.network_nodes.NewscastNode.get_degree}}\pysiglinewithargsret{\sphinxbfcode{\sphinxupquote{get\_degree}}}{}{}
Counts the number of descriptors in the node’s view.
\begin{quote}\begin{description}
\item[{Returns}] \leavevmode
The degree of the \sphinxcode{\sphinxupquote{NewscastNode}} instance.

\item[{Return type}] \leavevmode
\sphinxhref{https://docs.python.org/3.7/library/functions.html\#int}{int}

\end{description}\end{quote}

\end{fulllineitems}

\index{get\_node() (NewscastNode method)@\spxentry{get\_node()}\spxextra{NewscastNode method}}

\begin{fulllineitems}
\phantomsection\label{\detokenize{app.domain:app.domain.network_nodes.NewscastNode.get_node}}\pysiglinewithargsret{\sphinxbfcode{\sphinxupquote{get\_node}}}{}{}
Gets a random node from the current network view.

Each candidate {\hyperref[\detokenize{app.domain:app.domain.network_nodes.NewscastNode}]{\sphinxcrossref{\sphinxcode{\sphinxupquote{NewscastNode}}}}} to be returned is first pinged,
if no answer is obtained, another node is selected as a candidate by
iterating a list representation of {\hyperref[\detokenize{app.domain:app.domain.network_nodes.NewscastNode.view}]{\sphinxcrossref{\sphinxcode{\sphinxupquote{view}}}}} and the previous
candidate is removed from the {\hyperref[\detokenize{app.domain:app.domain.network_nodes.NewscastNode.view}]{\sphinxcrossref{\sphinxcode{\sphinxupquote{view}}}}}.

\begin{sphinxadmonition}{note}{Note:}
Newscast should always return a random node, thus iteration
should not be used, but this search is more efficient and readable.
\end{sphinxadmonition}
\begin{quote}\begin{description}
\item[{Returns}] \leavevmode
The selected \sphinxcode{\sphinxupquote{NewscastNode}}.

\item[{Return type}] \leavevmode
Optional{[}{\hyperref[\detokenize{app.domain:app.domain.network_nodes.NewscastNode}]{\sphinxcrossref{app.domain.network\_nodes.NewscastNode}}}{]}

\end{description}\end{quote}

\end{fulllineitems}

\index{replicate\_part() (NewscastNode method)@\spxentry{replicate\_part()}\spxextra{NewscastNode method}}

\begin{fulllineitems}
\phantomsection\label{\detokenize{app.domain:app.domain.network_nodes.NewscastNode.replicate_part}}\pysiglinewithargsret{\sphinxbfcode{\sphinxupquote{replicate\_part}}}{\emph{\DUrole{n}{cluster}}, \emph{\DUrole{n}{replica}}}{}
Attempts to restore the replication level of the specified file
block replica.

Similar to \sphinxcode{\sphinxupquote{send\_part()}} but with
slightly different instructions. In particular new \sphinxcode{\sphinxupquote{replicas}}
can not be corrupted at the current node, at the current epoch.

\begin{sphinxadmonition}{note}{Note:}
There are no guarantees that
{\hyperref[\detokenize{app:app.environment_settings.REPLICATION_LEVEL}]{\sphinxcrossref{\sphinxcode{\sphinxupquote{REPLICATION\_LEVEL}}}}} will be
completely restored during the execution of this method.
\end{sphinxadmonition}
\begin{quote}\begin{description}
\item[{Parameters}] \leavevmode\begin{itemize}
\item {} 
\sphinxstyleliteralstrong{\sphinxupquote{cluster}} ({\hyperref[\detokenize{app:app.type_hints.ClusterType}]{\sphinxcrossref{\sphinxcode{\sphinxupquote{ClusterType}}}}}) \textendash{} A reference to the
{\hyperref[\detokenize{app.domain:app.domain.cluster_groups.Cluster}]{\sphinxcrossref{\sphinxcode{\sphinxupquote{Cluster}}}}} that will
deliver the new \sphinxcode{\sphinxupquote{replica}}.

\item {} 
\sphinxstyleliteralstrong{\sphinxupquote{replica}} ({\hyperref[\detokenize{app.domain.helpers:app.domain.helpers.smart_dataclasses.FileBlockData}]{\sphinxcrossref{\sphinxcode{\sphinxupquote{FileBlockData}}}}}) \textendash{} The {\hyperref[\detokenize{app.domain.helpers:app.domain.helpers.smart_dataclasses.FileBlockData}]{\sphinxcrossref{\sphinxcode{\sphinxupquote{file block replica}}}}}
to be delivered.

\end{itemize}

\item[{Raises}] \leavevmode
\sphinxhref{https://docs.python.org/3.7/library/exceptions.html\#NotImplementedError}{\sphinxstyleliteralstrong{\sphinxupquote{NotImplementedError}}} \textendash{} When children of \sphinxcode{\sphinxupquote{Node}} do not implement the abstract method.

\item[{Return type}] \leavevmode
\sphinxhref{https://docs.python.org/3.7/library/constants.html\#None}{None}

\end{description}\end{quote}

\end{fulllineitems}

\index{shuffle() (NewscastNode method)@\spxentry{shuffle()}\spxextra{NewscastNode method}}

\begin{fulllineitems}
\phantomsection\label{\detokenize{app.domain:app.domain.network_nodes.NewscastNode.shuffle}}\pysiglinewithargsret{\sphinxbfcode{\sphinxupquote{shuffle}}}{\emph{\DUrole{n}{node}}}{}
Starts a shuffle process that merges and crops two nodes’ views at
the current node and at the destination node.

The final view consists of most up to date descriptors from both
{\hyperref[\detokenize{app.domain:app.domain.network_nodes.NewscastNode.view}]{\sphinxcrossref{\sphinxcode{\sphinxupquote{views}}}}} up to a maximum of \sphinxcode{\sphinxupquote{max\_view\_size}}
descriptors.
\begin{quote}\begin{description}
\item[{Parameters}] \leavevmode
\sphinxstyleliteralstrong{\sphinxupquote{node}} ({\hyperref[\detokenize{app.domain:app.domain.network_nodes.NewscastNode}]{\sphinxcrossref{\sphinxstyleliteralemphasis{\sphinxupquote{app.domain.network\_nodes.NewscastNode}}}}}) \textendash{} The node to be contacted for shuffling.

\item[{Return type}] \leavevmode
\sphinxhref{https://docs.python.org/3.7/library/constants.html\#None}{None}

\end{description}\end{quote}

\end{fulllineitems}

\index{shuffle\_request() (NewscastNode method)@\spxentry{shuffle\_request()}\spxextra{NewscastNode method}}

\begin{fulllineitems}
\phantomsection\label{\detokenize{app.domain:app.domain.network_nodes.NewscastNode.shuffle_request}}\pysiglinewithargsret{\sphinxbfcode{\sphinxupquote{shuffle\_request}}}{\emph{\DUrole{n}{senders\_view}}}{}
Merges and crops two nodes’ views at the current node.

The final view consists of most up to date descriptors from both
{\hyperref[\detokenize{app.domain:app.domain.network_nodes.NewscastNode.view}]{\sphinxcrossref{\sphinxcode{\sphinxupquote{views}}}}} up to a maximum of \sphinxcode{\sphinxupquote{max\_view\_size}}
descriptors.
\begin{quote}\begin{description}
\item[{Parameters}] \leavevmode
\sphinxstyleliteralstrong{\sphinxupquote{senders\_view}} (\sphinxstyleliteralemphasis{\sphinxupquote{\_NetworkView}}) \textendash{} A dictionary where keys are {\hyperref[\detokenize{app.domain:app.domain.network_nodes.Node}]{\sphinxcrossref{\sphinxcode{\sphinxupquote{network nodes}}}}}
and values are their respective age in the view.

\item[{Returns}] \leavevmode
A {\hyperref[\detokenize{app.domain:app.domain.network_nodes.NewscastNode.view}]{\sphinxcrossref{\sphinxcode{\sphinxupquote{view}}}}} and a fresh \sphinxcode{\sphinxupquote{descriptor}}
from the \sphinxcode{\sphinxupquote{NewscastNode}} instance, before it is
merged with the requestor’s view.

\item[{Return type}] \leavevmode
\_NetworkView

\end{description}\end{quote}

\end{fulllineitems}

\index{update\_status() (NewscastNode method)@\spxentry{update\_status()}\spxextra{NewscastNode method}}

\begin{fulllineitems}
\phantomsection\label{\detokenize{app.domain:app.domain.network_nodes.NewscastNode.update_status}}\pysiglinewithargsret{\sphinxbfcode{\sphinxupquote{update\_status}}}{}{}
Used to update the time to live of the node instance.

When invoked, the network node decides if it should remain online or
change some other state.
\begin{description}
\item[{Overrides:}] \leavevmode
{\hyperref[\detokenize{app.domain:app.domain.network_nodes.Node.update_status}]{\sphinxcrossref{\sphinxcode{\sphinxupquote{app.domain.network\_nodes.Node.update\_status()}}}}}.

\end{description}
\begin{quote}\begin{description}
\item[{Returns}] \leavevmode
The the status of the \sphinxcode{\sphinxupquote{Node}}.

\item[{Return type}] \leavevmode
{\hyperref[\detokenize{app.domain.helpers:app.domain.helpers.enums.Status}]{\sphinxcrossref{\sphinxcode{\sphinxupquote{Status}}}}}

\end{description}\end{quote}

\end{fulllineitems}


\end{fulllineitems}

\index{Node (class in app.domain.network\_nodes)@\spxentry{Node}\spxextra{class in app.domain.network\_nodes}}

\begin{fulllineitems}
\phantomsection\label{\detokenize{app.domain:app.domain.network_nodes.Node}}\pysiglinewithargsret{\sphinxbfcode{\sphinxupquote{class }}\sphinxbfcode{\sphinxupquote{Node}}}{\emph{\DUrole{n}{uid}}, \emph{\DUrole{n}{uptime}}}{}
Bases: \sphinxhref{https://docs.python.org/3.7/library/functions.html\#object}{\sphinxcode{\sphinxupquote{object}}}

This class contains basic network node functionality that should
always be useful.
\index{id (Node attribute)@\spxentry{id}\spxextra{Node attribute}}

\begin{fulllineitems}
\phantomsection\label{\detokenize{app.domain:app.domain.network_nodes.Node.id}}\pysigline{\sphinxbfcode{\sphinxupquote{id}}}
A unique identifier for the \sphinxcode{\sphinxupquote{Node}} instance.
\begin{quote}\begin{description}
\item[{Type}] \leavevmode
\sphinxhref{https://docs.python.org/3.7/library/stdtypes.html\#str}{str}

\end{description}\end{quote}

\end{fulllineitems}

\index{uptime (Node attribute)@\spxentry{uptime}\spxextra{Node attribute}}

\begin{fulllineitems}
\phantomsection\label{\detokenize{app.domain:app.domain.network_nodes.Node.uptime}}\pysigline{\sphinxbfcode{\sphinxupquote{uptime}}}
The amount of time the \sphinxcode{\sphinxupquote{Node}} is expected to remain online
without disconnecting. Current uptime implementation is based on
availability percentages.

\begin{sphinxadmonition}{note}{Note:}
Current implementation expects \sphinxcode{\sphinxupquote{network nodes}} joining a
{\hyperref[\detokenize{app.domain:app.domain.cluster_groups.Cluster}]{\sphinxcrossref{\sphinxcode{\sphinxupquote{cluster group}}}}}
to remain online for approximately:
\begin{quote}

\sphinxcode{\sphinxupquote{time\_to\_live}} =
{\hyperref[\detokenize{app.domain:app.domain.network_nodes.Node.uptime}]{\sphinxcrossref{\sphinxcode{\sphinxupquote{uptime}}}}}
*
{\hyperref[\detokenize{app.domain:app.domain.master_servers.Master.MAX_EPOCHS}]{\sphinxcrossref{\sphinxcode{\sphinxupquote{MAX\_EPOCHS}}}}}.
\end{quote}

However, a \sphinxcode{\sphinxupquote{network node}} who belongs to multiple
{\hyperref[\detokenize{app.domain:app.domain.cluster_groups.Cluster}]{\sphinxcrossref{\sphinxcode{\sphinxupquote{cluster groups}}}}}
may disconnect earlier than that, i.e.,
\sphinxcode{\sphinxupquote{network nodes}} remain online \sphinxcode{\sphinxupquote{time\_to\_live}} after
their first operation on the distributed backup system.
\end{sphinxadmonition}
\begin{quote}\begin{description}
\item[{Type}] \leavevmode
\sphinxhref{https://docs.python.org/3.7/library/functions.html\#float}{float}

\end{description}\end{quote}

\end{fulllineitems}

\index{status (Node attribute)@\spxentry{status}\spxextra{Node attribute}}

\begin{fulllineitems}
\phantomsection\label{\detokenize{app.domain:app.domain.network_nodes.Node.status}}\pysigline{\sphinxbfcode{\sphinxupquote{status}}}
Indicates if the \sphinxcode{\sphinxupquote{Node}} instance is online or offline. In later
releases this could also contain a ‘suspect’ status.
\begin{quote}\begin{description}
\item[{Type}] \leavevmode
{\hyperref[\detokenize{app.domain.helpers:app.domain.helpers.enums.Status}]{\sphinxcrossref{\sphinxcode{\sphinxupquote{app.domain.helpers.enums.Status}}}}}

\end{description}\end{quote}

\end{fulllineitems}

\index{suspicious\_replies (Node attribute)@\spxentry{suspicious\_replies}\spxextra{Node attribute}}

\begin{fulllineitems}
\phantomsection\label{\detokenize{app.domain:app.domain.network_nodes.Node.suspicious_replies}}\pysigline{\sphinxbfcode{\sphinxupquote{suspicious\_replies}}}
Collection that contains
{\hyperref[\detokenize{app.domain.helpers:app.domain.helpers.enums.HttpCodes}]{\sphinxcrossref{\sphinxcode{\sphinxupquote{http codes}}}}}
that when received, trigger complaints to monitors about the
replier.
\begin{quote}\begin{description}
\item[{Type}] \leavevmode
\sphinxhref{https://docs.python.org/3.7/library/stdtypes.html\#set}{\sphinxcode{\sphinxupquote{set}}}

\end{description}\end{quote}

\end{fulllineitems}

\index{files (Node attribute)@\spxentry{files}\spxextra{Node attribute}}

\begin{fulllineitems}
\phantomsection\label{\detokenize{app.domain:app.domain.network_nodes.Node.files}}\pysigline{\sphinxbfcode{\sphinxupquote{files}}}
A dictionary mapping file names to dictionaries of file block
identifiers and their respective contents, i.e.,
the {\hyperref[\detokenize{app.domain.helpers:app.domain.helpers.smart_dataclasses.FileBlockData}]{\sphinxcrossref{\sphinxcode{\sphinxupquote{file block replicas}}}}}
hosted at the \sphinxcode{\sphinxupquote{Node}}.
\begin{quote}\begin{description}
\item[{Type}] \leavevmode
Dict{[}str, {\hyperref[\detokenize{app:app.type_hints.ReplicasDict}]{\sphinxcrossref{\sphinxcode{\sphinxupquote{ReplicasDict}}}}}{]}

\end{description}\end{quote}

\end{fulllineitems}

\index{\_\_init\_\_() (Node method)@\spxentry{\_\_init\_\_()}\spxextra{Node method}}

\begin{fulllineitems}
\phantomsection\label{\detokenize{app.domain:app.domain.network_nodes.Node.__init__}}\pysiglinewithargsret{\sphinxbfcode{\sphinxupquote{\_\_init\_\_}}}{\emph{\DUrole{n}{uid}}, \emph{\DUrole{n}{uptime}}}{}
Instantiates a \sphinxcode{\sphinxupquote{Node}} object.

These are network nodes responsible for persisting
\sphinxcode{\sphinxupquote{file block replicas}}.
\begin{quote}\begin{description}
\item[{Parameters}] \leavevmode\begin{itemize}
\item {} 
\sphinxstyleliteralstrong{\sphinxupquote{uid}} (\sphinxhref{https://docs.python.org/3.7/library/stdtypes.html\#str}{\sphinxstyleliteralemphasis{\sphinxupquote{str}}}) \textendash{} An unique identifier for the \sphinxcode{\sphinxupquote{Node}} instance.

\item {} 
\sphinxstyleliteralstrong{\sphinxupquote{uptime}} (\sphinxhref{https://docs.python.org/3.7/library/functions.html\#float}{\sphinxstyleliteralemphasis{\sphinxupquote{float}}}) \textendash{} The availability of the \sphinxcode{\sphinxupquote{Node}} instance.

\end{itemize}

\item[{Return type}] \leavevmode
\sphinxhref{https://docs.python.org/3.7/library/constants.html\#None}{None}

\end{description}\end{quote}

\end{fulllineitems}

\index{discard\_part() (Node method)@\spxentry{discard\_part()}\spxextra{Node method}}

\begin{fulllineitems}
\phantomsection\label{\detokenize{app.domain:app.domain.network_nodes.Node.discard_part}}\pysiglinewithargsret{\sphinxbfcode{\sphinxupquote{discard\_part}}}{\emph{\DUrole{n}{fid}}, \emph{\DUrole{n}{number}}, \emph{\DUrole{n}{corrupt}\DUrole{o}{=}\DUrole{default_value}{False}}, \emph{\DUrole{n}{cluster}\DUrole{o}{=}\DUrole{default_value}{None}}}{}
Safely deletes a part from the SGNode instance’s disk.
\begin{quote}\begin{description}
\item[{Parameters}] \leavevmode\begin{itemize}
\item {} 
\sphinxstyleliteralstrong{\sphinxupquote{fid}} (\sphinxhref{https://docs.python.org/3.7/library/stdtypes.html\#str}{\sphinxstyleliteralemphasis{\sphinxupquote{str}}}) \textendash{} Name of the file the file block replica belongs to.

\item {} 
\sphinxstyleliteralstrong{\sphinxupquote{number}} (\sphinxhref{https://docs.python.org/3.7/library/functions.html\#int}{\sphinxstyleliteralemphasis{\sphinxupquote{int}}}) \textendash{} The part number that uniquely identifies the file block.

\item {} 
\sphinxstyleliteralstrong{\sphinxupquote{corrupt}} (\sphinxhref{https://docs.python.org/3.7/library/functions.html\#bool}{\sphinxstyleliteralemphasis{\sphinxupquote{bool}}}) \textendash{} If discard is being invoked due to identified file
block corruption, e.g., Sha256 does not match the expected.

\item {} 
\sphinxstyleliteralstrong{\sphinxupquote{cluster}} ({\hyperref[\detokenize{app:app.type_hints.ClusterType}]{\sphinxcrossref{\sphinxcode{\sphinxupquote{ClusterType}}}}}) \textendash{} {\hyperref[\detokenize{app.domain:app.domain.cluster_groups.Cluster}]{\sphinxcrossref{\sphinxcode{\sphinxupquote{Cluster}}}}} that
will {\hyperref[\detokenize{app.domain:app.domain.cluster_groups.Cluster.set_replication_epoch}]{\sphinxcrossref{\sphinxcode{\sphinxupquote{set the replication epoch}}}}}
or mark the simulation as failed.

\end{itemize}

\item[{Return type}] \leavevmode
\sphinxhref{https://docs.python.org/3.7/library/constants.html\#None}{None}

\end{description}\end{quote}

\end{fulllineitems}

\index{execute\_epoch() (Node method)@\spxentry{execute\_epoch()}\spxextra{Node method}}

\begin{fulllineitems}
\phantomsection\label{\detokenize{app.domain:app.domain.network_nodes.Node.execute_epoch}}\pysiglinewithargsret{\sphinxbfcode{\sphinxupquote{execute\_epoch}}}{\emph{\DUrole{n}{cluster}}, \emph{\DUrole{n}{fid}}}{}
Instructs the \sphinxcode{\sphinxupquote{Node}} instance to execute the epoch.
\begin{quote}\begin{description}
\item[{Parameters}] \leavevmode\begin{itemize}
\item {} 
\sphinxstyleliteralstrong{\sphinxupquote{cluster}} ({\hyperref[\detokenize{app:app.type_hints.ClusterType}]{\sphinxcrossref{\sphinxcode{\sphinxupquote{ClusterType}}}}}) \textendash{} A reference to the
{\hyperref[\detokenize{app.domain:app.domain.cluster_groups.Cluster}]{\sphinxcrossref{\sphinxcode{\sphinxupquote{Cluster}}}}} that invoked
the \sphinxcode{\sphinxupquote{Node}} method.

\item {} 
\sphinxstyleliteralstrong{\sphinxupquote{fid}} (\sphinxhref{https://docs.python.org/3.7/library/stdtypes.html\#str}{\sphinxstyleliteralemphasis{\sphinxupquote{str}}}) \textendash{} The {\hyperref[\detokenize{app.domain.helpers:app.domain.helpers.smart_dataclasses.FileData.name}]{\sphinxcrossref{\sphinxcode{\sphinxupquote{file name identifier}}}}}
of the file being simulated.

\end{itemize}

\item[{Raises}] \leavevmode
\sphinxhref{https://docs.python.org/3.7/library/exceptions.html\#NotImplementedError}{\sphinxstyleliteralstrong{\sphinxupquote{NotImplementedError}}} \textendash{} When children of \sphinxcode{\sphinxupquote{Node}} do not implement the abstract method.

\item[{Return type}] \leavevmode
\sphinxhref{https://docs.python.org/3.7/library/constants.html\#None}{None}

\end{description}\end{quote}

\end{fulllineitems}

\index{get\_file\_parts() (Node method)@\spxentry{get\_file\_parts()}\spxextra{Node method}}

\begin{fulllineitems}
\phantomsection\label{\detokenize{app.domain:app.domain.network_nodes.Node.get_file_parts}}\pysiglinewithargsret{\sphinxbfcode{\sphinxupquote{get\_file\_parts}}}{\emph{\DUrole{n}{fid}}}{}
Gets collection of file parts that correspond to the named file.
\begin{quote}\begin{description}
\item[{Parameters}] \leavevmode
\sphinxstyleliteralstrong{\sphinxupquote{fid}} (\sphinxhref{https://docs.python.org/3.7/library/stdtypes.html\#str}{\sphinxstyleliteralemphasis{\sphinxupquote{str}}}) \textendash{} The {\hyperref[\detokenize{app.domain.helpers:app.domain.helpers.smart_dataclasses.FileData.name}]{\sphinxcrossref{\sphinxcode{\sphinxupquote{file name identifier}}}}} that
designates the {\hyperref[\detokenize{app.domain.helpers:app.domain.helpers.smart_dataclasses.FileBlockData}]{\sphinxcrossref{\sphinxcode{\sphinxupquote{file block replicas}}}}}
to be retrieved.

\item[{Returns}] \leavevmode
A dictionary where keys are {\hyperref[\detokenize{app.domain.helpers:app.domain.helpers.smart_dataclasses.FileBlockData.number}]{\sphinxcrossref{\sphinxcode{\sphinxupquote{file block numbers}}}}} and
values are {\hyperref[\detokenize{app.domain.helpers:app.domain.helpers.smart_dataclasses.FileBlockData}]{\sphinxcrossref{\sphinxcode{\sphinxupquote{file block replicas}}}}}

\item[{Return type}] \leavevmode
{\hyperref[\detokenize{app:app.type_hints.ReplicasDict}]{\sphinxcrossref{\sphinxcode{\sphinxupquote{ReplicasDict}}}}}

\end{description}\end{quote}

\end{fulllineitems}

\index{get\_file\_parts\_count() (Node method)@\spxentry{get\_file\_parts\_count()}\spxextra{Node method}}

\begin{fulllineitems}
\phantomsection\label{\detokenize{app.domain:app.domain.network_nodes.Node.get_file_parts_count}}\pysiglinewithargsret{\sphinxbfcode{\sphinxupquote{get\_file\_parts\_count}}}{\emph{\DUrole{n}{fid}}}{}
Counts the number of file block replicas of a specific file owned
by the \sphinxcode{\sphinxupquote{Node}}.
\begin{quote}\begin{description}
\item[{Parameters}] \leavevmode
\sphinxstyleliteralstrong{\sphinxupquote{fid}} (\sphinxhref{https://docs.python.org/3.7/library/stdtypes.html\#str}{\sphinxstyleliteralemphasis{\sphinxupquote{str}}}) \textendash{} The {\hyperref[\detokenize{app.domain.helpers:app.domain.helpers.smart_dataclasses.FileData.name}]{\sphinxcrossref{\sphinxcode{\sphinxupquote{file name identifier}}}}} that
designates the {\hyperref[\detokenize{app.domain.helpers:app.domain.helpers.smart_dataclasses.FileBlockData}]{\sphinxcrossref{\sphinxcode{\sphinxupquote{file block replicas}}}}}
to be counted.

\item[{Returns}] \leavevmode
The number of counted replicas.

\item[{Return type}] \leavevmode
\sphinxhref{https://docs.python.org/3.7/library/functions.html\#int}{int}

\end{description}\end{quote}

\end{fulllineitems}

\index{is\_suspect() (Node method)@\spxentry{is\_suspect()}\spxextra{Node method}}

\begin{fulllineitems}
\phantomsection\label{\detokenize{app.domain:app.domain.network_nodes.Node.is_suspect}}\pysiglinewithargsret{\sphinxbfcode{\sphinxupquote{is\_suspect}}}{}{}
Returns \sphinxcode{\sphinxupquote{True}} if the node is behaving suspiciously,
else \sphinxcode{\sphinxupquote{False}}.
\begin{quote}\begin{description}
\item[{Return type}] \leavevmode
\sphinxhref{https://docs.python.org/3.7/library/functions.html\#bool}{bool}

\end{description}\end{quote}

\end{fulllineitems}

\index{is\_up() (Node method)@\spxentry{is\_up()}\spxextra{Node method}}

\begin{fulllineitems}
\phantomsection\label{\detokenize{app.domain:app.domain.network_nodes.Node.is_up}}\pysiglinewithargsret{\sphinxbfcode{\sphinxupquote{is\_up}}}{}{}
Returns \sphinxcode{\sphinxupquote{True}} if the node is online, else \sphinxcode{\sphinxupquote{False}}.
\begin{quote}\begin{description}
\item[{Return type}] \leavevmode
\sphinxhref{https://docs.python.org/3.7/library/functions.html\#bool}{bool}

\end{description}\end{quote}

\end{fulllineitems}

\index{receive\_part() (Node method)@\spxentry{receive\_part()}\spxextra{Node method}}

\begin{fulllineitems}
\phantomsection\label{\detokenize{app.domain:app.domain.network_nodes.Node.receive_part}}\pysiglinewithargsret{\sphinxbfcode{\sphinxupquote{receive\_part}}}{\emph{\DUrole{n}{replica}}}{}
Endpoint for file block replica reception.

The \sphinxcode{\sphinxupquote{Node}} stores a new {\hyperref[\detokenize{app.domain.helpers:app.domain.helpers.smart_dataclasses.FileBlockData}]{\sphinxcrossref{\sphinxcode{\sphinxupquote{file block replica}}}}} in
{\hyperref[\detokenize{app.domain:app.domain.network_nodes.Node.files}]{\sphinxcrossref{\sphinxcode{\sphinxupquote{files}}}}} if he does not have a replica with same
{\hyperref[\detokenize{app.domain.helpers:app.domain.helpers.smart_dataclasses.FileBlockData.id}]{\sphinxcrossref{\sphinxcode{\sphinxupquote{identifier}}}}}.
\begin{quote}\begin{description}
\item[{Parameters}] \leavevmode
\sphinxstyleliteralstrong{\sphinxupquote{replica}} ({\hyperref[\detokenize{app.domain.helpers:app.domain.helpers.smart_dataclasses.FileBlockData}]{\sphinxcrossref{\sphinxstyleliteralemphasis{\sphinxupquote{domain.helpers.smart\_dataclasses.FileBlockData}}}}}) \textendash{} The {\hyperref[\detokenize{app.domain.helpers:app.domain.helpers.smart_dataclasses.FileBlockData}]{\sphinxcrossref{\sphinxcode{\sphinxupquote{file block replica}}}}} to be
received by \sphinxcode{\sphinxupquote{Node}}.

\item[{Returns}] \leavevmode
If upon integrity verification the \sphinxcode{\sphinxupquote{sha256}}
hashvalue differs from the expected, the worker replies with
a BAD\_REQUEST. If the \sphinxcode{\sphinxupquote{Node}} already owns a replica with the
same {\hyperref[\detokenize{app.domain.helpers:app.domain.helpers.smart_dataclasses.FileBlockData.id}]{\sphinxcrossref{\sphinxcode{\sphinxupquote{identifier}}}}} it
replies with NOT\_ACCEPTABLE. Otherwise it replies with a OK,
i.e., the delivery is successful.

\item[{Return type}] \leavevmode
{\hyperref[\detokenize{app.domain.helpers:app.domain.helpers.enums.HttpCodes}]{\sphinxcrossref{\sphinxcode{\sphinxupquote{HttpCodes}}}}}

\end{description}\end{quote}

\end{fulllineitems}

\index{replicate\_part() (Node method)@\spxentry{replicate\_part()}\spxextra{Node method}}

\begin{fulllineitems}
\phantomsection\label{\detokenize{app.domain:app.domain.network_nodes.Node.replicate_part}}\pysiglinewithargsret{\sphinxbfcode{\sphinxupquote{replicate\_part}}}{\emph{\DUrole{n}{cluster}}, \emph{\DUrole{n}{replica}}}{}
Attempts to restore the replication level of the specified file
block replica.

Similar to {\hyperref[\detokenize{app.domain:app.domain.network_nodes.Node.send_part}]{\sphinxcrossref{\sphinxcode{\sphinxupquote{send\_part()}}}}} but with
slightly different instructions. In particular new \sphinxcode{\sphinxupquote{replicas}}
can not be corrupted at the current node, at the current epoch.

\begin{sphinxadmonition}{note}{Note:}
There are no guarantees that
{\hyperref[\detokenize{app:app.environment_settings.REPLICATION_LEVEL}]{\sphinxcrossref{\sphinxcode{\sphinxupquote{REPLICATION\_LEVEL}}}}} will be
completely restored during the execution of this method.
\end{sphinxadmonition}
\begin{quote}\begin{description}
\item[{Parameters}] \leavevmode\begin{itemize}
\item {} 
\sphinxstyleliteralstrong{\sphinxupquote{cluster}} ({\hyperref[\detokenize{app:app.type_hints.ClusterType}]{\sphinxcrossref{\sphinxcode{\sphinxupquote{ClusterType}}}}}) \textendash{} A reference to the
{\hyperref[\detokenize{app.domain:app.domain.cluster_groups.Cluster}]{\sphinxcrossref{\sphinxcode{\sphinxupquote{Cluster}}}}} that will
deliver the new \sphinxcode{\sphinxupquote{replica}}.

\item {} 
\sphinxstyleliteralstrong{\sphinxupquote{replica}} ({\hyperref[\detokenize{app.domain.helpers:app.domain.helpers.smart_dataclasses.FileBlockData}]{\sphinxcrossref{\sphinxcode{\sphinxupquote{FileBlockData}}}}}) \textendash{} The {\hyperref[\detokenize{app.domain.helpers:app.domain.helpers.smart_dataclasses.FileBlockData}]{\sphinxcrossref{\sphinxcode{\sphinxupquote{file block replica}}}}}
to be delivered.

\end{itemize}

\item[{Raises}] \leavevmode
\sphinxhref{https://docs.python.org/3.7/library/exceptions.html\#NotImplementedError}{\sphinxstyleliteralstrong{\sphinxupquote{NotImplementedError}}} \textendash{} When children of \sphinxcode{\sphinxupquote{Node}} do not implement the abstract method.

\item[{Return type}] \leavevmode
\sphinxhref{https://docs.python.org/3.7/library/constants.html\#None}{None}

\end{description}\end{quote}

\end{fulllineitems}

\index{send\_part() (Node method)@\spxentry{send\_part()}\spxextra{Node method}}

\begin{fulllineitems}
\phantomsection\label{\detokenize{app.domain:app.domain.network_nodes.Node.send_part}}\pysiglinewithargsret{\sphinxbfcode{\sphinxupquote{send\_part}}}{\emph{\DUrole{n}{cluster}}, \emph{\DUrole{n}{destination}}, \emph{\DUrole{n}{replica}}}{}
Attempts to send a replica to some other network node.
\begin{quote}\begin{description}
\item[{Parameters}] \leavevmode\begin{itemize}
\item {} 
\sphinxstyleliteralstrong{\sphinxupquote{cluster}} ({\hyperref[\detokenize{app:app.type_hints.ClusterType}]{\sphinxcrossref{\sphinxcode{\sphinxupquote{ClusterType}}}}}) \textendash{} A reference to the
{\hyperref[\detokenize{app.domain:app.domain.cluster_groups.Cluster}]{\sphinxcrossref{\sphinxcode{\sphinxupquote{Cluster}}}}} that will
deliver the new \sphinxcode{\sphinxupquote{replica}}. In a real
world implementation this argument would not make sense,
but we use it to facilitate simulation management and
environment logging.

\item {} 
\sphinxstyleliteralstrong{\sphinxupquote{destination}} (\sphinxhref{https://docs.python.org/3.7/library/stdtypes.html\#str}{\sphinxstyleliteralemphasis{\sphinxupquote{str}}}) \textendash{} The name, address or another unique identifier of the node
that will receive the file block \sphinxtitleref{replica}.

\item {} 
\sphinxstyleliteralstrong{\sphinxupquote{replica}} ({\hyperref[\detokenize{app.domain.helpers:app.domain.helpers.smart_dataclasses.FileBlockData}]{\sphinxcrossref{\sphinxcode{\sphinxupquote{FileBlockData}}}}}) \textendash{} The file block container to be sent to some other worker.

\end{itemize}

\item[{Returns}] \leavevmode
An http code.

\item[{Return type}] \leavevmode
{\hyperref[\detokenize{app.domain.helpers:app.domain.helpers.enums.HttpCodes}]{\sphinxcrossref{\sphinxcode{\sphinxupquote{HttpCodes}}}}}

\end{description}\end{quote}

\end{fulllineitems}

\index{update\_status() (Node method)@\spxentry{update\_status()}\spxextra{Node method}}

\begin{fulllineitems}
\phantomsection\label{\detokenize{app.domain:app.domain.network_nodes.Node.update_status}}\pysiglinewithargsret{\sphinxbfcode{\sphinxupquote{update\_status}}}{}{}
Used to update the time to live of the node instance.

When invoked, the network node decides if it should remain online or
change some other state.
\begin{quote}\begin{description}
\item[{Returns}] \leavevmode
The the status of the \sphinxcode{\sphinxupquote{Node}}.

\item[{Return type}] \leavevmode
{\hyperref[\detokenize{app.domain.helpers:app.domain.helpers.enums.Status}]{\sphinxcrossref{\sphinxcode{\sphinxupquote{Status}}}}}

\end{description}\end{quote}

\end{fulllineitems}


\end{fulllineitems}

\index{SGNode (class in app.domain.network\_nodes)@\spxentry{SGNode}\spxextra{class in app.domain.network\_nodes}}

\begin{fulllineitems}
\phantomsection\label{\detokenize{app.domain:app.domain.network_nodes.SGNode}}\pysiglinewithargsret{\sphinxbfcode{\sphinxupquote{class }}\sphinxbfcode{\sphinxupquote{SGNode}}}{\emph{\DUrole{n}{uid}}, \emph{\DUrole{n}{uptime}}}{}
Bases: {\hyperref[\detokenize{app.domain:app.domain.network_nodes.Node}]{\sphinxcrossref{\sphinxcode{\sphinxupquote{app.domain.network\_nodes.Node}}}}}

Represents a network node that executes a Swarm Guidance algorithm.
\index{clusters (SGNode attribute)@\spxentry{clusters}\spxextra{SGNode attribute}}

\begin{fulllineitems}
\phantomsection\label{\detokenize{app.domain:app.domain.network_nodes.SGNode.clusters}}\pysigline{\sphinxbfcode{\sphinxupquote{clusters}}}
A collection of {\hyperref[\detokenize{app.domain:app.domain.cluster_groups.SGCluster}]{\sphinxcrossref{\sphinxcode{\sphinxupquote{cluster groups}}}}} the \sphinxcode{\sphinxupquote{SGNode}} is a
member of.

\end{fulllineitems}

\index{routing\_table (SGNode attribute)@\spxentry{routing\_table}\spxextra{SGNode attribute}}

\begin{fulllineitems}
\phantomsection\label{\detokenize{app.domain:app.domain.network_nodes.SGNode.routing_table}}\pysigline{\sphinxbfcode{\sphinxupquote{routing\_table}}}
Contains the information required to appropriately route file
block blocks to other SGNode instances.
\begin{quote}\begin{description}
\item[{Type}] \leavevmode
Dict{[}str, \sphinxhref{https://pandas.pydata.org/docs/reference/api/pandas.DataFrame.html\#pandas.DataFrame}{\sphinxcode{\sphinxupquote{DataFrame}}}{]}

\end{description}\end{quote}

\end{fulllineitems}

\index{\_\_init\_\_() (SGNode method)@\spxentry{\_\_init\_\_()}\spxextra{SGNode method}}

\begin{fulllineitems}
\phantomsection\label{\detokenize{app.domain:app.domain.network_nodes.SGNode.__init__}}\pysiglinewithargsret{\sphinxbfcode{\sphinxupquote{\_\_init\_\_}}}{\emph{\DUrole{n}{uid}}, \emph{\DUrole{n}{uptime}}}{}
Instantiates a \sphinxcode{\sphinxupquote{Node}} object.

These are network nodes responsible for persisting
\sphinxcode{\sphinxupquote{file block replicas}}.
\begin{quote}\begin{description}
\item[{Parameters}] \leavevmode\begin{itemize}
\item {} 
\sphinxstyleliteralstrong{\sphinxupquote{uid}} (\sphinxhref{https://docs.python.org/3.7/library/stdtypes.html\#str}{\sphinxstyleliteralemphasis{\sphinxupquote{str}}}) \textendash{} An unique identifier for the \sphinxcode{\sphinxupquote{Node}} instance.

\item {} 
\sphinxstyleliteralstrong{\sphinxupquote{uptime}} (\sphinxhref{https://docs.python.org/3.7/library/functions.html\#float}{\sphinxstyleliteralemphasis{\sphinxupquote{float}}}) \textendash{} The availability of the \sphinxcode{\sphinxupquote{Node}} instance.

\end{itemize}

\item[{Return type}] \leavevmode
\sphinxhref{https://docs.python.org/3.7/library/constants.html\#None}{None}

\end{description}\end{quote}

\end{fulllineitems}

\index{execute\_epoch() (SGNode method)@\spxentry{execute\_epoch()}\spxextra{SGNode method}}

\begin{fulllineitems}
\phantomsection\label{\detokenize{app.domain:app.domain.network_nodes.SGNode.execute_epoch}}\pysiglinewithargsret{\sphinxbfcode{\sphinxupquote{execute\_epoch}}}{\emph{\DUrole{n}{cluster}}, \emph{\DUrole{n}{fid}}}{}
Instructs the \sphinxcode{\sphinxupquote{Node}} instance to execute the epoch.

The method iterates all file block blocks in \sphinxcode{\sphinxupquote{files}} and
independently decides if they should be sent to another \sphinxcode{\sphinxupquote{SGNode}}
by following the probabilities in {\hyperref[\detokenize{app.domain:app.domain.network_nodes.SGNode.routing_table}]{\sphinxcrossref{\sphinxcode{\sphinxupquote{routing\_table}}}}} column
vectors.
\begin{description}
\item[{Overrides:}] \leavevmode
{\hyperref[\detokenize{app.domain:app.domain.network_nodes.Node.execute_epoch}]{\sphinxcrossref{\sphinxcode{\sphinxupquote{app.domain.network\_nodes.Node.execute\_epoch()}}}}}.

\end{description}
\begin{quote}\begin{description}
\item[{Parameters}] \leavevmode\begin{itemize}
\item {} 
\sphinxstyleliteralstrong{\sphinxupquote{cluster}} ({\hyperref[\detokenize{app:app.type_hints.ClusterType}]{\sphinxcrossref{\sphinxcode{\sphinxupquote{ClusterType}}}}}) \textendash{} A reference to the
{\hyperref[\detokenize{app.domain:app.domain.cluster_groups.Cluster}]{\sphinxcrossref{\sphinxcode{\sphinxupquote{Cluster}}}}} that invoked
the \sphinxcode{\sphinxupquote{Node}} method.

\item {} 
\sphinxstyleliteralstrong{\sphinxupquote{fid}} (\sphinxhref{https://docs.python.org/3.7/library/stdtypes.html\#str}{\sphinxstyleliteralemphasis{\sphinxupquote{str}}}) \textendash{} The {\hyperref[\detokenize{app.domain.helpers:app.domain.helpers.smart_dataclasses.FileData.name}]{\sphinxcrossref{\sphinxcode{\sphinxupquote{file name identifier}}}}}
of the file being simulated.

\end{itemize}

\item[{Return type}] \leavevmode
\sphinxhref{https://docs.python.org/3.7/library/constants.html\#None}{None}

\end{description}\end{quote}

\end{fulllineitems}

\index{remove\_file\_routing() (SGNode method)@\spxentry{remove\_file\_routing()}\spxextra{SGNode method}}

\begin{fulllineitems}
\phantomsection\label{\detokenize{app.domain:app.domain.network_nodes.SGNode.remove_file_routing}}\pysiglinewithargsret{\sphinxbfcode{\sphinxupquote{remove\_file\_routing}}}{\emph{\DUrole{n}{fid}}}{}
Removes a file name from the \sphinxcode{\sphinxupquote{SGNode}} routing table.

This method is called when a \sphinxcode{\sphinxupquote{SGNode}} is evicted from the
{\hyperref[\detokenize{app.domain:app.domain.cluster_groups.SGCluster}]{\sphinxcrossref{\sphinxcode{\sphinxupquote{cluster group}}}}} and
results in the deletion from disk of all {\hyperref[\detokenize{app.domain.helpers:app.domain.helpers.smart_dataclasses.FileBlockData}]{\sphinxcrossref{\sphinxcode{\sphinxupquote{file block replicas}}}}} with
identifier \sphinxcode{\sphinxupquote{fid}}.
\begin{quote}\begin{description}
\item[{Parameters}] \leavevmode
\sphinxstyleliteralstrong{\sphinxupquote{fid}} (\sphinxhref{https://docs.python.org/3.7/library/stdtypes.html\#str}{\sphinxstyleliteralemphasis{\sphinxupquote{str}}}) \textendash{} The {\hyperref[\detokenize{app.domain.helpers:app.domain.helpers.smart_dataclasses.FileData.name}]{\sphinxcrossref{\sphinxcode{\sphinxupquote{file name identifier}}}}}
of the file whose routing is being eliminated.

\item[{Return type}] \leavevmode
\sphinxhref{https://docs.python.org/3.7/library/constants.html\#None}{None}

\end{description}\end{quote}

\end{fulllineitems}

\index{replicate\_part() (SGNode method)@\spxentry{replicate\_part()}\spxextra{SGNode method}}

\begin{fulllineitems}
\phantomsection\label{\detokenize{app.domain:app.domain.network_nodes.SGNode.replicate_part}}\pysiglinewithargsret{\sphinxbfcode{\sphinxupquote{replicate\_part}}}{\emph{\DUrole{n}{cluster}}, \emph{\DUrole{n}{replica}}}{}
Attempts to restore the replication level of the specified file
block replica.

Similar to {\hyperref[\detokenize{app.domain:app.domain.network_nodes.Node.send_part}]{\sphinxcrossref{\sphinxcode{\sphinxupquote{send\_part()}}}}} but with
slightly different instructions. In particular new \sphinxcode{\sphinxupquote{replicas}}
can not be corrupted at the current node, at the current epoch. The
replicas are also sent selectively in descending order to the
most reliable Nodes in the \sphinxcode{\sphinxupquote{Cluster}} down to the least
reliable. Whereas \sphinxcode{\sphinxupquote{send\_part()}}. follows
stochastic swarm guidance routing.
\begin{description}
\item[{Overrides:}] \leavevmode
{\hyperref[\detokenize{app.domain:app.domain.network_nodes.Node.replicate_part}]{\sphinxcrossref{\sphinxcode{\sphinxupquote{app.domain.network\_nodes.Node.replicate\_part()}}}}}.

\end{description}

\begin{sphinxadmonition}{note}{Note:}
There are no guarantees that
{\hyperref[\detokenize{app:app.environment_settings.REPLICATION_LEVEL}]{\sphinxcrossref{\sphinxcode{\sphinxupquote{REPLICATION\_LEVEL}}}}} will be
completely restored during the execution of this method.
\end{sphinxadmonition}
\begin{quote}\begin{description}
\item[{Parameters}] \leavevmode\begin{itemize}
\item {} 
\sphinxstyleliteralstrong{\sphinxupquote{cluster}} ({\hyperref[\detokenize{app:app.type_hints.ClusterType}]{\sphinxcrossref{\sphinxcode{\sphinxupquote{ClusterType}}}}}) \textendash{} A reference to the
{\hyperref[\detokenize{app.domain:app.domain.cluster_groups.Cluster}]{\sphinxcrossref{\sphinxcode{\sphinxupquote{Cluster}}}}} that will
deliver the new \sphinxcode{\sphinxupquote{replica}}.

\item {} 
\sphinxstyleliteralstrong{\sphinxupquote{replica}} ({\hyperref[\detokenize{app.domain.helpers:app.domain.helpers.smart_dataclasses.FileBlockData}]{\sphinxcrossref{\sphinxcode{\sphinxupquote{FileBlockData}}}}}) \textendash{} The {\hyperref[\detokenize{app.domain.helpers:app.domain.helpers.smart_dataclasses.FileBlockData}]{\sphinxcrossref{\sphinxcode{\sphinxupquote{file block replica}}}}}
to be delivered.

\end{itemize}

\item[{Return type}] \leavevmode
\sphinxhref{https://docs.python.org/3.7/library/constants.html\#None}{None}

\end{description}\end{quote}

\end{fulllineitems}

\index{select\_destination() (SGNode method)@\spxentry{select\_destination()}\spxextra{SGNode method}}

\begin{fulllineitems}
\phantomsection\label{\detokenize{app.domain:app.domain.network_nodes.SGNode.select_destination}}\pysiglinewithargsret{\sphinxbfcode{\sphinxupquote{select\_destination}}}{\emph{\DUrole{n}{fid}}}{}
Selects a random message destination according to \sphinxtitleref{routing\_table}
probabilities for the specified file name.
\begin{quote}\begin{description}
\item[{Parameters}] \leavevmode
\sphinxstyleliteralstrong{\sphinxupquote{fid}} (\sphinxhref{https://docs.python.org/3.7/library/stdtypes.html\#str}{\sphinxstyleliteralemphasis{\sphinxupquote{str}}}) \textendash{} The {\hyperref[\detokenize{app.domain.helpers:app.domain.helpers.smart_dataclasses.FileData.name}]{\sphinxcrossref{\sphinxcode{\sphinxupquote{file name identifier}}}}}
to obtain the proper {\hyperref[\detokenize{app.domain:app.domain.network_nodes.SGNode.routing_table}]{\sphinxcrossref{\sphinxcode{\sphinxupquote{routing\_table}}}}} for
destination selection.

\item[{Returns}] \leavevmode
The name or address of the selected destination.

\item[{Return type}] \leavevmode
\sphinxhref{https://docs.python.org/3.7/library/stdtypes.html\#str}{str}

\end{description}\end{quote}

\end{fulllineitems}

\index{set\_file\_routing() (SGNode method)@\spxentry{set\_file\_routing()}\spxextra{SGNode method}}

\begin{fulllineitems}
\phantomsection\label{\detokenize{app.domain:app.domain.network_nodes.SGNode.set_file_routing}}\pysiglinewithargsret{\sphinxbfcode{\sphinxupquote{set\_file\_routing}}}{\emph{\DUrole{n}{fid}}, \emph{\DUrole{n}{v\_}}}{}
Maps a file name identifier with a transition column vector used
for file block replica routing.
\begin{quote}\begin{description}
\item[{Parameters}] \leavevmode\begin{itemize}
\item {} 
\sphinxstyleliteralstrong{\sphinxupquote{fid}} (\sphinxhref{https://docs.python.org/3.7/library/stdtypes.html\#str}{\sphinxstyleliteralemphasis{\sphinxupquote{str}}}) \textendash{} The {\hyperref[\detokenize{app.domain.helpers:app.domain.helpers.smart_dataclasses.FileData.name}]{\sphinxcrossref{\sphinxcode{\sphinxupquote{file name identifier}}}}}
of the file whose routing is being configured.

\item {} 
\sphinxstyleliteralstrong{\sphinxupquote{v\_}} (Union{[}\sphinxhref{https://pandas.pydata.org/docs/reference/api/pandas.Series.html\#pandas.Series}{\sphinxcode{\sphinxupquote{Series}}}, \sphinxhref{https://pandas.pydata.org/docs/reference/api/pandas.DataFrame.html\#pandas.DataFrame}{\sphinxcode{\sphinxupquote{DataFrame}}}{]}) \textendash{} A column vector with probabilities that dictate the odds of
sending file block blocks belonging to the file with
specified id to other Cluster members also working on the
persistence of the file block blocks.

\end{itemize}

\item[{Raises}] \leavevmode
\sphinxhref{https://docs.python.org/3.7/library/exceptions.html\#ValueError}{\sphinxstyleliteralstrong{\sphinxupquote{ValueError}}} \textendash{} If \sphinxcode{\sphinxupquote{transition\_vector}} is not a
    \sphinxhref{https://pandas.pydata.org/docs/reference/api/pandas.DataFrame.html\#pandas.DataFrame}{\sphinxcode{\sphinxupquote{DataFrame}}} and cannot be casted to it.

\item[{Return type}] \leavevmode
\sphinxhref{https://docs.python.org/3.7/library/constants.html\#None}{None}

\end{description}\end{quote}

\end{fulllineitems}


\end{fulllineitems}

\index{SGNodeExt (class in app.domain.network\_nodes)@\spxentry{SGNodeExt}\spxextra{class in app.domain.network\_nodes}}

\begin{fulllineitems}
\phantomsection\label{\detokenize{app.domain:app.domain.network_nodes.SGNodeExt}}\pysiglinewithargsret{\sphinxbfcode{\sphinxupquote{class }}\sphinxbfcode{\sphinxupquote{SGNodeExt}}}{\emph{\DUrole{n}{uid}}, \emph{\DUrole{n}{uptime}}}{}
Bases: {\hyperref[\detokenize{app.domain:app.domain.network_nodes.SGNode}]{\sphinxcrossref{\sphinxcode{\sphinxupquote{app.domain.network\_nodes.SGNode}}}}}

Represents a network node that executes a Swarm Guidance algorithm.

\sphinxcode{\sphinxupquote{SGNodeExt}} instances differ from {\hyperref[\detokenize{app.domain:app.domain.network_nodes.SGNode}]{\sphinxcrossref{\sphinxcode{\sphinxupquote{SGNode}}}}} in the sense
that the latter does not monitor the peers belonging to his
{\hyperref[\detokenize{app.domain:app.domain.cluster_groups.SGClusterExt}]{\sphinxcrossref{\sphinxcode{\sphinxupquote{cluster groups}}}}},
concerning their connectivity {\hyperref[\detokenize{app.domain:app.domain.network_nodes.Node.status}]{\sphinxcrossref{\sphinxcode{\sphinxupquote{status}}}}} or suspicious
behaviours.
\index{update\_status() (SGNodeExt method)@\spxentry{update\_status()}\spxextra{SGNodeExt method}}

\begin{fulllineitems}
\phantomsection\label{\detokenize{app.domain:app.domain.network_nodes.SGNodeExt.update_status}}\pysiglinewithargsret{\sphinxbfcode{\sphinxupquote{update\_status}}}{}{}
Used to update the time to live of the node instance.

When invoked, the network node decides if it should remain online or
change some other state.
\begin{description}
\item[{Overrides:}] \leavevmode
{\hyperref[\detokenize{app.domain:app.domain.network_nodes.Node.update_status}]{\sphinxcrossref{\sphinxcode{\sphinxupquote{app.domain.network\_nodes.Node.update\_status()}}}}}.

\end{description}
\begin{quote}\begin{description}
\item[{Returns}] \leavevmode
The the status of the \sphinxcode{\sphinxupquote{Node}}.

\item[{Return type}] \leavevmode
{\hyperref[\detokenize{app.domain.helpers:app.domain.helpers.enums.Status}]{\sphinxcrossref{\sphinxcode{\sphinxupquote{Status}}}}}

\end{description}\end{quote}

\end{fulllineitems}


\end{fulllineitems}

\index{\_NetworkView (in module app.domain.network\_nodes)@\spxentry{\_NetworkView}\spxextra{in module app.domain.network\_nodes}}

\begin{fulllineitems}
\phantomsection\label{\detokenize{app.domain:app.domain.network_nodes._NetworkView}}\pysigline{\sphinxbfcode{\sphinxupquote{\_NetworkView}}\sphinxbfcode{\sphinxupquote{: Dict\DUrole{p}{{[}}Union\DUrole{p}{{[}}\sphinxhref{https://docs.python.org/3.7/library/stdtypes.html\#str}{str}\DUrole{p}{, }{\hyperref[\detokenize{app.domain:app.domain.network_nodes.Node}]{\sphinxcrossref{app.domain.network\_nodes.Node}}}\DUrole{p}{{]}}\DUrole{p}{, }\sphinxhref{https://docs.python.org/3.7/library/functions.html\#int}{int}\DUrole{p}{{]}}}}}
\end{fulllineitems}



\subsection{app.utils}
\label{\detokenize{app.utils:module-app.utils}}\label{\detokenize{app.utils:app-utils}}\label{\detokenize{app.utils::doc}}\index{module@\spxentry{module}!app.utils@\spxentry{app.utils}}\index{app.utils@\spxentry{app.utils}!module@\spxentry{module}}

\subsubsection{Submodules}
\label{\detokenize{app.utils:submodules}}

\subsubsection{app.utils.convertions}
\label{\detokenize{app.utils:module-app.utils.convertions}}\label{\detokenize{app.utils:app-utils-convertions}}\index{module@\spxentry{module}!app.utils.convertions@\spxentry{app.utils.convertions}}\index{app.utils.convertions@\spxentry{app.utils.convertions}!module@\spxentry{module}}
This module includes various data\sphinxhyphen{}type convertion utilities.

Some functions include representing a string or sequence of bytes
as base64\sphinxhyphen{}encoded strings or serialization objects into JSON strings.
\index{base64\_bytelike\_obj\_to\_bytes() (in module app.utils.convertions)@\spxentry{base64\_bytelike\_obj\_to\_bytes()}\spxextra{in module app.utils.convertions}}

\begin{fulllineitems}
\phantomsection\label{\detokenize{app.utils:app.utils.convertions.base64_bytelike_obj_to_bytes}}\pysiglinewithargsret{\sphinxbfcode{\sphinxupquote{base64\_bytelike\_obj\_to\_bytes}}}{\emph{\DUrole{n}{obj}}}{}
Converts a byte\sphinxhyphen{}like object to a sequence of bytes.
\begin{quote}\begin{description}
\item[{Parameters}] \leavevmode
\sphinxstyleliteralstrong{\sphinxupquote{obj}} (\sphinxhref{https://docs.python.org/3.7/library/stdtypes.html\#bytes}{\sphinxstyleliteralemphasis{\sphinxupquote{bytes}}}) \textendash{} The object to be converted to base64\sphinxhyphen{}encoded string.

\item[{Returns}] \leavevmode
A sequence of bytes representation of the given \sphinxcode{\sphinxupquote{obj}} or \sphinxcode{\sphinxupquote{None}}
if \sphinxcode{\sphinxupquote{obj}} is not a bytes type.

\item[{Return type}] \leavevmode
Optional{[}\sphinxhref{https://docs.python.org/3.7/library/stdtypes.html\#bytes}{bytes}{]}

\end{description}\end{quote}

\end{fulllineitems}

\index{base64\_string\_to\_bytes() (in module app.utils.convertions)@\spxentry{base64\_string\_to\_bytes()}\spxextra{in module app.utils.convertions}}

\begin{fulllineitems}
\phantomsection\label{\detokenize{app.utils:app.utils.convertions.base64_string_to_bytes}}\pysiglinewithargsret{\sphinxbfcode{\sphinxupquote{base64\_string\_to\_bytes}}}{\emph{\DUrole{n}{string}}}{}
Converts a base64 string to a sequence of bytes.
\begin{quote}\begin{description}
\item[{Parameters}] \leavevmode
\sphinxstyleliteralstrong{\sphinxupquote{string}} (\sphinxhref{https://docs.python.org/3.7/library/stdtypes.html\#str}{\sphinxstyleliteralemphasis{\sphinxupquote{str}}}) \textendash{} A base64\sphinxhyphen{}encoded string to be converted to a byte sequence.

\item[{Returns}] \leavevmode
A sequence of bytes converted from the given base64\sphinxhyphen{}encoded \sphinxcode{\sphinxupquote{string}}
or \sphinxcode{\sphinxupquote{None}} string is not a str type.

\item[{Return type}] \leavevmode
Optional{[}\sphinxhref{https://docs.python.org/3.7/library/stdtypes.html\#bytes}{bytes}{]}

\end{description}\end{quote}

\end{fulllineitems}

\index{bytes\_to\_base64\_string() (in module app.utils.convertions)@\spxentry{bytes\_to\_base64\_string()}\spxextra{in module app.utils.convertions}}

\begin{fulllineitems}
\phantomsection\label{\detokenize{app.utils:app.utils.convertions.bytes_to_base64_string}}\pysiglinewithargsret{\sphinxbfcode{\sphinxupquote{bytes\_to\_base64\_string}}}{\emph{\DUrole{n}{data}}}{}
Converts a byte sequence or non base64 string to a base64\sphinxhyphen{}encoded string.
\begin{quote}\begin{description}
\item[{Parameters}] \leavevmode
\sphinxstyleliteralstrong{\sphinxupquote{data}} (\sphinxstyleliteralemphasis{\sphinxupquote{Union}}\sphinxstyleliteralemphasis{\sphinxupquote{{[}}}\sphinxhref{https://docs.python.org/3.7/library/stdtypes.html\#bytes}{\sphinxstyleliteralemphasis{\sphinxupquote{bytes}}}\sphinxstyleliteralemphasis{\sphinxupquote{, }}\sphinxhref{https://docs.python.org/3.7/library/stdtypes.html\#str}{\sphinxstyleliteralemphasis{\sphinxupquote{str}}}\sphinxstyleliteralemphasis{\sphinxupquote{{]}}}) \textendash{} Sequence of bytes to be converted.

\item[{Returns}] \leavevmode
A base64\sphinxhyphen{}encoded string representation of \sphinxcode{\sphinxupquote{data}} or \sphinxcode{\sphinxupquote{None}} if
\sphinxcode{\sphinxupquote{data}} is not a str or bytes type.

\item[{Return type}] \leavevmode
Optional{[}\sphinxhref{https://docs.python.org/3.7/library/stdtypes.html\#str}{str}{]}

\end{description}\end{quote}

\end{fulllineitems}

\index{bytes\_to\_utf8string() (in module app.utils.convertions)@\spxentry{bytes\_to\_utf8string()}\spxextra{in module app.utils.convertions}}

\begin{fulllineitems}
\phantomsection\label{\detokenize{app.utils:app.utils.convertions.bytes_to_utf8string}}\pysiglinewithargsret{\sphinxbfcode{\sphinxupquote{bytes\_to\_utf8string}}}{\emph{\DUrole{n}{data}}}{}
Converts a sequence of bytes a utf\sphinxhyphen{}8 string.
\begin{quote}\begin{description}
\item[{Parameters}] \leavevmode
\sphinxstyleliteralstrong{\sphinxupquote{data}} (\sphinxhref{https://docs.python.org/3.7/library/stdtypes.html\#bytes}{\sphinxstyleliteralemphasis{\sphinxupquote{bytes}}}) \textendash{} Sequence of bytes to be converted.

\item[{Returns}] \leavevmode
A utf\sphinxhyphen{}8 string representation of \sphinxcode{\sphinxupquote{data}} or \sphinxcode{\sphinxupquote{None}} if \sphinxcode{\sphinxupquote{data}}
is not a bytes type.

\item[{Return type}] \leavevmode
Optional{[}\sphinxhref{https://docs.python.org/3.7/library/stdtypes.html\#str}{str}{]}

\end{description}\end{quote}

\end{fulllineitems}

\index{class\_name\_to\_obj() (in module app.utils.convertions)@\spxentry{class\_name\_to\_obj()}\spxextra{in module app.utils.convertions}}

\begin{fulllineitems}
\phantomsection\label{\detokenize{app.utils:app.utils.convertions.class_name_to_obj}}\pysiglinewithargsret{\sphinxbfcode{\sphinxupquote{class\_name\_to\_obj}}}{\emph{\DUrole{n}{module\_name}}, \emph{\DUrole{n}{class\_name}}, \emph{\DUrole{n}{args}}}{}
Uses reflection to instanciate a class by name.
\begin{quote}
\begin{description}
\item[{Examples:}] \leavevmode
The next two code snippets are equivalent:

\begin{sphinxVerbatim}[commandchars=\\\{\}]
\PYG{g+gp}{\PYGZgt{}\PYGZgt{}\PYGZgt{} }\PYG{n}{class\PYGZus{}name\PYGZus{}to\PYGZus{}obj}\PYG{p}{(}\PYG{n}{MASTER\PYGZus{}SERVERS}\PYG{p}{,} \PYG{l+s+s2}{\PYGZdq{}}\PYG{l+s+s2}{Master}\PYG{l+s+s2}{\PYGZdq{}}\PYG{p}{,} \PYG{p}{[}\PYG{l+s+s2}{\PYGZdq{}}\PYG{l+s+s2}{f.jpg}\PYG{l+s+s2}{\PYGZdq{}}\PYG{p}{,} \PYG{l+m+mi}{1}\PYG{p}{,} \PYG{l+m+mi}{80}\PYG{p}{]}\PYG{p}{)}
\end{sphinxVerbatim}

\begin{sphinxVerbatim}[commandchars=\\\{\}]
\PYG{g+gp}{\PYGZgt{}\PYGZgt{}\PYGZgt{} }\PYG{k+kn}{import} \PYG{n+nn}{app}\PYG{n+nn}{.}\PYG{n+nn}{domain}\PYG{n+nn}{.}\PYG{n+nn}{master\PYGZus{}servers} \PYG{k}{as} \PYG{n+nn}{ms}
\PYG{g+gp}{\PYGZgt{}\PYGZgt{}\PYGZgt{} }\PYG{n}{h} \PYG{o}{=} \PYG{n}{ms}\PYG{o}{.}\PYG{n}{Master}\PYG{p}{(}\PYG{l+s+s2}{\PYGZdq{}}\PYG{l+s+s2}{f.jpg}\PYG{l+s+s2}{\PYGZdq{}}\PYG{p}{,} \PYG{l+m+mi}{1}\PYG{p}{,} \PYG{l+m+mi}{80}\PYG{p}{)}
\end{sphinxVerbatim}

\end{description}
\end{quote}
\begin{quote}\begin{description}
\item[{Parameters}] \leavevmode\begin{itemize}
\item {} 
\sphinxstyleliteralstrong{\sphinxupquote{module\_name}} (\sphinxhref{https://docs.python.org/3.7/library/stdtypes.html\#str}{\sphinxstyleliteralemphasis{\sphinxupquote{str}}}) \textendash{} The fully qualified path of the module the class is defined in.
The name of the module must be included.

\item {} 
\sphinxstyleliteralstrong{\sphinxupquote{class\_name}} (\sphinxhref{https://docs.python.org/3.7/library/stdtypes.html\#str}{\sphinxstyleliteralemphasis{\sphinxupquote{str}}}) \textendash{} The name of the class to be instanciated.

\item {} 
\sphinxstyleliteralstrong{\sphinxupquote{args}} (\sphinxstyleliteralemphasis{\sphinxupquote{List}}\sphinxstyleliteralemphasis{\sphinxupquote{{[}}}\sphinxstyleliteralemphasis{\sphinxupquote{Any}}\sphinxstyleliteralemphasis{\sphinxupquote{{]}}}) \textendash{} The arguments expected by the named class as an iterable list.

\end{itemize}

\item[{Returns}] \leavevmode
An object of the named class.

\item[{Raises}] \leavevmode\begin{itemize}
\item {} 
\sphinxhref{https://docs.python.org/3.7/library/exceptions.html\#AttributeError}{\sphinxstyleliteralstrong{\sphinxupquote{AttributeError}}} \textendash{} When \sphinxcode{\sphinxupquote{class\_name}} does not exist or when \sphinxcode{\sphinxupquote{module\_name}} to be
    imported causes cyclic import errors.

\item {} 
\sphinxhref{https://docs.python.org/3.7/library/exceptions.html\#ImportError}{\sphinxstyleliteralstrong{\sphinxupquote{ImportError}}} \textendash{} When \sphinxcode{\sphinxupquote{module\_name}} is not a valid module.

\end{itemize}

\item[{Return type}] \leavevmode
Any

\end{description}\end{quote}

\end{fulllineitems}

\index{json\_string\_to\_obj() (in module app.utils.convertions)@\spxentry{json\_string\_to\_obj()}\spxextra{in module app.utils.convertions}}

\begin{fulllineitems}
\phantomsection\label{\detokenize{app.utils:app.utils.convertions.json_string_to_obj}}\pysiglinewithargsret{\sphinxbfcode{\sphinxupquote{json\_string\_to\_obj}}}{\emph{\DUrole{n}{json\_string}}}{}
Deserializes a JSON string to a a python object.
\begin{quote}\begin{description}
\item[{Parameters}] \leavevmode
\sphinxstyleliteralstrong{\sphinxupquote{json\_string}} (\sphinxhref{https://docs.python.org/3.7/library/stdtypes.html\#str}{\sphinxstyleliteralemphasis{\sphinxupquote{str}}}) \textendash{} The string to be deserialized into a python object.

\item[{Returns}] \leavevmode
A python object obtained from the processing \sphinxcode{\sphinxupquote{json\_string}}.

\item[{Return type}] \leavevmode
Any

\end{description}\end{quote}

\end{fulllineitems}

\index{obj\_to\_json\_string() (in module app.utils.convertions)@\spxentry{obj\_to\_json\_string()}\spxextra{in module app.utils.convertions}}

\begin{fulllineitems}
\phantomsection\label{\detokenize{app.utils:app.utils.convertions.obj_to_json_string}}\pysiglinewithargsret{\sphinxbfcode{\sphinxupquote{obj\_to\_json\_string}}}{\emph{\DUrole{n}{obj}}}{}
Serializes a python object to a JSON string.
\begin{quote}\begin{description}
\item[{Parameters}] \leavevmode
\sphinxstyleliteralstrong{\sphinxupquote{obj}} (\sphinxstyleliteralemphasis{\sphinxupquote{Any}}) \textendash{} The object to be serialized.

\item[{Returns}] \leavevmode
A string representation of the \sphinxcode{\sphinxupquote{obj}} in JSON format.

\item[{Return type}] \leavevmode
\sphinxhref{https://docs.python.org/3.7/library/stdtypes.html\#str}{str}

\end{description}\end{quote}

\end{fulllineitems}

\index{str\_copy() (in module app.utils.convertions)@\spxentry{str\_copy()}\spxextra{in module app.utils.convertions}}

\begin{fulllineitems}
\phantomsection\label{\detokenize{app.utils:app.utils.convertions.str_copy}}\pysiglinewithargsret{\sphinxbfcode{\sphinxupquote{str\_copy}}}{\emph{\DUrole{n}{string}}}{}
Hard copies a string
\begin{quote}
\begin{description}
\item[{Note:}] \leavevmode
Python’s builtin copy.deepcopy() does not deep copy strings.

\end{description}
\end{quote}
\begin{quote}\begin{description}
\item[{Parameters}] \leavevmode
\sphinxstyleliteralstrong{\sphinxupquote{string}} (\sphinxhref{https://docs.python.org/3.7/library/stdtypes.html\#str}{\sphinxstyleliteralemphasis{\sphinxupquote{str}}}) \textendash{} The string to be copied.

\item[{Returns}] \leavevmode
An deep copy of the \sphinxcode{\sphinxupquote{string}} or \sphinxcode{\sphinxupquote{None}} if the \sphinxcode{\sphinxupquote{string}}
is not a str type.

\item[{Return type}] \leavevmode
Optional{[}\sphinxhref{https://docs.python.org/3.7/library/stdtypes.html\#str}{str}{]}

\end{description}\end{quote}

\end{fulllineitems}

\index{truncate\_float\_value() (in module app.utils.convertions)@\spxentry{truncate\_float\_value()}\spxextra{in module app.utils.convertions}}

\begin{fulllineitems}
\phantomsection\label{\detokenize{app.utils:app.utils.convertions.truncate_float_value}}\pysiglinewithargsret{\sphinxbfcode{\sphinxupquote{truncate\_float\_value}}}{\emph{\DUrole{n}{f}}, \emph{\DUrole{n}{d}}}{}
Truncates a float value without rounding.
\begin{quote}\begin{description}
\item[{Parameters}] \leavevmode\begin{itemize}
\item {} 
\sphinxstyleliteralstrong{\sphinxupquote{f}} (\sphinxhref{https://docs.python.org/3.7/library/functions.html\#float}{\sphinxstyleliteralemphasis{\sphinxupquote{float}}}) \textendash{} The float value to truncate.

\item {} 
\sphinxstyleliteralstrong{\sphinxupquote{d}} (\sphinxhref{https://docs.python.org/3.7/library/functions.html\#int}{\sphinxstyleliteralemphasis{\sphinxupquote{int}}}) \textendash{} The number of decimal places the float can have.

\end{itemize}

\item[{Returns}] \leavevmode
The truncated float value of \sphinxcode{\sphinxupquote{f}}.

\item[{Return type}] \leavevmode
\sphinxhref{https://docs.python.org/3.7/library/functions.html\#float}{float}

\end{description}\end{quote}

\end{fulllineitems}

\index{utf8string\_to\_bytes() (in module app.utils.convertions)@\spxentry{utf8string\_to\_bytes()}\spxextra{in module app.utils.convertions}}

\begin{fulllineitems}
\phantomsection\label{\detokenize{app.utils:app.utils.convertions.utf8string_to_bytes}}\pysiglinewithargsret{\sphinxbfcode{\sphinxupquote{utf8string\_to\_bytes}}}{\emph{\DUrole{n}{string}}}{}
Converts utf\sphinxhyphen{}8 string to a sequence of bytes.
\begin{quote}\begin{description}
\item[{Parameters}] \leavevmode
\sphinxstyleliteralstrong{\sphinxupquote{string}} (\sphinxhref{https://docs.python.org/3.7/library/stdtypes.html\#str}{\sphinxstyleliteralemphasis{\sphinxupquote{str}}}) \textendash{} A utf\sphinxhyphen{}8 string to be converted to bytes.

\item[{Returns}] \leavevmode
The bytes of the utf\sphinxhyphen{}8 \sphinxcode{\sphinxupquote{string}} or \sphinxcode{\sphinxupquote{None}} if \sphinxcode{\sphinxupquote{string}}
is not a str type.

\item[{Return type}] \leavevmode
Optional{[}\sphinxhref{https://docs.python.org/3.7/library/stdtypes.html\#bytes}{bytes}{]}

\end{description}\end{quote}

\end{fulllineitems}



\subsubsection{app.utils.crypto}
\label{\detokenize{app.utils:module-app.utils.crypto}}\label{\detokenize{app.utils:app-utils-crypto}}\index{module@\spxentry{module}!app.utils.crypto@\spxentry{app.utils.crypto}}\index{app.utils.crypto@\spxentry{app.utils.crypto}!module@\spxentry{module}}
Utility module that includes confidentiality, integrity and authentication
functions.

\begin{sphinxadmonition}{note}{Note:}
As of current release this module only includes one function
{\hyperref[\detokenize{app.utils:app.utils.crypto.sha256}]{\sphinxcrossref{\sphinxcode{\sphinxupquote{sha256()}}}}}, but if you need to test a swarm guidance algorithm
that does not assume the communication channels to be secure or
trustworthy you should implement the crypto related functions here.
\end{sphinxadmonition}
\index{sha256() (in module app.utils.crypto)@\spxentry{sha256()}\spxextra{in module app.utils.crypto}}

\begin{fulllineitems}
\phantomsection\label{\detokenize{app.utils:app.utils.crypto.sha256}}\pysiglinewithargsret{\sphinxbfcode{\sphinxupquote{sha256}}}{\emph{\DUrole{n}{data}}}{}
Calculates the \sphinxcode{\sphinxupquote{sha256}} hash of \sphinxcode{\sphinxupquote{data}}. Data can be anything.
\begin{quote}\begin{description}
\item[{Parameters}] \leavevmode
\sphinxstyleliteralstrong{\sphinxupquote{data}} (\sphinxstyleliteralemphasis{\sphinxupquote{Any}}) \textendash{} The data to get the hash from. If \sphinxcode{\sphinxupquote{data}} is not of type bytes,
it will be converted to bytes before the data is digested.

\item[{Returns}] \leavevmode
The hashvalue of \sphinxcode{\sphinxupquote{data}} using \sphinxcode{\sphinxupquote{sha256}} algorithm.

\item[{Return type}] \leavevmode
\sphinxhref{https://docs.python.org/3.7/library/stdtypes.html\#str}{str}

\end{description}\end{quote}

\end{fulllineitems}



\subsubsection{app.utils.randoms}
\label{\detokenize{app.utils:module-app.utils.randoms}}\label{\detokenize{app.utils:app-utils-randoms}}\index{module@\spxentry{module}!app.utils.randoms@\spxentry{app.utils.randoms}}\index{app.utils.randoms@\spxentry{app.utils.randoms}!module@\spxentry{module}}
This module implements some functions related with random number generation.
\index{excluding\_randrange() (in module app.utils.randoms)@\spxentry{excluding\_randrange()}\spxextra{in module app.utils.randoms}}

\begin{fulllineitems}
\phantomsection\label{\detokenize{app.utils:app.utils.randoms.excluding_randrange}}\pysiglinewithargsret{\sphinxbfcode{\sphinxupquote{excluding\_randrange}}}{\emph{\DUrole{n}{start}}, \emph{\DUrole{n}{stop}}, \emph{\DUrole{n}{start\_again}}, \emph{\DUrole{n}{stop\_again}}, \emph{\DUrole{n}{step}\DUrole{o}{=}\DUrole{default_value}{1}}}{}
Generates a random number within two different intervals.”
\begin{quote}\begin{description}
\item[{Parameters}] \leavevmode\begin{itemize}
\item {} 
\sphinxstyleliteralstrong{\sphinxupquote{start}} \textendash{} Number consideration for generation starts from this.

\item {} 
\sphinxstyleliteralstrong{\sphinxupquote{stop}} \textendash{} Numbers less than this are generated unless they are bigger or
equal than \sphinxcode{\sphinxupquote{start\_again}}.

\item {} 
\sphinxstyleliteralstrong{\sphinxupquote{start\_again}} \textendash{} Number consideration for generation starts again from this.

\item {} 
\sphinxstyleliteralstrong{\sphinxupquote{stop\_again}} \textendash{} Number consideration stops here and does not include the inputed
value.

\item {} 
\sphinxstyleliteralstrong{\sphinxupquote{step}} \textendash{} Step point of range, this won’t be included.

\end{itemize}

\item[{Returns}] \leavevmode
A randomly selected element from in the interval \sphinxcode{\sphinxupquote{{[}start, stop)}} or in
\sphinxcode{\sphinxupquote{{[}start\_again, stop\_again)}}.

\end{description}\end{quote}

\end{fulllineitems}

\index{random\_index() (in module app.utils.randoms)@\spxentry{random\_index()}\spxextra{in module app.utils.randoms}}

\begin{fulllineitems}
\phantomsection\label{\detokenize{app.utils:app.utils.randoms.random_index}}\pysiglinewithargsret{\sphinxbfcode{\sphinxupquote{random\_index}}}{\emph{\DUrole{n}{i}}, \emph{\DUrole{n}{size}}}{}
Generates a random number that can be used as a iterables’ index.
\begin{quote}\begin{description}
\item[{Parameters}] \leavevmode\begin{itemize}
\item {} 
\sphinxstyleliteralstrong{\sphinxupquote{i}} (\sphinxhref{https://docs.python.org/3.7/library/functions.html\#int}{\sphinxstyleliteralemphasis{\sphinxupquote{int}}}) \textendash{} An index;

\item {} 
\sphinxstyleliteralstrong{\sphinxupquote{size}} (\sphinxhref{https://docs.python.org/3.7/library/functions.html\#int}{\sphinxstyleliteralemphasis{\sphinxupquote{int}}}) \textendash{} The size of the matrix

\end{itemize}

\item[{Returns}] \leavevmode
A random index that is different than \sphinxcode{\sphinxupquote{i}} and belongs to \sphinxcode{\sphinxupquote{{[}0, size)}}.

\item[{Return type}] \leavevmode
\sphinxhref{https://docs.python.org/3.7/library/functions.html\#int}{int}

\end{description}\end{quote}

\end{fulllineitems}



\section{Submodules}
\label{\detokenize{app:submodules}}

\section{app.environment\_settings}
\label{\detokenize{app:module-app.environment_settings}}\label{\detokenize{app:app-environment-settings}}\index{module@\spxentry{module}!app.environment\_settings@\spxentry{app.environment\_settings}}\index{app.environment\_settings@\spxentry{app.environment\_settings}!module@\spxentry{module}}
Module with simulation and project related variables.

This module demonstrates holds multiple constant variables that are used
through out the simulation’s lifetime including initialization and execution.

\begin{sphinxadmonition}{note}{Note:}
To configure the amount of available
{\hyperref[\detokenize{app.domain:app.domain.network_nodes.Node}]{\sphinxcrossref{\sphinxcode{\sphinxupquote{Network Nodes}}}}} system,
the initial size of a file
{\hyperref[\detokenize{app.domain:app.domain.cluster_groups.Cluster}]{\sphinxcrossref{\sphinxcode{\sphinxupquote{Cluster Group}}}}} that
work on the durability of a file, the way files are
{\hyperref[\detokenize{app.domain:app.domain.cluster_groups.Cluster.spread_files}]{\sphinxcrossref{\sphinxcode{\sphinxupquote{distributed}}}}}
among the clusters’ nodes at the start of a simulation and, the actual
name of the file whose persistence is being simulated, you should create
a simulation file using this {\hyperref[\detokenize{app:module-app.simfile_generator}]{\sphinxcrossref{\sphinxcode{\sphinxupquote{script}}}}} and
follow the respective instructions. To run the script type in your
command line terminal:

\begin{DUlineblock}{0em}
\item[] 
\end{DUlineblock}

\begin{sphinxVerbatim}[commandchars=\\\{\}]
\PYGZdl{} python simfile\PYGZus{}generator.py \PYGZhy{}\PYGZhy{}file=filename.json
\end{sphinxVerbatim}

\begin{DUlineblock}{0em}
\item[] 
\end{DUlineblock}

It is also strongly recommended that the user does not alter any
undocumented attributes or module variables unless they are absolutely
sure of what they do and the consequence of their changes. These include
variables such as {\hyperref[\detokenize{app:app.environment_settings.SHARED_ROOT}]{\sphinxcrossref{\sphinxcode{\sphinxupquote{SHARED\_ROOT}}}}} and
{\hyperref[\detokenize{app:app.environment_settings.SIMULATION_ROOT}]{\sphinxcrossref{\sphinxcode{\sphinxupquote{SIMULATION\_ROOT}}}}}.
\end{sphinxadmonition}
\index{get\_disk\_error\_chances() (in module app.environment\_settings)@\spxentry{get\_disk\_error\_chances()}\spxextra{in module app.environment\_settings}}

\begin{fulllineitems}
\phantomsection\label{\detokenize{app:app.environment_settings.get_disk_error_chances}}\pysiglinewithargsret{\sphinxbfcode{\sphinxupquote{get\_disk\_error\_chances}}}{\emph{\DUrole{n}{simulation\_epochs}}}{}
Defines the probability of a file block being corrupted while stored
at the disk of a {\hyperref[\detokenize{app.domain:app.domain.network_nodes.Node}]{\sphinxcrossref{\sphinxcode{\sphinxupquote{network node}}}}}.

\begin{sphinxadmonition}{note}{Note:}
Recommended value should be based on the paper named
\sphinxhref{http://www.cs.toronto.edu/bianca/papers/fast08.pdf}{An Analysis of Data Corruption in the Storage Stack}. Thus
the current implementation follows this formula:
\begin{quote}

({\hyperref[\detokenize{app.domain:app.domain.master_servers.Master.MAX_EPOCHS}]{\sphinxcrossref{\sphinxcode{\sphinxupquote{MAX\_EPOCHS}}}}} / {\hyperref[\detokenize{app:app.environment_settings.MONTH_EPOCHS}]{\sphinxcrossref{\sphinxcode{\sphinxupquote{MONTH\_EPOCHS}}}}}) * \sphinxcode{\sphinxupquote{P(Xt ≥ L)}})
\end{quote}

The notation \sphinxcode{\sphinxupquote{P(Xt ≥ L)}} denotes the probability of a disk
developing at least L checksum mismatches within T months since
the disk’s first use in the field. As described in linked paper.
\end{sphinxadmonition}
\begin{quote}\begin{description}
\item[{Parameters}] \leavevmode
\sphinxstyleliteralstrong{\sphinxupquote{simulation\_epochs}} (\sphinxhref{https://docs.python.org/3.7/library/functions.html\#int}{\sphinxstyleliteralemphasis{\sphinxupquote{int}}}) \textendash{} The number of epochs the simuulation is expected to run
assuming no failures occur.

\item[{Returns}] \leavevmode
A two element list with respectively, the probability of losing
and the probability of not losing a file block due to disk
errors, at an epoch basis.

\item[{Return type}] \leavevmode
List{[}\sphinxhref{https://docs.python.org/3.7/library/functions.html\#float}{float}{]}

\end{description}\end{quote}

\end{fulllineitems}

\index{set\_blocks\_count() (in module app.environment\_settings)@\spxentry{set\_blocks\_count()}\spxextra{in module app.environment\_settings}}

\begin{fulllineitems}
\phantomsection\label{\detokenize{app:app.environment_settings.set_blocks_count}}\pysiglinewithargsret{\sphinxbfcode{\sphinxupquote{set\_blocks\_count}}}{\emph{\DUrole{n}{n}}}{}
Changes {\hyperref[\detokenize{app:app.environment_settings.BLOCKS_COUNT}]{\sphinxcrossref{\sphinxcode{\sphinxupquote{BLOCKS\_COUNT}}}}} constant value at run time.
\begin{quote}\begin{description}
\item[{Parameters}] \leavevmode
\sphinxstyleliteralstrong{\sphinxupquote{n}} (\sphinxhref{https://docs.python.org/3.7/library/functions.html\#int}{\sphinxstyleliteralemphasis{\sphinxupquote{int}}}) \textendash{} 

\item[{Return type}] \leavevmode
\sphinxhref{https://docs.python.org/3.7/library/constants.html\#None}{None}

\end{description}\end{quote}

\end{fulllineitems}

\index{set\_blocks\_size() (in module app.environment\_settings)@\spxentry{set\_blocks\_size()}\spxextra{in module app.environment\_settings}}

\begin{fulllineitems}
\phantomsection\label{\detokenize{app:app.environment_settings.set_blocks_size}}\pysiglinewithargsret{\sphinxbfcode{\sphinxupquote{set\_blocks\_size}}}{\emph{\DUrole{n}{n}}}{}
Changes {\hyperref[\detokenize{app:app.environment_settings.BLOCKS_SIZE}]{\sphinxcrossref{\sphinxcode{\sphinxupquote{BLOCKS\_SIZE}}}}} constant value at run time to the given n bytes.
\begin{quote}\begin{description}
\item[{Parameters}] \leavevmode
\sphinxstyleliteralstrong{\sphinxupquote{n}} (\sphinxhref{https://docs.python.org/3.7/library/functions.html\#int}{\sphinxstyleliteralemphasis{\sphinxupquote{int}}}) \textendash{} 

\item[{Return type}] \leavevmode
\sphinxhref{https://docs.python.org/3.7/library/constants.html\#None}{None}

\end{description}\end{quote}

\end{fulllineitems}

\index{set\_loss\_chance() (in module app.environment\_settings)@\spxentry{set\_loss\_chance()}\spxextra{in module app.environment\_settings}}

\begin{fulllineitems}
\phantomsection\label{\detokenize{app:app.environment_settings.set_loss_chance}}\pysiglinewithargsret{\sphinxbfcode{\sphinxupquote{set\_loss\_chance}}}{\emph{\DUrole{n}{v}}}{}
Changes {\hyperref[\detokenize{app:app.environment_settings.LOSS_CHANCE}]{\sphinxcrossref{\sphinxcode{\sphinxupquote{LOSS\_CHANCE}}}}} constant value at run time.
\begin{quote}\begin{description}
\item[{Parameters}] \leavevmode
\sphinxstyleliteralstrong{\sphinxupquote{v}} (\sphinxhref{https://docs.python.org/3.7/library/functions.html\#float}{\sphinxstyleliteralemphasis{\sphinxupquote{float}}}) \textendash{} 

\item[{Return type}] \leavevmode
\sphinxhref{https://docs.python.org/3.7/library/constants.html\#None}{None}

\end{description}\end{quote}

\end{fulllineitems}

\index{set\_replication\_level() (in module app.environment\_settings)@\spxentry{set\_replication\_level()}\spxextra{in module app.environment\_settings}}

\begin{fulllineitems}
\phantomsection\label{\detokenize{app:app.environment_settings.set_replication_level}}\pysiglinewithargsret{\sphinxbfcode{\sphinxupquote{set\_replication\_level}}}{\emph{\DUrole{n}{n}}}{}
Changes {\hyperref[\detokenize{app:app.environment_settings.REPLICATION_LEVEL}]{\sphinxcrossref{\sphinxcode{\sphinxupquote{REPLICATION\_LEVEL}}}}} constant value at run time.
\begin{quote}\begin{description}
\item[{Parameters}] \leavevmode
\sphinxstyleliteralstrong{\sphinxupquote{n}} (\sphinxhref{https://docs.python.org/3.7/library/functions.html\#int}{\sphinxstyleliteralemphasis{\sphinxupquote{int}}}) \textendash{} 

\item[{Return type}] \leavevmode
\sphinxhref{https://docs.python.org/3.7/library/constants.html\#None}{None}

\end{description}\end{quote}

\end{fulllineitems}

\index{ATOL (in module app.environment\_settings)@\spxentry{ATOL}\spxextra{in module app.environment\_settings}}

\begin{fulllineitems}
\phantomsection\label{\detokenize{app:app.environment_settings.ATOL}}\pysigline{\sphinxbfcode{\sphinxupquote{ATOL}}\sphinxbfcode{\sphinxupquote{: \sphinxhref{https://docs.python.org/3.7/library/functions.html\#float}{float}}}\sphinxbfcode{\sphinxupquote{ = 0.05}}}
Defines the maximum amount of absolute positive or negative deviation that a
current distribution {\hyperref[\detokenize{app.domain:app.domain.cluster_groups.SGCluster.cv_}]{\sphinxcrossref{\sphinxcode{\sphinxupquote{cv\_}}}}} can
have from the desired steady state
{\hyperref[\detokenize{app.domain:app.domain.cluster_groups.SGCluster.v_}]{\sphinxcrossref{\sphinxcode{\sphinxupquote{v\_}}}}}, in order for the
distributions to be considered equal and thus marking the epoch as convergent.

This constant will be used by
{\hyperref[\detokenize{app.domain:app.domain.cluster_groups.SGCluster.equal_distributions}]{\sphinxcrossref{\sphinxcode{\sphinxupquote{app.domain.cluster\_groups.SGCluster.equal\_distributions()}}}}} along
with a relative tolerance that is the minimum value in
{\hyperref[\detokenize{app.domain:app.domain.cluster_groups.SGCluster.v_}]{\sphinxcrossref{\sphinxcode{\sphinxupquote{v\_}}}}}.

\end{fulllineitems}

\index{BLOCKS\_COUNT (in module app.environment\_settings)@\spxentry{BLOCKS\_COUNT}\spxextra{in module app.environment\_settings}}

\begin{fulllineitems}
\phantomsection\label{\detokenize{app:app.environment_settings.BLOCKS_COUNT}}\pysigline{\sphinxbfcode{\sphinxupquote{BLOCKS\_COUNT}}\sphinxbfcode{\sphinxupquote{: \sphinxhref{https://docs.python.org/3.7/library/functions.html\#int}{int}}}\sphinxbfcode{\sphinxupquote{ = 333}}}
Defines into how many
{\hyperref[\detokenize{app.domain.helpers:app.domain.helpers.smart_dataclasses.FileBlockData}]{\sphinxcrossref{\sphinxcode{\sphinxupquote{FileBlockData}}}}} instances a file
is divided into. Either use this or {\hyperref[\detokenize{app:app.environment_settings.BLOCKS_SIZE}]{\sphinxcrossref{\sphinxcode{\sphinxupquote{BLOCKS\_SIZE}}}}} but not both.

\end{fulllineitems}

\index{BLOCKS\_SIZE (in module app.environment\_settings)@\spxentry{BLOCKS\_SIZE}\spxextra{in module app.environment\_settings}}

\begin{fulllineitems}
\phantomsection\label{\detokenize{app:app.environment_settings.BLOCKS_SIZE}}\pysigline{\sphinxbfcode{\sphinxupquote{BLOCKS\_SIZE}}\sphinxbfcode{\sphinxupquote{: \sphinxhref{https://docs.python.org/3.7/library/functions.html\#int}{int}}}\sphinxbfcode{\sphinxupquote{ = 1048576}}}
Defines the raw size of each file block before it’s wrapped in a
{\hyperref[\detokenize{app.domain.helpers:app.domain.helpers.smart_dataclasses.FileBlockData}]{\sphinxcrossref{\sphinxcode{\sphinxupquote{FileBlockData}}}}} instance
object.

Some possible values include \{ 32KB = 32768B; 128KB = 131072B; 512KB = 524288B;
1MB = 1048576B; 20MB = 20971520B \}.

\end{fulllineitems}

\index{DEBUG (in module app.environment\_settings)@\spxentry{DEBUG}\spxextra{in module app.environment\_settings}}

\begin{fulllineitems}
\phantomsection\label{\detokenize{app:app.environment_settings.DEBUG}}\pysigline{\sphinxbfcode{\sphinxupquote{DEBUG}}\sphinxbfcode{\sphinxupquote{: \sphinxhref{https://docs.python.org/3.7/library/functions.html\#bool}{bool}}}\sphinxbfcode{\sphinxupquote{ = True}}}
Indicates if some debug related actions or prints to the terminal should
be performed.

\end{fulllineitems}

\index{DELIVER\_CHANCE (in module app.environment\_settings)@\spxentry{DELIVER\_CHANCE}\spxextra{in module app.environment\_settings}}

\begin{fulllineitems}
\phantomsection\label{\detokenize{app:app.environment_settings.DELIVER_CHANCE}}\pysigline{\sphinxbfcode{\sphinxupquote{DELIVER\_CHANCE}}\sphinxbfcode{\sphinxupquote{: \sphinxhref{https://docs.python.org/3.7/library/functions.html\#float}{float}}}\sphinxbfcode{\sphinxupquote{ = 0.96}}}
Defines the probability of a message being delivered to a destination,
in the simulation environment.

\end{fulllineitems}

\index{LOSS\_CHANCE (in module app.environment\_settings)@\spxentry{LOSS\_CHANCE}\spxextra{in module app.environment\_settings}}

\begin{fulllineitems}
\phantomsection\label{\detokenize{app:app.environment_settings.LOSS_CHANCE}}\pysigline{\sphinxbfcode{\sphinxupquote{LOSS\_CHANCE}}\sphinxbfcode{\sphinxupquote{: \sphinxhref{https://docs.python.org/3.7/library/functions.html\#float}{float}}}\sphinxbfcode{\sphinxupquote{ = 0.04}}}
Defines the probability of a message not being delivered to a destination
due to network link problems, in the simulation environment.

\end{fulllineitems}

\index{MATLAB\_DIR (in module app.environment\_settings)@\spxentry{MATLAB\_DIR}\spxextra{in module app.environment\_settings}}

\begin{fulllineitems}
\phantomsection\label{\detokenize{app:app.environment_settings.MATLAB_DIR}}\pysigline{\sphinxbfcode{\sphinxupquote{MATLAB\_DIR}}\sphinxbfcode{\sphinxupquote{: \sphinxhref{https://docs.python.org/3.7/library/stdtypes.html\#str}{str}}}\sphinxbfcode{\sphinxupquote{ = \textquotesingle{}C:\textbackslash{}\textbackslash{}GitHub\textbackslash{}\textbackslash{}hive\sphinxhyphen{}msc\sphinxhyphen{}thesis\textbackslash{}\textbackslash{}hive\textbackslash{}\textbackslash{}docs\textbackslash{}\textbackslash{}scripts\textbackslash{}\textbackslash{}matlab\textquotesingle{}}}}
Path the folder where matlab scripts are located. Used by
{\hyperref[\detokenize{app.domain.helpers:app.domain.helpers.matlab_utils.MatlabEngineContainer}]{\sphinxcrossref{\sphinxcode{\sphinxupquote{MatlabEngineContainer}}}}}

\end{fulllineitems}

\index{MAX\_REPLICATION\_DELAY (in module app.environment\_settings)@\spxentry{MAX\_REPLICATION\_DELAY}\spxextra{in module app.environment\_settings}}

\begin{fulllineitems}
\phantomsection\label{\detokenize{app:app.environment_settings.MAX_REPLICATION_DELAY}}\pysigline{\sphinxbfcode{\sphinxupquote{MAX\_REPLICATION\_DELAY}}\sphinxbfcode{\sphinxupquote{: \sphinxhref{https://docs.python.org/3.7/library/functions.html\#int}{int}}}\sphinxbfcode{\sphinxupquote{ = 3}}}
The maximum amount of epoch time steps replica file block blocks take to
be regenerated after their are lost.

\end{fulllineitems}

\index{MIN\_CONVERGENCE\_THRESHOLD (in module app.environment\_settings)@\spxentry{MIN\_CONVERGENCE\_THRESHOLD}\spxextra{in module app.environment\_settings}}

\begin{fulllineitems}
\phantomsection\label{\detokenize{app:app.environment_settings.MIN_CONVERGENCE_THRESHOLD}}\pysigline{\sphinxbfcode{\sphinxupquote{MIN\_CONVERGENCE\_THRESHOLD}}\sphinxbfcode{\sphinxupquote{: \sphinxhref{https://docs.python.org/3.7/library/functions.html\#int}{int}}}\sphinxbfcode{\sphinxupquote{ = 0}}}
The number of consecutive epoch time steps that a
{\hyperref[\detokenize{app.domain:app.domain.cluster_groups.SGCluster}]{\sphinxcrossref{\sphinxcode{\sphinxupquote{SGCluster}}}}} must converge before epochs
start being marked with verified convergence in
{\hyperref[\detokenize{app.domain.helpers:app.domain.helpers.smart_dataclasses.LoggingData.convergence_set}]{\sphinxcrossref{\sphinxcode{\sphinxupquote{app.domain.helpers.smart\_dataclasses.LoggingData.convergence\_set}}}}}.

\end{fulllineitems}

\index{MIN\_REPLICATION\_DELAY (in module app.environment\_settings)@\spxentry{MIN\_REPLICATION\_DELAY}\spxextra{in module app.environment\_settings}}

\begin{fulllineitems}
\phantomsection\label{\detokenize{app:app.environment_settings.MIN_REPLICATION_DELAY}}\pysigline{\sphinxbfcode{\sphinxupquote{MIN\_REPLICATION\_DELAY}}\sphinxbfcode{\sphinxupquote{: \sphinxhref{https://docs.python.org/3.7/library/functions.html\#int}{int}}}\sphinxbfcode{\sphinxupquote{ = 1}}}
The minimum amount of epoch time steps replica file block blocks take to
be regenerated after their are lost.

\end{fulllineitems}

\index{MONTH\_EPOCHS (in module app.environment\_settings)@\spxentry{MONTH\_EPOCHS}\spxextra{in module app.environment\_settings}}

\begin{fulllineitems}
\phantomsection\label{\detokenize{app:app.environment_settings.MONTH_EPOCHS}}\pysigline{\sphinxbfcode{\sphinxupquote{MONTH\_EPOCHS}}\sphinxbfcode{\sphinxupquote{: \sphinxhref{https://docs.python.org/3.7/library/functions.html\#int}{int}}}\sphinxbfcode{\sphinxupquote{ = 21600}}}
Defines how many epochs (discrete time steps) a month is represented with.
With the default value of 21600 each epoch would represent two minutes. See
{\hyperref[\detokenize{app:app.environment_settings.get_disk_error_chances}]{\sphinxcrossref{\sphinxcode{\sphinxupquote{get\_disk\_error\_chances()}}}}}.

\end{fulllineitems}

\index{NEWSCAST\_CACHE\_SIZE (in module app.environment\_settings)@\spxentry{NEWSCAST\_CACHE\_SIZE}\spxextra{in module app.environment\_settings}}

\begin{fulllineitems}
\phantomsection\label{\detokenize{app:app.environment_settings.NEWSCAST_CACHE_SIZE}}\pysigline{\sphinxbfcode{\sphinxupquote{NEWSCAST\_CACHE\_SIZE}}\sphinxbfcode{\sphinxupquote{: \sphinxhref{https://docs.python.org/3.7/library/functions.html\#int}{int}}}\sphinxbfcode{\sphinxupquote{ = 20}}}
attr:\sphinxtitleref{NewscastNode view
\textless{}app.domain.network\_nodes.NewscastNode\textgreater{}} can have at any given time.
\begin{quote}\begin{description}
\item[{Type}] \leavevmode
The maximum amount of neighbors a

\item[{Type}] \leavevmode
py

\end{description}\end{quote}

\end{fulllineitems}

\index{OUTFILE\_ROOT (in module app.environment\_settings)@\spxentry{OUTFILE\_ROOT}\spxextra{in module app.environment\_settings}}

\begin{fulllineitems}
\phantomsection\label{\detokenize{app:app.environment_settings.OUTFILE_ROOT}}\pysigline{\sphinxbfcode{\sphinxupquote{OUTFILE\_ROOT}}\sphinxbfcode{\sphinxupquote{: \sphinxhref{https://docs.python.org/3.7/library/stdtypes.html\#str}{str}}}\sphinxbfcode{\sphinxupquote{ = \textquotesingle{}C:\textbackslash{}\textbackslash{}GitHub\textbackslash{}\textbackslash{}hive\sphinxhyphen{}msc\sphinxhyphen{}thesis\textbackslash{}\textbackslash{}hive\textbackslash{}\textbackslash{}docs\textbackslash{}\textbackslash{}static\textbackslash{}\textbackslash{}outfiles\textquotesingle{}}}}
Path to the folder where simulation output files are located.

\end{fulllineitems}

\index{REPLICATION\_LEVEL (in module app.environment\_settings)@\spxentry{REPLICATION\_LEVEL}\spxextra{in module app.environment\_settings}}

\begin{fulllineitems}
\phantomsection\label{\detokenize{app:app.environment_settings.REPLICATION_LEVEL}}\pysigline{\sphinxbfcode{\sphinxupquote{REPLICATION\_LEVEL}}\sphinxbfcode{\sphinxupquote{: \sphinxhref{https://docs.python.org/3.7/library/functions.html\#int}{int}}}\sphinxbfcode{\sphinxupquote{ = 3}}}
The amount of replicas each file block has.

\end{fulllineitems}

\index{RESOURCES\_ROOT (in module app.environment\_settings)@\spxentry{RESOURCES\_ROOT}\spxextra{in module app.environment\_settings}}

\begin{fulllineitems}
\phantomsection\label{\detokenize{app:app.environment_settings.RESOURCES_ROOT}}\pysigline{\sphinxbfcode{\sphinxupquote{RESOURCES\_ROOT}}\sphinxbfcode{\sphinxupquote{: \sphinxhref{https://docs.python.org/3.7/library/stdtypes.html\#str}{str}}}\sphinxbfcode{\sphinxupquote{ = \textquotesingle{}C:\textbackslash{}\textbackslash{}GitHub\textbackslash{}\textbackslash{}hive\sphinxhyphen{}msc\sphinxhyphen{}thesis\textbackslash{}\textbackslash{}hive\textbackslash{}\textbackslash{}docs\textbackslash{}\textbackslash{}static\textbackslash{}\textbackslash{}resources\textquotesingle{}}}}
Path to the folder where miscellaneous files are located.

\end{fulllineitems}

\index{RTOL (in module app.environment\_settings)@\spxentry{RTOL}\spxextra{in module app.environment\_settings}}

\begin{fulllineitems}
\phantomsection\label{\detokenize{app:app.environment_settings.RTOL}}\pysigline{\sphinxbfcode{\sphinxupquote{RTOL}}\sphinxbfcode{\sphinxupquote{: \sphinxhref{https://docs.python.org/3.7/library/functions.html\#float}{float}}}\sphinxbfcode{\sphinxupquote{ = 0.05}}}
Defines the maximum amount of relative positive or negative deviation that a
current distribution {\hyperref[\detokenize{app.domain:app.domain.cluster_groups.SGCluster.cv_}]{\sphinxcrossref{\sphinxcode{\sphinxupquote{cv\_}}}}} can
have from the desired steady state
{\hyperref[\detokenize{app.domain:app.domain.cluster_groups.SGCluster.v_}]{\sphinxcrossref{\sphinxcode{\sphinxupquote{v\_}}}}}, in order for the
distributions to be considered equal and thus marking the epoch as convergent.

This constant will be used by
{\hyperref[\detokenize{app.domain:app.domain.cluster_groups.SGCluster.equal_distributions}]{\sphinxcrossref{\sphinxcode{\sphinxupquote{app.domain.cluster\_groups.SGCluster.equal\_distributions()}}}}} along
with a relative tolerance that is the minimum value in
{\hyperref[\detokenize{app.domain:app.domain.cluster_groups.SGCluster.v_}]{\sphinxcrossref{\sphinxcode{\sphinxupquote{v\_}}}}}.

\end{fulllineitems}

\index{SHARED\_ROOT (in module app.environment\_settings)@\spxentry{SHARED\_ROOT}\spxextra{in module app.environment\_settings}}

\begin{fulllineitems}
\phantomsection\label{\detokenize{app:app.environment_settings.SHARED_ROOT}}\pysigline{\sphinxbfcode{\sphinxupquote{SHARED\_ROOT}}\sphinxbfcode{\sphinxupquote{: \sphinxhref{https://docs.python.org/3.7/library/stdtypes.html\#str}{str}}}\sphinxbfcode{\sphinxupquote{ = \textquotesingle{}C:\textbackslash{}\textbackslash{}GitHub\textbackslash{}\textbackslash{}hive\sphinxhyphen{}msc\sphinxhyphen{}thesis\textbackslash{}\textbackslash{}hive\textbackslash{}\textbackslash{}docs\textbackslash{}\textbackslash{}static\textbackslash{}\textbackslash{}shared\textquotesingle{}}}}
Path to the folder where files to be persisted during the simulation are
located.

\end{fulllineitems}

\index{SIMULATION\_ROOT (in module app.environment\_settings)@\spxentry{SIMULATION\_ROOT}\spxextra{in module app.environment\_settings}}

\begin{fulllineitems}
\phantomsection\label{\detokenize{app:app.environment_settings.SIMULATION_ROOT}}\pysigline{\sphinxbfcode{\sphinxupquote{SIMULATION\_ROOT}}\sphinxbfcode{\sphinxupquote{: \sphinxhref{https://docs.python.org/3.7/library/stdtypes.html\#str}{str}}}\sphinxbfcode{\sphinxupquote{ = \textquotesingle{}C:\textbackslash{}\textbackslash{}GitHub\textbackslash{}\textbackslash{}hive\sphinxhyphen{}msc\sphinxhyphen{}thesis\textbackslash{}\textbackslash{}hive\textbackslash{}\textbackslash{}docs\textbackslash{}\textbackslash{}static\textbackslash{}\textbackslash{}simfiles\textquotesingle{}}}}
Path to the folder where simulation files to be executed by
{\hyperref[\detokenize{app:module-app.hive_simulation}]{\sphinxcrossref{\sphinxcode{\sphinxupquote{app.hive\_simulation}}}}} are located.

\end{fulllineitems}



\section{app.hive\_simulation}
\label{\detokenize{app:module-app.hive_simulation}}\label{\detokenize{app:app-hive-simulation}}\index{module@\spxentry{module}!app.hive\_simulation@\spxentry{app.hive\_simulation}}\index{app.hive\_simulation@\spxentry{app.hive\_simulation}!module@\spxentry{module}}
This scripts’s functions are used to start simulations.

You can start a simulation by executing the following command:

\begin{sphinxVerbatim}[commandchars=\\\{\}]
\PYGZdl{} python hive\PYGZus{}simulation.py \PYGZhy{}\PYGZhy{}file=a\PYGZus{}simulation\PYGZus{}name.json \PYGZhy{}\PYGZhy{}iters=30
\end{sphinxVerbatim}

You can also execute all simulation file that exist in
{\hyperref[\detokenize{app:app.environment_settings.SIMULATION_ROOT}]{\sphinxcrossref{\sphinxcode{\sphinxupquote{SIMULATION\_ROOT}}}}} by instead executing:

\begin{sphinxVerbatim}[commandchars=\\\{\}]
\PYGZdl{} python hive\PYGZus{}simulation.py \PYGZhy{}d \PYGZhy{}i 24
\end{sphinxVerbatim}

If you wish to execute multiple simulations in parallel (to save time) you
can use the \sphinxhyphen{}t or \textendash{}threading flag in either of the previously specified
commands. The threading flag expects an integer that specifies the max
working threads. For example:

\begin{sphinxVerbatim}[commandchars=\\\{\}]
\PYGZdl{} python hive\PYGZus{}simulation.py \PYGZhy{}d \PYGZhy{}\PYGZhy{}iters=1 \PYGZhy{}\PYGZhy{}threading=12
\end{sphinxVerbatim}

If you don’t have a simulation file yet, run the following instead:

\begin{sphinxVerbatim}[commandchars=\\\{\}]
\PYGZdl{} python simfile\PYGZus{}generator.py \PYGZhy{}\PYGZhy{}file=filename.json
\end{sphinxVerbatim}

\begin{sphinxadmonition}{note}{Note:}
For the simulation to run without errors you must ensure that:
\begin{enumerate}
\sphinxsetlistlabels{\arabic}{enumi}{enumii}{}{.}%
\item {} 
The specified simulation files exist in         {\hyperref[\detokenize{app:app.environment_settings.SIMULATION_ROOT}]{\sphinxcrossref{\sphinxcode{\sphinxupquote{SIMULATION\_ROOT}}}}}.

\item {} 
Any file used by the simulation, e.g., a picture or a .pptx         document is accessible in         {\hyperref[\detokenize{app:app.environment_settings.SHARED_ROOT}]{\sphinxcrossref{\sphinxcode{\sphinxupquote{SHARED\_ROOT}}}}}.

\item {} 
An output file simdirectory exists with default path being:         {\hyperref[\detokenize{app:app.environment_settings.OUTFILE_ROOT}]{\sphinxcrossref{\sphinxcode{\sphinxupquote{OUTFILE\_ROOT}}}}}.

\end{enumerate}
\end{sphinxadmonition}
\index{\_\_can\_exec\_simfile\_\_() (in module app.hive\_simulation)@\spxentry{\_\_can\_exec\_simfile\_\_()}\spxextra{in module app.hive\_simulation}}

\begin{fulllineitems}
\phantomsection\label{\detokenize{app:app.hive_simulation.__can_exec_simfile__}}\pysiglinewithargsret{\sphinxbfcode{\sphinxupquote{\_\_can\_exec\_simfile\_\_}}}{\emph{\DUrole{n}{sname}}}{}
Asserts if simulation can proceed with user specified file.
\begin{quote}\begin{description}
\item[{Parameters}] \leavevmode
\sphinxstyleliteralstrong{\sphinxupquote{sname}} (\sphinxhref{https://docs.python.org/3.7/library/stdtypes.html\#str}{\sphinxstyleliteralemphasis{\sphinxupquote{str}}}) \textendash{} The name of the simulation file, including extension,
whose existence inside
{\hyperref[\detokenize{app:app.environment_settings.SIMULATION_ROOT}]{\sphinxcrossref{\sphinxcode{\sphinxupquote{SIMULATION\_ROOT}}}}} will be
checked.

\item[{Return type}] \leavevmode
\sphinxhref{https://docs.python.org/3.7/library/constants.html\#None}{None}

\end{description}\end{quote}

\end{fulllineitems}

\index{\_\_makedirs\_\_() (in module app.hive\_simulation)@\spxentry{\_\_makedirs\_\_()}\spxextra{in module app.hive\_simulation}}

\begin{fulllineitems}
\phantomsection\label{\detokenize{app:app.hive_simulation.__makedirs__}}\pysiglinewithargsret{\sphinxbfcode{\sphinxupquote{\_\_makedirs\_\_}}}{}{}
Helper method that reates required simulation working directories if
they do not exist.
\begin{quote}\begin{description}
\item[{Return type}] \leavevmode
\sphinxhref{https://docs.python.org/3.7/library/constants.html\#None}{None}

\end{description}\end{quote}

\end{fulllineitems}

\index{\_\_start\_simulation\_\_() (in module app.hive\_simulation)@\spxentry{\_\_start\_simulation\_\_()}\spxextra{in module app.hive\_simulation}}

\begin{fulllineitems}
\phantomsection\label{\detokenize{app:app.hive_simulation.__start_simulation__}}\pysiglinewithargsret{\sphinxbfcode{\sphinxupquote{\_\_start\_simulation\_\_}}}{\emph{\DUrole{n}{sname}}, \emph{\DUrole{n}{sid}}, \emph{\DUrole{n}{epochs}}}{}
Helper method that orders execution of one simulation instance.
\begin{quote}\begin{description}
\item[{Parameters}] \leavevmode\begin{itemize}
\item {} 
\sphinxstyleliteralstrong{\sphinxupquote{sname}} (\sphinxhref{https://docs.python.org/3.7/library/stdtypes.html\#str}{\sphinxstyleliteralemphasis{\sphinxupquote{str}}}) \textendash{} The name of the simulation file to be executed.

\item {} 
\sphinxstyleliteralstrong{\sphinxupquote{sid}} (\sphinxhref{https://docs.python.org/3.7/library/functions.html\#int}{\sphinxstyleliteralemphasis{\sphinxupquote{int}}}) \textendash{} A sequence number that identifies the simulation execution instance.

\item {} 
\sphinxstyleliteralstrong{\sphinxupquote{epochs}} (\sphinxhref{https://docs.python.org/3.7/library/functions.html\#int}{\sphinxstyleliteralemphasis{\sphinxupquote{int}}}) \textendash{} The number of discrete time steps the simulation lasts.

\end{itemize}

\item[{Return type}] \leavevmode
\sphinxhref{https://docs.python.org/3.7/library/constants.html\#None}{None}

\end{description}\end{quote}

\end{fulllineitems}

\index{\_parallel\_main() (in module app.hive\_simulation)@\spxentry{\_parallel\_main()}\spxextra{in module app.hive\_simulation}}

\begin{fulllineitems}
\phantomsection\label{\detokenize{app:app.hive_simulation._parallel_main}}\pysiglinewithargsret{\sphinxbfcode{\sphinxupquote{\_parallel\_main}}}{\emph{\DUrole{n}{threads\_count}}, \emph{\DUrole{n}{sdir}}, \emph{\DUrole{n}{sname}}, \emph{\DUrole{n}{iters}}, \emph{\DUrole{n}{epochs}}}{}
Helper method that initializes a multi\sphinxhyphen{}threaded simulation.
\begin{quote}\begin{description}
\item[{Parameters}] \leavevmode\begin{itemize}
\item {} 
\sphinxstyleliteralstrong{\sphinxupquote{threads\_count}} (\sphinxhref{https://docs.python.org/3.7/library/functions.html\#int}{\sphinxstyleliteralemphasis{\sphinxupquote{int}}}) \textendash{} Number of worker threads that will consume jobs from the Task Pool.

\item {} 
\sphinxstyleliteralstrong{\sphinxupquote{sdir}} (\sphinxhref{https://docs.python.org/3.7/library/functions.html\#bool}{\sphinxstyleliteralemphasis{\sphinxupquote{bool}}}) \textendash{} Indicates whether or not the program will proceed by executing
all simulations files inside
{\hyperref[\detokenize{app:app.environment_settings.SIMULATION_ROOT}]{\sphinxcrossref{\sphinxcode{\sphinxupquote{SIMULATION\_ROOT}}}}} folder or
if will run with the specified file \sphinxcode{\sphinxupquote{sname}}.

\item {} 
\sphinxstyleliteralstrong{\sphinxupquote{sname}} (\sphinxstyleliteralemphasis{\sphinxupquote{Optional}}\sphinxstyleliteralemphasis{\sphinxupquote{{[}}}\sphinxhref{https://docs.python.org/3.7/library/stdtypes.html\#str}{\sphinxstyleliteralemphasis{\sphinxupquote{str}}}\sphinxstyleliteralemphasis{\sphinxupquote{{]}}}) \textendash{} The name of the simulation file to be executed or \sphinxcode{\sphinxupquote{None}} if
\sphinxcode{\sphinxupquote{sdir}} is set to \sphinxcode{\sphinxupquote{True}}.

\item {} 
\sphinxstyleliteralstrong{\sphinxupquote{iters}} (\sphinxhref{https://docs.python.org/3.7/library/functions.html\#int}{\sphinxstyleliteralemphasis{\sphinxupquote{int}}}) \textendash{} How many times each simulation file is executed.

\item {} 
\sphinxstyleliteralstrong{\sphinxupquote{epochs}} (\sphinxhref{https://docs.python.org/3.7/library/functions.html\#int}{\sphinxstyleliteralemphasis{\sphinxupquote{int}}}) \textendash{} Number of discrete time steps (epochs) each executed simulation
lasts.

\end{itemize}

\item[{Return type}] \leavevmode
\sphinxhref{https://docs.python.org/3.7/library/constants.html\#None}{None}

\end{description}\end{quote}

\end{fulllineitems}

\index{\_single\_main() (in module app.hive\_simulation)@\spxentry{\_single\_main()}\spxextra{in module app.hive\_simulation}}

\begin{fulllineitems}
\phantomsection\label{\detokenize{app:app.hive_simulation._single_main}}\pysiglinewithargsret{\sphinxbfcode{\sphinxupquote{\_single\_main}}}{\emph{\DUrole{n}{sdir}}, \emph{\DUrole{n}{sname}}, \emph{\DUrole{n}{iters}}, \emph{\DUrole{n}{epochs}}}{}
Helper function that initializes a single\sphinxhyphen{}threaded simulation.
\begin{quote}\begin{description}
\item[{Parameters}] \leavevmode\begin{itemize}
\item {} 
\sphinxstyleliteralstrong{\sphinxupquote{sdir}} (\sphinxhref{https://docs.python.org/3.7/library/functions.html\#bool}{\sphinxstyleliteralemphasis{\sphinxupquote{bool}}}) \textendash{} Indicates whether or not the program will proceed by executing
all simulations files inside
{\hyperref[\detokenize{app:app.environment_settings.SIMULATION_ROOT}]{\sphinxcrossref{\sphinxcode{\sphinxupquote{SIMULATION\_ROOT}}}}} folder or
if will run with the specified file \sphinxcode{\sphinxupquote{sname}}.

\item {} 
\sphinxstyleliteralstrong{\sphinxupquote{sname}} (\sphinxstyleliteralemphasis{\sphinxupquote{Optional}}\sphinxstyleliteralemphasis{\sphinxupquote{{[}}}\sphinxhref{https://docs.python.org/3.7/library/stdtypes.html\#str}{\sphinxstyleliteralemphasis{\sphinxupquote{str}}}\sphinxstyleliteralemphasis{\sphinxupquote{{]}}}) \textendash{} The name of the simulation file to be executed or \sphinxcode{\sphinxupquote{None}} if
\sphinxcode{\sphinxupquote{sdir}} is set to \sphinxcode{\sphinxupquote{True}}.

\item {} 
\sphinxstyleliteralstrong{\sphinxupquote{iters}} (\sphinxhref{https://docs.python.org/3.7/library/functions.html\#int}{\sphinxstyleliteralemphasis{\sphinxupquote{int}}}) \textendash{} How many times each simulation file is executed.

\item {} 
\sphinxstyleliteralstrong{\sphinxupquote{epochs}} (\sphinxhref{https://docs.python.org/3.7/library/functions.html\#int}{\sphinxstyleliteralemphasis{\sphinxupquote{int}}}) \textendash{} Number of discrete time steps (epochs) each executed simulation
lasts.

\end{itemize}

\item[{Return type}] \leavevmode
\sphinxhref{https://docs.python.org/3.7/library/constants.html\#None}{None}

\end{description}\end{quote}

\end{fulllineitems}

\index{get\_next\_scenario() (in module app.hive\_simulation)@\spxentry{get\_next\_scenario()}\spxextra{in module app.hive\_simulation}}

\begin{fulllineitems}
\phantomsection\label{\detokenize{app:app.hive_simulation.get_next_scenario}}\pysiglinewithargsret{\sphinxbfcode{\sphinxupquote{get\_next\_scenario}}}{\emph{\DUrole{n}{k}}}{}~\begin{quote}\begin{description}
\item[{Parameters}] \leavevmode
\sphinxstyleliteralstrong{\sphinxupquote{k}} (\sphinxhref{https://docs.python.org/3.7/library/stdtypes.html\#str}{\sphinxstyleliteralemphasis{\sphinxupquote{str}}}) \textendash{} 

\item[{Return type}] \leavevmode
Tuple{[}\sphinxhref{https://numpy.org/doc/stable/reference/generated/numpy.ndarray.html\#numpy.ndarray}{numpy.ndarray}, \sphinxhref{https://numpy.org/doc/stable/reference/generated/numpy.ndarray.html\#numpy.ndarray}{numpy.ndarray}{]}

\end{description}\end{quote}

\end{fulllineitems}

\index{main() (in module app.hive\_simulation)@\spxentry{main()}\spxextra{in module app.hive\_simulation}}

\begin{fulllineitems}
\phantomsection\label{\detokenize{app:app.hive_simulation.main}}\pysiglinewithargsret{\sphinxbfcode{\sphinxupquote{main}}}{\emph{\DUrole{n}{threads\_count}}, \emph{\DUrole{n}{sdir}}, \emph{\DUrole{n}{sname}}, \emph{\DUrole{n}{iters}}, \emph{\DUrole{n}{epochs}}}{}
Receives user input and initializes the simulation process.
\begin{quote}\begin{description}
\item[{Parameters}] \leavevmode\begin{itemize}
\item {} 
\sphinxstyleliteralstrong{\sphinxupquote{threads\_count}} (\sphinxhref{https://docs.python.org/3.7/library/functions.html\#int}{\sphinxstyleliteralemphasis{\sphinxupquote{int}}}) \textendash{} Indicates if multiple simulation instances should run in parallel
(default results in running the simulation in a
single thread).

\item {} 
\sphinxstyleliteralstrong{\sphinxupquote{sdir}} (\sphinxhref{https://docs.python.org/3.7/library/functions.html\#bool}{\sphinxstyleliteralemphasis{\sphinxupquote{bool}}}) \textendash{} Indicates if the user wishes to execute all simulation files
that exist in
{\hyperref[\detokenize{app:app.environment_settings.SIMULATION_ROOT}]{\sphinxcrossref{\sphinxcode{\sphinxupquote{SIMULATION\_ROOT}}}}} or
if he wishes to run one single simulation file, which must be
explicitly specified in \sphinxtitleref{sname}.

\item {} 
\sphinxstyleliteralstrong{\sphinxupquote{sname}} (\sphinxstyleliteralemphasis{\sphinxupquote{Optional}}\sphinxstyleliteralemphasis{\sphinxupquote{{[}}}\sphinxhref{https://docs.python.org/3.7/library/stdtypes.html\#str}{\sphinxstyleliteralemphasis{\sphinxupquote{str}}}\sphinxstyleliteralemphasis{\sphinxupquote{{]}}}) \textendash{} When \sphinxtitleref{sdir} is set to \sphinxcode{\sphinxupquote{False}}, \sphinxtitleref{sname} needs to be specified as a
non blank string containing the name of the simulation file to
be executed. The named file must exist in
{\hyperref[\detokenize{app:app.environment_settings.SIMULATION_ROOT}]{\sphinxcrossref{\sphinxcode{\sphinxupquote{SIMULATION\_ROOT}}}}}.

\item {} 
\sphinxstyleliteralstrong{\sphinxupquote{iters}} (\sphinxhref{https://docs.python.org/3.7/library/functions.html\#int}{\sphinxstyleliteralemphasis{\sphinxupquote{int}}}) \textendash{} The number of times the same simulation file should be executed.

\item {} 
\sphinxstyleliteralstrong{\sphinxupquote{epochs}} (\sphinxhref{https://docs.python.org/3.7/library/functions.html\#int}{\sphinxstyleliteralemphasis{\sphinxupquote{int}}}) \textendash{} The number of discrete time steps each iteration of each instance
of a simulation lasts.

\end{itemize}

\item[{Return type}] \leavevmode
\sphinxhref{https://docs.python.org/3.7/library/constants.html\#None}{None}

\end{description}\end{quote}

\end{fulllineitems}



\section{app.mixing\_rate\_sampler}
\label{\detokenize{app:module-app.mixing_rate_sampler}}\label{\detokenize{app:app-mixing-rate-sampler}}\index{module@\spxentry{module}!app.mixing\_rate\_sampler@\spxentry{app.mixing\_rate\_sampler}}\index{app.mixing\_rate\_sampler@\spxentry{app.mixing\_rate\_sampler}!module@\spxentry{module}}
This is a non\sphinxhyphen{}essential module used for convex optimization prototyping.

This functionality tests and compares the mixing rate of various
markov matrices.

You can start a test by executing the following command:

\begin{sphinxVerbatim}[commandchars=\\\{\}]
\PYGZdl{} python mixing\PYGZus{}rate\PYGZus{}sampler.py \PYGZhy{}\PYGZhy{}samples=1000
\end{sphinxVerbatim}

You can also specify the names of the functions used to generate markov
matrices like so:

\begin{sphinxVerbatim}[commandchars=\\\{\}]
\PYGZdl{} python mixing\PYGZus{}rate\PYGZus{}sampler.py \PYGZhy{}s 10 \PYGZhy{}f afunc,anotherfunc,yetanotherfunc
\end{sphinxVerbatim}

\begin{sphinxadmonition}{note}{Note:}
Default functions set \{ “new\_mh\_transition\_matrix”,
“new\_sdp\_mh\_transition\_matrix”, “new\_go\_transition\_matrix”,
“new\_mgo\_transition\_matrix” \}
\end{sphinxadmonition}
\index{main() (in module app.mixing\_rate\_sampler)@\spxentry{main()}\spxextra{in module app.mixing\_rate\_sampler}}

\begin{fulllineitems}
\phantomsection\label{\detokenize{app:app.mixing_rate_sampler.main}}\pysiglinewithargsret{\sphinxbfcode{\sphinxupquote{main}}}{}{}
Compares the mixing rate of the markov matrices generated by all
specified \sphinxtitleref{functions}, \sphinxtitleref{samples} times.

The execution of the main method results in a JSON file outputed to
\sphinxcode{\sphinxupquote{MIXING\_RATE\_SAMPLE\_ROOT}} folder.

\end{fulllineitems}

\index{\_ResultsDict (in module app.mixing\_rate\_sampler)@\spxentry{\_ResultsDict}\spxextra{in module app.mixing\_rate\_sampler}}

\begin{fulllineitems}
\phantomsection\label{\detokenize{app:app.mixing_rate_sampler._ResultsDict}}\pysigline{\sphinxbfcode{\sphinxupquote{\_ResultsDict}}\sphinxbfcode{\sphinxupquote{: OrderedDict\DUrole{p}{{[}}\sphinxhref{https://docs.python.org/3.7/library/stdtypes.html\#str}{str}\DUrole{p}{, }{\hyperref[\detokenize{app:app.mixing_rate_sampler._SizeResultsDict}]{\sphinxcrossref{\_SizeResultsDict}}}\DUrole{p}{{]}}}}}
\end{fulllineitems}

\index{\_SizeResultsDict (in module app.mixing\_rate\_sampler)@\spxentry{\_SizeResultsDict}\spxextra{in module app.mixing\_rate\_sampler}}

\begin{fulllineitems}
\phantomsection\label{\detokenize{app:app.mixing_rate_sampler._SizeResultsDict}}\pysigline{\sphinxbfcode{\sphinxupquote{\_SizeResultsDict}}\sphinxbfcode{\sphinxupquote{: OrderedDict\DUrole{p}{{[}}\sphinxhref{https://docs.python.org/3.7/library/stdtypes.html\#str}{str}\DUrole{p}{, }List\DUrole{p}{{[}}\sphinxhref{https://docs.python.org/3.7/library/functions.html\#float}{float}\DUrole{p}{{]}}\DUrole{p}{{]}}}}}
\end{fulllineitems}



\section{app.simfile\_generator}
\label{\detokenize{app:module-app.simfile_generator}}\label{\detokenize{app:app-simfile-generator}}\index{module@\spxentry{module}!app.simfile\_generator@\spxentry{app.simfile\_generator}}\index{app.simfile\_generator@\spxentry{app.simfile\_generator}!module@\spxentry{module}}
This scripts’s functions are used to create a simulation file for the user.

You can create a simulation file by following the instructions that
appear in your terminal when running the following command:

\begin{sphinxVerbatim}[commandchars=\\\{\}]
\PYGZdl{} python simfile\PYGZus{}generator.py \PYGZhy{}\PYGZhy{}file=filename.json
\end{sphinxVerbatim}

\begin{sphinxadmonition}{note}{Note:}
Simulation files are placed inside
{\hyperref[\detokenize{app:app.environment_settings.SIMULATION_ROOT}]{\sphinxcrossref{\sphinxcode{\sphinxupquote{SIMULATION\_ROOT}}}}} directory. Any file
used to simulate persistance must be inside
{\hyperref[\detokenize{app:app.environment_settings.SHARED_ROOT}]{\sphinxcrossref{\sphinxcode{\sphinxupquote{SHARED\_ROOT}}}}} directory.
\end{sphinxadmonition}
\index{\_in\_yes\_no() (in module app.simfile\_generator)@\spxentry{\_in\_yes\_no()}\spxextra{in module app.simfile\_generator}}

\begin{fulllineitems}
\phantomsection\label{\detokenize{app:app.simfile_generator._in_yes_no}}\pysiglinewithargsret{\sphinxbfcode{\sphinxupquote{\_in\_yes\_no}}}{\emph{\DUrole{n}{message}}}{}
Asks the user to reply with yes or no to a message.
\begin{quote}\begin{description}
\item[{Parameters}] \leavevmode
\sphinxstyleliteralstrong{\sphinxupquote{message}} (\sphinxhref{https://docs.python.org/3.7/library/stdtypes.html\#str}{\sphinxstyleliteralemphasis{\sphinxupquote{str}}}) \textendash{} The message to be printed to the user upon first input request.

\item[{Returns}] \leavevmode
\sphinxcode{\sphinxupquote{True}} if user presses yes, otherwise \sphinxcode{\sphinxupquote{False}}.

\item[{Return type}] \leavevmode
\sphinxhref{https://docs.python.org/3.7/library/functions.html\#bool}{bool}

\end{description}\end{quote}

\end{fulllineitems}

\index{\_init\_nodes\_uptime() (in module app.simfile\_generator)@\spxentry{\_init\_nodes\_uptime()}\spxextra{in module app.simfile\_generator}}

\begin{fulllineitems}
\phantomsection\label{\detokenize{app:app.simfile_generator._init_nodes_uptime}}\pysiglinewithargsret{\sphinxbfcode{\sphinxupquote{\_init\_nodes\_uptime}}}{}{}
Creates a record containing network nodes’ uptime.
\begin{quote}\begin{description}
\item[{Returns}] \leavevmode
A dictionary where keys are
{\hyperref[\detokenize{app.domain:app.domain.network_nodes.Node.id}]{\sphinxcrossref{\sphinxcode{\sphinxupquote{network node identifiers}}}}}
and values are their respective uptimes
{\hyperref[\detokenize{app.domain:app.domain.network_nodes.Node.uptime}]{\sphinxcrossref{\sphinxcode{\sphinxupquote{uptime}}}}} values.

\item[{Return type}] \leavevmode
Dict{[}\sphinxhref{https://docs.python.org/3.7/library/stdtypes.html\#str}{str}, \sphinxhref{https://docs.python.org/3.7/library/functions.html\#float}{float}{]}

\end{description}\end{quote}

\end{fulllineitems}

\index{\_init\_persisting\_dict() (in module app.simfile\_generator)@\spxentry{\_init\_persisting\_dict()}\spxextra{in module app.simfile\_generator}}

\begin{fulllineitems}
\phantomsection\label{\detokenize{app:app.simfile_generator._init_persisting_dict}}\pysiglinewithargsret{\sphinxbfcode{\sphinxupquote{\_init\_persisting\_dict}}}{}{}
Creates the “persisting” key of simulation file.
\begin{quote}\begin{description}
\item[{Returns}] \leavevmode
A dictionary containing data respecting files to be shared in the system

\item[{Return type}] \leavevmode
Dict{[}\sphinxhref{https://docs.python.org/3.7/library/stdtypes.html\#str}{str}, Any{]}

\end{description}\end{quote}

\end{fulllineitems}

\index{\_input\_bounded\_float() (in module app.simfile\_generator)@\spxentry{\_input\_bounded\_float()}\spxextra{in module app.simfile\_generator}}

\begin{fulllineitems}
\phantomsection\label{\detokenize{app:app.simfile_generator._input_bounded_float}}\pysiglinewithargsret{\sphinxbfcode{\sphinxupquote{\_input\_bounded\_float}}}{\emph{\DUrole{n}{message}}, \emph{\DUrole{n}{lower\_bound}\DUrole{o}{=}\DUrole{default_value}{0.0}}, \emph{\DUrole{n}{upper\_bound}\DUrole{o}{=}\DUrole{default_value}{100.0}}}{}
Obtains a user inputed integer within the specified closed interval.
\begin{quote}\begin{description}
\item[{Parameters}] \leavevmode\begin{itemize}
\item {} 
\sphinxstyleliteralstrong{\sphinxupquote{message}} (\sphinxhref{https://docs.python.org/3.7/library/stdtypes.html\#str}{\sphinxstyleliteralemphasis{\sphinxupquote{str}}}) \textendash{} The message to be printed to the user upon first input request.

\item {} 
\sphinxstyleliteralstrong{\sphinxupquote{lower\_bound}} (\sphinxhref{https://docs.python.org/3.7/library/functions.html\#float}{\sphinxstyleliteralemphasis{\sphinxupquote{float}}}) \textendash{} Any input smaller than\textasciigrave{}lower\_bound\textasciigrave{} is rejected.

\item {} 
\sphinxstyleliteralstrong{\sphinxupquote{upper\_bound}} (\sphinxhref{https://docs.python.org/3.7/library/functions.html\#float}{\sphinxstyleliteralemphasis{\sphinxupquote{float}}}) \textendash{} Any input bigger than \sphinxtitleref{upper\_bound} is rejected.

\end{itemize}

\item[{Returns}] \leavevmode
An float inputed by the user.

\item[{Return type}] \leavevmode
\sphinxhref{https://docs.python.org/3.7/library/functions.html\#float}{float}

\end{description}\end{quote}

\end{fulllineitems}

\index{\_input\_bounded\_integer() (in module app.simfile\_generator)@\spxentry{\_input\_bounded\_integer()}\spxextra{in module app.simfile\_generator}}

\begin{fulllineitems}
\phantomsection\label{\detokenize{app:app.simfile_generator._input_bounded_integer}}\pysiglinewithargsret{\sphinxbfcode{\sphinxupquote{\_input\_bounded\_integer}}}{\emph{\DUrole{n}{message}}, \emph{\DUrole{n}{lower\_bound}\DUrole{o}{=}\DUrole{default_value}{2}}, \emph{\DUrole{n}{upper\_bound}\DUrole{o}{=}\DUrole{default_value}{10000000}}}{}
Obtains a user inputed integer within the specified closed interval.
\begin{quote}\begin{description}
\item[{Parameters}] \leavevmode\begin{itemize}
\item {} 
\sphinxstyleliteralstrong{\sphinxupquote{message}} (\sphinxhref{https://docs.python.org/3.7/library/stdtypes.html\#str}{\sphinxstyleliteralemphasis{\sphinxupquote{str}}}) \textendash{} The message to be printed to the user upon first input request.

\item {} 
\sphinxstyleliteralstrong{\sphinxupquote{lower\_bound}} (\sphinxhref{https://docs.python.org/3.7/library/functions.html\#int}{\sphinxstyleliteralemphasis{\sphinxupquote{int}}}) \textendash{} Any input equal or smaller than \sphinxtitleref{lower\_bound} is
rejected.

\item {} 
\sphinxstyleliteralstrong{\sphinxupquote{upper\_bound}} (\sphinxhref{https://docs.python.org/3.7/library/functions.html\#int}{\sphinxstyleliteralemphasis{\sphinxupquote{int}}}) \textendash{} Any input equal or bigger than \sphinxtitleref{upper\_bound} is rejected.

\end{itemize}

\item[{Returns}] \leavevmode
An integer inputed by the user.

\item[{Return type}] \leavevmode
\sphinxhref{https://docs.python.org/3.7/library/functions.html\#int}{int}

\end{description}\end{quote}

\end{fulllineitems}

\index{\_input\_character\_option() (in module app.simfile\_generator)@\spxentry{\_input\_character\_option()}\spxextra{in module app.simfile\_generator}}

\begin{fulllineitems}
\phantomsection\label{\detokenize{app:app.simfile_generator._input_character_option}}\pysiglinewithargsret{\sphinxbfcode{\sphinxupquote{\_input\_character\_option}}}{\emph{\DUrole{n}{message}}, \emph{\DUrole{n}{white\_list}}}{}
Obtains a user inputed character within a predefined set.
\begin{quote}\begin{description}
\item[{Parameters}] \leavevmode\begin{itemize}
\item {} 
\sphinxstyleliteralstrong{\sphinxupquote{message}} (\sphinxhref{https://docs.python.org/3.7/library/stdtypes.html\#str}{\sphinxstyleliteralemphasis{\sphinxupquote{str}}}) \textendash{} The message to be printed to the user upon first input request.

\item {} 
\sphinxstyleliteralstrong{\sphinxupquote{white\_list}} (\sphinxstyleliteralemphasis{\sphinxupquote{List}}\sphinxstyleliteralemphasis{\sphinxupquote{{[}}}\sphinxhref{https://docs.python.org/3.7/library/stdtypes.html\#str}{\sphinxstyleliteralemphasis{\sphinxupquote{str}}}\sphinxstyleliteralemphasis{\sphinxupquote{{]}}}) \textendash{} A list of valid option characters.

\end{itemize}

\item[{Returns}] \leavevmode
The character that represents the initial distribution of files in a
\sphinxcode{\sphinxupquote{domain.cluster\_groups}}’s class instance desired by the user.

\item[{Return type}] \leavevmode
\sphinxhref{https://docs.python.org/3.7/library/stdtypes.html\#str}{str}

\end{description}\end{quote}

\end{fulllineitems}

\index{\_input\_filename() (in module app.simfile\_generator)@\spxentry{\_input\_filename()}\spxextra{in module app.simfile\_generator}}

\begin{fulllineitems}
\phantomsection\label{\detokenize{app:app.simfile_generator._input_filename}}\pysiglinewithargsret{\sphinxbfcode{\sphinxupquote{\_input\_filename}}}{\emph{\DUrole{n}{message}}}{}
Asks the user to input the name of a file in the command line terminal.

A warning message is displayed if the specified file does not exist inside
{\hyperref[\detokenize{app:app.environment_settings.SHARED_ROOT}]{\sphinxcrossref{\sphinxcode{\sphinxupquote{SHARED\_ROOT}}}}}

\begin{sphinxadmonition}{note}{Note:}
Defaults to \sphinxcode{\sphinxupquote{"FBZ\_0134.NEF"}} when input is blank. This file should
be present inside {\hyperref[\detokenize{app:app.environment_settings.SHARED_ROOT}]{\sphinxcrossref{\sphinxcode{\sphinxupquote{SHARED\_ROOT}}}}}
unless it was previously deleted by the user.
\end{sphinxadmonition}
\begin{quote}\begin{description}
\item[{Parameters}] \leavevmode
\sphinxstyleliteralstrong{\sphinxupquote{message}} (\sphinxhref{https://docs.python.org/3.7/library/stdtypes.html\#str}{\sphinxstyleliteralemphasis{\sphinxupquote{str}}}) \textendash{} The message to be printed to the user upon first input request.

\item[{Returns}] \leavevmode
A file name with extension.

\item[{Return type}] \leavevmode
\sphinxhref{https://docs.python.org/3.7/library/stdtypes.html\#str}{str}

\end{description}\end{quote}

\end{fulllineitems}

\index{main() (in module app.simfile\_generator)@\spxentry{main()}\spxextra{in module app.simfile\_generator}}

\begin{fulllineitems}
\phantomsection\label{\detokenize{app:app.simfile_generator.main}}\pysiglinewithargsret{\sphinxbfcode{\sphinxupquote{main}}}{\emph{\DUrole{n}{simfile\_name}}}{}
Creates a JSON file within the user’s file system that is used by
\sphinxcode{\sphinxupquote{hive\_simulation}}.

\begin{sphinxadmonition}{note}{Note:}
The name of the created file concerns the name of the simulation file.
It does not concern the name or names of the files whose persistence
is being simulated.
\end{sphinxadmonition}
\begin{quote}\begin{description}
\item[{Parameters}] \leavevmode
\sphinxstyleliteralstrong{\sphinxupquote{simfile\_name}} (\sphinxhref{https://docs.python.org/3.7/library/stdtypes.html\#str}{\sphinxstyleliteralemphasis{\sphinxupquote{str}}}) \textendash{} Name to be assigned to JSON file in the user’s file system.

\item[{Return type}] \leavevmode
\sphinxhref{https://docs.python.org/3.7/library/constants.html\#None}{None}

\end{description}\end{quote}

\end{fulllineitems}

\index{yield\_label() (in module app.simfile\_generator)@\spxentry{yield\_label()}\spxextra{in module app.simfile\_generator}}

\begin{fulllineitems}
\phantomsection\label{\detokenize{app:app.simfile_generator.yield_label}}\pysiglinewithargsret{\sphinxbfcode{\sphinxupquote{yield\_label}}}{}{}
Used to generate an arbrirary numbers of unique labels.
\begin{quote}
\begin{description}
\item[{Examples:}] \leavevmode
The following code snippets illustrate the result of calling this
method \sphinxcode{\sphinxupquote{n}} times.

\begin{sphinxVerbatim}[commandchars=\\\{\}]
 \PYG{o}{\PYGZgt{}\PYGZgt{}}\PYG{o}{\PYGZgt{}} \PYG{n}{n} \PYG{o}{=} \PYG{l+m+mi}{4}
 \PYG{o}{\PYGZgt{}\PYGZgt{}}\PYG{o}{\PYGZgt{}} \PYG{k}{for} \PYG{n}{s} \PYG{o+ow}{in} \PYG{n}{itertools}\PYG{o}{.}\PYG{n}{islice}\PYG{p}{(}\PYG{n}{yield\PYGZus{}label}\PYG{p}{(}\PYG{p}{)}\PYG{p}{,} \PYG{n}{n}\PYG{p}{)}\PYG{p}{:}
 \PYG{o}{.}\PYG{o}{.}\PYG{o}{.}     \PYG{k}{return} \PYG{n}{s}
 \PYG{p}{[}\PYG{n}{a}\PYG{p}{,} \PYG{n}{b}\PYG{p}{,} \PYG{n}{c}\PYG{p}{,} \PYG{n}{d}\PYG{p}{]}

\PYG{o}{\PYGZgt{}\PYGZgt{}}\PYG{o}{\PYGZgt{}} \PYG{n}{n} \PYG{o}{=} \PYG{l+m+mi}{4} \PYG{o}{+} \PYG{l+m+mi}{26}
 \PYG{o}{\PYGZgt{}\PYGZgt{}}\PYG{o}{\PYGZgt{}} \PYG{k}{for} \PYG{n}{s} \PYG{o+ow}{in} \PYG{n}{itertools}\PYG{o}{.}\PYG{n}{islice}\PYG{p}{(}\PYG{n}{yield\PYGZus{}label}\PYG{p}{(}\PYG{p}{)}\PYG{p}{,} \PYG{n}{n}\PYG{p}{)}\PYG{p}{:}
 \PYG{o}{.}\PYG{o}{.}\PYG{o}{.}     \PYG{k}{return} \PYG{n}{s}
 \PYG{p}{[}\PYG{n}{a}\PYG{p}{,} \PYG{n}{b}\PYG{p}{,} \PYG{n}{c}\PYG{p}{,} \PYG{n}{d}\PYG{p}{,} \PYG{o}{.}\PYG{o}{.}\PYG{o}{.}\PYG{p}{,} \PYG{n}{aa}\PYG{p}{,} \PYG{n}{ab}\PYG{p}{,} \PYG{n}{ac}\PYG{p}{,} \PYG{n}{ad}\PYG{p}{]}
\end{sphinxVerbatim}

\end{description}
\end{quote}
\begin{quote}\begin{description}
\item[{Yields}] \leavevmode
The next string label in the sequence.

\item[{Return type}] \leavevmode
\sphinxhref{https://docs.python.org/3.7/library/stdtypes.html\#str}{str}

\end{description}\end{quote}

\end{fulllineitems}



\section{app.type\_hints}
\label{\detokenize{app:module-app.type_hints}}\label{\detokenize{app:app-type-hints}}\index{module@\spxentry{module}!app.type\_hints@\spxentry{app.type\_hints}}\index{app.type\_hints@\spxentry{app.type\_hints}!module@\spxentry{module}}\index{ClusterDict (in module app.type\_hints)@\spxentry{ClusterDict}\spxextra{in module app.type\_hints}}

\begin{fulllineitems}
\phantomsection\label{\detokenize{app:app.type_hints.ClusterDict}}\pysigline{\sphinxbfcode{\sphinxupquote{ClusterDict}}\sphinxbfcode{\sphinxupquote{: Dict\DUrole{p}{{[}}\sphinxhref{https://docs.python.org/3.7/library/stdtypes.html\#str}{str}\DUrole{p}{, }{\hyperref[\detokenize{app:app.type_hints.ClusterType}]{\sphinxcrossref{ClusterType}}}\DUrole{p}{{]}}}}}
\end{fulllineitems}

\index{ClusterType (in module app.type\_hints)@\spxentry{ClusterType}\spxextra{in module app.type\_hints}}

\begin{fulllineitems}
\phantomsection\label{\detokenize{app:app.type_hints.ClusterType}}\pysigline{\sphinxbfcode{\sphinxupquote{ClusterType}}\sphinxbfcode{\sphinxupquote{: Union\DUrole{p}{{[}}cg.Cluster\DUrole{p}{, }cg.SGCluster\DUrole{p}{, }cg.SGClusterExt\DUrole{p}{, }cg.HDFSCluster\DUrole{p}{, }cg.NewscastCluster\DUrole{p}{{]}}}}}
\end{fulllineitems}

\index{HttpResponse (in module app.type\_hints)@\spxentry{HttpResponse}\spxextra{in module app.type\_hints}}

\begin{fulllineitems}
\phantomsection\label{\detokenize{app:app.type_hints.HttpResponse}}\pysigline{\sphinxbfcode{\sphinxupquote{HttpResponse}}\sphinxbfcode{\sphinxupquote{: Union\DUrole{p}{{[}}\sphinxhref{https://docs.python.org/3.7/library/functions.html\#int}{int}\DUrole{p}{, }e.HttpCodes\DUrole{p}{{]}}}}}
\end{fulllineitems}

\index{MasterType (in module app.type\_hints)@\spxentry{MasterType}\spxextra{in module app.type\_hints}}

\begin{fulllineitems}
\phantomsection\label{\detokenize{app:app.type_hints.MasterType}}\pysigline{\sphinxbfcode{\sphinxupquote{MasterType}}\sphinxbfcode{\sphinxupquote{: Union\DUrole{p}{{[}}ms.Master\DUrole{p}{, }ms.SGMaster\DUrole{p}{, }ms.HDFSMaster\DUrole{p}{, }ms.NewscastMaster\DUrole{p}{{]}}}}}
\end{fulllineitems}

\index{NodeDict (in module app.type\_hints)@\spxentry{NodeDict}\spxextra{in module app.type\_hints}}

\begin{fulllineitems}
\phantomsection\label{\detokenize{app:app.type_hints.NodeDict}}\pysigline{\sphinxbfcode{\sphinxupquote{NodeDict}}\sphinxbfcode{\sphinxupquote{: Dict\DUrole{p}{{[}}\sphinxhref{https://docs.python.org/3.7/library/stdtypes.html\#str}{str}\DUrole{p}{, }{\hyperref[\detokenize{app:app.type_hints.NodeType}]{\sphinxcrossref{NodeType}}}\DUrole{p}{{]}}}}}
\end{fulllineitems}

\index{NodeType (in module app.type\_hints)@\spxentry{NodeType}\spxextra{in module app.type\_hints}}

\begin{fulllineitems}
\phantomsection\label{\detokenize{app:app.type_hints.NodeType}}\pysigline{\sphinxbfcode{\sphinxupquote{NodeType}}\sphinxbfcode{\sphinxupquote{: Union\DUrole{p}{{[}}nn.Node\DUrole{p}{, }nn.SGNode\DUrole{p}{, }nn.SGNodeExt\DUrole{p}{, }nn.HDFSNode\DUrole{p}{, }nn.NewscastNode\DUrole{p}{{]}}}}}
\end{fulllineitems}

\index{ReplicasDict (in module app.type\_hints)@\spxentry{ReplicasDict}\spxextra{in module app.type\_hints}}

\begin{fulllineitems}
\phantomsection\label{\detokenize{app:app.type_hints.ReplicasDict}}\pysigline{\sphinxbfcode{\sphinxupquote{ReplicasDict}}\sphinxbfcode{\sphinxupquote{: Dict\DUrole{p}{{[}}\sphinxhref{https://docs.python.org/3.7/library/functions.html\#int}{int}\DUrole{p}{, }sd.FileBlockData\DUrole{p}{{]}}}}}
\end{fulllineitems}



\chapter{Notes}
\label{\detokenize{notedocs:notes}}\label{\detokenize{notedocs::doc}}

\section{Future Releases}
\label{\detokenize{notedocs:future-releases}}
In the future, we will focus on improving each simulation’s thread performance,
in the current release, any P2P network with more than 16 peers and many files
can take a long time to complete due to the amount of \sphinxstyleemphasis{for} statements that
exist in the code. This is one reason why our program offers the possibility
of running multiple (different) simulations in different threads, allowing
researchers to complete more simulations, in less time, by fully utilizing the
CPU of their machines. Ideally, we would like to offer this speed up using
multi\sphinxhyphen{}threading and fast individual threads.


\chapter{Indices}
\label{\detokenize{indices:indices}}\label{\detokenize{indices::doc}}\begin{itemize}
\item {} 
\DUrole{xref,std,std-ref}{genindex}

\item {} 
\DUrole{xref,std,std-ref}{modindex}

\end{itemize}

This project was born during the development and writing of a master’s
dissertation in Computer Science and Engineering, through a research
sponsorship granted by \sphinxhref{https://welcome.isr.tecnico.ulisboa.pt/}{ISR} for the participation in the project
\sphinxstyleemphasis{UID/EEA/50009/2019 \sphinxhyphen{} 1801P.00920.1.02 DSOR}. ISR is research and
development institution affiliated with \sphinxhref{https://tecnico.ulisboa.pt/en/}{IST}, the university where said
dissertation is submitted at.

The project’s work consisted of adapting and optimizing a few algorithms widely
used and studied in robotics and control research fields to a P2P scenario
where the peers form the basis for a Distributed Backup System. Throughout this
process, due to the lack of other viable options, the researchers wrote their
own cycle based simulator, and the result was \sphinxstyleemphasis{Hives}.

Hives is a P2P Stochastic Swarm Guidance Simulator that facilitates
research by allowing developers to prototype P2P networks based on swarm
guidance behaviors quickly. The simulator is written in \sphinxhref{http://www.python.org/}{Python} (version 3.7.7),
which offers users easy access to powerful scientific libraries such as \sphinxhref{https://numpy.org/}{NumPy},
\sphinxhref{https://www.scipy.org/}{SciPy}, and \sphinxhref{https://pandas.pydata.org/}{Pandas}, which are not readily available in languages like \sphinxhref{https://www.oracle.com/java/technologies/javase-jdk14-downloads.html/}{Java}
and as a result of some of the best or most well\sphinxhyphen{}known simulators out there.

\sphinxincludegraphics[width=0.450\linewidth]{{ist-logo}.png} \sphinxincludegraphics[width=0.450\linewidth]{{isr-logo}.png}


\renewcommand{\indexname}{Python Module Index}
\begin{sphinxtheindex}
\let\bigletter\sphinxstyleindexlettergroup
\bigletter{a}
\item\relax\sphinxstyleindexentry{app}\sphinxstyleindexpageref{app:\detokenize{module-app}}
\item\relax\sphinxstyleindexentry{app.domain}\sphinxstyleindexpageref{app.domain:\detokenize{module-app.domain}}
\item\relax\sphinxstyleindexentry{app.domain.cluster\_groups}\sphinxstyleindexpageref{app.domain:\detokenize{module-app.domain.cluster_groups}}
\item\relax\sphinxstyleindexentry{app.domain.helpers}\sphinxstyleindexpageref{app.domain.helpers:\detokenize{module-app.domain.helpers}}
\item\relax\sphinxstyleindexentry{app.domain.helpers.enums}\sphinxstyleindexpageref{app.domain.helpers:\detokenize{module-app.domain.helpers.enums}}
\item\relax\sphinxstyleindexentry{app.domain.helpers.exceptions}\sphinxstyleindexpageref{app.domain.helpers:\detokenize{module-app.domain.helpers.exceptions}}
\item\relax\sphinxstyleindexentry{app.domain.helpers.matlab\_utils}\sphinxstyleindexpageref{app.domain.helpers:\detokenize{module-app.domain.helpers.matlab_utils}}
\item\relax\sphinxstyleindexentry{app.domain.helpers.matrices}\sphinxstyleindexpageref{app.domain.helpers:\detokenize{module-app.domain.helpers.matrices}}
\item\relax\sphinxstyleindexentry{app.domain.helpers.smart\_dataclasses}\sphinxstyleindexpageref{app.domain.helpers:\detokenize{module-app.domain.helpers.smart_dataclasses}}
\item\relax\sphinxstyleindexentry{app.domain.master\_servers}\sphinxstyleindexpageref{app.domain:\detokenize{module-app.domain.master_servers}}
\item\relax\sphinxstyleindexentry{app.domain.network\_nodes}\sphinxstyleindexpageref{app.domain:\detokenize{module-app.domain.network_nodes}}
\item\relax\sphinxstyleindexentry{app.environment\_settings}\sphinxstyleindexpageref{app:\detokenize{module-app.environment_settings}}
\item\relax\sphinxstyleindexentry{app.hive\_simulation}\sphinxstyleindexpageref{app:\detokenize{module-app.hive_simulation}}
\item\relax\sphinxstyleindexentry{app.mixing\_rate\_sampler}\sphinxstyleindexpageref{app:\detokenize{module-app.mixing_rate_sampler}}
\item\relax\sphinxstyleindexentry{app.simfile\_generator}\sphinxstyleindexpageref{app:\detokenize{module-app.simfile_generator}}
\item\relax\sphinxstyleindexentry{app.type\_hints}\sphinxstyleindexpageref{app:\detokenize{module-app.type_hints}}
\item\relax\sphinxstyleindexentry{app.utils}\sphinxstyleindexpageref{app.utils:\detokenize{module-app.utils}}
\item\relax\sphinxstyleindexentry{app.utils.convertions}\sphinxstyleindexpageref{app.utils:\detokenize{module-app.utils.convertions}}
\item\relax\sphinxstyleindexentry{app.utils.crypto}\sphinxstyleindexpageref{app.utils:\detokenize{module-app.utils.crypto}}
\item\relax\sphinxstyleindexentry{app.utils.randoms}\sphinxstyleindexpageref{app.utils:\detokenize{module-app.utils.randoms}}
\end{sphinxtheindex}

\renewcommand{\indexname}{Index}
\printindex
\end{document}