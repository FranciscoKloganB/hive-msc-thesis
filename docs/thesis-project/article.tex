%% fsip2pnupsg.tex
%% V0.1
%% 2019/11/27
%% by Francisco Barros

%%%%%%%%%%%%%%%%%
% BEGIN IMPORTS %
%%%%%%%%%%%%%%%%%
\documentclass[runningheads]{llncs}

\usepackage{graphicx}
\usepackage{cite}
\usepackage{amsmath}
\usepackage{mathtools}
\usepackage{enumerate}
\usepackage{optidef}
\usepackage{subfigure}
\usepackage{algorithmic}
\usepackage{algorithm}
%%%%%%%%%%%%%%%
% END IMPORTS %
%%%%%%%%%%%%%%%

%%%%%%%%%%%%%%%%%%
% BEGIN DOCUMENT %
%%%%%%%%%%%%%%%%%%
\begin{document}

%%%%%%%%%%%%%%%%
% BEGIN HEADER %
%%%%%%%%%%%%%%%%
\title {File Survivability in P2P Networks using Probabilistic Swarm Guidance
\thanks{This work was partially supported by the project MYRG2016-00097-FST of the University of Macau; by the Portuguese Fundação para a Ciência e a Tecnologia (FCT) through Institute for Systems and Robotics (ISR), under Laboratory for Robotics and Engineering Systems (LARSyS) project UID/EEA/50009/2019.}
}
\titlerunning{Hive}
\author{Francisco Barros, Daniel Silvestre \and Carlos Silvestre}
\authorrunning{F. Barros et al.}
\institute{Instituo Superior Técnico - Taguspark\newline Av. Prof. Doutor Cavaco Silva, 2744-016 Porto Salvo, Portugal
\email{fbarros@isr.ist.utl.pt, dsilvestre@isr.ist.utl.pt, csilvestre@umac.mo}\newline
\url{www.tecnico.ulisboa.pt}
}
\maketitle
%%%%%%%%%%%%%%
% END HEADER %
%%%%%%%%%%%%%%
%%%%%%%%%%%%%%%%%%
% BEGIN ABSTRACT %
%%%%%%%%%%%%%%%%%%
\begin{abstract}
The abstract should briefly summarize the contents of the paper in 150--250 words.
\begin{keywords}Agents-based Systems; Cooperative Control; Peer-to-Peer; Cloud Storage; Distributed control; File Availability; Swarm Guidance;\end{keywords}
\end{abstract}
%%%%%%%%%%%%%%%%
% END ABSTRACT %
%%%%%%%%%%%%%%%%

%%%%%%%%%%%%%%%
% BEGIN INDEX %
%%%%%%%%%%%%%%%
\section{Index}\label{sec:index}
The remainder of this paper is organized as follows. In Section \ref{sec:intro}...
%%%%%%%%%%%%%
% END INDEX %
%%%%%%%%%%%%%

%%%%%%%%%%%%%%%%%%%%%%
% BEGIN INTRODUCTION %
%%%%%%%%%%%%%%%%%%%%%%
\section{Introduction}\label{sec:intro}
With the growth of the internet and the emergence of distributed systems, two large-scale computing paradigms have gained popularity throughout the last two decades due to their promise of virtually, almost unlimited, scalability. On one hand,  Peer-to-Peer (P2P) networking can be broadly defined as a set of equally privileged peers that contribute with a portion of their resources in order to achieve common goals. These networks are very popular \cite{ssaroiu:msp2pfss} among file sharing and streaming applications, due to their self-organized behavior, lack of centralization and low-cost. P2P networks can be leveraged by communities for availability purposes as peers can replicate each others' files decreasing the odds of file loss due to the failure of a small number of machines, i.e. P2P based-systems can be used for file storage or file backups. On the other hand Cloud-based systems offer unmatched availability and reliability , which are currently very trendy with services such as Amazon S3, Microsoft Azure and Google-Cloud providing all types of IT
\subsection{Goals}\label{subsec:intro}
\textbf{What is usually done and how are we better:} todo\newline
\textbf{The main contributions of this paper can be summarized as follows:}
\begin{itemize}
	\item colocar contribuicoes
\end{itemize}
%%%%%%%%%%%%%%%%%%%%
% END INTRODUCTION %
%%%%%%%%%%%%%%%%%%%%

%%%%%%%%%%%%%%%%%%%%%%
% BEGIN RELATED WORK %
%%%%%%%%%%%%%%%%%%%%%%
\section{Related Work}\label{sec:relatedwork}
%%%%%%%%%%%%%%%%%%%%
% END RELATED WORK %
%%%%%%%%%%%%%%%%%%%%

%%%%%%%%%%%%%%%%%%%%%%%%%%%
% BEGIN SOLUTION PROPOSAL %
%%%%%%%%%%%%%%%%%%%%%%%%%%%
\section{Solution Proposal}\label{sec:proposal}
%%%%%%%%%%%%%%%%%%%%%%%%%
% END SOLUTION PROPOSAL %
%%%%%%%%%%%%%%%%%%%%%%%%%

%%%%%%%%%%%%%%%%%%%%%%%%%%%%%
% BEGIN TESTING METHODOLOGY %
%%%%%%%%%%%%%%%%%%%%%%%%%%%%%
\section{Testing Methodology}\label{sec:methodology}
%%%%%%%%%%%%%%%%%%%%%%%%%%%
% END TESTING METHODOLOGY %
%%%%%%%%%%%%%%%%%%%%%%%%%%%

%%%%%%%%%%%%%%%%%%%%%%%
% BEGIN WORK SCHEDULE %
%%%%%%%%%%%%%%%%%%%%%%%
\section{Work Schedule}\label{sec:workschedule}
%%%%%%%%%%%%%%%%%%%%%
% END WORK SCHEDULE %
%%%%%%%%%%%%%%%%%%%%%

%%%%%%%%%%%%%%%%%%%%%
% BEGIN CONCLUSIONS %
%%%%%%%%%%%%%%%%%%%%%
\section{Conclusions}\label{sec:conclusion}
%%%%%%%%%%%%%%%%%%%
% END CONCLUSIONS %
%%%%%%%%%%%%%%%%%%%

%%%%%%%%%%%%%%%%%%%%
% BEGIN REFERENCES %
%%%%%%%%%%%%%%%%%%%%
\bibliographystyle{splncs04}
\bibliography{bibliography}
%%%%%%%%%%%%%%%%%%
% END REFERENCES %
%%%%%%%%%%%%%%%%%%

\end{document}
%%%%%%%%%%%%%%%%
% END DOCUMENT %
%%%%%%%%%%%%%%%%
