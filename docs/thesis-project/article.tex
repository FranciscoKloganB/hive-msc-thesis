%% fsip2pnupsg.tex
%% V0.1
%% 2019/11/27
%% by Francisco Barros

%%%%%%%%%%%%%%%%%
% BEGIN IMPORTS %
%%%%%%%%%%%%%%%%%
\documentclass[runningheads]{llncs}
\usepackage{indentfirst}
\usepackage{graphicx}
\usepackage{cite}
\usepackage{amsmath}
\usepackage{mathtools}
\usepackage{enumerate}
\usepackage{optidef}
\usepackage{subfigure}
\usepackage{algorithmic}
\usepackage{algorithm}

\newcommand{\SubItem}[1]{{\setlength\itemindent{15pt} \item[-] #1}}
%%%%%%%%%%%%%%%
% END IMPORTS %
%%%%%%%%%%%%%%%

%%%%%%%%%%%%%%%%%%
% BEGIN DOCUMENT %
%%%%%%%%%%%%%%%%%%
\begin{document}

%%%%%%%%%%%%%%%%
% BEGIN HEADER %
%%%%%%%%%%%%%%%%
\title {Hives: File Survivability in P2P Networks using Probabilistic Swarm Guidance
\thanks{This work has the support of the project MYRG2016-00097-FST of the University of Macau; by the Portuguese Fundação para a Ciência e a Tecnologia (FCT) through Institute for Systems and Robotics (ISR), under Laboratory for Robotics and Engineering Systems (LARSyS) project UID/EEA/50009/2019.}
}
\titlerunning{Hives}
\author{Francisco Barros, Daniel Silvestre \and Carlos Silvestre}
\authorrunning{F. Barros et al.}
\institute{Instituto Superior Técnico - Taguspark\newline Av. Prof. Doutor Cavaco Silva, 2744-016 Porto Salvo, Portugal
\email{fbarros@isr.ist.utl.pt, dsilvestre@isr.ist.utl.pt, csilvestre@umac.mo}\newline
\url{www.tecnico.ulisboa.pt}
}
\maketitle
%%%%%%%%%%%%%%
% END HEADER %
%%%%%%%%%%%%%%
%%%%%%%%%%%%%%%%%%
% BEGIN ABSTRACT %
%%%%%%%%%%%%%%%%%%
\begin{abstract}
The abstract should briefly summarize the contents of the paper in 150--250 words.
\begin{keywords}Agents-based Systems; Cooperative Control; Peer-to-Peer; Cloud Storage; Distributed control; File Availability; Swarm Guidance;\end{keywords}
\end{abstract}
%%%%%%%%%%%%%%%%
% END ABSTRACT %
%%%%%%%%%%%%%%%%

%%%%%%%%%%%%%%%
% BEGIN INDEX %
%%%%%%%%%%%%%%%
\section{Index}\label{sec:index}\newpage
%%%%%%%%%%%%%
% END INDEX %
%%%%%%%%%%%%%

%%%%%%%%%%%%%%%%%%%%%%
% BEGIN INTRODUCTION %
%%%%%%%%%%%%%%%%%%%%%%
\section{Introduction}\label{sec:intro}

\subsection{Problem Description}
With the growth of the internet and the emergence of distributed systems, two large-scale computing paradigms have gained popularity due to their promise of virtually unlimited scalability. On the one hand, we have Peer-to-Peer (P2P) networking, which can be defined as a group of equally privileged peers, who contribute with a portion of their resources, to achieve common goals. These networks are popular among file sharing and streaming applications \cite{ssaroiu:msp2pfss}, due to their self-organized behavior, lack of centralization and, low-cost. On the other hand, Cloud platforms offer unmatched, on-demand, self-service, availability, and reliability, at a higher cost, and are currently trendy, with companies such as Amazon, Google, and Microsoft offering various ITaaS products to individuals and organizations alike. Cloud-based systems are centralized architectures, in which a large number of computers are clustered and managed by master entities, which possibly, become bottlenecks. Both paradigms can be used to deploy distributed file-storage or file-backup applications; despite the fact that P2P implementations are cheaper for both companies and their clients, these have a hard-time achieving 99.9\% availability seen Cloud-based implementations, furthermore, depending on the P2P application there may also exist the risk of permanent file loss, making them somewhat unappealing for clients who want to store or backup their files remotely.

\subsection{Our approach}
The focus of this thesis is to create a structured P2P overlay to be used on a distributed backup system, where clients upload their files to remote peers, who are selling their storage space, without persistence guarantees. Probabilistic Swarm Guidance (PSG) and Markov Chain theories \cite{markovchain_approach_to_pga} will be used, along with block-based file replication using erasure code algorithms and a reputation system, to hopefully create a network that is able to keep a file available at any given time, regardless of presence of attackers, network partitions and amount of churn, i.e., no matter how many peers leave/join an overlay.

\subsection
{Frequent approach}There has been much research on the topic of P2P overlays. There are two significant categories of overlays, Structured and Unstructured. In the former, the overlay has rigid rules regarding peer placement; they favor look-up speed but have higher maintenance overhead under churn, i.e., the process of peers leaving or joining the overlay. Some popular P2P Structured overlays include Chord \cite{chord}, Tapestry \cite{tapestry}, Kademlia \cite{kademlia}, and P-Grid \cite{pgrid}. In the latter, peer placement is random, which provides the overlay network with a higher degree of resilience and robustness in the advent of failures. Unstructured overlays disseminate information in a gossip-like fashion; thus, look-up operations are slower, resource management is less efficient, but overlay maintenance is often straightforward. Popular  Unstructured P2P overlays include FreeNet \cite{freenet}, Gnutella \cite{gnutella} and BitTorrent \cite{bittorrent}. Research on multi-layer overlays has not yielded satisfactory results, and implementation is difficult \cite{sotart}, but research on bio-inspired overlays did have promising results, some of these include Self-Chord \cite{selfchord}. Finally, there have also been some attempts of transparently integrating Cloud Systems into P2P networks, e.g., CLOUDCAST \cite{cloudcast}, and vice-versa, e.g., Spotify \cite{spotify} and Wuala \cite{wuala}. From all the mentioned research, perhaps the one that borrows most similarity with our proposal is CLOUDCAST; however, neither it nor any of the remaining literature, try to create a structured overlay composed of completely autonomous peers, who work together to keep the file alive without enforced persistence based on PSG.

\subsection{Proposal Goals}\label{subsec:intro}
\begin{itemize}
    \item Implement a Tracker server with light-weight responsibilities:
        \SubItem{Act as peer discovery service for clients;}
        \SubItem{Act as an indexer for clusters of peers (Hives);}
        \SubItem{Track peer reputations;}
    \item Implement an erasure-code algorithm to ensure file persists in the Hive;
    \item Implement a trust system where:
        \SubItem Each peer gives feedback to the Tracker about his neighbors;
        \SubItem Tracker synthesizes the feedback and presents it to the client;
    \item Calculate the ideal steady-state distribution of file blocks, for a Hive, when:
        \SubItem{No trust history on a new peer is available;}
        \SubItem{Peer availability, reliability and other trust factors are known;}
    \item Hives converge to ideal block distribution in bounded-times using PSG;
    \item Efficiently update the PSG's Markov Chain as it changes over time;
    \item Balance the in-degree and out-degree of each peer such that:
        \SubItem{Message exchange overhead is minimal;}
        \SubItem{Any changes to the probability of file loss are negligible;}
        \SubItem{Hive availability, reliability, and confidentiality are not compromised;}
        \SubItem{Hive is resilient to massive node failures and resistant to churn;}
    \item Extra miles may include:
        \SubItem{Peers can reconstruct Hives without Tracker intervention;}
        \SubItem{Peers trust their neighbors independently of Tracker;}
        \SubItem{Hives can leverage dedicated Cloud-Storage if they become unreliable;}
\end{itemize}
%%%%%%%%%%%%%%%%%%%%
% END INTRODUCTION %
%%%%%%%%%%%%%%%%%%%%

%%%%%%%%%%%%%%%%%%%%%%
% BEGIN RELATED WORK %
%%%%%%%%%%%%%%%%%%%%%%
\section{Related Work}\label{sec:relatedwork}
%%%%%%%%%%%%%%%%%%%%
% END RELATED WORK %
%%%%%%%%%%%%%%%%%%%%

%%%%%%%%%%%%%%%%%%%%%%%%%%%
% BEGIN SOLUTION PROPOSAL %
%%%%%%%%%%%%%%%%%%%%%%%%%%%
\section{Solution Proposal}\label{sec:proposal}
%%%%%%%%%%%%%%%%%%%%%%%%%
% END SOLUTION PROPOSAL %
%%%%%%%%%%%%%%%%%%%%%%%%%

%%%%%%%%%%%%%%%%%%%%%%%%%%%%%
% BEGIN TESTING METHODOLOGY %
%%%%%%%%%%%%%%%%%%%%%%%%%%%%%
\section{Testing Methodology}\label{sec:methodology}
%%%%%%%%%%%%%%%%%%%%%%%%%%%
% END TESTING METHODOLOGY %
%%%%%%%%%%%%%%%%%%%%%%%%%%%

%%%%%%%%%%%%%%%%%%%%%%%
% BEGIN WORK SCHEDULE %
%%%%%%%%%%%%%%%%%%%%%%%
\section{Work Schedule}\label{sec:workschedule}
%%%%%%%%%%%%%%%%%%%%%
% END WORK SCHEDULE %
%%%%%%%%%%%%%%%%%%%%%

%%%%%%%%%%%%%%%%%%%%%
% BEGIN CONCLUSIONS %
%%%%%%%%%%%%%%%%%%%%%
\section{Conclusions}\label{sec:conclusion}
%%%%%%%%%%%%%%%%%%%
% END CONCLUSIONS %
%%%%%%%%%%%%%%%%%%%

%%%%%%%%%%%%%%%%%%%%
% BEGIN REFERENCES %
%%%%%%%%%%%%%%%%%%%%
\bibliographystyle{splncs04}
\bibliography{bibliography}
%%%%%%%%%%%%%%%%%%
% END REFERENCES %
%%%%%%%%%%%%%%%%%%

\end{document}
%%%%%%%%%%%%%%%%
% END DOCUMENT %
%%%%%%%%%%%%%%%%
